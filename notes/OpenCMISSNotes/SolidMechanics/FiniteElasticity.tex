\section{Finite Elasticity}
\label{sec:FiniteElasticity}

%Deformation can be viewed three ways: A point to point transformation; A
%coordinate transformation; or as a transformation of metrics (convected coordinates).

\subsection{Kinematics}

As shown in \figref{fig:configurationsetting}, consider a \emph{material
  body} which is a three-dimensional smooth manifold with a boundary, $\manifold{B}$,
which consists of a set of points which are refered to as \emph{material
  points}. Consider also an ambient space manifold,
$\manifold{S}\in\rntopology{n}$. The material body is only accessible to
the observer when it moves through the ambient space. This motion is a
time-dependent embedding on the material body into the ambient space. The
embedding is known as a \emph{placement of the body}. It is given by the
mapping
\begin{equation}
  \mapping{\fnof{\kappa}{\mathcal{X},t}}{\manifold{B}}{\manifold{S}}
\end{equation}

The embedded submanifold occupying a location in the ambient space is is
called a \emph{configuration} of $\manifold{B}$ and is given by
\begin{equation}
  \embedmanifold{B}_{t}=\fnof{\kappa_{t}}{\manifold{B}}=\fnof{\kappa}{\manifold{B},t}
\end{equation}

The customary (but not necessary) \emph{reference placement} is given by
\begin{equation}
  \mapping{\kappa_{0}}{\manifold{B}}{\manifold{S}}
\end{equation}
and the region of space occupied by the reference placement \ie the
\emph{reference configuration} is given by
\begin{equation}
  \embedmanifold{B}_{0}=\fnof{\kappa_{0}}{\manifold{B}}
\end{equation}
Points in $\embedmanifold{B}_{0}$ are denoted by capital letters \ie $X, Y,
\ldots$. Points in $\embedmanifold{B}$ are denoted by lower case leters \ie
$x, y, \dots$.

\epstexfigure{SolidMechanics/svgs/setup.eps_tex}{}{}{fig:configurationsetting}{0.75}

A new configuration of $\manifold{B}$ is given by the deformation mapping
\begin{equation}
  \mapping{\chi}{\embedmanifold{B}}{\rntopology{3}}
\end{equation}
where a configuration represents a deformed state of the body. As the body
moves we obtain a family of configurations. If we hold $X\in\embedmanifold{B}$
fixed can write $\fnof{V_{t}}{X}=\fnof{V}{X,t}$. We then have
\begin{equation}
  \fnof{V_{t}}{X}=\fnof{V}{X,t}=\delby{\fnof{\chi}{X,t}}{t}=\dby{\fnof{\chi_{X}}{t}}{t}
\end{equation}

Here $V_{t}$ is called the \emph{material velocity} of the motion. The
\emph{material acceleration} of the body is defined as
\begin{equation}
  \fnof{A_{t}}{X}=\fnof{A}{X,t}=\delby{\fnof{V}{X,t}}{t}=\dby{\fnof{V_{X}}{t}}{t}
\end{equation}

The \emph{spatial velocity} of the motion is defined by $v_{t}$ and the
\emph{spatial acceleration} of the motion is defined by $a_{t}$.

FIX BELOW

Consider the following line, area and volume forms in the reference configuration given by
\begin{align}
  \exteriorderiv{\covectr{X}}&=\sqrt{\determinant{\materialmetrictensorsymbol_{IJ}}}dX^{1} \\
  \exteriorderiv{V}&=\sqrt{\determinant{\materialmetrictensorsymbol_{IJ}}}\wedgeprod{\wedgeprod{dX^{1}}{dX^{2}}}{dX^{3}}
\end{align}
the corresponding volume form in the current configuration
\begin{equation}
  \exteriorderiv{v}=\sqrt{\determinant{\spatialmetrictensorsymbol_{ij}}}\wedgeprod{\wedgeprod{dx^{1}}{\cdots}}{dx^{n}}
\end{equation}
is given by
\begin{equation}
  \pullback{\chi}{\exteriorderiv{v}}=J\exteriorderiv{V}
\end{equation}
where $J$ is the \emph{Jacobian} of the mapping and is given by
\begin{equation}
  \fnof{J}{X}=\sqrt{\dfrac{\determinant{\spatialmetrictensorsymbol_{ij}}}{\determinant{\materialmetrictensorsymbol_{IJ}}}}\determinant{\pbrac{\delby{\fnof{\chi^{i}}{X}}{X^{I}}}}
\end{equation}

\subsection{Deformation Gradient}

Let
$\mapping{\chi}{\embedmanifold{B}_{0}}{\fnof{\chi}{\embedmanifold{B}_{0}}\subset\manifold{S}}$
be a deformation configuration of $\embedmanifold{B}$ in $\manifold{S}$. The
tangent of the mapping \ie $\tangentbundle{\chi}$ is denoted as $\deformationgradienttensor$
and is called the \emph{deformation gradient} of $\chi$ \ie
$\deformationgradienttensor=\tangentbundle{\chi}$. For $X\in\embedmanifold{B}_{0}$ we
have\symbolat{$\deformationgradienttensor$}{Deformation gradient tensor}
\begin{equation}
  \deformationgradienttensor_{X}=\mapping{\fnof{\deformationgradienttensor}{X}}{\tangentspace{\embedmanifold{B}_{0}}{X}}{\tangentspace{\manifold{S}}{\fnof{\chi}{X}}}
\end{equation}
 
If $X^{I}$ and $x^{i}$ are the coordinates on $\embedmanifold{B}_{0}$ and
$\manifold{S}$ then the deformation gradient tensor with respect to the
coordinate bases are
\begin{equation}
  \fnof{\deformationgradienttensor}{X}=\fnof{\deformationgradienttensorsymbol^{i}_{I}}{X}\tensorprod{\spatialbasevector_{i}}{\materialbasevector^{I}}=\delby{\fnof{\chi^{i}}{X}}{X^{I}}\tensorprod{\spatialbasevector_{i}}{\materialbasevector^{I}}
\end{equation}

Note that $\deformationgradienttensor$ is a two-point tensor. 

The polar decomposition

\begin{diagram}
 & & \tangentspace{B}{X} & & \\
 & \ruTo^{\rightstretchtensor} & & \rdTo^{\tensor{R}} \\
\tangentspace{B}{X} & & \rTo^{\deformationgradienttensor} & & \tangentspace{S}{x}\\
 & \rdTo_{\tensor{R}} & & \ruTo_{\leftstretchtensor} \\
 & &  \tangentspace{S}{x} & &
\end{diagram}

is given by
\begin{equation}
  \deformationgradienttensor=\dotprod{\tensor{R}}{\rightstretchtensor}=\dotprod{\leftstretchtensor}{\tensor{R}}
\end{equation}
where $\rightstretchtensor$ is the \emph{right stretch tensor}, $\leftstretchtensor$ is the
\emph{left stretch tensor} and $\tensor{R}$ is the \emph{rotation
  tensor}.\symbolat{$\rightstretchtensor$}{Right stretch
  tensor}\symbolat{$\leftstretchtensor$}{Left stretch
  tensor}\symbolat{$\tensor{R}$}{Rotation tensor}

If we let the deformed coordinates be given by the position vector,
$\fnof{\vectr{z}}{\vectr{x},t}$ then the deformation gradient tensor with
respect to the undeformed $\vectr{X}$ coordinates is given by
\begin{equation}
  \fnof{\deformationgradienttensor}{\vectr{X}}=\delby{\vectr{z}}{\vectr{X}}
\end{equation} 
or, in component form,
\begin{equation}
  \deformationgradienttensorsymbol^{i}_{I}=\delby{z^{i}}{X^{I}}=\delby{z^{i}}{\xi^{r}}\delby{\xi^{r}}{X^{I}}
\end{equation}

The deformation gradient tensor can also be used to map lines, area and volume forms
between the reference and current configurations and vice versa \ie
\begin{align}
  \exteriorderiv{\covectr{x}} &= \deformationgradienttensor\exteriorderiv{\covectr{X}} \\
  \exteriorderiv{\covectr{a}} &= J\invtranspose{\deformationgradienttensor}\exteriorderiv{\covectr{A}} \\
  \exteriorderiv{v} &= J\exteriorderiv{V}
\end{align}
where $\exteriorderiv{\covectr{X}}$, $\exteriorderiv{\covectr{A}}$ and
$\exteriorderiv{V}$ are the line, area and volume forms in the reference
configuration and $\exteriorderiv{\covectr{x}}$, $\exteriorderiv{\covectr{a}}$ and
$\exteriorderiv{v}$ are the line, area and volume forms in the current
configuration. $J$ is the \emph{Jacobian} and is given by
\begin{equation}
  J=\sqrt{\dfrac{\determinant{\tensor{g}}}{\determinant{\materialmetrictensor}}}\determinant{\deformationgradienttensor}
\end{equation}

The formula for mapping areas is known as \emph{Nanson's formula} and is
given by
\begin{equation}
  \covectr{n}\exteriorderiv{\covectr{a}}=J\pushforward{\chi}{\covectr{N}\exteriorderiv{\covectr{A}}}
\end{equation}
or
\begin{equation}
  \covectr{n}\exteriorderiv{\covectr{a}}=J\invtranspose{\deformationgradienttensor}\covectr{N}\exteriorderiv{\covectr{A}}
\end{equation}
where $\covectr{N}$ and $\exteriorderiv{\covectr{A}}$ are the unit normal and
area form in the reference configuration,  $\covectr{n}$ and $\exteriorderiv{\covectr{a}}$ are the unit normal and
area form in the current configuration and $J$ is the Jacobian. 
  
\subsection{Strain and deformation tensors}

Strain and deformation tensors quantify the amount of strain or deformation
\ie the amount of stretch or distance between material points.  

There are three strain tensors in the \emph{material} coordinate system. The
\emph{right Cauchy-Green (or Green) deformation tensor}, $\rightcauchygreentensor$, is
defined by\symbolat{$\tensortwo{C}$}{Right Cauchy-Green (or Green) deformation
  tensor}
\begin{equation}
  \mapping{\fnof{\rightcauchygreentensor}{X}}{\tangentspace{\embedmanifold{B}}{X}}{\tangentspace{\embedmanifold{B}}{X}}
\end{equation}
as the pullback of the spatial metric tensor \ie
$\fnof{\rightcauchygreentensor}{X}=\transpose{\fnof{\deformationgradienttensor}{X}}\fnof{\spatialmetrictensor}{x}
\fnof{\deformationgradienttensor}{X}$ or
$\rightcauchygreentensor=\transpose{\deformationgradienttensor}\spatialmetrictensor\deformationgradienttensor$
where $x=\fnof{\chi}{X}$. In terms of coordinates we have
\begin{equation}
  \rightcauchygreentensor=\rightcauchygreentensorsymbol_{IJ}
  \tensorprod{\materialbasevector^{I}}{\materialbasevector^{J}}=
  \spatialmetrictensorsymbol_{ij}\deformationgradienttensorsymbol^{i}_{I}\deformationgradienttensorsymbol^{j}_{J}\,
  \dotprodthree{\tensorprod{\materialbasevector^{I}}{\spatialbasevector_{i}}}{\tensorprod{\spatialbasevector^{i}}{
      \spatialbasevector^{j}}}{\tensorprod{\spatialbasevector_{j}}{\materialbasevector^{J}}}=
  \transpose{\deformationgradienttensor}\spatialmetrictensor\deformationgradienttensor
\end{equation}

If $\rightcauchygreentensor$ is invertible we also have $\pioladeformationtensor=\inverse{\rightcauchygreentensor}$
where $\pioladeformationtensor$ is called the \emph{Piola deformation
  tensor}.\symbolat{$\pioladeformationtensor$}{Piola deformation tensor}

We also have the \emph{Green-Lagrange strain tensor} is given by the difference
in metric tensors\symbolat{$\greenlagrangestraintensor$}{Green-Lagrange strain tensor}
\begin{equation}
  \fnof{\greenlagrangestraintensor}{X}=\frac{1}{2}\pbrac{\fnof{\rightcauchygreentensor}{X}-\materialmetrictensor}
\end{equation}
of, in component form
\begin{equation}
  \greenlagrangestraintensor=\greenlagrangestraintensorsymbol_{IJ}\tensorprod{\materialbasevector^{I}}{\materialbasevector^{J}}=\frac{1}{2}\pbrac{\rightcauchygreentensorsymbol_{IJ}-\materialmetrictensorsymbol_{IJ}}\tensorprod{\materialbasevector^{I}}{\materialbasevector^{J}}
\end{equation}

There are also three strain tensors in the \emph{spatial} coordinate
sytsem. The \emph{left Cauchy-Green (or Finger) deformation tensor},
$\leftcauchygreentensor$, is defined by\symbolat{$\leftcauchygreentensor$}{Left Cauchy-Green (or
  Finger) deformation tensor}
\begin{equation}
  \mapping{\fnof{\leftcauchygreentensor}{x}}{\tangentspace{\fnof{\chi}{\embedmanifold{B}}}{x}}{\tangentspace{\fnof{\chi}{\embedmanifold{B}}}{x}}
\end{equation}
as the push forward of the material metric tensor \ie
$\fnof{\leftcauchygreentensor}{x}=\fnof{\deformationgradienttensor}{X}\fnof{\materialmetrictensor}{X}\transpose{\fnof{\deformationgradienttensor}{X}}$
or $\leftcauchygreentensor=\deformationgradienttensor\materialmetrictensor\transpose{\deformationgradienttensor}$ where
$X=\fnof{\inverse{\chi}}{x}$. In terms of coordinates we have
\begin{equation}
  \leftcauchygreentensor=\leftcauchygreentensorsymbol^{ij}\tensorprod{\spatialbasevector_{i}}{\spatialbasevector_{j}}=\materialmetrictensorsymbol^{IJ}\deformationgradienttensorsymbol^{i}_{I}\deformationgradienttensorsymbol^{j}_{J}\dotprodthree{\tensorprod{\spatialbasevector_{i}}{\materialbasevector^{I}}}{\tensorprod{\materialbasevector_{I}}{\materialbasevector_{J}}}{\tensorprod{\materialbasevector^{J}}{\spatialbasevector_{j}}}=\deformationgradienttensor\materialmetrictensor\transpose{\deformationgradienttensor}
\end{equation}

The left and right Cauchy-Green deformation tensors get their left and right
names from their relationship to the left and right stretch tensors \ie in
cartesian coordinates
\begin{equation}
  \rightstretchtensor=\sqrt{\rightcauchygreentensor}
\end{equation}
and
\begin{equation}
  \leftstretchtensor=\sqrt{\leftcauchygreentensor}
\end{equation}

We also have $\tensor{c}=\inverse{\leftcauchygreentensor}$ where $\tensor{c}$ is the
\emph{Cauchy deformation tensor} \ie \symbolat{$\tensor{c}$}{Cauchy deformation tensor}
\begin{equation}
	  \tensor{c}=c_{ij}\tensorprod{\spatialbasevector^{i}}{\spatialbasevector^{j}}
\end{equation}

The final spatial strain tensor is the \emph{Euler-Almansi strain tensor},
$\tensor{e}$, is defined as the difference in metrics
by\symbolat{$\tensor{e}$}{Euler-Almansi strain tensor}
\begin{equation}
  \mapping{\fnof{\tensor{e}}{x}}{\tangentspace{\fnof{\chi}{\embedmanifold{B}}}{x}}{\tangentspace{\fnof{\chi}{\embedmanifold{B}}}{x}}
\end{equation}
and
\begin{equation}
  \begin{split}
    \fnof{\tensor{e}}{x}&=\frac{1}{2}\pbrac{\tensor{g}-\inverse{\fnof{\tensor{b}}{x}}}\\
    &= \frac{1}{2}\pbrac{\tensor{g}-\fnof{\tensor{c}}{x}}
  \end{split}
\end{equation}
where $\tensor{c}=\inverse{\tensor{b}}$. In component form we have
\begin{equation}
  \tensor{e}=e_{ij}\tensorprod{\spatialbasevector^{i}}{\spatialbasevector^{j}}=\frac{1}{2}\pbrac{\spatialmetrictensorsymbol_{ij}-c_{ij}}\tensorprod{\spatialbasevector^{i}}{\spatialbasevector^{j}}
\end{equation}

Note that we have the following relationships between the deformation tensors
and metric tensors
\begin{alignat}{7}
  \flattensor{\rightcauchygreentensor}&=&\pullback{\chi}{\flattensor{\tensor{g}}} &&\quad&
  \flattensor{\tensor{c}}&=&\pushforward{\chi}{\flattensor{\materialmetrictensor}} \\
  \sharptensor{\pioladeformationtensor}&=&\pullback{\chi}{\sharptensor{\tensor{g}}} &&\quad&
  \sharptensor{\tensor{b}}&=&\pushforward{\chi}{\sharptensor{\materialmetrictensor}} \\
  \flattensor{\greenlagrangestraintensor}&=&\pullback{\chi}{\flattensor{\tensor{e}}} &&\quad&
  \flattensor{\tensor{e}}&=&\pushforward{\chi}{\flattensor{\greenlagrangestraintensor}} \\ 
  &=&\frac{1}{2}\pbrac{\pullback{\chi}{\flattensor{\tensor{g}}}-\flattensor{\materialmetrictensor}}&&\quad&
  &=&\frac{1}{2}\pbrac{\flattensor{\tensor{g}}-\pushforward{\chi}{\flattensor{\materialmetrictensor}}}
\end{alignat}
\ie
\begin{alignat}{7}
  \flattensor{\rightcauchygreentensor}&=&\transpose{\deformationgradienttensor}\flattensor{\tensor{g}}\deformationgradienttensor &\quad&
  \flattensor{\tensor{c}}&=&\invtranspose{\deformationgradienttensor}\flattensor{\materialmetrictensor}\inverse{\deformationgradienttensor} \\
  \sharptensor{\pioladeformationtensor}&=&\inverse{\deformationgradienttensor}\sharptensor{\tensor{g}}\invtranspose{\deformationgradienttensor} &\quad&
  \sharptensor{\tensor{b}}&=&\deformationgradienttensor\sharptensor{\materialmetrictensor}\transpose{\deformationgradienttensor} \\
  \flattensor{\greenlagrangestraintensor}&=&\transpose{\deformationgradienttensor}\flattensor{\tensor{e}}\deformationgradienttensor &\quad&
  \flattensor{\tensor{e}}&=&\invtranspose{\deformationgradienttensor}\flattensor{\greenlagrangestraintensor}\inverse{\deformationgradienttensor}   
\end{alignat}
or, in component form,
\begin{alignat}{3}
  C_{IJ}=\spatialmetrictensorsymbol_{ij}F^{i}_{I}F^{j}_{J} &\quad& 
  c_{ij}=\materialmetrictensorsymbol_{IJ}\pbrac{\inverse{F}}^{I}_{i}\pbrac{\inverse{F}}^{J}_{j} \\
  B^{IJ}=\spatialmetrictensorsymbol^{ij}\pbrac{\inverse{F}}^{I}_{i}\pbrac{\inverse{F}}^{J}_{j} &\quad&
  b^{ij}=\materialmetrictensorsymbol^{IJ}F^{i}_{I}F^{j}_{J} \\
  E_{IJ}=e_{ij}F^{i}_{I}F^{j}_{J} &\quad&
  e_{ij}=E_{IJ}\pbrac{\inverse{F}}^{I}_{i}\pbrac{\inverse{F}}^{J}_{j}   
\end{alignat}

\subsection{Stress tensors}

Stress is amount of force over an area. We can thus form a number of different
stress tensors depending on what configuration we base the force and the area
in.

The \emph{Cauchy stress tensor}\symbolat{$\tensortwo{\sigma}$}{Cauchy stress tensor}, $\cauchystresstensor$, has both the force and
area referred to the \emph{spatial} configuration. The Cauchy stress tensor is
important as it quantifies the physical stress that exists in current
configuration and can be measured. It is defined by
\begin{equation}
  \mapping{\fnof{\cauchystresstensor}{x}}{\spaceprod{\tangentspace{\embedmanifold{B}}{x}}{\tangentspace{\embedmanifold{B}}{x}}}{\realnums}
\end{equation}
\ie
\begin{equation}
  \fnof{\cauchystresstensor}{x}=\sigma^{ij}\tensorprod{\spatialbasevector_{i}}{\spatialbasevector_{j}}
\end{equation}

Cauchy's law....

We can also define the \emph{Kirchoff stress tensor}\symbolat{$\tensortwo{\tau}$}{Kirchoff stress tensor}, $\tensor{\tau}$, by
scaling the Cauchy stress by the Jacobian \ie
\begin{equation}
  \tensor{\tau}=J\cauchystresstensor
\end{equation}

To find other stress measures we need to make use of Piola's transform and
idenity.

The \emph{Piola transform} allows mapping of vector fields on the current
configuration to equivalent vector fields in the reference configuration. The
transform is given by
\begin{equation}
  \vectr{Y}=J\pullback{\chi}{\vectr{y}}=J\inverse{\deformationgradienttensor}\vectr{y}
\end{equation}
where $\vectr{y}$ is the vector field on the current configuration and
$\vectr{Y}$ is the equivalent vector field on the reference configuration. In
component form we have
\begin{equation}
  Y^{I}=J\pbrac{\inverse{F}}^{I}_{i}y^{i}
\end{equation}

Now if $\vectr{Y}$ is the Piola transform of $\vectr{y}$ then we have the
\emph{Piola identity} given by
\begin{equation}
  \Divop{\vectr{Y}}=J\divop{\circcomposition{\vectr{y}}{\chi}}
\end{equation}
where $\Divop$ is the divergence operator in the reference configuration and
$\divop$ is the divergence operator in the current configurtion. 

Using the Piola transform we can construct two new stress tensors. If we apply
the Piola transform to the second index of the Cauchy stress tensor we obtain
the \emph{First Piola-Kirchoff Stress Tensor}\symbolat{$\tensortwo{P}$}{First Piola-Kirchoff stress tensor} \ie
\begin{equation}
  \tensor{P}=P^{iI}\tensorprod{\spatialbasevector_{i}}{\materialbasevector_{I}}=J\pbrac{\inverse{F}}^{I}_{j}\sigma^{ij}\dotprod{\tensorprod{\spatialbasevector_{i}}{\spatialbasevector_{j}}}{\tensorprod{\spatialbasevector^{j}}{\materialbasevector_{I}}}=J\cauchystresstensor\invtranspose{\deformationgradienttensor}
\end{equation}

The first Piola-Kirchoff stress tensor relates forces in the current
configuration with areas in the reference configuration. Note that
$\tensor{P}$ is a two point tensor and is not symmetric.

The relationship between Cauchy stress and the First Piola-Kirchoff stress is
given by
\begin{equation}
  \cauchystresstensor=\inverse{J}\tensor{P}\transpose{\deformationgradienttensor}
\end{equation}

If we now apply the Piola transform to the first index of the First Piola-Kirchoff stress tensor we obtain
the \emph{Second Piola-Kirchoff Stress Tensor}\symbolat{$\tensortwo{S}$}{Second Piola-Kirchoff stress tensor} \ie
\begin{equation}
  \secondpiolakirchoffstresstensor=\inverse{\deformationgradienttensor}\tensor{P}
\end{equation}
or, in terms of the Cauchy stress,
\begin{equation}
  \secondpiolakirchoffstresstensor=S^{IJ}\tensorprod{\materialbasevector_{I}}{\materialbasevector_{J}}=J\pbrac{\inverse{F}}^{I}_{i}\pbrac{\inverse{F}}^{J}_{j}\sigma^{ij}\dotprod{\dotprod{\tensorprod{\materialbasevector_{I}}{\spatialbasevector^{i}}}{\tensorprod{\spatialbasevector_{i}}{\spatialbasevector_{j}}}}{\tensorprod{\spatialbasevector^{j}}{\materialbasevector_{J}}}=J\inverse{\deformationgradienttensor}\cauchystresstensor\invtranspose{\deformationgradienttensor}
\end{equation}

The second Piola-Kirchoff stress tensor relates forces in the reference
configuration with areas in the reference configuration.

Note that the Kirchoff stress is just the push forward of the second
Piola-Kirchoff stress \ie
\begin{equation}
  \tensor{\tau}=\deformationgradienttensor\secondpiolakirchoffstresstensor\transpose{\deformationgradienttensor}
\end{equation}
and
\begin{equation}
  \secondpiolakirchoffstresstensor=\inverse{\deformationgradienttensor}\tensor{\tau}\invtranspose{\deformationgradienttensor}
\end{equation}

BIOT STRESS

COVECTED STRESSS

RESTRICTIONS ON STRESS TENSOR E.G. SYMMETRY

The relationships between the stress tensors are given in \Tabref{tab:RelationshipBetweenStressTensors}.

\begin{table}[htb] \centering
  \begin{tabular}{|c|c|c|c|c|} \hline
    & $\cauchystresstensor$ & $\tensor{\tau}$ & $\tensor{P}$ & $\secondpiolakirchoffstresstensor$
    \\ \hline \hline
    $\cauchystresstensor$ & - & $\inverse{J}\tensor{\tau}$ &
    $\inverse{J}\tensor{P}\transpose{\deformationgradienttensor}$ &
    $\inverse{J}\deformationgradienttensor\secondpiolakirchoffstresstensor\transpose{\deformationgradienttensor}$ \\
    $\tensor{\tau}$ & $J\cauchystresstensor$ & - &
    $\tensor{P}\transpose{\deformationgradienttensor}$ &
    $\deformationgradienttensor\secondpiolakirchoffstresstensor\transpose{\deformationgradienttensor}$ \\
    $\tensor{P}$ & $J\cauchystresstensor\invtranspose{\deformationgradienttensor}$ &
    $\tensor{\tau}\invtranspose{\deformationgradienttensor}$ & - & $\deformationgradienttensor\secondpiolakirchoffstresstensor$ \\
    $\secondpiolakirchoffstresstensor$ &
    $J\inverse{\deformationgradienttensor}\cauchystresstensor\invtranspose{\deformationgradienttensor}$ &
    $\inverse{\deformationgradienttensor}\tensor{\tau}\invtranspose{\deformationgradienttensor}$ &
    $\inverse{\deformationgradienttensor}\tensor{P}$ & - \\ \hline
  \end{tabular}
  \caption{Relationships between stress tensors.}
  \label{tab:RelationshipBetweenStressTensors}
\end{table}

\subsection{Anisotropy}
\label{subsec:FiniteElasticityAnisotropy}

\subsubsection{Preferred Directions}
\label{subsubsec:FiniteElasticityAnisotropyPreferredDirections}

Most biological materials are, to an extent,
\emph{anisotropic}\index{anisotopy} \ie their material properties have
preferred directions. To deal with anisotropy \OpenCMISS defines an
anisotropic coordinate system which defines the directions of
anisotropy. From \secref{sec:MaterialCoordinateTransformations} the
principle axes of anisotopy are that of the fibre, sheet and
sheet-normal directions.

For solid mechanics problems we most consider whether or not we are
dealing with material or reference coordinate manifold or the spatial
or deformed coordinate manifold. The choice of the coordinate manifold
has important implications for the anisotropic coordinate system. In
the undeformed material coordinate system it is realitively easy to
define an orthogonal fibre, sheet, sheet-normal coordinate
system. However, under deformation, this orthogonal coordinate system
is distorted in the spatial or deformed coordiantes and is not
necessarily orthogonal. To deal with tensor properties it is far
easier to deal with orthogonal coordinates and so an orthogonal
coordinate system for defining anisotropy needs to be defined in
spatial or deformed coordinates.

To define an orthogonal deformed anisotropic fibre coordinate system
consider the deformation of a block of fibrous tissue as shown in
\figref{fig:FiniteElasticityDeformedFibres}.

\epstexfigure{SolidMechanics/svgs/DeformedFibres.eps_tex}{Defomed
  fibre coordinate system.}{Deformed fibre coordinate system. A block
  of fibrous tissue in material, $\materialcoordinatevector$,
  coordinates undergoes deformation via the deformation gradient
  tensor, $\deformationgradienttensor$, to a block of fibrous tissue
  in the spatial coordinate system, $\spatialcoordinatevector$. The
  undeformed block of tissue is defined by the orthogonal elemental
  coordinates given by the unit vectors,
  $\materialxicoordinatevector_{1}$,
  $\materialxicoordinatevector_{2}$, and
  $\materialxicoordinatevector_{3}$. From these orthogonal elemental
  coordinates the undeformed orthogonal fibre coordinate system can be
  found from the fibre, sheet and sheet-normal angles. The orthogonal
  undeformed fibre coordinate system is given by the unit vectors,
  $\materialfibrecoordinatevector_{1}$,
  $\materialfibrecoordinatevector_{2}$, and
  $\materialfibrecoordinatevector_{3}$. In the spatial coordinates the
  block deforms to give the (not necessarily unit or orthogonal)
  elemental coordinate system given by the vectors,
  $\spatialxicoordinatevector_{1}$, $\spatialxicoordinatevector_{2}$,
  and $\spatialxicoordinatevector_{3}$. The undeformed fibre
  coordinate system also deforms to give the (not necessarily unit or
  orthogonal) deformed fibre coordinate system given by the vectors,
  $\spatialfibrecoordinatevector^{'}_{1}$,
  $\spatialfibrecoordinatevector^{'}_{2}$, and
  $\spatialfibrecoordinatevector^{'}_{3}$. An unit orthogonal deformed
  fibre coordinate system can be constructed in spatial coordinates
  from the deformed fibre coordinates. Firstly, the unit orthogonal
  sheet-normal vector, $\spatialfibrecoordinatevector_{3}$, is
  constructed as unit and orthogonal to
  $\spatialfibrecoordinatevector^{'}_{1}$ and
  $\spatialfibrecoordinatevector^{'}_{2}$. Secondly, an unit
  orthogonal sheet vector, $\spatialfibrecoordinatevector_{2}$, is
  constructed as unit and orthogonal to
  $\spatialfibrecoordinatevector^{'}_{1}$ and
  $\spatialfibrecoordinatevector_{3}$. Finally, a unit vector,
  $\spatialfibrecoordinatevector_{1}$ can be found from normalising
  the vector $\spatialfibrecoordinatevector^{'}_{1}$. The unit
  orthogonal deformed fibre coordinates are then given by the vectors,
  $\spatialfibrecoordinatevector_{1}$,
  $\spatialfibrecoordinatevector_{2}$, and
  $\spatialfibrecoordinatevector_{3}$.}{fig:FiniteElasticityDeformedFibres}{0.66}

In \figref{fig:FiniteElasticityDeformedFibres} the material or reference geometric coordinate vector is given by
\begin{equation}
  \materialcoordinatevector=\materialcoordinatesymbol^{I}\materialbasevector_{I}
\end{equation}

From \secref{sec:MaterialCoordinateTransformations} we can define a
unit orthogonal elemental coordinate system
$\fnof{\materialxicoordinatevector_{R}}{\materialcoordinatevector}$
defined with respect to the material or undeformed coordinate system
\ie
\begin{equation}
  \fnof{\materialxicoordinatevector_{R}}{\materialcoordinatevector}=
  \fnof{\materialxicoordinatesymbol^{I}_{R}}{\materialcoordinatevector}\materialbasevector_{I}
\end{equation}
and then a unit orthogonal fibre coordinate system defined by the unit
vectors $\fnof{\materialfibrecoordinatevector_{A}}{\materialcoordinatevector}$ which are defined with
respect to the material or undeformed coordinate system \ie
\begin{equation}
  \fnof{\materialfibrecoordinatevector_{A}}{\materialcoordinatevector}=
  \fnof{\materialfibrecoordinatesymbol^{I}_{A}}{\materialcoordinatevector}\materialbasevector_{I}
\end{equation}

It is often useful to deal with physical material properties with
regard to the fibre coordinate system rather than the material or
undeformed coordinate system \eg material property tensors are often
diagonal in the fibre coordinate system. The material coordinate
vector in material or reference fibre coordinates,
$\tilde{\materialfibrecoordinatevector}$, is given by
\begin{equation}
  \tilde{\materialfibrecoordinatevector}=
  \tilde{\materialfibrecoordinatesymbol}^{A}\materialfibrecoordinatevector_{A}
\end{equation}
where the $\tilde{.}$ indicates that the quantity is with respect to fibre coordinates.

The transformation between material or reference coordinates and material fibre coordinates is given by
\begin{equation}
  \tilde{\materialfibrecoordinatevector}=\tilde{\materialfibrecoordinatesymbol}^{A}\materialfibrecoordinatevector_{A}=
  \tensortwo{Q}\materialcoordinatevector=Q^{A}_{I}\materialcoordinatesymbol^{I}
  \dotprod{\tensorprod{\materialfibrecoordinatevector_{A}}{\materialbasevector^{I}}}{\materialbasevector_{I}}
\end{equation}
where $\materialfibrecoordinatevector_{A}$ are the material fibre
coordinate base vectors and $\tensortwo{Q}$ is the tensor
tranformation between the material or reference coordinate system and
the material fibre coordinate system \ie
\begin{equation}
  \tensortwo{Q}=Q^{A}_{I}\tensorprod{\materialfibrecoordinatevector_{A}}{\materialbasevector^{I}}=
  \delby{\tilde{\materialfibrecoordinatesymbol}^{A}}{\materialxicoordinatesymbol^{R}}
  \delby{\materialxicoordinatesymbol^{R}}{\materialcoordinatesymbol^{I}}
  \tensorprod{\materialfibrecoordinatevector_{A}}{\materialbasevector^{I}}
\end{equation}

Now, the material or undeformed block of fibrous tissue in
\figref{fig:FiniteElasticityDeformedFibres} undergoes deformation as
defined by the deformation gradient tensor,
$\fnof{\deformationgradienttensor}{\materialcoordinatevector}$, such
that a point in the material or reference manifold,
$\materialcoordinatevector$, is mapped to a point in the spatial or
deformed manifold, $\spatialcoordinatevector$, where
\begin{equation}
  \spatialcoordinatevector=\spatialcoordinatesymbol^{i}\spatialbasevector_{i}
\end{equation}
and the mapping is given by
\begin{equation}
  \spatialcoordinatevector=\spatialcoordinatesymbol^{i}\spatialbasevector_{i}=
  \fnof{\deformationgradienttensor}{\materialcoordinatevector}\materialcoordinatevector
  =\fnof{\deformationgradienttensorsymbol^{i}_{I}}{\materialcoordinatevector}\materialcoordinatesymbol^{I}
  \dotprod{\tensorprod{\spatialbasevector_{i}}{\materialbasevector^{I}}}{\materialbasevector_{I}}
\end{equation}

We can also define the deformation gradient tensor as a map from the
material fibre coordinates to the spatial deformed coordinates \ie
\begin{equation}
  \fnof{\tilde{\deformationgradienttensor}}{\tilde{\materialfibrecoordinatevector}}=
  \fnof{\tilde{\deformationgradienttensorsymbol}^{i}_{A}}{\tilde{\materialfibrecoordinatevector}}
  \tensorprod{\spatialbasevector_{i}}{\materialfibrecoordinatevector^{A}}=
  \fnof{\deformationgradienttensor}{\materialcoordinatevector}\transpose{\tensortwo{Q}}=
  \fnof{\deformationgradienttensorsymbol^{i}_{I}}{\materialcoordinatevector}\pbrac{\transpose{Q}}^{I}_{A}
  \dotprod{\tensorprod{\spatialbasevector_{i}}{\materialbasevector^{I}}}{\tensorprod{\materialbasevector_{I}}{\materialfibrecoordinatevector^{A}}}
\end{equation}

Note that
$\fnof{\deformationgradienttensor}{\materialcoordinatevector}$ is a
two-point tensor and so transforms as
$\tensortwo{Q}\fnof{\deformationgradienttensor}{\materialcoordinatevector}$.

Also note that in \OpenCMISS we have
\begin{equation}
  \begin{aligned}
    \tilde{\deformationgradienttensorsymbol}^{i}_{A}&=\delby{\spatialcoordinatesymbol^{i}}{\materialxicoordinatesymbol^{R}}
    \delby{\materialxicoordinatesymbol^{R}}{\materialfibrecoordinatesymbol^{A}}\\
    &=\delby{\materialcoordinatesymbol^{J}}{\materialfibrecoordinatesymbol^{A}}\deformationgradienttensorsymbol^{i}_{J}=
    \delby{\materialcoordinatesymbol^{J}}{\materialfibrecoordinatesymbol^{A}}
    \delby{\materialxicoordinatesymbol^{R}}{\materialcoordinatesymbol^{J}}
    \delby{\spatialcoordinatesymbol^{i}}{\materialxicoordinatesymbol^{R}}\\
    &=\delby{\materialcoordinatesymbol^{J}}{\materialxicoordinatesymbol^{S}}
    \delby{\materialxicoordinatesymbol^{S}}{\materialfibrecoordinatesymbol^{A}}
    \deformationgradienttensorsymbol^{i}_{J}=
    \delby{\materialcoordinatesymbol^{J}}{\materialxicoordinatesymbol^{S}}
    \delby{\materialxicoordinatesymbol^{S}}{\materialfibrecoordinatesymbol^{A}}
    \delby{\materialxicoordinatesymbol^{R}}{\materialcoordinatesymbol^{J}}
    \delby{\spatialcoordinatesymbol^{i}}{\materialxicoordinatesymbol^{R}}
  \end{aligned}
\end{equation}
and
\begin{equation}
  \begin{aligned}
    \deformationgradienttensorsymbol^{i}_{J}&=\delby{\spatialcoordinatesymbol^{i}}{\materialxicoordinatesymbol^{R}}
    \delby{\materialxicoordinatesymbol^{R}}{\materialcoordinatesymbol^{J}}\\
    &=\delby{\materialfibrecoordinatesymbol^{A}}{\materialcoordinatesymbol^{J}}
    \tilde{\deformationgradienttensorsymbol}^{i}_{A}=\delby{\materialfibrecoordinatesymbol^{A}}{\materialcoordinatesymbol^{J}}    
    \delby{\materialxicoordinatesymbol^{R}}{\materialfibrecoordinatesymbol^{A}}
    \delby{\spatialcoordinatesymbol^{i}}{\materialxicoordinatesymbol^{R}}
  \end{aligned}
\end{equation}

Note that in \OpenCMISS if no fibres are defined the the material
fibre coordinate vectors are aligned with the material or reference
coordinate system.

Now, the material fibre coordinate system also undergoes deformation
along with the block of material. The material fibre coordinate system
is mapped under the deformation gradient tensor to the spatial or
deformed fibre coordinate system with respect to spatial or deformed
coordinates \ie
\begin{equation}
  \fnof{\spatialfibrecoordinatevector^{'}_{a}}{\spatialcoordinatevector}=
  \fnof{\spatialfibrecoordinatesymbol^{i'}_{a}}{\spatialcoordinatevector}\spatialbasevector_{i}=
  \fnof{\deformationgradienttensor}{\materialcoordinatevector}\materialfibrecoordinatevector_{a}=
  \fnof{\deformationgradienttensorsymbol^{i}_{I}}{\materialcoordinatevector}\materialfibrecoordinatesymbol^{I}_{a}
  \dotprod{\tensorprod{\spatialbasevector_{i}}{\materialfibrecoordinatevector^{I}}}{\materialfibrecoordinatevector_{I}}
\end{equation}

Now, the spatial or deformed fibre coordinate system,
$\fnof{\spatialfibrecoordinatevector^{'}_{a}}{\spatialcoordinatevector}$, are not necessarily of unit
length or are orthogonal to each other. Dealing with tensor properties
with respect to a non-orthogonal coordinate system can be problematic
and so it will be useful to construct an orthogonal spatial fibre
coordinate system. The first step to construct
$\fnof{\spatialfibrecoordinatevector_{3}}{\spatialcoordinatevector}$, a unit vector orthogonal to
$\fnof{\spatialfibrecoordinatevector^{'}_{1}}{\spatialcoordinatevector}$ and
$\fnof{\spatialfibrecoordinatevector^{'}_{2}}{\spatialcoordinatevector}$ \ie
\begin{equation}
  \fnof{\spatialfibrecoordinatevector_{3}}{\spatialcoordinatevector}=
  \dfrac{\crossprod{\fnof{\spatialfibrecoordinatevector^{'}_{1}}{\spatialcoordinatevector}}{\fnof{\spatialfibrecoordinatevector^{'}_{2}}{\spatialcoordinatevector}}}
        {\norm{\crossprod{\fnof{\spatialfibrecoordinatevector^{'}_{1}}{\spatialcoordinatevector}}{\fnof{\spatialfibrecoordinatevector^{'}_{2}}{\spatialcoordinatevector}}}}
\end{equation}

Next, we construct $\fnof{\spatialfibrecoordinatevector_{2}}{\spatialcoordinatevector}$, a unit vector orthogonal to
$\fnof{\spatialfibrecoordinatevector_{3}}{\spatialcoordinatevector}$ and
$\fnof{\spatialfibrecoordinatevector^{'}_{1}}{\spatialcoordinatevector}$ \ie
\begin{equation}
  \fnof{\spatialfibrecoordinatevector_{2}}{\spatialcoordinatevector}=
  \dfrac{\crossprod{\fnof{\spatialfibrecoordinatevector_{3}}{\spatialcoordinatevector}}{\fnof{\spatialfibrecoordinatevector^{'}_{1}}{\spatialcoordinatevector}}}
        {\norm{\crossprod{\fnof{\spatialfibrecoordinatevector_{3}}{\spatialcoordinatevector}}{\fnof{\spatialfibrecoordinatevector^{'}_{1}}{\spatialcoordinatevector}}}}
\end{equation}

And finally, we can complete the unit orthogonal spatial fibre
coordinate system by normalising
$\fnof{\spatialfibrecoordinatevector^{'}_{1}}{\spatialcoordinatevector}$ \ie
\begin{equation}
  \fnof{\spatialfibrecoordinatevector_{1}}{\spatialcoordinatevector}=
  \norm{\fnof{\spatialfibrecoordinatevector^{'}_{1}}{\spatialcoordinatevector}}
\end{equation}

Now, as with the material coordinate case, it is useful to define
tensors with respect to the spatial fibre coordinate system as they
may have useful properties with respect this coordinate system \eg a
diagonal tensor. We can define the spatial fibre coordinate vector as
\begin{equation}
  \tilde{\spatialfibrecoordinatevector}=\tilde{\spatialfibrecoordinatesymbol}^{a}\spatialfibrecoordinatevector_{a}
\end{equation}

The transformation between the spatial or deformed coordinates and the spatial fibre coordinates is given by
\begin{equation}
  \tilde{\spatialfibrecoordinatevector}=\tilde{\spatialfibrecoordinatesymbol}^{a}\spatialfibrecoordinatevector_{a}=
  \tensortwo{q}\spatialcoordinatevector=q^{a}_{i}\spatialcoordinatesymbol^{i}
  \dotprod{\tilde{\spatialfibrecoordinatevector_{a}}{\spatialbasevector^{i}}}{\spatialbasevector_{i}}
\end{equation}
where $\spatialfibrecoordinatevector_{a}$ are the spatial
fibre coordinate base vectors (with respect to the spatial fibre
coordinate system) and $\tensortwo{q}$ is the tensor tranformation
between the spatial or deformed coordinate system and the spatial
fibre coordinate system \ie
\begin{equation}
  \tensortwo{q}=q^{a}_{i}\tensorprod{\spatialfibrecoordinatevector_{a}}{\spatialbasevector^{i}}=
  \delby{\tilde{\spatialfibrecoordinatesymbol}^{a}}{\spatialcoordinatesymbol^{i}}
  \tensorprod{\spatialfibrecoordinatevector_{a}}{\spatialbasevector^{i}}
\end{equation}

By assuming that the magnitude of the spatial coordinate base vectors,
$\spatialbasevector_{i}$, and the spatial fibre coordinate base
vectors, $\spatialfibrecoordinatevector_{a}$, are equal then the
transformation tensor, $\tensortwo{q}$, can be found from the
components of the spatial fibre coordinate base vectors \ie
\begin{equation}
  \begin{bmatrix}
    \delby{\tilde{\spatialfibrecoordinatesymbol}^{1}}{\spatialcoordinatesymbol^{1}} &
    \delby{\tilde{\spatialfibrecoordinatesymbol}^{1}}{\spatialcoordinatesymbol^{2}} &
    \delby{\tilde{\spatialfibrecoordinatesymbol}^{1}}{\spatialcoordinatesymbol^{3}} \\
    \delby{\tilde{\spatialfibrecoordinatesymbol}^{2}}{\spatialcoordinatesymbol^{1}} &
    \delby{\tilde{\spatialfibrecoordinatesymbol}^{2}}{\spatialcoordinatesymbol^{2}} &
    \delby{\tilde{\spatialfibrecoordinatesymbol}^{2}}{\spatialcoordinatesymbol^{3}} \\
    \delby{\tilde{\spatialfibrecoordinatesymbol}^{3}}{\spatialcoordinatesymbol^{1}} &
    \delby{\tilde{\spatialfibrecoordinatesymbol}^{3}}{\spatialcoordinatesymbol^{2}} &
    \delby{\tilde{\spatialfibrecoordinatesymbol}^{3}}{\spatialcoordinatesymbol^{3}}
  \end{bmatrix} = \begin{bmatrix}
    \spatialfibrecoordinatesymbol^{1}_{1} & \spatialfibrecoordinatesymbol^{2}_{1} & \spatialfibrecoordinatesymbol^{3}_{1} \\
    \spatialfibrecoordinatesymbol^{1}_{2} & \spatialfibrecoordinatesymbol^{2}_{2} & \spatialfibrecoordinatesymbol^{3}_{2} \\
    \spatialfibrecoordinatesymbol^{1}_{3} & \spatialfibrecoordinatesymbol^{2}_{3} & \spatialfibrecoordinatesymbol^{3}_{3} \\
  \end{bmatrix}
\end{equation}


\subsubsection{Structural Tensor}
\label{subsubsec:FiniteElasticityAnisotropyStructuralTensor}

In \OpenCMISS the anisotropic directions are encapsulated using a
\emph{structural tensor}. Given a set of (orthogonal) preferred
direction vectors we can form a structural tensor using tensor product
of these vectors. For example, for the material fibre vectors,
$\materialfibrecoordinatevector_{I}$, we can form material structural
tensor components, $\materialstructuraltensorcomponent{I}$ via the
tensor product of the various material fibre vectors \ie
\begin{equation}
  \materialstructuraltensorcomponent{I}=\tensorprod{\materialfibrecoordinatevector_{I}}{\materialfibrecoordinatevector_{I}}
  \label{eqn:MaterialStructuralTensorComponentDefinition}
\end{equation}
and
\begin{equation}
  \materialstructuraltensorcomponent{IJ}=\tensorprod{\materialfibrecoordinatevector_{I}}{\materialfibrecoordinatevector_{J}}
  \label{eqn:MaterialStructuralTensorCrossComponentDefinition}
\end{equation}

The material structure tensor can the be found using a linear
combination of the material structure tensor components defined in
\eqnrefs{eqn:MaterialStructuralTensorComponentDefinition}{eqn:MaterialStructuralTensorCrossComponentDefinition} \ie
\begin{equation}
  \materialstructuraltensor=\gsum{I=1}{N}{\pbrac{a_{I}\materialstructuraltensorcomponent{I}
      +\gsum{J=1,J\neq I}{N}{a_{IJ}\materialstructuraltensorcomponent{IJ}}}}
  \label{eqn:MaterialStructuralTensorDefinition}
\end{equation}
where $a_{I}$ and $a_{IJ}$ are the coefficients for each component and
$N$ is the number of material dimensions.

Similarily, we can form a spatial structural tensor using the spatial
fibre vectors, $\spatialfibrecoordinatevector_{i}$ \ie the spatial
structural tensor components are
\begin{equation}
  \spatialstructuraltensorcomponent{i}=\tensorprod{\spatialfibrecoordinatevector_{i}}{\spatialfibrecoordinatevector_{i}}
  \label{eqn:SpatialStructuralTensorComponentDefinition}
\end{equation}
and
\begin{equation}
  \spatialstructuraltensorcomponent{ij}=\tensorprod{\spatialfibrecoordinatevector_{i}}{\spatialfibrecoordinatevector_{j}}
  \label{eqn:SpatialStructuralTensorCrossComponentDefinition}
\end{equation}

The spatial structure tensor can the be found using a linear
combination of the material structure tensor components defined in
\eqnrefs{eqn:SpatialStructuralTensorComponentDefinition}{eqn:SpatialStructuralTensorCrossComponentDefinition} \ie
\begin{equation}
  \spatialstructuraltensor=\gsum{i=1}{n}{\pbrac{a_{i}\spatialstructuraltensorcomponent{i}
      +\gsum{j=1,j\neq i}{n}{a_{ij}\spatialstructuraltensorcomponent{ij}}}}
  \label{eqn:SpatialStructuralTensorDefinition}
\end{equation}
where $a_{i}$ and $a_{ij}$ are the coefficients for each component and
$n$ is the number of spatial dimensions.

\subsection{Growth}
\label{subsec:FiniteElasticityGrowth}

\subsubsection{Multiplicative Decomposition}
\label{subsubsec:FiniteElasticityGrowthMultiplicativeDecomposition}

To allow for growth we use a multiplicative decomposition approach \ie
\begin{equation}
  \deformationgradienttensor=\deformationgradienttensor_{e}\deformationgradienttensor_{g}
\end{equation}
where $\deformationgradienttensor_{g}$ is the growth tensor
with respect to material coordinates and
$\deformationgradienttensor_{e}$ is the elastic component of
the deformation gradient tensor.

The elastic component of the deformation gradient tensor can be calculated
from
\begin{equation}
  \deformationgradienttensor_{e}=\deformationgradienttensor\inverse{\deformationgradienttensor_{g}}
\end{equation}

In component form we have
\begin{equation}
  \deformationgradienttensorsymbol^{i}_{J}=\pbrac{\deformationgradienttensorsymbol_{e}}^{i}_{K}
  \pbrac{\deformationgradienttensorsymbol_{g}}^{K}_{J}
\end{equation}
and
\begin{equation}
  \pbrac{\deformationgradienttensorsymbol_{e}}^{i}_{J}=\pbrac{\deformationgradienttensorsymbol}^{i}_{J}
  \pbrac{\inverse{\deformationgradienttensorsymbol}_{g}}^{J}_{K}
\end{equation}

The Jacobian of the growth component of the deformation is given by
$J_{g}=\determinant{\deformationgradienttensor_{g}}$ and the
Jacobian of the elastic component of the deformation is given by
$J_{e}=\determinant{\deformationgradienttensor_{e}}$.

The right Cauchy Green deformation tensor is now given by the pullback
of the spatial metric tensor, $\spatialmetrictensor$,
\begin{equation}
  \fnof{\tilde{\rightcauchygreentensor}}{\materialcoordinatevector}=
  \transpose{\fnof{\deformationgradienttensor_{e}}{\materialcoordinatevector}}\spatialmetrictensor
  \fnof{\deformationgradienttensor_{e}}{\materialcoordinatevector}
\end{equation}

In component form we have
\begin{equation}
  \rightcauchygreentensorsymbol_{JK}=\spatialmetrictensorsymbol_{ij}\pbrac{\deformationgradienttensorsymbol_{e}}^{i}_{J}
  \pbrac{\deformationgradienttensorsymbol_{e}}^{j}_{K}
\end{equation}

To find the stress tensors in spatial or deformed coordinates we need to push the second
Piola Kirchhoff tensor in the material or reference coordinates forward to the spatial
coordinates, $\spatialcoordinatevector$, to give the Kirchhoff stress tensor,
$\fnof{\kirchoffstresstensor}{\spatialcoordinatevector}$. The push foward is given by
\begin{equation}
  \fnof{\kirchoffstresstensor}{\spatialcoordinatevector}=
  \fnof{\deformationgradienttensor_{e}}{\materialcoordinatevector}
  \fnof{\secondpiolakirchoffstresstensor}{\materialcoordinatevector}
  \transpose{\fnof{\deformationgradienttensor_{e}}{\materialcoordinatevector}}
\end{equation}

The Cauchy stress tensor,
$\fnof{\cauchystresstensor}{\spatialcoordinatevector}$, can then be
calculated from the Kirchhoff stress tensor using the Jacobian of the
deformation \ie
\begin{equation}
  \fnof{\cauchystresstensor}{\spatialcoordinatevector}=\inverse{J_{e}}\fnof{\kirchoffstresstensor}{\spatialcoordinatevector}=
  \inverse{J_{e}}\fnof{\deformationgradienttensor_{e}}{\materialcoordinatevector}
  \fnof{\secondpiolakirchoffstresstensor}{\materialcoordinatevector}
  \transpose{\fnof{\deformationgradienttensor_{e}}{\materialcoordinatevector}}
\end{equation}

In component form we have
\begin{equation}
  \kirchoffstresstensorsymbol^{ij}=\pbrac{\deformationgradienttensorsymbol_{e}}^{i}_{J}
  \secondpiolakirchofftensorsymbol^{JK}\pbrac{\transpose{\deformationgradienttensorsymbol_{e}}}^{j}_{K}
\end{equation}
and
\begin{equation}
  \cauchystresstensorsymbol^{ij}=\inverse{J_{e}}\pbrac{\deformationgradienttensorsymbol_{e}}^{i}_{J}
  \secondpiolakirchofftensorsymbol^{JK}\pbrac{\transpose{\deformationgradienttensorsymbol_{e}}}^{j}_{K}
\end{equation}


\subsubsection{Growth Tensors}
\label{subsubsec:FiniteElasticityGrowthTensors}

In \OpenCMISS we allow for a number of different forms of growth
tensor. These are based on different forms of the material structural
tensor (see \subsubsecref{subsubsec:FiniteElasticityAnisotropyStructuralTensor}). These are

\subsubsubsection{Fibre Growth Tensor}
\label{subsubsubsec:FiniteElasticityFibreGrowthTensor}

The \emph{fibre growth tensor} is given by
\begin{equation}
  \deformationgradienttensor_{g}=\lambda_{f}\materialstructuraltensorcomponent{1}
  \label{eqn:OpenCMISSFibreGrowthTensorDefinition}
\end{equation}
where $\lambda_{f}$ is the growth factor in the fibre direction.

\subsubsubsection{Isotropic Growth Tensor}
\label{subsubsubsec:FiniteElasticityIsotropicGrowthTensor}

The \emph{isotropic growth tensor} or fibre growth tensor is given by
\begin{equation}
  \deformationgradienttensor_{g}=\lambda\materialstructuraltensorcomponent{1}
  +\lambda\materialstructuraltensorcomponent{2}
  +\lambda\materialstructuraltensorcomponent{3}
  \label{eqn:OpenCMISSIsotropicGrowthTensorDefinition}
\end{equation}
where $\lambda$ is the isotropic growth factor in the fibre direction.

\subsubsubsection{Transversely Isotropic Growth Tensor}
\label{subsubsubsec:FiniteElasticityIsotropicGrowthTensor}

The \emph{transversely isotropic growth tensor} is given by
\begin{equation}
  \deformationgradienttensor_{g}=\lambda_{n}\materialstructuraltensorcomponent{1}
  +\lambda_{t}\materialstructuraltensorcomponent{2}
  +\lambda_{t}\materialstructuraltensorcomponent{3}
  \label{eqn:OpenCMISSTransIsotropicGrowthTensorDefinition}
\end{equation}
where $\lambda_{f}$ is the growth factor in the fibre direction and
$\lambda_{t}$ is the growth factor in the plane orthogonal to the
fibre direction.

\subsubsubsection{Orthotropic Growth Tensor}
\label{subsubsubsec:FiniteElasticityOrthotropicGrowthTensor}

The \emph{orthotropic growth tensor} is given by
\begin{equation}
  \deformationgradienttensor_{g}=\lambda_{f}\materialstructuraltensorcomponent{1}
  +\lambda_{s}\materialstructuraltensorcomponent{2}
  +\lambda_{n}\materialstructuraltensorcomponent{3}
  \label{eqn:OpenCMISSOrthotropicGrowthTensorDefinition}
\end{equation}
where $\lambda_{f}$ is the growth factor in the fibre direction, $\lambda_{s}$ the growth factor in the sheet direction, and
$\lambda_{n}$ is the growth factor in the normal direction.

\subsubsubsection{Full Growth Tensor}
\label{subsubsubsec:FiniteElasticityFullGrowthTensor}

The \emph{full growth tensor} is given by
\begin{equation}
  \deformationgradienttensor_{g}=\lambda_{f}\materialstructuraltensorcomponent{1}
  +\lambda_{s}\materialstructuraltensorcomponent{2}
  +\lambda_{n}\materialstructuraltensorcomponent{3}
  +\lambda_{fs}\materialstructuraltensorcomponent{12}
  +\lambda_{fn}\materialstructuraltensorcomponent{13}
  +\lambda_{sn}\materialstructuraltensorcomponent{23}
  \label{eqn:OpenCMISSFullGrowthTensorDefinition}
\end{equation}
where $\lambda_{f}$ is the growth factor in the fibre direction,
$\lambda_{s}$ the growth factor in the sheet direction, $\lambda_{n}$
is the growth factor in the normal direction, $\lambda_{fs}$ the cross
growth factor in the fibre-sheet plane, $\lambda_{fn}$ the cross
growth factor in the fibre-normal plane, and $\lambda_{sn}$ is the
cross growth factor int the sheet-normal plane.

\subsection{Active Contraction}
\label{subsec:FiniteElasticityActiveContraction}

When dealing with biological materials like muscle we must deal with
not only passive elastic stresses but also stresses actively generated
by the muscle fibres, themselves, contracting. That is, the total
Cauchy stress that the material experiences is made up of the sum of
an elastic stress and an active stress \ie
\begin{equation}
  \fnof{\cauchystresstensor}{\spatialcoordinatevector}=\fnof{\cauchystresstensor_{e}}{\spatialcoordinatevector}+
  \fnof{\cauchystresstensor_{a}}{\spatialcoordinatevector}
\end{equation}
where $\fnof{\cauchystresstensor_{e}}{\spatialcoordinatevector}$ is
the \emph{elastic Cauchy stress tensor}\footnote{also known as the
passive Cauchy stress tensor} and
$\fnof{\cauchystresstensor_{a}}{\spatialcoordinatevector}$ is the
\emph{active Cauchy stress tensor}.

There are two main methods for calculating the active stress,
depending on whether material or spatial coordiantes are used. The
first, and simplest, method is to calculate the active stress in the
material or undeformed coordinate system. The advantages of using this
coordinate system are that the material fibre coordinate system is
orthogonal and fixed throughout deformation. Thus, the active stress
tensor is often diagonal in the material fibre coordinate system.

With active stress in material coordinates we have
\begin{equation}
  \fnof{\secondpiolakirchoffstresstensor}{\materialcoordinatevector}=
  \fnof{\secondpiolakirchoffstresstensor_{e}}{\materialcoordinatevector}+
  \fnof{\secondpiolakirchoffstresstensor_{a}}{\materialcoordinatevector}
\end{equation}
where
$\fnof{\secondpiolakirchoffstresstensor_{e}}{\materialcoordinatevector}$
is the \emph{elastic second Piola-Kirchoff stress tensor} and
$\fnof{\secondpiolakirchoffstresstensor_{a}}{\materialcoordinatevector}$
is the \emph{active second Piola-Kirchoff stress tensor} which has
been calculated using some model of active contraction.

Once the second Piola-Kirchoff stress tensor is calculated in can be
pushed forward to the spatial or deformed coordinate system as a
Cauchy stress. As mentioned above, the second Piola-Kirchoff stress
tensors are often calculated in the material or reference fibre
coordinate system. The push forward transformation is thus given using
the deformation gradient tensor in material fibre coordinates
\begin{equation}
  \begin{aligned}
    \cauchystresstensor&=\cauchystresstensorsymbol^{ij}\tensorprod{\spatialbasevector_{i}}{\spatialbasevector_{j}}=
    \inverse{J}\tilde{\deformationgradienttensor}\tilde{\secondpiolakirchoffstresstensor}
    \transpose{\tilde{\deformationgradienttensor}}=
    \tilde{\deformationgradienttensor}\pbrac{\tilde{\secondpiolakirchoffstresstensor}_{e}+
      \tilde{\secondpiolakirchoffstresstensor}_{a}}\transpose{\tilde{\deformationgradienttensor}}\\
    &=\inverse{J}\pbrac{\tilde{\deformationgradienttensorsymbol}}^{i}_{B}
    \pbrac{\tilde{\secondpiolakirchofftensorsymbol}^{BC}_{e}+\tilde{\secondpiolakirchofftensorsymbol}^{BC}_{a}}
    \pbrac{\transpose{\tilde{\deformationgradienttensorsymbol}}}^{j}_{C}\tensorprod{\spatialbasevector_{i}}{\spatialbasevector_{j}}
  \end{aligned}
\end{equation}

Whilst the first method for active contraction is relatively
straightforward to compute it does have the major drawback that,
because the active stress is calculated in the material domain, the
immediate physical environment an active cell might encounter is not
part of the calculation. For example an active muscle cell might sense
the amount of stretch the cell experiences and use that in the active
model via stretch receptors or myofillament overlap.

To overcome some of the limitations of the first method, the second
method for active contraction computes the active stress with respect
to the spatial fibre coordinate system \ie
\begin{equation}
  \tilde{\cauchystresstensor}_{a}=\tilde{\cauchystresstensorsymbol}^{bc}_{a}
  \tensorprod{\spatialfibrecoordinatevector_{b}}{\spatialfibrecoordinatevector_{b}}
\end{equation}

In order to add the active Cauchy stress to the elastic Cauchy stress
we must transform the active stress tensor from spatial fibre
coordinates to spatial coordinates. This can be achieved by using the
inverse transformation tensor from spatial to fibre coordinates
defined in \subsecref{subsec:FiniteElasticityAnisotropy} \ie
\begin{equation}
  \begin{aligned}
    \cauchystresstensor_{a}&=\transpose{\tensortwo{q}}\tilde{\cauchystresstensor}_{a}\tensortwo{q}\\
    &=\pbrac{q}^{i}_{b}\tilde{\cauchystresstensorsymbol}^{bc}\pbrac{\transpose{\tilde{q}}}^{j}_{c}
  \end{aligned}
\end{equation}

To calculate the stretch an active cell experiences in each direction
it is useful to compute the stretches relative to the spatial fibre
coordinate system. To do this we must first transform the
Euler-Almansi strain tensor from spatial to spatial fibre coordinates \ie
\begin{equation}
  \begin{aligned}
    \tilde{\euleralmansistraintensor}&=\tilde{\euleralmansistraintensorsymbol}_{bc}
    \tensorprod{\spatialfibrecoordinatevector^{b}}{\spatialfibrecoordinatevector^{c}}=
    \inverse{\tensortwo{q}}\euleralmansistraintensor\invtranspose{\tensortwo{q}}\\
    &=\dfrac{1}{2}\inverse{\tensortwo{q}}\pbrac{\flattensor{\spatialmetrictensor}-\flattensor{\cauchydeformationtensor}}
    \invtranspose{\tensortwo{q}}\\
    &=\dfrac{1}{2}\pbrac{\inverse{q}}^{i}_{b}\pbrac{\spatialmetrictensorsymbol_{ij}-\cauchydeformationtensorsymbol_{ij}}
    \pbrac{\invtranspose{q}}^{j}_{c}\tensorprod{\spatialfibrecoordinatevector^{b}}{\spatialfibrecoordinatevector^{c}}
  \end{aligned}
\end{equation}
where
\begin{equation}
  \flattensor{\cauchydeformationtensor}=\cauchydeformationtensorsymbol_{ij}
  \tensorprod{\spatialbasevector^{i}}{\spatialbasevector^{j}}=
  \invtranspose{\deformationgradienttensor}\flattensor{\materialmetrictensor}\inverse{\deformationgradienttensor}
\end{equation}

The principal stretches in spatial fibre coordinates are then given by
\begin{align}
  \tilde{\lambda}_{1}&=\dfrac{1}{\sqrt{1-2\tilde{\euleralmansistraintensorsymbol}_{11}}} \\
  \tilde{\lambda}_{2}&=\dfrac{1}{\sqrt{1-2\tilde{\euleralmansistraintensorsymbol}_{22}}} \\
  \tilde{\lambda}_{3}&=\dfrac{1}{\sqrt{1-2\tilde{\euleralmansistraintensorsymbol}_{33}}} 
\end{align}

\subsection{Elasticity Tensors}
\label{subsec:FiniteElasticityElasticityTensors}

If $\fnof{\tensor{P}}{\vectr{X},\deformationgradienttensor}$ is the first Piola-Kirchoff
constitutive function that depends on $\vectr{X}$ and the deformation
gradient tensor $\deformationgradienttensor$. The \emph{first elasticity tensor}\symbolat{$\tensorfour{A}$}{First elasticity
  tensor},
$\tensorfour{A}$, is given by 
\begin{equation}
  \tensorfour{A}=\delby{\tensor{P}}{\deformationgradienttensor}
\end{equation}
or
\begin{equation}
  A^{iAB}_{j}=\delby{P^{iA}}{F^{j}_{B}}
\end{equation}

Note that $\tensorfour{A}$ is a two-point tensor. 

If $\fnof{\secondpiolakirchoffstresstensor}{\vectr{X},\rightcauchygreentensor}$ is the second Piola-Kirchoff
constitutive function that depends on $\vectr{X}$ and the right Cauchy-Green
strain tensor $\rightcauchygreentensor$. The \emph{second elasticity tensor}\symbolat{$\spatialsecondelasticitytensor$}{Second elasticity
  tensor},
$\materialsecondelasticitytensor$, is given by 
\begin{equation}
  \materialsecondelasticitytensor=\delby{\secondpiolakirchoffstresstensor}{\rightcauchygreentensor}
\end{equation}
or
\begin{equation}
  C^{ABCD}=\delby{S^{AB}}{C_{CD}}
\end{equation}

We can derive a relationship between $\tensorfour{A}$ and $\materialsecondelasticitytensor$. Differentiating
\begin{equation}
  \tensor{P}=\deformationgradienttensor\secondpiolakirchoffstresstensor
\end{equation}
or
\begin{equation}
  P^{iA}=F^{i}_{B}S^{BA}
\end{equation}
with respect to the deformation gradient tensor gives
\begin{equation}
  \delby{P^{iA}}{F^{j}_{B}}=F^{i}_{C}\delby{S^{CA}}{C_{DE}}\delby{C_{DE}}{F^{i}_{C}}+\delby{F^{i}_{C}}{F^{j}_{B}}\secondpiolakirchofftensorsymbol^{CA}
\end{equation}

Now
\begin{equation}
  C_{DE}=F^{l}_{D}F^{k}_{E}\spatialmetrictensorsymbol_{lk}
\end{equation}
and so
\begin{equation}
  \delby{C_{DE}}{F^{j}_{B}}=\mixedkronecker{B}{D}F^{k}_{E}\spatialmetrictensorsymbol_{jk}+F^{l}_{D}\mixedkronecker{B}{E}\spatialmetrictensorsymbol_{lj}
\end{equation}

Substituting this into the above equation gives
\begin{equation}
  \begin{split}
    \delby{P^{iA}}{F^{j}_{B}}&=C^{CADE}\pbrac{\mixedkronecker{B}{D}F^{k}_{E}\spatialmetrictensorsymbol_{jk}+
      F^{l}_{D}\mixedkronecker{B}{E}\spatialmetrictensorsymbol_{lj}}F^{i}_{C}+T^{CA}\mixedkronecker{j}{i}\mixedkronecker{B}{C}\\
    &=C^{CABE}F^{k}_{E}F^{i}_{C}\spatialmetrictensorsymbol_{jk}+C^{CADE}F^{l}_{D}F^{i}_{C}\spatialmetrictensorsymbol_{lj}+T^{BA}\mixedkronecker{i}{j}
  \end{split}
\end{equation}

Now, using the symmetries $C^{CABE}=C^{CAEB}$ and $T^{AB}=T^{BA}$ we have
\begin{equation}
  \tensorfour{A}=2\dotprod{\materialsecondelasticitytensor}{\dotprod{\deformationgradienttensor}{\dotprod{\deformationgradienttensor}{\tensortwo{g}}}}+
  \tensorprod{\tensortwo{S}}{\tensortwo{I}}
\end{equation}
or
\begin{equation}
  A^{iAB}_{j}=2C^{CADB}F^{k}_{D}F^{i}_{C}\spatialmetrictensorsymbol_{kj}+T^{AB}\mixedkronecker{i}{j}
\end{equation}

Now, we can also define the \emph{first spatial elasticity tensor}\symbolat{$\tensorfour{a}$}{First spatial
elasticity tensor},
$\tensorfour{a}$, and the \emph{second spatial elasticity tensor}\symbolat{$\tensorfour{c}$}{Second spatial elasticity
tensor},
$\tensorfour{c}$, using push forwards and Piola transforms of $\tensorfour{A}$
and $\materialsecondelasticitytensor$ respectively \ie
\begin{equation}
  \begin{split}
    \tensorfour{a}&=\frac{1}{J}\pushforward{\chi}\tensorfour{A} \\
    \tensorfour{c}&=\frac{2}{J}\pushforward{\chi}\materialsecondelasticitytensor \\
  \end{split}
\end{equation}
or
\begin{equation}
  \begin{split}
    a^{ikl}_{j}&=\frac{1}{J}F^{k}_{A}F^{l}_{B}A^{iAB}_{j} \\
    c^{ijkl}&=\frac{2}{J}F^{i}_{A}F^{j}_{B}F^{k}_{C}F^{l}_{D}C^{ABCD} \\
  \end{split}
\end{equation}

The relationship between $\tensorfour{a}$ and $\tensorfour{c}$ is given by
\begin{equation}
  \tensorfour{a}=2\dotprod{\tensorfour{c}}{\tensortwo{g}}+\tensorprod{\tensortwo{\sigma}}{\tensortwo{I}}
\end{equation}
or
\begin{equation}
  a^{ikl}_{j}=c^{ikml}\spatialmetrictensorsymbol_{mj}+\sigma^{kl}\mixedkronecker{i}{j}
\end{equation}

There is also a relationship between $\tensorfour{A}$ and $\tensorfour{a}$
known as the \emph{Piola identity} \ie
\begin{equation}
  \divergence{\vectr{X}}{\pbrac{\dotprod{\tensorfour{A}}{\vectr{U}}}}=
  J\divergence{\vectr{x}}{\pbrac{\dotprod{\tensorfour{a}}{\vectr{u}}}}
\end{equation}
where
\begin{equation}
  \vectr{u}=\pushforward{\chi}{\vectr{U}}
\end{equation}

Note that the tensor $\spatialsecondelasticitytensor$ is different to $\materialsecondelasticitytensor$ in that
it is \emph{not} given by $\delby{\sigma^{ij}}{c_{kl}}$ and $\sigma^{ij}\neq
2\delby{W}{c_{ij}}$. They are instead given by
\begin{equation}
  c^{ijkl}=\delby{\sigma^{ij}}{\spatialmetrictensorsymbol_{kl}}
\end{equation}
and
\begin{equation}
  \sigma^{ij}=2\delby{W}{\spatialmetrictensorsymbol_{ij}}
\end{equation}

\subsection{Constitutive Relationships}

The relationship between stress and strain is known as a \emph{constitutive
  relationship}. Materials for which the constitutive relationship is just a
function of the current state of deformation are known as \emph{elastic}.

\subsubsection{Hyperelasticity}

For the special case whereby the work done by stresses during deformation is
just a function of the initial configuration and the current configuration are
known as \emph{hyperelastic}. As a consequence hyperelastic materials are
independent of the path of deformation and just depend on a \emph{stored
  energy function} or \emph{elastic potential}, $W$.

WORK CONJUGATES: P and F, S and C/E.

The first Piola-Kirchoff stress tensor is thus a function of postion and the
deformation gradient tensor \ie
\begin{equation}
  \tensor{P}=\fnof{\tensor{P}}{\vectr{X},\fnof{\deformationgradienttensor}{\vectr{X}}}
\end{equation}

In terms of the stored energy function we have
\begin{equation}
  \fnof{\tensor{P}}{\vectr{X}}=\delby{\fnof{\psi}{\fnof{\deformationgradienttensor}{\vectr{X}},\vectr{X}}}{\fnof{\deformationgradienttensor}{\vectr{X}}}
\end{equation}
or, in component form,
\begin{equation}
  \tensor{P}=P^{iI}\tensorprod{\spatialbasevector_{i}}{\materialbasevector_{I}}=\delby{\psi}{F^{i}_{I}}\tensorprod{\spatialbasevector_{i}}{\materialbasevector_{I}}
\end{equation}
ABOVE IS NOT RIGHT IN TERMS OF POSITION OF INDICES.

The second Piola-Kirchoff stress tensor is thus a function of position and
the right Cauchy-Green deformation tensor (or, equivalently, the
Green-Lagrange strain tensor) \ie
\begin{equation}
  \secondpiolakirchoffstresstensor=\fnof{\secondpiolakirchoffstresstensor}{\vectr{X},\fnof{\rightcauchygreentensor}{\vectr{X}}}
\end{equation}


The constitutive law can then be used to derive the second Piola Kirchhoff
stress tensor in fibre coordinates, $\fnof{\secondpiolakirchoffstresstensor}{\vectr{N}}$, from
either the right Cauchy-Green deformation tensor or the Green-Lagrange strain
tensor \ie
\begin{equation}
  \fnof{\secondpiolakirchoffstresstensor}{\vectr{N}}=2\delby{\fnof{W}{\fnof{\rightcauchygreentensor}{\vectr{N}}}}{\fnof{\rightcauchygreentensor}{\vectr{N}}}
\end{equation}
or
\begin{equation}
  \fnof{\secondpiolakirchoffstresstensor}{\vectr{N}}=\delby{\fnof{W}{\fnof{\greenlagrangestraintensor}{\vectr{N}}}}{\fnof{\greenlagrangestraintensor}{\vectr{N}}}
\end{equation}
where $\fnof{W}{\fnof{\rightcauchygreentensor}{\vectr{N}}}$ or
$\fnof{W}{\fnof{\greenlagrangestraintensor}{\vectr{N}}}$ is the strain energy
function. In component form we have
\begin{equation}
  \secondpiolakirchoffstresstensor=S^{AB}\tensorprod{\vectr{N}_{A}}{\vectr{N}_{B}}=2\delby{\fnof{W}{\rightcauchygreentensor}}{\rightcauchygreentensor}=2\delby{\fnof{W}{\rightcauchygreentensor}}{C_{AB}}\tensorprod{\vectr{N}_{A}}{\vectr{N}_{B}}
\end{equation}
or
\begin{equation}
  \secondpiolakirchoffstresstensor=S^{AB}\tensorprod{\vectr{N}_{A}}{\vectr{N}_{B}}=\delby{\fnof{W}{\greenlagrangestraintensor}}{\greenlagrangestraintensor}=\delby{\fnof{W}{\greenlagrangestraintensor}}{E_{AB}}\tensorprod{\vectr{N}_{A}}{\vectr{N}_{B}}
\end{equation}

TIDY UP ABOVE.

Constituative relationships in the spatial frame.






\subsection{Compressible and Nearly-Incompressibile Materials}
\label{subsec:FiniteElasticityCompressibleNearlyIncompressibleMat}

JUST AND IN A SPHERICAL PART TO THE STRAIN ENERGY FUNCTION

LIST THE DIFFERENT TYPES OF COMPRESSIBLY/NEAR MODELS HEAR OR LATER?

\subsection{Incompressibile Materials}
\label{subsec:FiniteElasticityIncompressibleMat}

A number of materials are incompressible \ie any deformation does not
result in a change of volume. A common incompressible material is
water and, due to the high proportion of water in their composition,
biological materials.

For incompressible materials it is often useful to consider the strain
energy (or stress) in terms of a component that is purely volumetric
(or spherical) and a component that is purely isochoric (or
deviatoric). Following \citet{federico:2012} we start with a
\emph{modified deformation gradient
tensor}\symbolat{$\bar{\deformationgradienttensor}$}{modified deformation gradient
  tensor} \ie
\begin{equation}
  \deformationgradienttensor=J^{\frac{1}{3}}\bar{\deformationgradienttensor}
  \quad\Rightarrow\quad \bar{\deformationgradienttensor}
  =J^{-\frac{1}{3}}\deformationgradienttensor
  \label{eqn:modifiedDeformationGradientTensor}
\end{equation}
Note that $\determinant{\bar{\deformationgradienttensor}}=1$ and so
represents deformation that is purely distortional rather than
dialational.

From the modified deformation gradient tensor we can construct the
various deformation and strain tensors as before. In the reference
configuration we have
\begin{equation}
  \rightcauchygreentensor =
  J^{\frac{1}{3}}J^{\frac{1}{3}}\transpose{\bar{\deformationgradienttensor}}\spatialmetrictensor\bar{\deformationgradienttensor}
  =J^{\frac{2}{3}}\bar{\rightcauchygreentensor}
  \quad\Rightarrow\quad \bar{\rightcauchygreentensor}
  =J^{-\frac{2}{3}}\rightcauchygreentensor
\end{equation}
where $\bar{\rightcauchygreentensor}$ is the \emph{modified right
Cauchy-Green deformation
tensor}\symbolat{$\bar{\tensortwo{C}}$}{modified right Cauchy-Green
  deformation tensor} and
\begin{equation}
  \greenlagrangestraintensor =
  J^{\frac{2}{3}}{\bar{\greenlagrangestraintensor}}+\dfrac{1}{2}\pbrac{J^{\frac{2}{3}}-1}\materialmetrictensor
  \quad\Rightarrow\quad \bar{\greenlagrangestraintensor} = \dfrac{1}{2}\pbrac{\bar{\rightcauchygreentensor}-\materialmetrictensor}
\end{equation}
where $\bar{\greenlagrangestraintensor}$ is the \emph{modified right
Green-Lagrange strain
tensor}\symbolat{$\bar{\greenlagrangestraintensor}$}{modified
  Green-Lagrange strain tensor}. Note that we also have
$\bar{\pioladeformationtensor}=\inverse{\bar{\rightcauchygreentensor}}$
the \emph{modified Piola deformation
tensor}\symbolat{$\bar{\pioladeformationtensor}$}{modified Piola
  deformation tensor}.

The modified strain tensors in the current configuration are 
\begin{equation}
  \leftcauchygreentensor =
  J^{\frac{1}{3}}J^{\frac{1}{3}}\bar{\deformationgradienttensor}
  \materialmetrictensor\transpose{\bar{\deformationgradienttensor}}=J^{\frac{2}{3}}\bar{\leftcauchygreentensor}
  \quad\Rightarrow\quad \bar{\leftcauchygreentensor}
  =J^{-\frac{2}{3}}\leftcauchygreentensor
\end{equation}
where $\bar{\leftcauchygreentensor}$ is the \emph{modified left Cauchy-Green deformation
  tensor}\symbolat{$\bar{\leftcauchygreentensor}$}{modified left Cauchy-Green deformation tensor} and
\begin{equation}
  \euleralmansistraintensor =
  J^{-\frac{2}{3}}{\bar{\euleralmansistraintensor}}+\dfrac{1}{2}\pbrac{1-J^{-\frac{2}{3}}}\spatialmetrictensor
  \quad\Rightarrow\quad \bar{\euleralmansistraintensor} = \dfrac{1}{2}\pbrac{\spatialmetrictensor-\bar{\cauchydeformationtensor}}
\end{equation}
where $\bar{\euleralmansistraintensor}$ is the \emph{modified Almansi
strain tensor}\symbolat{$\bar{\euleralmansistraintensor}$}{modified
  Almansi strain tensor}. Note that we also have
$\bar{\cauchydeformationtensor}=\inverse{\bar{\leftcauchygreentensor}}$
the \emph{modified Cauchy deformation
tensor}\symbolat{$\bar{\euleralmansistraintensor}$}{modified Cauchy
  deformation tensor}.

Consider now decomposing the strain energy function into a
isochoric/deviatoric part and a volumetric/spherical part \ie
\begin{equation}
  \begin{split}
    \fnof{W}{\rightcauchygreentensor} &=
    \fnof{W_{\Devop}}{\rightcauchygreentensor}+\fnof{W_{\Sphop}}{\rightcauchygreentensor} \\
    &=\fnof{\bar{W}_{\Devop}}{\fnof{\bar{\rightcauchygreentensor}}{\rightcauchygreentensor}}+
    \fnof{\bar{W}_{\Sphop}}{\fnof{J}{\rightcauchygreentensor}}\\
    &=\fnof{\bar{W}}{\fnof{\bar{\rightcauchygreentensor}}{\rightcauchygreentensor},\fnof{J}{\rightcauchygreentensor}}
  \end{split}
  \label{eqn:FiniteElasticityDecomposedStrainEnergy}
\end{equation}
where
$\fnof{\bar{W}}{\fnof{\bar{\rightcauchygreentensor}}{\rightcauchygreentensor},\fnof{J}{\rightcauchygreentensor}}$
is the \emph{modified strain energy function}, $\fnof{J}{\rightcauchygreentensor}$ is the
Jacobian which quantifies volume change and
$\fnof{\bar{\rightcauchygreentensor}}{\rightcauchygreentensor}$, is the modified right Cauchy-Green
deformation tensor.

The second Piola-Kirchoff stress can now be obtained from
\begin{equation}
  \begin{split}
    \fnof{\secondpiolakirchoffstresstensor}{\rightcauchygreentensor}
    &=2\delby{\fnof{W}{\rightcauchygreentensor}}{\rightcauchygreentensor} \\
    &=2\delby{\fnof{\bar{W}}{\fnof{\bar{\rightcauchygreentensor}}{\rightcauchygreentensor},\fnof{J}{\rightcauchygreentensor}}}{\rightcauchygreentensor}\\
    &=2\doubledotprod{
      \delby{\fnof{\bar{W}_{\Devop}}{\fnof{\bar{\rightcauchygreentensor}}{\rightcauchygreentensor}}}{\bar{\rightcauchygreentensor}}
    }{\delby{\fnof{\bar{\rightcauchygreentensor}}{\rightcauchygreentensor}}{\rightcauchygreentensor}}
      +2\delby{\fnof{\bar{W}_{\Sphop}}{\fnof{J}{\rightcauchygreentensor}}}{J}\delby{\fnof{J}{\rightcauchygreentensor}}{\rightcauchygreentensor}
  \end{split}
\end{equation}
or equivalently
\begin{equation}
  \begin{split}
    \fnof{\secondpiolakirchoffstresstensor}{\greenlagrangestraintensor} &= \delby{\fnof{W}{\greenlagrangestraintensor}}{\greenlagrangestraintensor} \\
    &=\delby{\fnof{\bar{W}}{\fnof{\bar{\greenlagrangestraintensor}}{\greenlagrangestraintensor},\fnof{J}{\greenlagrangestraintensor}}}{\greenlagrangestraintensor}\\
    &=\doubledotprod{\delby{\fnof{\bar{W}_{\Devop}}{\fnof{\bar{\greenlagrangestraintensor}}{\greenlagrangestraintensor}}}{\bar{\greenlagrangestraintensor}}}{\delby{\fnof{\bar{\greenlagrangestraintensor}}{\greenlagrangestraintensor}}{\greenlagrangestraintensor}}+\delby{\fnof{\bar{W}_{\Sphop}}{\fnof{J}{\greenlagrangestraintensor}}}{J}\delby{\fnof{J}{\greenlagrangestraintensor}}{\greenlagrangestraintensor}
  \end{split}
\end{equation}

We now define $\bar{\secondpiolakirchoffstresstensor}$ as the \emph{second Piola-Kirchoff psuedo
  stress tensor}\symbolat{$\bar{\tensortwo{S}}$}{second Piola-Kirchoff
  psuedo stress tensor} \ie 
\begin{equation}
  \fnof{\bar{\secondpiolakirchoffstresstensor}}{\fnof{\bar{\rightcauchygreentensor}}{\rightcauchygreentensor}}=
  2\delby{\fnof{\bar{W}_{\Devop}}{\fnof{\bar{\rightcauchygreentensor}}{\rightcauchygreentensor}}}{\bar{\rightcauchygreentensor}}
  =\delby{\fnof{\bar{W}_{\Devop}}{\fnof{\bar{\greenlagrangestraintensor}}{\greenlagrangestraintensor}}}{\bar{\greenlagrangestraintensor}}
  \label{eqn:SecondPiolaKirchoffPsuedoStressDefinition}
\end{equation}
and
\begin{equation}
  \fnof{p}{\fnof{J}{\rightcauchygreentensor}} = -\delby{\fnof{\bar{W}_{\Sphop}}{\fnof{J}{\rightcauchygreentensor}}}{J}
  = -\delby{\fnof{\bar{W}_{\Sphop}}{\fnof{J}{\greenlagrangestraintensor}}}{J}
  \label{eqn:HydrostaticPressureDefinition}
\end{equation}
as the \emph{hydrostatic pressure} to give
\begin{align}
  \fnof{\secondpiolakirchoffstresstensor}{\rightcauchygreentensor}
  &=\doubledotprod{\fnof{\bar{\secondpiolakirchoffstresstensor}}{\fnof{\bar{\rightcauchygreentensor}}{\rightcauchygreentensor}}}{\delby{\fnof{\bar{\rightcauchygreentensor}}{\rightcauchygreentensor}}{\rightcauchygreentensor}}
  -2p\delby{\fnof{J}{\rightcauchygreentensor}}{\rightcauchygreentensor}
  \label{eqn:PK2StressC1} \\
  \fnof{\secondpiolakirchoffstresstensor}{\greenlagrangestraintensor}
  &=\doubledotprod{\fnof{\bar{\secondpiolakirchoffstresstensor}}{\fnof{\bar{\greenlagrangestraintensor}}{\greenlagrangestraintensor}}}{\delby{\fnof{\bar{\greenlagrangestraintensor}}{\greenlagrangestraintensor}}{\greenlagrangestraintensor}}
  -p\delby{\fnof{J}{\greenlagrangestraintensor}}{\euleralmansistraintensor}
  \label{eqn:PK2StressE1}
\end{align}

Now, as $\fnof{J}{\rightcauchygreentensor}=\sqrt{\determinant{\rightcauchygreentensor}}$ and using
\eqnref{eqn:DerivDeterminantTensorSecondOrder} we have
\begin{equation}
  \begin{split}
    \delby{\fnof{J}{\rightcauchygreentensor}}{\rightcauchygreentensor}
    &=\dfrac{1}{2\sqrt{\determinant{\rightcauchygreentensor}}}\delby{\pbrac{\determinant{\rightcauchygreentensor}}}{\rightcauchygreentensor}\\
    &=\dfrac{1}{2\sqrt{\determinant{\rightcauchygreentensor}}}\determinant{\rightcauchygreentensor}\invtranspose{\rightcauchygreentensor}\\
    &=\dfrac{1}{2}\sqrt{\determinant{\rightcauchygreentensor}}\invtranspose{\rightcauchygreentensor}\\
    &=\dfrac{\fnof{J}{\rightcauchygreentensor}}{2}\invtranspose{\rightcauchygreentensor}
  \end{split}
\end{equation}
or by using \eqnref{eqn:InverseTransposeInverseTensorTwo} and since
$\rightcauchygreentensor$ is symmetric, we have
\begin{equation}
  \begin{split}
    \delby{\fnof{J}{\rightcauchygreentensor}}{\rightcauchygreentensor}
    &=\dfrac{\fnof{J}{\rightcauchygreentensor}}{2}\inverse{\rightcauchygreentensor}\\
    &=\dfrac{\fnof{J}{\rightcauchygreentensor}}{2}\pioladeformationtensor
  \end{split}
  \label{eqn:DelJacobianDelC}
\end{equation}

Now, as
$\fnof{\bar{\rightcauchygreentensor}}{\rightcauchygreentensor}=\fnof{J^{-\frac{2}{3}}}{\rightcauchygreentensor}\rightcauchygreentensor$
we have
\begin{equation}
  \delby{\fnof{\bar{\rightcauchygreentensor}}{\rightcauchygreentensor}}{\rightcauchygreentensor}=\tensorprod{\delby{\pbrac{\fnof{J^{-\frac{2}{3}}}{\rightcauchygreentensor}}}{\rightcauchygreentensor}}{\rightcauchygreentensor}+\fnof{J^{-\frac{2}{3}}}{\rightcauchygreentensor}\delby{\rightcauchygreentensor}{\rightcauchygreentensor}
  \label{eqn:DelCBarDelCFirst}
\end{equation}

Similar to above we have
\begin{equation}
  \begin{split}
    \delby{\pbrac{\fnof{J^{-\frac{2}{3}}}{\rightcauchygreentensor}}}{\rightcauchygreentensor}&=\delby{\pbrac{\pbrac{\sqrt{\determinant{\rightcauchygreentensor}}}^{-\frac{2}{3}}}}{\rightcauchygreentensor}\\
    &=\delby{\pbrac{\pbrac{\determinant{\rightcauchygreentensor}}^{-\frac{1}{3}}}}{\rightcauchygreentensor}\\
    &=\dfrac{-1}{3\pbrac{\determinant{\rightcauchygreentensor}}^{\frac{4}{3}}}\determinant{\rightcauchygreentensor}\invtranspose{\rightcauchygreentensor}\\
    &=\dfrac{-1}{3\pbrac{\determinant{\rightcauchygreentensor}}^{\frac{1}{3}}}\invtranspose{\rightcauchygreentensor}\\
    &=\dfrac{-\fnof{J^{-\frac{2}{3}}}{\rightcauchygreentensor}\invtranspose{\rightcauchygreentensor}}{3}
  \end{split}
\end{equation}
or, as $\rightcauchygreentensor$ is symmetric
\begin{equation}
  \begin{split}
    \delby{\pbrac{\fnof{J^{-\frac{2}{3}}}{\rightcauchygreentensor}}}{\rightcauchygreentensor}
    &=\dfrac{-\fnof{J^{-\frac{2}{3}}}{\rightcauchygreentensor}\inverse{\rightcauchygreentensor}}{3}\\
    &=\dfrac{-\fnof{J^{-\frac{2}{3}}}{\rightcauchygreentensor}\pioladeformationtensor}{3}
  \end{split}
  \label{eqn:DelTwoThirdsJacobianDelC}
\end{equation}

From \eqnref{eqn:DerivSelfSymmetricTensorSecondOrder} we also have
\begin{equation}
  \delby{\rightcauchygreentensor}{\rightcauchygreentensor}=\symidentitytensorfour
  \label{eqn:DelCBarDelCDeformation}
\end{equation}

Now substituting \eqnref{eqn:DelTwoThirdsJacobianDelC} and
\eqnref{eqn:DelCBarDelCDeformation} into \eqnref{eqn:DelCBarDelCFirst} and
rearranging we obtain
\begin{equation}
  \begin{split}
    \delby{\fnof{\bar{\rightcauchygreentensor}}{\rightcauchygreentensor}}{\rightcauchygreentensor}&=\tensorprod{\delby{\pbrac{\fnof{J^{-\frac{2}{3}}}{\rightcauchygreentensor}}}{\rightcauchygreentensor}}{\rightcauchygreentensor}+\fnof{J^{-\frac{2}{3}}}{\rightcauchygreentensor}\delby{\rightcauchygreentensor}{\rightcauchygreentensor}\\
    &=\tensorprod{\dfrac{-\fnof{J^{-\frac{2}{3}}}{\rightcauchygreentensor}\pioladeformationtensor}{3}}{\rightcauchygreentensor}+\fnof{J^{-\frac{2}{3}}}{\rightcauchygreentensor}\symidentitytensorfour\\
    &=\fnof{J^{-\frac{2}{3}}}{\rightcauchygreentensor}\pbrac{\symidentitytensorfour-\dfrac{\tensorprod{\pioladeformationtensor}{\rightcauchygreentensor}}{3}}
  \end{split}
  \label{eqn:DelCBarDelC}
\end{equation}

Similarily, as $\rightcauchygreentensor\sim
2\greenlagrangestraintensor$, we have
\begin{equation}
  \delby{\fnof{J}{\greenlagrangestraintensor}}{\greenlagrangestraintensor}
  =\fnof{J}{\greenlagrangestraintensor}\pioladeformationtensor
  \label{eqn:DelJacobianDelE}
\end{equation}
and
\begin{equation}
  \delby{\fnof{\bar{\greenlagrangestraintensor}}{\greenlagrangestraintensor}}{\greenlagrangestraintensor}=
  \fnof{J^{-\frac{2}{3}}}{\rightcauchygreentensor}\pbrac{\symidentitytensorfour-\dfrac{\tensorprod{\pioladeformationtensor}{\rightcauchygreentensor}}{3}}
  \label{eqn:DelEBarDelE}
\end{equation}

Substituting \eqnref{eqn:DelJacobianDelC} and \eqnref{eqn:DelCBarDelC} into
\eqnref{eqn:PK2StressC1} and substituting \eqnref{eqn:DelJacobianDelE} and
\eqnref{eqn:DelEBarDelE} into \eqnref{eqn:PK2StressE1} we can thus find that
\begin{equation}
  \begin{split}
    \fnof{\secondpiolakirchoffstresstensor}{\rightcauchygreentensor}
    &=\doubledotprod{\fnof{\bar{\secondpiolakirchoffstresstensor}}{\fnof{\bar{\rightcauchygreentensor}}{\rightcauchygreentensor}}}{\delby{\fnof{\bar{\rightcauchygreentensor}}{\rightcauchygreentensor}}{\rightcauchygreentensor}}-2p\delby{\fnof{J}{\rightcauchygreentensor}}{\rightcauchygreentensor}\\
    &=\doubledotprod{\fnof{\bar{\secondpiolakirchoffstresstensor}}{\fnof{\bar{\rightcauchygreentensor}}{\rightcauchygreentensor}}}{\fnof{J^{-\frac{2}{3}}}{\rightcauchygreentensor}\pbrac{\symidentitytensorfour-\dfrac{\tensorprod{\pioladeformationtensor}{\rightcauchygreentensor}}{3}}}-pJ\pioladeformationtensor\\
    &=\fnof{\secondpiolakirchoffstresstensor_{\pullback{\Devop}{}}}{\fnof{\bar{\rightcauchygreentensor}}{\rightcauchygreentensor}}
    +\fnof{\secondpiolakirchoffstresstensor_{\pullback{\Sphop}{}}}{\fnof{J}{\rightcauchygreentensor}}
  \end{split}
  \label{eqn:FiniteElasticityDecomposedSecondPKinC}
\end{equation}
or
\begin{equation}
  \begin{split}
    \fnof{\secondpiolakirchoffstresstensor}{\tensor{E}}
    &=\doubledotprod{\fnof{\bar{\secondpiolakirchoffstresstensor}}{\fnof{\bar{\greenlagrangestraintensor}}{\greenlagrangestraintensor}}}{\delby{\fnof{\bar{\tensor{E}}}{\greenlagrangestraintensor}}{\tensor{E}}}-p\delby{\fnof{J}{\greenlagrangestraintensor}}{\tensor{E}}\\
    &=\doubledotprod{\fnof{\bar{\secondpiolakirchoffstresstensor}}{\fnof{\bar{\greenlagrangestraintensor}}{\greenlagrangestraintensor}}}{\fnof{J^{-\frac{2}{3}}}{\greenlagrangestraintensor}\pbrac{\symidentitytensorfour-\dfrac{\tensorprod{\pioladeformationtensor}{\rightcauchygreentensor}}{3}}}-pJ\pioladeformationtensor\\
    &=\fnof{\secondpiolakirchoffstresstensor_{\pullback{\Devop}{}}}{\fnof{\bar{\greenlagrangestraintensor}}{\greenlagrangestraintensor}}
    +\fnof{\secondpiolakirchoffstresstensor_{\pullback{\Sphop}{}}}{\fnof{J}{\greenlagrangestraintensor}}
  \end{split}
  \label{eqn:FiniteElasticityDecomposedSecondPKinE}
\end{equation}

It should be noted that $\secondpiolakirchoffstresstensor_{\pullback{\Devop}{}}$ and
$\secondpiolakirchoffstresstensor_{\pullback{\Sphop}{}}$ are not the deviatoric and spherical parts
of $\secondpiolakirchoffstresstensor$. However, we do not require the deviatoric and spherical parts
of the second Piola-Kirchoff stress tensor in the reference configuration (\ie
with respect to the metric $\materialmetrictensor$) but rather we require that when we
push $\secondpiolakirchoffstresstensor_{\pullback{\Devop}{}}$ and
$\secondpiolakirchoffstresstensor_{\pullback{\Sphop}{}}$ forward to give $\cauchystresstensor_{\devop}$
and $\cauchystresstensor_{\sphop}$ that these tensors are deviatoric and
spherical respectively (\ie with respect to the metric $\spatialmetrictensor$).

In order to deal with the deviatoric and spherical tensors in the
various configurations we need to define some projection tensors. For
the current configuration (\ie with respect to the metric tensor
$\spatialmetrictensor$) we can define the fourth order identity,
$\spatialidentitytensorfour$, spherical,
$\spatialsphericaltensorfour$, and deviatoric,
$\spatialdeviatorictensorfour$, tensors as
\begin{align}
  \spatialsymidentitytensorfour=\symtensorprod{\spatialidentitytensortwo}{\tensortwo{\spatialidentitytensortwo}}
  &=\spatialidentitytensorsymbol^{ab..}_{..cd}\,
  \tensorprodfour{\spatialbasevector_{a}}{\spatialbasevector_{b}}{\spatialbasevector^{c}}{\spatialbasevector^{d}}
  \label{eqn:SpatialSymIdentityTensorFour} \\ \nonumber
  &=\dfrac{1}{2}\pbrac{\mixedkronecker{a}{c}\mixedkronecker{b}{d}+\mixedkronecker{a}{d}\mixedkronecker{b}{c}}
  \tensorprodfour{\spatialbasevector_{a}}{\spatialbasevector_{b}}{\spatialbasevector^{c}}{\spatialbasevector^{d}} \\
  \spatialsphericaltensorfour=\dfrac{1}{3}\tensorprod{\sharptensor{\spatialmetrictensor}}{\flattensor{\spatialmetrictensor}}
  &=\spatialsphericaltensorsymbol^{ab..}_{..cd}\,
  \tensorprodfour{\spatialbasevector_{a}}{\spatialbasevector_{b}}{\spatialbasevector^{c}}{\spatialbasevector^{d}}
  \label{eqn:SpatialSphericalTensorFour} \\ \nonumber
  &=\dfrac{1}{3}\spatialmetrictensorsymbol^{ab}\spatialmetrictensorsymbol_{cd}\,
  \tensorprodfour{\spatialbasevector_{a}}{\spatialbasevector_{b}}{\spatialbasevector^{c}}{\spatialbasevector^{d}}\\
  \spatialdeviatorictensorfour=\spatialsymidentitytensorfour-\spatialsphericaltensorfour
  &=\spatialdeviatorictensorsymbol^{ab..}_{..cd}\,
  \tensorprodfour{\spatialbasevector_{a}}{\spatialbasevector_{b}}{\spatialbasevector^{c}}{\spatialbasevector^{d}}
  \label{eqn:SpatialDeviatoricTensorFour} \\ \nonumber
  &=\pbrac{\dfrac{1}{2}\pbrac{\mixedkronecker{a}{c}\mixedkronecker{b}{d}+\mixedkronecker{a}{d}\mixedkronecker{b}{c}}-\dfrac{1}{3}\spatialmetrictensorsymbol^{ab}\spatialmetrictensorsymbol_{cd}}
  \tensorprodfour{\spatialbasevector_{a}}{\spatialbasevector_{b}}{\spatialbasevector^{c}}{\spatialbasevector^{d}}
\end{align}
where $\spatialidentitytensortwo$ is the spatial identity tensor \ie
$\spatialidentitytensortwo=\sharptensor{\spatialmetrictensor}\flattensor{\spatialmetrictensor}$.

NOTE: THE ORDER OF THE INDICES ABOVE ARE DIFFERENT TO THAT IN THE DIFFERENTIAL GEOMETRY SECTION???

Note that these tensors act on a second order symmetric (contravariant) tensor
in the current configuration, $\sharptensor{\tensortwo{q}}$, such that
\begin{align}
  \doubledotprod{\spatialsymidentitytensorfour}{\sharptensor{\tensortwo{q}}}&=\sharptensor{\tensortwo{q}}\\
  \doubledotprod{\spatialsphericaltensorfour}{\sharptensor{\tensortwo{q}}}
  &=\dfrac{1}{3}\trace{\spatialmetrictensor}{\sharptensor{\tensortwo{q}}}\sharptensor{\tensortwo{g}}\\
  \doubledotprod{\spatialdeviatorictensorfour}{\sharptensor{\tensortwo{q}}}
  &=\sharptensor{\tensortwo{q}}-\dfrac{1}{3}\trace{\spatialmetrictensor}{\sharptensor{\tensortwo{q}}}\sharptensor{\spatialmetrictensor}
\end{align}
where the \emph{current configuration trace operator} is defined as
\begin{equation}
  \trace{\spatialmetrictensor}{\sharptensor{\tensortwo{q}}}=\trace{}{\sharptensor{\tensortwo{q}}}=\doubledotprod{\flattensor{\spatialmetrictensor}}{\sharptensor{\tensortwo{q}}}
  \label{eqn:SpatialTraceOperatorDefinition}
\end{equation}
\ie the trace with respect to the current configuration metric tensor, $\flattensor{\spatialmetrictensor}$.

The musical isomorphisms of these tensors are
\begin{align}
  \sharptensor{\spatialsymidentitytensorfour}=
  \symtensorprod{\sharptensor{\spatialmetrictensor}}{\sharptensor{\spatialmetrictensor}}
  &=\spatialidentitytensorsymbol^{abcd}\,
  \tensorprodfour{\spatialbasevector_{a}}{\spatialbasevector_{b}}{\spatialbasevector_{c}}{\spatialbasevector_{d}}\\ \nonumber
  &=\dfrac{1}{2}\pbrac{\spatialmetrictensorsymbol^{ac}\spatialmetrictensorsymbol^{bd}+\spatialmetrictensorsymbol^{ad}\spatialmetrictensorsymbol^{bc}}
  \tensorprodfour{\spatialbasevector_{a}}{\spatialbasevector_{b}}{\spatialbasevector_{c}}{\spatialbasevector_{d}}\\
  \sharptensor{\spatialsphericaltensorfour}=
  \dfrac{1}{3}\tensorprod{\sharptensor{\spatialmetrictensor}}{\sharptensor{\spatialmetrictensor}}
  &=\spatialsphericaltensorsymbol^{abcd}\,
  \tensorprodfour{\spatialbasevector_{a}}{\spatialbasevector_{b}}{\spatialbasevector_{c}}{\spatialbasevector_{d}}\\ \nonumber
  &=\dfrac{1}{3}\spatialmetrictensorsymbol^{ab}\spatialmetrictensorsymbol^{cd}\,
  \tensorprodfour{\spatialbasevector_{a}}{\spatialbasevector_{b}}{\spatialbasevector_{c}}{\spatialbasevector_{d}}\\
  \sharptensor{\spatialdeviatorictensorfour}=
  \sharptensor{\spatialsymidentitytensorfour}-\sharptensor{\spatialsphericaltensorfour}
  &=\spatialdeviatorictensorsymbol^{abcd}\,
  \tensorprodfour{\spatialbasevector_{a}}{\spatialbasevector_{b}}{\spatialbasevector_{c}}{\spatialbasevector_{d}}\\ \nonumber
  &=\pbrac{\dfrac{1}{2}\pbrac{\spatialmetrictensorsymbol^{ac}\spatialmetrictensorsymbol^{bd}+\spatialmetrictensorsymbol^{ad}\spatialmetrictensorsymbol^{bc}}-\dfrac{1}{3}\spatialmetrictensorsymbol^{ab}\spatialmetrictensorsymbol^{cd}}\tensorprodfour{\spatialbasevector_{a}}{\spatialbasevector_{b}}{\spatialbasevector_{c}}{\spatialbasevector_{d}}
\end{align}
and
\begin{align}
  \flattensor{\spatialsymidentitytensorfour}=
  \symtensorprod{\flattensor{\spatialmetrictensor}}{\flattensor{\spatialmetrictensor}}
  &=\spatialidentitytensorsymbol_{abcd}\,
  \tensorprodfour{\spatialbasevector^{a}}{\spatialbasevector^{b}}{\spatialbasevector^{c}}{\spatialbasevector^{d}} \\ \nonumber
  &=\dfrac{1}{2}\pbrac{\spatialmetrictensorsymbol_{ac}\spatialmetrictensorsymbol_{bd}+\spatialmetrictensorsymbol_{ad}\spatialmetrictensorsymbol_{bc}}
  \tensorprodfour{\spatialbasevector^{a}}{\spatialbasevector^{b}}{\spatialbasevector^{c}}{\spatialbasevector^{d}}\\
  \flattensor{\spatialsphericaltensorfour}=
  \dfrac{1}{3}\tensorprod{\flattensor{\spatialmetrictensor}}{\flattensor{\spatialmetrictensor}}
  &=\spatialsphericaltensorsymbol_{abcd}\,
  \tensorprodfour{\spatialbasevector^{a}}{\spatialbasevector^{b}}{\spatialbasevector^{c}}{\spatialbasevector^{d}}\\ \nonumber
  &=\dfrac{1}{3}\spatialmetrictensorsymbol_{ab}\spatialmetrictensorsymbol_{cd}\,
  \tensorprodfour{\spatialbasevector^{a}}{\spatialbasevector^{b}}{\spatialbasevector^{c}}{\spatialbasevector^{d}}\\
  \flattensor{\spatialdeviatorictensorfour}=
  \flattensor{\spatialsymidentitytensorfour}-\flattensor{\spatialsphericaltensorfour}
  &=\spatialdeviatorictensorsymbol_{abcd}\,
  \tensorprodfour{\spatialbasevector^{a}}{\spatialbasevector^{b}}{\spatialbasevector^{c}}{\spatialbasevector^{d}}\\ \nonumber
  &=\pbrac{\dfrac{1}{2}\pbrac{\spatialmetrictensorsymbol_{ac}\spatialmetrictensorsymbol_{bd}+\spatialmetrictensorsymbol_{ad}\spatialmetrictensorsymbol_{bc}}-\dfrac{1}{3}\spatialmetrictensorsymbol_{ab}\spatialmetrictensorsymbol_{cd}}
  \tensorprodfour{\spatialbasevector^{a}}{\spatialbasevector^{b}}{\spatialbasevector^{c}}{\spatialbasevector^{d}}
\end{align}

We can pull-back these spatial tensors into the material configuration \ie
\begin{align}
  \materialsymidentitytensorfour
  &=\pullback{\chi}{\spatialsymidentitytensorfour}=\symtensorprod{\materialidentitytensortwo}{\materialidentitytensortwo}
  \label{eqn:MaterialSymIdentityTensorFourPB} \\
  \pullback{\materialsphericaltensorfour}{}
  &=\pullback{\chi}{\spatialsphericaltensorfour}
  =\dfrac{1}{3}\tensorprod{\sharptensor{\pioladeformationtensor}}{\flattensor{\rightcauchygreentensor}}
  \label{eqn:MaterialSphericalTensorFourPB}\\
  \pullback{\materialdeviatorictensorfour}{}
  &=\pullback{\chi}{\spatialdeviatorictensorfour}
  =\materialsymidentitytensorfour-\pullback{}{\materialsphericaltensorfour}
  \label{eqn:MaterialDeviatoricTensorFourPB}
\end{align}
where $\materialidentitytensortwo$ is the material identity tensor \ie
$\materialidentitytensortwo=\sharptensor{\materialmetrictensor}\flattensor{\materialmetrictensor}$.

The operation of these tensors on
$\sharptensor{\tensortwo{Q}}=\pullback{\chi}{\sharptensor{\tensortwo{q}}}=\inverse{\deformationgradienttensor}\sharptensor{\tensortwo{q}}\invtranspose{\deformationgradienttensor}$,
where $\sharptensor{\tensortwo{Q}}$ is a symmetric contravariant tensor, is given by
\begin{align}
  \doubledotprod{\materialsymidentitytensorfour}{\sharptensor{\tensortwo{Q}}}&=\sharptensor{\tensortwo{Q}} \\
  \doubledotprod{\pullback{\materialsphericaltensorfour}{}}{\sharptensor{\tensortwo{Q}}}&=\dfrac{1}{3}\Trop^{\pullbacksymbol}_{\rightcauchygreentensor}\sharptensor{\tensortwo{Q}}\sharptensor{\pioladeformationtensor} \\
  \doubledotprod{\pullback{\materialdeviatorictensorfour}{}}{\sharptensor{\tensortwo{Q}}}&=\sharptensor{\tensortwo{Q}}-\dfrac{1}{3}\Trop^{\pullbacksymbol}_{\rightcauchygreentensor}\sharptensor{\tensortwo{Q}}\sharptensor{\pioladeformationtensor}
\end{align}
where the \emph{pulled back reference configuration trace operator} is definedas 
\begin{equation}
  \Trop^{\pullbacksymbol}_{\rightcauchygreentensor}\sharptensor{\tensortwo{Q}}=\pullback{\Trop}{\sharptensor{\tensortwo{Q}}}=\doubledotprod{\flattensor{\rightcauchygreentensor}}{\sharptensor{\tensortwo{Q}}}
  \label{eqn:PBMaterialTraceOperatorDefinition}
\end{equation}
\ie the trace with respect to the pulled back current configuration metric tensor,
$\flattensor{\rightcauchygreentensor}=\pullback{\chi}{\flattensor{\tensortwo{g}}}$. In other words
$\doubledotprod{\pullback{\materialsphericaltensorfour}{}}{\sharptensor{\tensortwo{Q}}}$ and
$\doubledotprod{\pullback{\materialdeviatorictensorfour}{}}{\sharptensor{\tensortwo{Q}}}$ are the pulled
back spherical and deviatoric components of $\sharptensor{\tensor{Q}}$ with respect to the
pulled back current configuration metric tensor $\flattensor{\rightcauchygreentensor}=\pullback{\chi}{\flattensor{\tensortwo{g}}}$.

The musical isomorphisms of these pulled back tensors are
\begin{align}
  \materialsymidentitytensorfour^{\pullbacksymbol\sharptensorsymbol}
  &=\pullback{\chi}{\sharptensor{\spatialsymidentitytensorfour}}
  =\symtensorprod{\sharptensor{\pioladeformationtensor}}{\sharptensor{\pioladeformationtensor}}
  \label{eqn:MaterialSymIdentityTensorFourPBSharp} \\
  \materialsphericaltensorfour^{\pullbacksymbol\sharptensorsymbol}
  &=\pullback{\chi}{\sharptensor{\spatialsphericaltensorfour}}
  =\dfrac{1}{3}\tensorprod{\sharptensor{\pioladeformationtensor}}{\sharptensor{\pioladeformationtensor}}
  \label{eqn:MaterialSphericalTensorFourPBSharp} \\\
  \materialdeviatorictensorfour^{\pullbacksymbol\sharptensorsymbol}
  &=\pullback{\chi}{\sharptensor{\spatialdeviatorictensorfour}}
  =\materialsymidentitytensorfour^{\pullbacksymbol\sharptensorsymbol}-\materialsphericaltensorfour^{\pullbacksymbol\sharptensorsymbol}
  \label{eqn:MaterialDeviatoricTensorFourPBSharp}
\end{align}
and 
\begin{align}
  \materialsymidentitytensorfour^{\pullbacksymbol\flattensorsymbol}
  &=\pullback{\chi}{\flattensor{\spatialsymidentitytensorfour}}
  =\symtensorprod{\flattensor{\rightcauchygreentensor}}{\flattensor{\rightcauchygreentensor}}
  \label{eqn:MaterialSymIdentityTensorFourPBFlat}\\
  \materialsphericaltensorfour^{\pullbacksymbol\flattensorsymbol}
  &=\pullback{\chi}{\flattensor{\spatialsphericaltensorfour}}
  =\dfrac{1}{3}\tensorprod{\flattensor{\rightcauchygreentensor}}{\flattensor{\rightcauchygreentensor}}
  \label{eqn:MaterialSphericalTensorFourPBFlat}\\
  \materialdeviatorictensorfour^{\pullbacksymbol\flattensorsymbol}
  &=\pullback{\chi}{\flattensor{\spatialdeviatorictensorfour}}
  =\materialsymidentitytensorfour^{\pullbacksymbol\flattensorsymbol}
  -\materialsphericaltensorfour^{\pullbacksymbol\flattensorsymbol}
  \label{eqn:MaterialDeviatoricTensorFourPBFlat}
\end{align}

Note that
\begin{align}
  \materialsymidentitytensorfour^{\pullbacksymbol\sharptensorsymbol}
  &=-\delby{\sharptensor{\pioladeformationtensor}}{\flattensor{\rightcauchygreentensor}}
  =-\delby{\inverse{\rightcauchygreentensor}}{\rightcauchygreentensor}
  \label{eqn:MaterialSymIdentityTensorFourPBSharpRelationship} \\
  \materialsymidentitytensorfour^{\pullbacksymbol\flattensorsymbol}
  &=-\delby{\flattensor{\rightcauchygreentensor}}{\sharptensor{\pioladeformationtensor}}
  =-\delby{\inverse{\pioladeformationtensor}}{\pioladeformationtensor}
  \label{eqn:MaterialSymIdentityTensorFourPBFlatRelationship} \\
\end{align}

We can now define for the reference configuration (\ie with respect to the
metric tensor $\flattensor{\materialmetrictensor}$) the fourth order identity,
$\materialsymidentitytensorfour$, spherical, $\materialsphericaltensorfour$, and deviatoric,
$\materialdeviatorictensorfour$, tensors as
\begin{align}
  \materialsymidentitytensorfour= \symtensorprod{\materialidentitytensortwo}{\materialidentitytensortwo}
  &=\materialidentitytensorsymbol^{AB..}_{..CD}\,
  \tensorprodfour{\materialbasevector_{A}}{\materialbasevector_{B}}{\materialbasevector^{C}}{\materialbasevector^{D}}\\ \nonumber
  &=\dfrac{1}{2}\pbrac{\mixedkronecker{A}{C}\mixedkronecker{B}{D}+\mixedkronecker{D}{A}\mixedkronecker{C}{B}}
  \tensorprodfour{\materialbasevector_{A}}{\materialbasevector_{B}}{\materialbasevector^{C}}{\materialbasevector^{D}}\\
  \materialsphericaltensorfour=\dfrac{1}{3}\tensorprod{\sharptensor{\materialmetrictensor}}{\flattensor{\materialmetrictensor}}
  &=\materialsphericaltensorsymbol^{AB..}_{..CD}\,
  \tensorprodfour{\materialbasevector_{A}}{\materialbasevector_{B}}{\materialbasevector^{C}}{\materialbasevector^{D}}\\ \nonumber
  &=\dfrac{1}{3}\materialmetrictensorsymbol^{AB}\materialmetrictensorsymbol_{CD}\,
  \tensorprodfour{\materialbasevector_{A}}{\materialbasevector_{B}}{\materialbasevector^{C}}{\materialbasevector^{D}}\\
  \materialdeviatorictensorfour=\materialsymidentitytensorfour-\materialsphericaltensorfour
  &=\materialdeviatorictensorsymbol^{AB..}_{..CD}\,
  \tensorprodfour{\materialbasevector_{A}}{\materialbasevector_{B}}{\materialbasevector^{C}}{\materialbasevector^{D}}\\ \nonumber
  &= \pbrac{\dfrac{1}{2}\pbrac{\mixedkronecker{A}{C}\mixedkronecker{B}{D}+\mixedkronecker{A}{D}\mixedkronecker{B}{C}}-\dfrac{1}{3}\materialmetrictensorsymbol^{AB}\materialmetrictensorsymbol_{CD}}
  \tensorprodfour{\materialbasevector_{A}}{\materialbasevector_{B}}{\materialbasevector^{C}}{\materialbasevector^{D}}
\end{align}

Note that these tensors act on a second order symmetric (contravariant) tensor
in the reference configuration, $\sharptensor{\tensortwo{Q}}$, such that
\begin{align}
  \doubledotprod{\materialsymidentitytensorfour}{\sharptensor{\tensortwo{Q}}}
  &=\sharptensor{\tensortwo{Q}} \\
  \doubledotprod{\materialsphericaltensorfour}{\sharptensor{\tensortwo{Q}}}
  &=\dfrac{1}{3}\Trace{\materialmetrictensor}{\sharptensor{\tensortwo{Q}}}\sharptensor{\materialmetrictensor} \\
  \doubledotprod{\materialdeviatorictensorfour}{\sharptensor{\tensortwo{Q}}}
  &=\sharptensor{\tensortwo{Q}}-\dfrac{1}{3}\Trace{{\materialmetrictensor}}{\sharptensor{\tensortwo{Q}}}\sharptensor{\materialmetrictensor}
\end{align}
where the \emph{reference configuration trace operator} is defined as
\begin{equation}
  \Trace{\materialmetrictensor}{\sharptensor{\tensortwo{Q}}}=\Trace{}{\sharptensor{\tensortwo{Q}}}=\doubledotprod{\flattensor{\materialmetrictensor}}{\sharptensor{\tensortwo{Q}}}
  \label{eqn:MaterialTraceOperatorDefinition}
\end{equation}
\ie the trace with respect to the reference configuration metric tensor, $\flattensor{\materialmetrictensor}$.

The musical isomorphisms of these tensors are
\begin{align}
  \sharptensor{\materialsymidentitytensorfour}=
  \symtensorprod{\sharptensor{\materialmetrictensor}}{\sharptensor{\materialmetrictensor}}
  &=\materialidentitytensorsymbol^{ABCD}\,
  \tensorprodfour{\materialbasevector_{A}}{\materialbasevector_{B}}{\materialbasevector_{C}}{\materialbasevector_{D}}\\ \nonumber
  &=\dfrac{1}{2}\pbrac{\materialmetrictensorsymbol^{AC}\materialmetrictensorsymbol^{BD}+\materialmetrictensorsymbol^{AD}\materialmetrictensorsymbol^{BC}}
  \tensorprodfour{\materialbasevector_{A}}{\materialbasevector_{B}}{\materialbasevector_{C}}{\materialbasevector_{D}}\\
  \sharptensor{\materialsphericaltensorfour}
  =\dfrac{1}{3}\tensorprod{\sharptensor{\materialmetrictensor}}{\sharptensor{\materialmetrictensor}}&=
  \materialsphericaltensorsymbol^{ABCD}\,
  \tensorprodfour{\materialbasevector_{A}}{\materialbasevector_{B}}{\materialbasevector_{B}}{\materialbasevector_{D}}\\ \nonumber
  &=\dfrac{1}{3}\materialmetrictensorsymbol^{AB}\materialmetrictensorsymbol^{CD}\,
  \tensorprodfour{\materialbasevector_{A}}{\materialbasevector_{B}}{\materialbasevector_{C}}{\materialbasevector_{D}}\\
  \sharptensor{\materialdeviatorictensorfour}=
  \sharptensor{\materialidentitytensorfour}-\sharptensor{\materialsphericaltensorfour}
  &=\materialdeviatorictensorsymbol^{ABCD}\,
  \tensorprodfour{\materialbasevector_{A}}{\materialbasevector_{B}}{\materialbasevector_{C}}{\materialbasevector_{D}}\\ \nonumber
  &=\pbrac{\dfrac{1}{2}\pbrac{\materialmetrictensorsymbol^{AC}\materialmetrictensorsymbol^{BD}+\materialmetrictensorsymbol^{AD}\materialmetrictensorsymbol^{BC}}-\dfrac{1}{3}\materialmetrictensorsymbol^{AB}\materialmetrictensorsymbol^{CD}}
  \tensorprodfour{\materialbasevector_{A}}{\materialbasevector_{B}}{\materialbasevector_{C}}{\materialbasevector_{D}}
\end{align}
and
\begin{align}
  \flattensor{\materialsymidentitytensorfour}=
  \symtensorprod{\flattensor{\materialmetrictensor}}{\flattensor{\materialmetrictensor}}
  &=\materialidentitytensorsymbol_{ABCD}\,
  \tensorprodfour{\materialbasevector^{A}}{\materialbasevector^{B}}{\materialbasevector^{C}}{\materialbasevector^{D}}\\ \nonumber
  &=\dfrac{1}{2}\pbrac{\materialmetrictensorsymbol_{AC}\materialmetrictensorsymbol_{BD}+\materialmetrictensorsymbol_{AD}\materialmetrictensorsymbol_{BC}}
  \tensorprodfour{\materialbasevector^{A}}{\materialbasevector^{B}}{\materialbasevector^{C}}{\materialbasevector^{D}}\\
  \flattensor{\materialsphericaltensorfour}=
  \dfrac{1}{3}\tensorprod{\flattensor{\materialmetrictensor}}{\flattensor{\materialmetrictensor}}&=
  \materialsphericaltensorfour_{ABCD}\,
  \tensorprodfour{\materialbasevector^{A}}{\materialbasevector^{B}}{\materialbasevector^{C}}{\materialbasevector^{D}}\\ \nonumber
  &=\dfrac{1}{3}\materialmetrictensorsymbol_{AB}\materialmetrictensorsymbol_{CD}\,
  \tensorprodfour{\materialbasevector^{A}}{\materialbasevector^{B}}{\materialbasevector^{C}}{\materialbasevector^{D}}\\
  \flattensor{\materialdeviatorictensorfour}=
  \flattensor{\materialsymidentitytensorfour}-\flattensor{\materialsphericaltensorfour}
  &=\materialdeviatorictensorsymbol_{ABCD}\,
  \tensorprodfour{\materialbasevector^{A}}{\materialbasevector^{B}}{\materialbasevector^{C}}{\materialbasevector^{D}}\\ \nonumber
  &=\pbrac{\dfrac{1}{2}\pbrac{\materialmetrictensorsymbol_{AC}\materialmetrictensorsymbol_{BD}+\materialmetrictensorsymbol_{AD}\materialmetrictensorsymbol_{BC}}-\dfrac{1}{3}\materialmetrictensorsymbol_{AB}\materialmetrictensorsymbol_{CD}}
  \tensorprodfour{\materialbasevector^{A}}{\materialbasevector^{B}}{\materialbasevector^{C}}{\materialbasevector^{D}}
\end{align}

We can push-forward these material tensors into the spatial configuration \ie
\begin{align}
  \spatialsymidentitytensorfour&=
  \pushforward{\chi}{\materialsymidentitytensorfour}
  =\symtensorprod{\spatialidentitytensortwo}{\spatialidentitytensortwo}\\
  \pushforward{\spatialsphericaltensorfour}{}
  &=\pushforward{\chi}{\materialsphericaltensorfour}
  =\dfrac{1}{3}\tensorprod{\sharptensor{\leftcauchygreentensor}}{\flattensor{\tensortwo{c}}}\\
  \pushforward{\spatialdeviatorictensorfour}{}
  &=\pushforward{\chi}{\materialdeviatorictensorfour}
  =\spatialsymidentitytensorfour-\pushforward{\spatialsphericaltensorfour}{}
\end{align}

The operation of these tensors on
$\sharptensor{\tensortwo{q}}=\pushforward{\chi}{\sharptensor{\tensortwo{Q}}}=\deformationgradienttensor\sharptensor{\tensortwo{Q}}\transpose{\deformationgradienttensor}$
is given by
\begin{align}
  \doubledotprod{\spatialsymidentitytensorfour}{\sharptensor{\tensortwo{q}}}
  &=\sharptensor{\tensortwo{q}} \\
  \doubledotprod{\pushforward{\spatialsphericaltensorfour}{}}{\sharptensor{\tensortwo{q}}}
  &=\dfrac{1}{3}\trop_{\pushforwardsymbol\cauchydeformationtensor}\sharptensor{\tensortwo{q}}\sharptensor{\leftcauchygreentensor} \\
  \doubledotprod{\pushforward{\spatialdeviatorictensorfour}{}}{\sharptensor{\tensortwo{q}}}
  &=\sharptensor{\tensortwo{q}}-\dfrac{1}{3}\trop_{\pushforwardsymbol\cauchydeformationtensor}\sharptensor{\tensortwo{q}}\sharptensor{\leftcauchygreentensor}
\end{align}
where the \emph{pushed forward current configuration trace operator} is
defined as
\begin{equation}
  \trop_{\pushforwardsymbol\cauchydeformationtensor}\sharptensor{\tensortwo{q}}=\pushforward{\trop}{\sharptensor{\tensortwo{q}}}=\doubledotprod{\flattensor{\cauchydeformationtensor}}{\sharptensor{\tensortwo{q}}}
  \label{eqn:PFSpatialTraceOperatorDefinition}
\end{equation}
\ie the trace with respect to the pushed forward reference configuration metric tensor,
$\flattensor{\cauchydeformationtensor}=\inverse{\sharptensor{\leftcauchygreentensor}}=\pushforward{\chi}{\flattensor{\materialmetrictensor}}$. In other words
$\doubledotprod{\pushforward{\spatialsphericaltensorfour}{}}{\sharptensor{\tensortwo{q}}}$ and
$\doubledotprod{\pushforward{\spatialdeviatorictensorfour}{}}{\sharptensor{\tensortwo{q}}}$ are the push
forward spherical and deviatoric components of $\sharptensor{\tensor{q}}$ with respect to
the pushed forward reference configuration metric tensor
$\flattensor{\cauchydeformationtensor}=\inverse{\leftcauchygreentensor}=\pushforward{\chi}{\flattensor{\materialmetrictensor}}$.

The musical isomorphisms of these pushed forward tensors are
\begin{align}
  \sharptensor{\spatialsymidentitytensorfour_{\pushforwardsymbol}}&=
  \pushforward{\chi}{\sharptensor{\materialsymidentitytensorfour}}=
  \symtensorprod{\sharptensor{\leftcauchygreentensor}}{\sharptensor{\leftcauchygreentensor}}\\
  \sharptensor{\spatialsphericaltensorfour_{\pushforwardsymbol}}&=
  \pushforward{\chi}{\sharptensor{\materialsphericaltensorfour}}=
  \dfrac{1}{3}\tensorprod{\sharptensor{\leftcauchygreentensor}}{\sharptensor{\leftcauchygreentensor}}\\
  \sharptensor{\spatialdeviatorictensorfour_{\pushforwardsymbol}}&=
  \pushforward{\chi}{\sharptensor{\materialdeviatorictensorfour}}=
  \sharptensor{\spatialsymidentitytensorfour_{\pushforwardsymbol}}-\sharptensor{\spatialsphericaltensorfour_{\pushforwardsymbol}}
\end{align}
and
\begin{align}
  \flattensor{\spatialsymidentitytensorfour_{\pushforwardsymbol}}&=
  \pushforward{\chi}{\flattensor{\materialsymidentitytensorfour}}=
  \symtensorprod{\cauchydeformationtensor}{\cauchydeformationtensor}\\
  \flattensor{\spatialsphericaltensorfour_{\pushforwardsymbol}}&=
  \pushforward{\chi}{\flattensor{\materialsphericaltensorfour}}=
  \dfrac{1}{3}\tensorprod{\cauchydeformationtensor}{\cauchydeformationtensor}\\
  \flattensor{\spatialdeviatorictensorfour_{\pushforwardsymbol}}&=
  \pushforward{\chi}{\flattensor{\materialdeviatorictensorfour}}=
  \flattensor{\spatialsymidentitytensorfour_{\pushforwardsymbol}}-\flattensor{\spatialsphericaltensorfour_{\pushforwardsymbol}}
\end{align}

Note that
\begin{align}
  \sharptensor{\spatialsymidentitytensorfour_{\pushforwardsymbol}}
  &=-\delby{\leftcauchygreentensor}{\leftcauchygreentensor}
  =-\delby{\inverse{\cauchydeformationtensor}}{\cauchydeformationtensor} \\
  \flattensor{\spatialsymidentitytensorfour_{\pushforwardsymbol}}
  &=-\delby{\cauchydeformationtensor}{\leftcauchygreentensor}
  =-\delby{\inverse{\leftcauchygreentensor}}{\leftcauchygreentensor} \\
\end{align}

From \eqnref{eqn:MaterialSphericalTensorFourPB} we have
\begin{equation}
  \fnof{\pullback{\materialsphericaltensorfour}{}}{\rightcauchygreentensor}=\dfrac{\tensorprod{\pioladeformationtensor}{\rightcauchygreentensor}}{3}
\end{equation}
and from \eqnref{eqn:MaterialDeviatoricTensorFourPB} we have \todo{WHY TRANSPOSE??? DO COMPONENT FORM? WHICH FORTH ORDER TRANSPOSE OPERATOR? CHECK???}
\begin{equation}
  \fnof{\pullback{\materialdeviatorictensorfour}{}}{\rightcauchygreentensor}=\materialsymidentitytensorfour-\dfrac{\tensorprod{\pioladeformationtensor}{\rightcauchygreentensor}}{3}  
\end{equation}
and so \eqnref{eqn:DelCBarDelC} becomes
\begin{equation}
  \delby{\bar{\rightcauchygreentensor}}{\rightcauchygreentensor}=\fnof{J^{-\frac{2}{3}}}{\rightcauchygreentensor}\pbrac{\materialsymidentitytensorfour-\dfrac{\tensorprod{\pioladeformationtensor}{\rightcauchygreentensor}}{3}}=\fnof{J^{-\frac{2}{3}}}{\rightcauchygreentensor}\transpose{\fnof{\materialdeviatorictensorfour^{\pullbacksymbol}}{\rightcauchygreentensor}}
  \label{eqn:DelCBarDelCDeviatoricTensorForm}
\end{equation}

\Eqnrefs{eqn:FiniteElasticityDecomposedSecondPKinC}{eqn:FiniteElasticityDecomposedSecondPKinE} now become
\begin{equation}
  \begin{split}
    \fnof{\secondpiolakirchoffstresstensor}{\rightcauchygreentensor}
    &=\doubledotprod{\fnof{\bar{\secondpiolakirchoffstresstensor}}{\fnof{\bar{\rightcauchygreentensor}}{\rightcauchygreentensor}}}{\delby{\fnof{\bar{\rightcauchygreentensor}}{\rightcauchygreentensor}}{\rightcauchygreentensor}}
    -2p\delby{\fnof{J}{\rightcauchygreentensor}}{\rightcauchygreentensor}
    =\doubledotprod{\fnof{\bar{\secondpiolakirchoffstresstensor}}{\fnof{\bar{\greenlagrangestraintensor}}{\greenlagrangestraintensor}}}{\delby{\fnof{\bar{\greenlagrangestraintensor}}{\greenlagrangestraintensor}}{\greenlagrangestraintensor}}
    -p\delby{\fnof{J}{\greenlagrangestraintensor}}{\greenlagrangestraintensor}\\
    &=\doubledotprod{\fnof{\bar{\secondpiolakirchoffstresstensor}}{\fnof{\bar{\rightcauchygreentensor}}{\rightcauchygreentensor}}}{\fnof{J^{-\frac{2}{3}}}{\rightcauchygreentensor}\pbrac{\materialsymidentitytensorfour-\dfrac{\tensorprod{\pioladeformationtensor}{\rightcauchygreentensor}}{3}}}
    -p\fnof{J}{\rightcauchygreentensor}\pioladeformationtensor\\
    &=\doubledotprod{\fnof{\bar{\secondpiolakirchoffstresstensor}}{\fnof{\bar{\rightcauchygreentensor}}{\rightcauchygreentensor}}}{\fnof{J^{-\frac{2}{3}}}{\rightcauchygreentensor}\transpose{\fnof{\materialdeviatorictensorfour^{\pullbacksymbol}}{\rightcauchygreentensor}}}
    -p\fnof{J}{\rightcauchygreentensor}\pioladeformationtensor\\
    &=\fnof{J^{-\frac{2}{3}}}{\rightcauchygreentensor}\pbrac{\doubledotprod{\fnof{\pullback{\materialdeviatorictensorfour}{}}{\rightcauchygreentensor}}{\bar{\secondpiolakirchoffstresstensor}}}
    -p\fnof{J}{\rightcauchygreentensor}\pioladeformationtensor\\
    &=\doubledotprod{\fnof{\pullback{\materialdeviatorictensorfour}{}}{\rightcauchygreentensor}}{\fnof{\secondpiolakirchoffstresstensor}{\rightcauchygreentensor}}+
    \doubledotprod{\fnof{\pullback{\materialsphericaltensorfour}{}}{\rightcauchygreentensor}}{\fnof{\secondpiolakirchoffstresstensor}{\rightcauchygreentensor}}\\
    &=\fnof{\secondpiolakirchoffstresstensor_{\pullback{\Devop}{}}}{\rightcauchygreentensor}+\fnof{\secondpiolakirchoffstresstensor_{\pullback{\Sphop}{}}}{\rightcauchygreentensor}
  \end{split}
\end{equation}
where
\begin{equation}
  \fnof{\secondpiolakirchoffstresstensor_{\pullback{\Devop}{}}}{\rightcauchygreentensor}
  =\doubledotprod{\fnof{\pullback{\materialdeviatorictensorfour}{}}{\rightcauchygreentensor}}{\fnof{\secondpiolakirchoffstresstensor}{\rightcauchygreentensor}}
  =\fnof{J^{-\frac{2}{3}}}{\rightcauchygreentensor}\pbrac{\doubledotprod{\fnof{\pullback{\materialdeviatorictensorfour}{}}{\rightcauchygreentensor}}{\fnof{\bar{\secondpiolakirchoffstresstensor}}{\fnof{\bar{\rightcauchygreentensor}}{\rightcauchygreentensor}}}}
  \label{eqn:SecondPiolaKirchoffDeviatoricPBDefinition}
\end{equation}
and
\begin{equation}
  \fnof{\secondpiolakirchoffstresstensor_{\pullback{\Sphop}{}}}{\rightcauchygreentensor}
  =\doubledotprod{\fnof{\pullback{\materialsphericaltensorfour}{}}{\rightcauchygreentensor}}{\fnof{\secondpiolakirchoffstresstensor}{\rightcauchygreentensor}}
  =-p\fnof{J}{\rightcauchygreentensor}\inverse{\rightcauchygreentensor}=-p\fnof{J}{\rightcauchygreentensor}\pioladeformationtensor
  \label{eqn:SecondPiolaKirchoffSphericalPBDefinition}
\end{equation}

We can now push the second Piola-Kirchoff stress forward to give the Cauchy
stress \ie
\begin{equation}
  \begin{split}
    \fnof{\cauchystresstensor}{\spatialmetrictensor}&=\fnof{\inverse{J}}{\rightcauchygreentensor}\pushforward{\chi}{\fnof{\secondpiolakirchoffstresstensor}{\rightcauchygreentensor}}
    =\fnof{\inverse{J}}{\rightcauchygreentensor}\deformationgradienttensor\fnof{\secondpiolakirchoffstresstensor}{\rightcauchygreentensor}\transpose{\deformationgradienttensor}\\
    &=\fnof{\inverse{J}}{\rightcauchygreentensor}\pushforward{\chi}{\pbrac{\fnof{\secondpiolakirchoffstresstensor_{\pullback{\Devop}{}}}{\rightcauchygreentensor}
        +\fnof{\secondpiolakirchoffstresstensor_{\pullback{\Sphop}{}}}{\rightcauchygreentensor}}}\\
    &=\fnof{J^{-\frac{2}{3}}}{\rightcauchygreentensor}\doubledotprod{\fnof{\spatialdeviatorictensorfour}{\spatialmetrictensor}}{\fnof{\bar{\cauchystresstensor}}{\fnof{\bar{\leftcauchygreentensor}}{\spatialmetrictensor}}}-p\inverse{\spatialmetrictensor}\\
    &=\doubledotprod{\fnof{\spatialdeviatorictensorfour}{\spatialmetrictensor}}{\fnof{\cauchystresstensor}{\spatialmetrictensor}}+
    \doubledotprod{\fnof{\spatialsphericaltensorfour}{\spatialmetrictensor}}{\fnof{\cauchystresstensor}{\spatialmetrictensor}}\\
    &=\fnof{\cauchystresstensor_{\devop}}{\spatialmetrictensor}+\fnof{\cauchystresstensor_{\sphop}}{\spatialmetrictensor}
  \end{split}
\end{equation}
where
\begin{equation}
  \fnof{\cauchystresstensor_{\devop}}{\spatialmetrictensor}=\doubledotprod{\fnof{\spatialdeviatorictensorfour}{\spatialmetrictensor}}{\fnof{\cauchystresstensor}{\spatialmetrictensor}}
  =\fnof{J^{-\frac{2}{3}}}{\rightcauchygreentensor}\doubledotprod{\fnof{\spatialdeviatorictensorfour}{\spatialmetrictensor}}{\fnof{\bar{\cauchystresstensor}}{\fnof{\bar{\leftcauchygreentensor}}{\spatialmetrictensor}}}
  \label{eqn:DeviatoricCauchyStress}
\end{equation}
and
\begin{equation}
  \fnof{\cauchystresstensor_{\sphop}}{\spatialmetrictensor}=\doubledotprod{\fnof{\spatialsphericaltensorfour}{\spatialmetrictensor}}{\fnof{\cauchystresstensor}{\spatialmetrictensor}}=-p\inverse{\spatialmetrictensor}
  \label{eqn:SphericalCauchyStress}
\end{equation}
where
\begin{equation}
  \fnof{\bar{\cauchystresstensor}}{\fnof{\bar{\leftcauchygreentensor}}{\spatialmetrictensor}}=\fnof{\inverse{J}}{\rightcauchygreentensor}\pushforward{\chi}{\fnof{\bar{\secondpiolakirchoffstresstensor}}{\fnof{\bar{\rightcauchygreentensor}}{\rightcauchygreentensor}}}
\end{equation}
defines the \emph{Cauchy psuedo stress
tensor}\symbolat{$\bar{\cauchystresstensor}$}{Cauchy psuedo stress
  tensor}.

We can also use the constitutive law to obtain the elasticity
tensor. Following \citet{holzapfel:2000} we obtain
\begin{equation}
  \begin{split}
    \fnof{\materialsecondelasticitytensor}{\rightcauchygreentensor}&=2\delby{\fnof{\secondpiolakirchoffstresstensor}{\rightcauchygreentensor}}{\rightcauchygreentensor}\\
    &=2\pbrac{\delby{\fnof{\secondpiolakirchoffstresstensor_{\pullback{\Devop}{}}}{\fnof{\bar{\rightcauchygreentensor}}{\rightcauchygreentensor}}}{\rightcauchygreentensor}+\delby{\fnof{\secondpiolakirchoffstresstensor_{\pullback{\Sphop}{}}}{\fnof{J}{\rightcauchygreentensor}}}{\rightcauchygreentensor}}\\
    &=\fnof{\materialsecondelasticitytensor_{\pullback{\Devop}{}}}{\fnof{\bar{\rightcauchygreentensor}}{\rightcauchygreentensor}}+\fnof{\materialsecondelasticitytensor_{\pullback{\Sphop}{}}}{\fnof{J}{\rightcauchygreentensor}}
  \end{split}
\end{equation}
or 
\begin{equation}
  \begin{split}
    \fnof{\materialsecondelasticitytensor}{\greenlagrangestraintensor}&=\delby{\fnof{\secondpiolakirchoffstresstensor}{\greenlagrangestraintensor}}{\greenlagrangestraintensor}\\
    &=\delby{\fnof{\secondpiolakirchoffstresstensor_{\pullback{\Devop}{}}}{\fnof{\bar{\greenlagrangestraintensor}}{\greenlagrangestraintensor}}}{\greenlagrangestraintensor}+\delby{\fnof{\secondpiolakirchoffstresstensor_{\pullback{\Sphop}{}}}{\fnof{J}{\greenlagrangestraintensor}}}{\greenlagrangestraintensor}\\
   &=\fnof{\materialsecondelasticitytensor_{\pullback{\Devop}{}}}{\fnof{\bar{\greenlagrangestraintensor}}{\greenlagrangestraintensor}}+\fnof{\materialsecondelasticitytensor_{\pullback{\Sphop}{}}}{\fnof{J}{\greenlagrangestraintensor}}
  \end{split}
\end{equation}

If we consider the deviatoric component of the second elasticity tensor \ie
\begin{equation}
  \begin{split}
    \fnof{\materialsecondelasticitytensor_{\pullback{\Devop}{}}}{\fnof{\bar{\rightcauchygreentensor}}{\rightcauchygreentensor}}&=2\delby{\fnof{\secondpiolakirchoffstresstensor_{\pullback{\Devop}{}}}{\fnof{\bar{\rightcauchygreentensor}}{\rightcauchygreentensor}}}{\rightcauchygreentensor}\\
    &=2\delby{}{\rightcauchygreentensor}\pbrac{2\delby{\fnof{\bar{W}_{\Devop}}{\fnof{\bar{\rightcauchygreentensor}}{\rightcauchygreentensor}}}{\rightcauchygreentensor}} \\
    &=2\delby{}{\rightcauchygreentensor}\pbrac{\doubledotprod{2\delby{\fnof{\bar{W}_{\Devop}}{\fnof{\bar{\rightcauchygreentensor}}{\rightcauchygreentensor}}}{\bar{\rightcauchygreentensor}}}{\delby{\fnof{\bar{\rightcauchygreentensor}}{\rightcauchygreentensor}}{\rightcauchygreentensor}}}
  \end{split}
  \label{eqn:MaterialDeviatoricSecondElasticityTensor1}
\end{equation}

Now substituting \eqnrefs{eqn:SecondPiolaKirchoffPsuedoStressDefinition}{eqn:DelCBarDelCDeviatoricTensorForm} into \eqnref{eqn:MaterialDeviatoricSecondElasticityTensor1} gives
\begin{equation}
  \begin{split}
    \fnof{\materialsecondelasticitytensor_{\pullback{\Devop}{}}}{\fnof{\bar{\rightcauchygreentensor}}{\rightcauchygreentensor}}
    &=2\delby{}{\rightcauchygreentensor}\pbrac{\doubledotprod{2\delby{\fnof{\bar{W}_{\Devop}}{\fnof{\bar{\rightcauchygreentensor}}{\rightcauchygreentensor}}}{\bar{\rightcauchygreentensor}}}{\delby{\fnof{\bar{\rightcauchygreentensor}}{\rightcauchygreentensor}}{\rightcauchygreentensor}}}\\
    &=2\delby{}{\rightcauchygreentensor}\pbrac{\doubledotprod{\fnof{\bar{\secondpiolakirchoffstresstensor}}{\fnof{\bar{\rightcauchygreentensor}}{\rightcauchygreentensor}}}{\fnof{J^{-\frac{2}{3}}}{\rightcauchygreentensor}\transpose{\fnof{\materialdeviatorictensorfour^{\pullbacksymbol}}{\rightcauchygreentensor}}}}\\
    &=2\delby{}{\rightcauchygreentensor}\pbrac{\fnof{J^{-\frac{2}{3}}}{\rightcauchygreentensor}\pbrac{\doubledotprod{\fnof{\materialdeviatorictensorfour^{\pullbacksymbol}}{\rightcauchygreentensor}}{\fnof{\bar{\secondpiolakirchoffstresstensor}}{\fnof{\bar{\rightcauchygreentensor}}{\rightcauchygreentensor}}}}}\\
    &=\tensorprod{2\pbrac{\doubledotprod{\fnof{\materialdeviatorictensorfour^{\pullbacksymbol}}{\rightcauchygreentensor}}{\fnof{\bar{\secondpiolakirchoffstresstensor}}{\fnof{\bar{\rightcauchygreentensor}}{\rightcauchygreentensor}}}}}{\delby{\pbrac{\fnof{J^{-\frac{2}{3}}}{\rightcauchygreentensor}}}{\rightcauchygreentensor}}\\
    &\qquad+2\fnof{J^{-\frac{2}{3}}}{\rightcauchygreentensor}\delby{\pbrac{\doubledotprod{\fnof{\materialdeviatorictensorfour^{\pullbacksymbol}}{\rightcauchygreentensor}}{\fnof{\bar{\secondpiolakirchoffstresstensor}}{\fnof{\bar{\rightcauchygreentensor}}{\rightcauchygreentensor}}}}}{\rightcauchygreentensor}
  \end{split}
  \label{eqn:MaterialDeviatoricSecondElasticityTensor2}
\end{equation}

Substituting
\eqnrefs{eqn:DelTwoThirdsJacobianDelC}{eqn:SecondPiolaKirchoffDeviatoricPBDefinition}
into the first part of the right hand side of
\eqnref{eqn:MaterialDeviatoricSecondElasticityTensor2} gives
\begin{equation}
  \begin{split}
    \tensorprod{2\pbrac{\doubledotprod{\fnof{\materialdeviatorictensorfour^{\pullbacksymbol}}{\rightcauchygreentensor}}{\fnof{\bar{\secondpiolakirchoffstresstensor}}{\fnof{\bar{\rightcauchygreentensor}}{\rightcauchygreentensor}}}}}{\delby{\pbrac{\fnof{J^{-\frac{2}{3}}}{\rightcauchygreentensor}}}{\rightcauchygreentensor}}
    &=\tensorprod{2\pbrac{\doubledotprod{\fnof{\materialdeviatorictensorfour^{\pullbacksymbol}}{\rightcauchygreentensor}}{\fnof{\bar{\secondpiolakirchoffstresstensor}}{\fnof{\bar{\rightcauchygreentensor}}{\rightcauchygreentensor}}}}}{\dfrac{-\fnof{J^{-\frac{2}{3}}}{\rightcauchygreentensor}\pioladeformationtensor}{3}}\\
    &=-\dfrac{2}{3}\tensorprod{\fnof{J^{-\frac{2}{3}}}{\rightcauchygreentensor}\pbrac{\doubledotprod{\fnof{\materialdeviatorictensorfour^{\pullbacksymbol}}{\rightcauchygreentensor}}{\fnof{\bar{\secondpiolakirchoffstresstensor}}{\fnof{\bar{\rightcauchygreentensor}}{\rightcauchygreentensor}}}}}{\pioladeformationtensor}\\
    &=-\dfrac{2}{3}\tensorprod{\fnof{\secondpiolakirchoffstresstensor_{\pullback{\Devop}{}}}{\rightcauchygreentensor}}{\pioladeformationtensor}
  \end{split}
  \label{eqn:FirstPartRHSMaterialDeviatoricSecondElasticityTensor2}
\end{equation}

Substituting \eqnrefs{eqn:MaterialSphericalTensorFourPB}{eqn:MaterialDeviatoricTensorFourPB} into the second part on the right hand side of \eqnref{eqn:MaterialDeviatoricSecondElasticityTensor2} gives
\begin{equation}
  \begin{split}
    2\fnof{J^{-\frac{2}{3}}}{\rightcauchygreentensor}&\delby{\pbrac{\doubledotprod{\fnof{\materialdeviatorictensorfour^{\pullbacksymbol}}{\rightcauchygreentensor}}{\fnof{\bar{\secondpiolakirchoffstresstensor}}{\fnof{\bar{\rightcauchygreentensor}}{\rightcauchygreentensor}}}}}{\rightcauchygreentensor}\\
    &= 2\fnof{J^{-\frac{2}{3}}}{\rightcauchygreentensor}\delby{}{\rightcauchygreentensor}\pbrac{\doubledotprod{\pbrac{\materialsymidentitytensorfour-\dfrac{\tensorprod{\pioladeformationtensor}{\rightcauchygreentensor}}{3}}}{\fnof{\bar{\secondpiolakirchoffstresstensor}}{\fnof{\bar{\rightcauchygreentensor}}{\rightcauchygreentensor}}}}\\
    &= 2\fnof{J^{-\frac{2}{3}}}{\rightcauchygreentensor}\delby{}{\rightcauchygreentensor}\pbrac{\doubledotprod{\materialsymidentitytensorfour}{\fnof{\bar{\secondpiolakirchoffstresstensor}}{\fnof{\bar{\rightcauchygreentensor}}{\rightcauchygreentensor}}}-\doubledotprod{\pbrac{\dfrac{\tensorprod{\pioladeformationtensor}{\rightcauchygreentensor}}{3}}}{\fnof{\bar{\secondpiolakirchoffstresstensor}}{\fnof{\bar{\rightcauchygreentensor}}{\rightcauchygreentensor}}}}
  \end{split}
  \label{eqn:SecondPartRHSMaterialDeviatoricSecondElasticityTensor1}
\end{equation}

Now by applying \eqnrefs{eqn:DoubleDotSymIdentityFourTwoTensor}{eqn:DoubleDotProdFourTwoProperty1} to \eqnref{eqn:SecondPartRHSMaterialDeviatoricSecondElasticityTensor1} gives
\begin{equation}
  \begin{split}
    2\fnof{J^{-\frac{2}{3}}}{\rightcauchygreentensor}&\delby{\pbrac{\doubledotprod{\fnof{\materialdeviatorictensorfour^{\pullbacksymbol}}{\rightcauchygreentensor}}{\fnof{\bar{\secondpiolakirchoffstresstensor}}{\fnof{\bar{\rightcauchygreentensor}}{\rightcauchygreentensor}}}}}{\rightcauchygreentensor}\\
    &=2\fnof{J^{-\frac{2}{3}}}{\rightcauchygreentensor}\delby{}{\rightcauchygreentensor}\pbrac{\fnof{\bar{\secondpiolakirchoffstresstensor}}{\fnof{\bar{\rightcauchygreentensor}}{\rightcauchygreentensor}}-\dfrac{\pbrac{\doubledotprod{\rightcauchygreentensor}{\fnof{\bar{\secondpiolakirchoffstresstensor}}{\fnof{\bar{\rightcauchygreentensor}}{\rightcauchygreentensor}}}}\pioladeformationtensor}{3}}\\
    &=2\fnof{J^{-\frac{2}{3}}}{\rightcauchygreentensor}\pbrac{\delby{\fnof{\bar{\secondpiolakirchoffstresstensor}}{\fnof{\bar{\rightcauchygreentensor}}{\rightcauchygreentensor}}}{\rightcauchygreentensor}-\dfrac{1}{3}\delby{\pbrac{\pbrac{\doubledotprod{\rightcauchygreentensor}{\fnof{\bar{\secondpiolakirchoffstresstensor}}{\fnof{\bar{\rightcauchygreentensor}}{\rightcauchygreentensor}}}}\pioladeformationtensor}}{\rightcauchygreentensor}}\\
   &=2\fnof{J^{-\frac{2}{3}}}{\rightcauchygreentensor}\doubledotprod{\pbrac{\delby{\fnof{\bar{\secondpiolakirchoffstresstensor}}{\fnof{\bar{\rightcauchygreentensor}}{\rightcauchygreentensor}}}{\bar{\rightcauchygreentensor}}-\dfrac{1}{3}\delby{\pbrac{\pbrac{\doubledotprod{\rightcauchygreentensor}{\fnof{\bar{\secondpiolakirchoffstresstensor}}{\fnof{\bar{\rightcauchygreentensor}}{\rightcauchygreentensor}}}}\pioladeformationtensor}}{\bar{\rightcauchygreentensor}}}}{\delby{\fnof{\bar{\rightcauchygreentensor}}{\rightcauchygreentensor}}{\rightcauchygreentensor}}
  \end{split}
  \label{eqn:SecondPartRHSMaterialDeviatoricSecondElasticityTensor2}
\end{equation}

Now, substituting \eqnref{eqn:DelCBarDelCDeviatoricTensorForm} into \eqnref{eqn:SecondPartRHSMaterialDeviatoricSecondElasticityTensor2} and applying the chain rule (TODO: PUT IN AND REFERENCE) gives
\begin{equation}
  \begin{split}
    2\fnof{J^{-\frac{2}{3}}}{\rightcauchygreentensor}&\delby{\pbrac{\doubledotprod{\fnof{\materialdeviatorictensorfour^{\pullbacksymbol}}{\rightcauchygreentensor}}{\fnof{\bar{\secondpiolakirchoffstresstensor}}{\fnof{\bar{\rightcauchygreentensor}}{\rightcauchygreentensor}}}}}{\rightcauchygreentensor}\\
    &=2\fnof{J^{-\frac{2}{3}}}{\rightcauchygreentensor}\doubledotprod{\pbrac{\delby{\fnof{\bar{\secondpiolakirchoffstresstensor}}{\fnof{\bar{\rightcauchygreentensor}}{\rightcauchygreentensor}}}{\bar{\rightcauchygreentensor}}-\dfrac{1}{3}\delby{\pbrac{\pbrac{\doubledotprod{\rightcauchygreentensor}{\fnof{\bar{\secondpiolakirchoffstresstensor}}{\fnof{\bar{\rightcauchygreentensor}}{\rightcauchygreentensor}}}}\pioladeformationtensor}}{\bar{\rightcauchygreentensor}}}}{\fnof{J^{-\frac{2}{3}}}{\rightcauchygreentensor}\transpose{\fnof{\materialdeviatorictensorfour^{\pullbacksymbol}}{\rightcauchygreentensor}}}\\
    &=2\fnof{J^{-\frac{4}{3}}}{\rightcauchygreentensor}\doubledotprod{\pbrac{\delby{\fnof{\bar{\secondpiolakirchoffstresstensor}}{\fnof{\bar{\rightcauchygreentensor}}{\rightcauchygreentensor}}}{\bar{\rightcauchygreentensor}}-\dfrac{1}{3}\pbrac{\tensorprod{\pioladeformationtensor}{\delby{\pbrac{\pbrac{\doubledotprod{\rightcauchygreentensor}{\fnof{\bar{\secondpiolakirchoffstresstensor}}{\fnof{\bar{\rightcauchygreentensor}}{\rightcauchygreentensor}}}}}}{\bar{\rightcauchygreentensor}}}+\pbrac{\doubledotprod{\rightcauchygreentensor}{\fnof{\bar{\secondpiolakirchoffstresstensor}}{\fnof{\bar{\rightcauchygreentensor}}{\rightcauchygreentensor}}}}\delby{\pioladeformationtensor}{\bar{\rightcauchygreentensor}}}}}{\transpose{\fnof{\materialdeviatorictensorfour^{\pullbacksymbol}}{\rightcauchygreentensor}}}\\
    &=2\fnof{J^{-\frac{4}{3}}}{\rightcauchygreentensor}\doubledotprod{\delby{\fnof{\bar{\secondpiolakirchoffstresstensor}}{\fnof{\bar{\rightcauchygreentensor}}{\rightcauchygreentensor}}}{\bar{\rightcauchygreentensor}}}{\transpose{\fnof{\materialdeviatorictensorfour^{\pullbacksymbol}}{\rightcauchygreentensor}}}\\
    &\qquad-\dfrac{2}{3}\fnof{J^{-\frac{4}{3}}}{\rightcauchygreentensor}\doubledotprod{\pbrac{\tensorprod{\pioladeformationtensor}{\delby{\pbrac{\pbrac{\doubledotprod{\rightcauchygreentensor}{\fnof{\bar{\secondpiolakirchoffstresstensor}}{\fnof{\bar{\rightcauchygreentensor}}{\rightcauchygreentensor}}}}}}{\bar{\rightcauchygreentensor}}}+\pbrac{\doubledotprod{\rightcauchygreentensor}{\fnof{\bar{\secondpiolakirchoffstresstensor}}{\fnof{\bar{\rightcauchygreentensor}}{\rightcauchygreentensor}}}}\delby{\pioladeformationtensor}{\bar{\rightcauchygreentensor}}}}{\transpose{\fnof{\materialdeviatorictensorfour^{\pullbacksymbol}}{\rightcauchygreentensor}}}\\
    &=2\fnof{J^{-\frac{4}{3}}}{\rightcauchygreentensor}\doubledotprod{\delby{\fnof{\bar{\secondpiolakirchoffstresstensor}}{\fnof{\bar{\rightcauchygreentensor}}{\rightcauchygreentensor}}}{\bar{\rightcauchygreentensor}}}{\transpose{\fnof{\materialdeviatorictensorfour^{\pullbacksymbol}}{\rightcauchygreentensor}}}\\
    &\qquad-\dfrac{2}{3}\fnof{J^{-\frac{4}{3}}}{\rightcauchygreentensor}\doubledotprod{\pbrac{\tensorprod{\pioladeformationtensor}{\delby{\pbrac{\pbrac{\doubledotprod{\rightcauchygreentensor}{\fnof{\bar{\secondpiolakirchoffstresstensor}}{\fnof{\bar{\rightcauchygreentensor}}{\rightcauchygreentensor}}}}}}{\bar{\rightcauchygreentensor}}}}}{\transpose{\fnof{\materialdeviatorictensorfour^{\pullbacksymbol}}{\rightcauchygreentensor}}}\\
    &\qquad\qquad-\dfrac{2}{3}\fnof{J^{-\frac{4}{3}}}{\rightcauchygreentensor}\doubledotprod{\pbrac{
        \doubledotprod{\rightcauchygreentensor}{\fnof{\bar{\secondpiolakirchoffstresstensor}}{\fnof{\bar{\rightcauchygreentensor}}{\rightcauchygreentensor}}}
        }\delby{\pioladeformationtensor}{\bar{\rightcauchygreentensor}}}{\transpose{\fnof{\materialdeviatorictensorfour^{\pullbacksymbol}}{\rightcauchygreentensor}}}
  \end{split}
  \label{eqn:SecondPartRHSMaterialDeviatoricSecondElasticityTensor3}
\end{equation}

We now define
\begin{equation}
  \fnof{\bar{\materialsecondelasticitytensor}}{\fnof{\bar{\rightcauchygreentensor}}{\rightcauchygreentensor}}=2\fnof{J^{-\frac{4}{3}}}{\rightcauchygreentensor}\delby{\fnof{\bar{\secondpiolakirchoffstresstensor}}{\fnof{\bar{\rightcauchygreentensor}}{\rightcauchygreentensor}}}{\bar{\rightcauchygreentensor}}=4\fnof{J^{-\frac{4}{3}}}{\rightcauchygreentensor}\deltwosqby{\fnof{\bar{W}_{\Devop}}{\fnof{\bar{\rightcauchygreentensor}}{\rightcauchygreentensor}}}{\bar{\rightcauchygreentensor}}=2\fnof{J^{-\frac{4}{3}}}{\rightcauchygreentensor}\deltwosqby{\fnof{\bar{W}_{\Devop}}{\fnof{\bar{\greenlagrangestraintensor}}{\greenlagrangestraintensor}}}{\bar{\greenlagrangestraintensor}}
  \label{eqn:MaterialPsuedoSecondElasticityTensorDefinition}
\end{equation}
where $\fnof{\bar{\materialsecondelasticitytensor}}{\fnof{\bar{\rightcauchygreentensor}}{\rightcauchygreentensor}}$ is the \emph{material psuedo second elasticity
tensor}\symbolat{$\bar{\materialsecondelasticitytensor}$}{Material pseudo second elasticity tensor} and note that from \eqnref{eqn:PBMaterialTraceOperatorDefinition} we have
\begin{equation}
  \doubledotprod{\rightcauchygreentensor}{\fnof{\bar{\secondpiolakirchoffstresstensor}}{\fnof{\bar{\rightcauchygreentensor}}{\rightcauchygreentensor}}}=\pullback{\Trop}{\fnof{\bar{\secondpiolakirchoffstresstensor}}{\fnof{\bar{\rightcauchygreentensor}}{\rightcauchygreentensor}}}
  \label{eqn:PsuedoStressPBTraceDefinition}
\end{equation}

Subsituting
\eqnrefs{eqn:MaterialPsuedoSecondElasticityTensorDefinition}{eqn:PsuedoStressPBTraceDefinition}
into
\eqnref{eqn:SecondPartRHSMaterialDeviatoricSecondElasticityTensor3}
gives
\begin{equation}
  \begin{split}
    2\fnof{J^{-\frac{2}{3}}}{\rightcauchygreentensor}&
    \delby{
      \pbrac{
        \doubledotprod{
          \fnof{\materialdeviatorictensorfour^{\pullbacksymbol}}{\rightcauchygreentensor}
        }{
          \fnof{\bar{\secondpiolakirchoffstresstensor}}{\fnof{\bar{\rightcauchygreentensor}}{\rightcauchygreentensor}}
        }
      }
    }{
      \rightcauchygreentensor
    }=
    \doubledotprod{
      \fnof{\bar{\materialsecondelasticitytensor}}{\fnof{\bar{\rightcauchygreentensor}}{\rightcauchygreentensor}}
    }{
      \transpose{
        \fnof{\materialdeviatorictensorfour^{\pullbacksymbol}}{\rightcauchygreentensor}
      }
    }\\    
    &-\dfrac{2}{3}\fnof{J^{-\frac{4}{3}}}{\rightcauchygreentensor}
    \doubledotprod{
      \pbrac{
        \tensorprod{
          \pioladeformationtensor
        }{
          \delby{
            \pbrac{
              \doubledotprod{
                \rightcauchygreentensor
              }{
                \fnof{\bar{\secondpiolakirchoffstresstensor}}{\fnof{\bar{\rightcauchygreentensor}}{\rightcauchygreentensor}}
              }
            }
          }{
            \bar{\rightcauchygreentensor}
          }
        }
      }
    }{
      \transpose{
        \fnof{\materialdeviatorictensorfour^{\pullbacksymbol}}{\rightcauchygreentensor}
      }
    }\\
    &\qquad-\dfrac{2}{3}
    \fnof{J^{-\frac{4}{3}}}{\rightcauchygreentensor}
    \pullback{\Trop}{
      \fnof{\bar{\secondpiolakirchoffstresstensor}}{\fnof{\bar{\rightcauchygreentensor}}{\rightcauchygreentensor}}
    }
    \doubledotprod{
      \delby{
        \pioladeformationtensor
      }{
        \bar{\rightcauchygreentensor}
      }
    }{
      \transpose{
        \fnof{\materialdeviatorictensorfour^{\pullbacksymbol}}{\rightcauchygreentensor}
      }
    }\\
    &=\doubledotprod{
      \fnof{\bar{\materialsecondelasticitytensor}}{\fnof{\bar{\rightcauchygreentensor}}{\rightcauchygreentensor}}
    }{
      \transpose{
        \fnof{\materialdeviatorictensorfour^{\pullbacksymbol}}{\rightcauchygreentensor}
      }
    }
    -\dfrac{2}{3}\fnof{J^{-\frac{4}{3}}}{\rightcauchygreentensor}
    \pullback{\Trop}{
      \fnof{\bar{\secondpiolakirchoffstresstensor}}{\fnof{\bar{\rightcauchygreentensor}}{\rightcauchygreentensor}}
    }
    \doubledotprod{
      \delby{
        \pioladeformationtensor
      }{
        \bar{\rightcauchygreentensor}
      }
    }{
      \transpose{
        \fnof{\materialdeviatorictensorfour^{\pullbacksymbol}}{\rightcauchygreentensor}
      }
    }\\
    &\qquad-\dfrac{2}{3}\fnof{J^{-\frac{4}{3}}}{\rightcauchygreentensor}
    \doubledotprod{
      \pbrac{
        \tensorprod{
          \pioladeformationtensor
        }{
          \pbrac{
            \doubledotprod{
              \rightcauchygreentensor
            }{
              \delby{
                \fnof{\bar{\secondpiolakirchoffstresstensor}}{\fnof{\bar{\rightcauchygreentensor}}{\rightcauchygreentensor}}
              }{
                \bar{\rightcauchygreentensor}
              }
            }+\doubledotprod{
              \fnof{\bar{\secondpiolakirchoffstresstensor}}{\fnof{\bar{\rightcauchygreentensor}
                }{
                  \rightcauchygreentensor}
              }
            }{
              \delby{
                \rightcauchygreentensor
              }{
                \bar{\rightcauchygreentensor}
              }
            }
          }
        }
      }
    }{
      \transpose{
        \fnof{\materialdeviatorictensorfour^{\pullbacksymbol}}{\rightcauchygreentensor}
      }
    }\\
    &=\doubledotprod{
      \fnof{\bar{\materialsecondelasticitytensor}}{\fnof{\bar{\rightcauchygreentensor}}{\rightcauchygreentensor}}
    }{
      \transpose{
        \fnof{\materialdeviatorictensorfour^{\pullbacksymbol}}{\rightcauchygreentensor}
      }
    }
    -\dfrac{2}{3}\fnof{J^{-\frac{4}{3}}}{\rightcauchygreentensor}
    \pullback{\Trop}{
      \fnof{\bar{\secondpiolakirchoffstresstensor}}{\fnof{\bar{\rightcauchygreentensor}}{\rightcauchygreentensor}}
    }
    \doubledotprod{
      \delby{
        \pioladeformationtensor
      }{
        \bar{\rightcauchygreentensor}
      }
    }{
      \transpose{
        \fnof{\materialdeviatorictensorfour^{\pullbacksymbol}}{\rightcauchygreentensor}
      }
    } \\
    &\qquad-\dfrac{1}{3}\doubledotprod{
      \pbrac{
        \tensorprod{
          \pioladeformationtensor
        }{
          \pbrac{
            \doubledotprod{
              \rightcauchygreentensor
            }{
              2\fnof{J^{-\frac{4}{3}}}{\rightcauchygreentensor}\delby{
                \fnof{\bar{\secondpiolakirchoffstresstensor}}{\fnof{\bar{\rightcauchygreentensor}}{\rightcauchygreentensor}}
              }{
                \bar{\rightcauchygreentensor}
              }
            }
            +2\fnof{J^{-\frac{4}{3}}}{\rightcauchygreentensor}
            \doubledotprod{
              \fnof{\bar{\secondpiolakirchoffstresstensor}}{\fnof{\bar{\rightcauchygreentensor}}{\rightcauchygreentensor}}
            }{
              \delby{
                \rightcauchygreentensor
              }{
                \bar{\rightcauchygreentensor}
              }
            }
          }
        }
      }
    }{
      \transpose{
        \fnof{\materialdeviatorictensorfour^{\pullbacksymbol}}{\rightcauchygreentensor}
      }
    }\\
    &=\doubledotprod{
      \fnof{\bar{\materialsecondelasticitytensor}}{\fnof{\bar{\rightcauchygreentensor}}{\rightcauchygreentensor}}
    }{
      \transpose{
        \fnof{\materialdeviatorictensorfour^{\pullbacksymbol}}{\rightcauchygreentensor}
      }
    }
    -\dfrac{2}{3}\fnof{J^{-\frac{4}{3}}}{\rightcauchygreentensor}
    \pullback{\Trop}{
      \fnof{\bar{\secondpiolakirchoffstresstensor}}{\fnof{\bar{\rightcauchygreentensor}}{\rightcauchygreentensor}}
    }
    \doubledotprod{
      \delby{
        \pioladeformationtensor
      }{
        \bar{\rightcauchygreentensor}
      }
    }{
      \transpose{
        \fnof{\materialdeviatorictensorfour^{\pullbacksymbol}}{\rightcauchygreentensor}
      }
    } \\
    &\qquad-\dfrac{1}{3}\doubledotprod{
      \pbrac{
        \tensorprod{
          \pioladeformationtensor
        }{
          \pbrac{
            \doubledotprod{
              \rightcauchygreentensor
            }{
              \fnof{\bar{\materialsecondelasticitytensor}}{\fnof{\bar{\rightcauchygreentensor}}{\rightcauchygreentensor}}             
             }
            +2\fnof{J^{-\frac{4}{3}}}{\rightcauchygreentensor}
            \doubledotprod{
              \fnof{\bar{\secondpiolakirchoffstresstensor}}{\fnof{\bar{\rightcauchygreentensor}}{\rightcauchygreentensor}}
            }{
              \delby{
                \rightcauchygreentensor
              }{
                \bar{\rightcauchygreentensor}
              }
            }
          }
        }
      }
    }{
      \transpose{
        \fnof{\materialdeviatorictensorfour^{\pullbacksymbol}}{\rightcauchygreentensor}
      }
    }
  \end{split}
  \label{eqn:SecondPartRHSMaterialDeviatoricSecondElasticityTensor4}
\end{equation}

Now, we have \todo{WHY???}
\begin{equation}
  \delby{\rightcauchygreentensor}{\bar{\rightcauchygreentensor}}=\fnof{J^{\frac{2}{3}}}{\rightcauchygreentensor}\transpose{\fnof{\pullback{\materialdeviatorictensorfour}{}}{\rightcauchygreentensor}}
  \label{eqn:DelCDelCBar}
\end{equation}
and \todo{WHY???}
\begin{equation}
  \delby{\pioladeformationtensor}{\bar{\rightcauchygreentensor}}=\fnof{J^{\frac{2}{3}}}{\rightcauchygreentensor}\materialsymidentitytensorfour^{\pullbacksymbol\sharptensorsymbol}
  \label{eqn:DelBDelCBar}
\end{equation}

If we now substitute \eqnrefs{eqn:DelCDelCBar}{eqn:DelBDelCBar} into \eqnref{eqn:SecondPartRHSMaterialDeviatoricSecondElasticityTensor4} we obtain
\begin{equation}
  \begin{split}
    2\fnof{J^{-\frac{2}{3}}}{\rightcauchygreentensor}&
    \delby{
      \pbrac{
        \doubledotprod{
          \fnof{\materialdeviatorictensorfour^{\pullbacksymbol}}{\rightcauchygreentensor}
        }{
          \fnof{\bar{\secondpiolakirchoffstresstensor}}{\fnof{\bar{\rightcauchygreentensor}}{\rightcauchygreentensor}}
        }
      }
    }{
      \rightcauchygreentensor
    }\\
    &=\doubledotprod{
      \fnof{\bar{\materialsecondelasticitytensor}}{\fnof{\bar{\rightcauchygreentensor}}{\rightcauchygreentensor}}
    }{
      \transpose{
        \fnof{\materialdeviatorictensorfour^{\pullbacksymbol}}{\rightcauchygreentensor}
      }
    } 
    -\dfrac{2}{3}\fnof{J^{-\frac{2}{3}}}{\rightcauchygreentensor}
    \pullback{\Trop}{
      \fnof{\bar{\secondpiolakirchoffstresstensor}}{\fnof{\bar{\rightcauchygreentensor}}{\rightcauchygreentensor}}
    }
    \doubledotprod{
      \materialsymidentitytensorfour^{\pullbacksymbol\sharptensorsymbol}
    }{
      \transpose{
        \fnof{\materialdeviatorictensorfour^{\pullbacksymbol}}{\rightcauchygreentensor}
      }
    } \\
    &\qquad-\dfrac{1}{3}\doubledotprod{
      \pbrac{
        \tensorprod{
          \pioladeformationtensor
        }{
          \pbrac{
            \doubledotprod{
              \rightcauchygreentensor
            }{
              \fnof{\bar{\materialsecondelasticitytensor}}{\fnof{\bar{\rightcauchygreentensor}}{\rightcauchygreentensor}}             
            }
            +2\fnof{J^{-\frac{2}{3}}}{\rightcauchygreentensor}
            \doubledotprod{
              \fnof{\bar{\secondpiolakirchoffstresstensor}}{\fnof{\bar{\rightcauchygreentensor}}{\rightcauchygreentensor}}
            }{
               \transpose{
                \fnof{\materialdeviatorictensorfour^{\pullbacksymbol}}{\rightcauchygreentensor}
              }
            }
          }
        }
      }
    }{
      \transpose{
        \fnof{\materialdeviatorictensorfour^{\pullbacksymbol}}{\rightcauchygreentensor}
      }
    } \\
    &=\doubledotprod{
      \pbrac{
        \doubledotprod{
          \pullback{\materialsymidentitytensorfour}{}
          }{
          \fnof{\bar{\materialsecondelasticitytensor}}{\fnof{\bar{\rightcauchygreentensor}}{\rightcauchygreentensor}}
          }
        }
    }{
      \transpose{
        \fnof{\materialdeviatorictensorfour^{\pullbacksymbol}}{\rightcauchygreentensor}
      }
    } 
    -\dfrac{2}{3}\fnof{J^{-\frac{2}{3}}}{\rightcauchygreentensor}
    \pullback{\Trop}{
      \fnof{\bar{\secondpiolakirchoffstresstensor}}{\fnof{\bar{\rightcauchygreentensor}}{\rightcauchygreentensor}}
    }
    \doubledotprod{
      \materialsymidentitytensorfour^{\pullbacksymbol\sharptensorsymbol}
    }{
      \transpose{
        \fnof{\materialdeviatorictensorfour^{\pullbacksymbol}}{\rightcauchygreentensor}
      }
    } \\
    &\qquad-\doubledotprod{
      \doubledotprod{
        \pbrac{
          \dfrac{1}{3}
          \tensorprod{
            \pioladeformationtensor
          }{
            \rightcauchygreentensor
          }
        }
      }{
        \fnof{\bar{\materialsecondelasticitytensor}}{\fnof{\bar{\rightcauchygreentensor}}{\rightcauchygreentensor}}             
      }
    }{
      \transpose{
        \fnof{\materialdeviatorictensorfour^{\pullbacksymbol}}{\rightcauchygreentensor}
      }
    }\\
    &\qquad\qquad
    -\dfrac{2}{3}\doubledotprod{
      \pbrac{
        \tensorprod{
          \pioladeformationtensor
        }{
          \pbrac{
            \fnof{J^{-\frac{2}{3}}}{\rightcauchygreentensor}
            \doubledotprod{
              \fnof{\materialdeviatorictensorfour^{\pullbacksymbol}}{\rightcauchygreentensor}
            }{
              \fnof{\bar{\secondpiolakirchoffstresstensor}}{\fnof{\bar{\rightcauchygreentensor}}{\rightcauchygreentensor}}
            }
          }
        }
      }      
    }{
      \transpose{
        \fnof{\materialdeviatorictensorfour^{\pullbacksymbol}}{\rightcauchygreentensor}
      }
    }   
  \end{split}
  \label{eqn:SecondPartRHSMaterialDeviatoricSecondElasticityTensor5}
\end{equation}

Substituting \eqnrefs{eqn:MaterialSphericalTensorFourPB}{eqn:SecondPiolaKirchoffDeviatoricPBDefinition} into \eqnref{eqn:SecondPartRHSMaterialDeviatoricSecondElasticityTensor5} 
gives
\begin{equation}
   \begin{split}
    2\fnof{J^{-\frac{2}{3}}}{\rightcauchygreentensor}&
    \delby{
      \pbrac{
        \doubledotprod{
          \fnof{\materialdeviatorictensorfour^{\pullbacksymbol}}{\rightcauchygreentensor}
        }{
          \fnof{\bar{\secondpiolakirchoffstresstensor}}{\fnof{\bar{\rightcauchygreentensor}}{\rightcauchygreentensor}}
        }
      }
    }{
      \rightcauchygreentensor
    }\\
    &=\doubledotprod{
      \doubledotprod{
        \pullback{\materialsymidentitytensorfour}{}
      }{
        \fnof{\bar{\materialsecondelasticitytensor}}{\fnof{\bar{\rightcauchygreentensor}}{\rightcauchygreentensor}}
      }
    }{
      \transpose{
        \fnof{\materialdeviatorictensorfour^{\pullbacksymbol}}{\rightcauchygreentensor}
      }
    } 
    -\dfrac{2}{3}\fnof{J^{-\frac{2}{3}}}{\rightcauchygreentensor}
    \pullback{\Trop}{
      \fnof{\bar{\secondpiolakirchoffstresstensor}}{\fnof{\bar{\rightcauchygreentensor}}{\rightcauchygreentensor}}
    }
    \doubledotprod{
      \materialsymidentitytensorfour^{\pullbacksymbol\sharptensorsymbol}
    }{
      \transpose{
        \fnof{\materialdeviatorictensorfour^{\pullbacksymbol}}{\rightcauchygreentensor}
      }
    } \\
    &\qquad-\doubledotprod{
      \doubledotprod{
        \fnof{\pullback{\materialsphericaltensorfour}{}}{\rightcauchygreentensor}
      }{
        \fnof{\bar{\materialsecondelasticitytensor}}{\fnof{\bar{\rightcauchygreentensor}}{\rightcauchygreentensor}}             
      }
    }{
      \transpose{
        \fnof{\materialdeviatorictensorfour^{\pullbacksymbol}}{\rightcauchygreentensor}
      }
    }
    -\dfrac{2}{3}\doubledotprod{
      \pbrac{
        \tensorprod{
          \pioladeformationtensor
        }{
          \fnof{\secondpiolakirchoffstresstensor_{\pullback{\Devop}{}}}{\rightcauchygreentensor}
        }
      }      
    }{
      \transpose{
        \fnof{\materialdeviatorictensorfour^{\pullbacksymbol}}{\rightcauchygreentensor}
      }
    } \\
    &=\doubledotprod{
      \doubledotprod{
        \pbrac{
          \fnof{\pullback{\materialsymidentitytensorfour}{}}{\rightcauchygreentensor}-\fnof{\pullback{\materialsphericaltensorfour}}{\rightcauchygreentensor}
        }
      }{
        \fnof{\bar{\materialsecondelasticitytensor}}{\fnof{\bar{\rightcauchygreentensor}}{\rightcauchygreentensor}}
      }
    }{
      \transpose{
        \fnof{\materialdeviatorictensorfour^{\pullbacksymbol}}{\rightcauchygreentensor}
      }
    }\\
    &\qquad
    -\dfrac{2}{3}\fnof{J^{-\frac{2}{3}}}{\rightcauchygreentensor}
    \pullback{\Trop}{
      \fnof{\bar{\secondpiolakirchoffstresstensor}}{\fnof{\bar{\rightcauchygreentensor}}{\rightcauchygreentensor}}
    }
    \doubledotprod{
      \materialsymidentitytensorfour^{\pullbacksymbol\sharptensorsymbol}
    }{
      \transpose{
        \fnof{\materialdeviatorictensorfour^{\pullbacksymbol}}{\rightcauchygreentensor}
      }
    } \\
    &\qquad\qquad
    -\dfrac{2}{3}\doubledotprod{
      \pbrac{
        \tensorprod{
          \pioladeformationtensor
        }{
          \fnof{\secondpiolakirchoffstresstensor_{\pullback{\Devop}{}}}{\rightcauchygreentensor}
        }
      }      
    }{
      \transpose{
        \fnof{\materialdeviatorictensorfour^{\pullbacksymbol}}{\rightcauchygreentensor}
      }
    }  \\
   &=\doubledotprod{
      \doubledotprod{
         \fnof{\pullback{\materialdeviatorictensorfour}}{\rightcauchygreentensor}
       }{
        \fnof{\bar{\materialsecondelasticitytensor}}{\fnof{\bar{\rightcauchygreentensor}}{\rightcauchygreentensor}}
      }
    }{
      \transpose{
        \fnof{\materialdeviatorictensorfour^{\pullbacksymbol}}{\rightcauchygreentensor}
      }
    }
    -\dfrac{2}{3}\fnof{J^{-\frac{2}{3}}}{\rightcauchygreentensor}
    \pullback{\Trop}{
      \fnof{\bar{\secondpiolakirchoffstresstensor}}{\fnof{\bar{\rightcauchygreentensor}}{\rightcauchygreentensor}}
    }
    \doubledotprod{
      \materialsymidentitytensorfour^{\pullbacksymbol\sharptensorsymbol}
    }{
      \transpose{
        \fnof{\materialdeviatorictensorfour^{\pullbacksymbol}}{\rightcauchygreentensor}
      }
    } \\
    &\qquad
    -\dfrac{2}{3}\doubledotprod{
      \pbrac{
        \tensorprod{
          \pioladeformationtensor
        }{
          \fnof{\secondpiolakirchoffstresstensor_{\pullback{\Devop}{}}}{\rightcauchygreentensor}
        }
      }      
    }{
      \transpose{
        \fnof{\materialdeviatorictensorfour^{\pullbacksymbol}}{\rightcauchygreentensor}
      }
    }  
   \end{split}
  \label{eqn:SecondPartRHSMaterialDeviatoricSecondElasticityTensor6}
\end{equation}

Now, combining \eqnrefs{eqn:FirstPartRHSMaterialDeviatoricSecondElasticityTensor2}{eqn:SecondPartRHSMaterialDeviatoricSecondElasticityTensor6} gives \todo{WHAT HAPPENS TO THE FINAL DEVIATORIC TENSOR ON THE LAST TERM OF THE FIRST LINE???}
\begin{equation}
  \begin{split}
    \fnof{\materialsecondelasticitytensor_{\pullback{\Devop}{}}}{\fnof{\bar{\rightcauchygreentensor}}{\rightcauchygreentensor}}&=
    -\dfrac{2}{3}\tensorprod{
      \fnof{\secondpiolakirchoffstresstensor_{\pullback{\Devop}{}}}{\rightcauchygreentensor}
    }{
      \pioladeformationtensor
    }
    +\doubledotprod{
      \doubledotprod{
        \fnof{\pullback{\materialdeviatorictensorfour}{}}{\rightcauchygreentensor}
      }{
        \fnof{\bar{\materialsecondelasticitytensor}}{\fnof{\bar{\rightcauchygreentensor}}{\rightcauchygreentensor}}
      }
    }{
      \transpose{
        \fnof{\pullback{\materialdeviatorictensorfour}{}}{\rightcauchygreentensor}
      }
    }\\
    &\qquad
    -\dfrac{2}{3}\fnof{J^{-\frac{2}{3}}}{\rightcauchygreentensor}
    \pullback{\Trop}{
      \fnof{\bar{\secondpiolakirchoffstresstensor}}{\fnof{\bar{\rightcauchygreentensor}}{\rightcauchygreentensor}}
    }
    \doubledotprod{
      \materialsymidentitytensorfour^{\pullbacksymbol\sharptensorsymbol}
    }{
      \transpose{
        \fnof{\materialdeviatorictensorfour^{\pullbacksymbol}}{\rightcauchygreentensor}
      }
    } 
    -\dfrac{2}{3}\doubledotprod{
      \pbrac{
        \tensorprod{
          \pioladeformationtensor
        }{
          \fnof{\secondpiolakirchoffstresstensor_{\pullback{\Devop}{}}}{\rightcauchygreentensor}
        }
      }      
    }{
      \transpose{
        \fnof{\materialdeviatorictensorfour^{\pullbacksymbol}}{\rightcauchygreentensor}
      }
    } \\
    &=\doubledotprod{
      \doubledotprod{
        \fnof{\pullback{\materialdeviatorictensorfour}{}}{\rightcauchygreentensor}
      }{
        \fnof{\bar{\materialsecondelasticitytensor}}{\fnof{\bar{\rightcauchygreentensor}}{\rightcauchygreentensor}}
      }
    }{
      \transpose{
        \fnof{\pullback{\materialdeviatorictensorfour}{}}{\rightcauchygreentensor}
      }
    }
    -\dfrac{2}{3}\fnof{J^{-\frac{2}{3}}}{\rightcauchygreentensor}
    \pullback{\Trop}{
      \fnof{\bar{\secondpiolakirchoffstresstensor}}{\fnof{\bar{\rightcauchygreentensor}}{\rightcauchygreentensor}}
    }
    \doubledotprod{
      \materialsymidentitytensorfour^{\pullbacksymbol\sharptensorsymbol}
    }{
      \transpose{
        \fnof{\materialdeviatorictensorfour^{\pullbacksymbol}}{\rightcauchygreentensor}
      }
    }\\
    &\qquad-\dfrac{2}{3}\pbrac{
      \tensorprod{
        \fnof{\secondpiolakirchoffstresstensor_{\pullback{\Devop}{}}}{\rightcauchygreentensor}
      }{
        \pioladeformationtensor
      }
      +\tensorprod{
        \pioladeformationtensor
      }{
        \fnof{\secondpiolakirchoffstresstensor_{\pullback{\Devop}{}}}{\rightcauchygreentensor}
      } 
    }
  \end{split}
  \label{eqn:MaterialSDeviatoricSecondElasticityTensorDefinition}
\end{equation}

If we now consider the spherical component of the second elasticity tensor \ie
\begin{equation}
  \begin{split}
    \fnof{\materialsecondelasticitytensor_{\pullback{\Sphop}{}}}{\rightcauchygreentensor}&=2\delby{\fnof{\secondpiolakirchoffstresstensor_{\pullback{\Sphop}{}}}{\fnof{J}{\rightcauchygreentensor}}}{\rightcauchygreentensor}\\
    &=2\delby{}{\rightcauchygreentensor}\pbrac{2\delby{\fnof{\bar{W}_{\Sphop}}{\fnof{J}{\rightcauchygreentensor}}}{\rightcauchygreentensor}} \\
    &=2\delby{}{\rightcauchygreentensor}\pbrac{2\delby{\fnof{\bar{W}_{\Sphop}}{\fnof{J}{\rightcauchygreentensor}}}{J}\delby{\fnof{J}{\rightcauchygreentensor}}{\rightcauchygreentensor}}\\
    &=4\deltwoby{\fnof{\bar{W}_{\Sphop}}{\fnof{J}{\rightcauchygreentensor}}}{J}{\rightcauchygreentensor}\delby{\fnof{J}{\rightcauchygreentensor}}{\rightcauchygreentensor}+4\delby{\fnof{\bar{W}_{\Sphop}}{\fnof{J}{\rightcauchygreentensor}}}{J}\deltwosqby{\fnof{J}{\rightcauchygreentensor}}{\rightcauchygreentensor}\\
    &=4\deltwosqby{\fnof{\bar{W}_{\Sphop}}{\fnof{J}{\rightcauchygreentensor}}}{J}\tensorprod{\delby{\fnof{J}{\rightcauchygreentensor}}{\rightcauchygreentensor}}{\delby{\fnof{J}{\rightcauchygreentensor}}{\rightcauchygreentensor}}+4\delby{\fnof{\bar{W}_{\Sphop}}{\fnof{J}{\rightcauchygreentensor}}}{J}\deltwosqby{\fnof{J}{\rightcauchygreentensor}}{\rightcauchygreentensor}
  \end{split}
  \label{eqn:MaterialSphericalSecondElasticityTensor1}
\end{equation}

Now, from \eqnrefs{eqn:DelJacobianDelC}{eqn:MaterialSphericalTensorFourPBSharp}, we have
\begin{equation}
  \begin{split}
    4\tensorprod{\delby{\fnof{J}{\rightcauchygreentensor}}{\rightcauchygreentensor}}{\delby{\fnof{J}{\rightcauchygreentensor}}{\rightcauchygreentensor}}
    &=4\tensorprod{\pbrac{\dfrac{-\fnof{J}{\rightcauchygreentensor}\pioladeformationtensor}{2}}}{\pbrac{\dfrac{-\fnof{J}{\rightcauchygreentensor}\pioladeformationtensor}{2}}}\\
    &=\fnof{J^{2}}{\rightcauchygreentensor}\tensorprod{\pioladeformationtensor}{\pioladeformationtensor}\\
    &=3\fnof{J^{2}}{\rightcauchygreentensor}\materialsphericaltensorfour^{\pullbacksymbol\sharptensorsymbol}
  \end{split}
  \label{eqn:DelJacobianDelCCrossDelJacobianDelC}
\end{equation}

And, from \eqnrefsthree{eqn:DelJacobianDelC}{eqn:MaterialSphericalTensorFourPBSharp}{eqn:MaterialSymIdentityTensorFourPBSharpRelationship}, we also have
\begin{equation}
  \begin{split}
    4\deltwosqby{\fnof{J}{\rightcauchygreentensor}}{\rightcauchygreentensor}
    &=4\delby{}{\rightcauchygreentensor}\pbrac{\delby{\fnof{J}{\rightcauchygreentensor}}{\rightcauchygreentensor}}\\
    &=4\delby{}{\rightcauchygreentensor}\pbrac{\dfrac{-\fnof{J}{\rightcauchygreentensor}\pioladeformationtensor}{2}}\\
    &=-2\tensorprod{\delby{\fnof{J}{\rightcauchygreentensor}}{\rightcauchygreentensor}}{\pioladeformationtensor}-2\fnof{J}{\rightcauchygreentensor}\delby{\pioladeformationtensor}{\rightcauchygreentensor}\\
    &=-2\dfrac{-\fnof{J}{\rightcauchygreentensor}}{2}\tensorprod{\pioladeformationtensor}{\pioladeformationtensor}-2\dfrac{\fnof{J}{\rightcauchygreentensor}}{2}\delby{\pioladeformationtensor}{\rightcauchygreentensor}\\
    &=3\fnof{J}{\rightcauchygreentensor}\materialsphericaltensorfour^{\pullbacksymbol\sharptensorsymbol}+2\fnof{J}{\rightcauchygreentensor}\materialsymidentitytensorfour^{\pullbacksymbol\sharptensorsymbol}
  \end{split}
  \label{eqn:DelSqJacobianDelCSq}
\end{equation}

Now we define
\begin{equation}
  \bulkstrainmodulussymbol=\deltwosqby{\fnof{\bar{W}_{\Sphop}}{\fnof{J}{\rightcauchygreentensor}}}{J}
  \label{eqn:LargeStrainBulkModulusDefinition}
\end{equation}
where $\bulkstrainmodulussymbol$ is the \emph{large strain bulk
modulus}.\symbolat{$\bulkstrainmodulussymbol$}{Large strain bulk
  modulus}

Substituting \eqnrefsfour{eqn:DelJacobianDelCCrossDelJacobianDelC}{eqn:DelSqJacobianDelCSq}{eqn:LargeStrainBulkModulusDefinition}{eqn:HydrostaticPressureDefinition} into \eqnref{eqn:MaterialSphericalSecondElasticityTensor1} gives
\begin{equation}
  \begin{split}
    \fnof{\materialsecondelasticitytensor_{\pullback{\Sphop}{}}}{\rightcauchygreentensor}
    &=4\deltwosqby{\fnof{\bar{W}_{\Sphop}}{\fnof{J}{\rightcauchygreentensor}}}{J}\tensorprod{\delby{\fnof{J}{\rightcauchygreentensor}}{\rightcauchygreentensor}}{\delby{\fnof{J}{\rightcauchygreentensor}}{\rightcauchygreentensor}}+4\delby{\fnof{\bar{W}_{\Sphop}}{\fnof{J}{\rightcauchygreentensor}}}{J}\deltwosqby{\fnof{J}{\rightcauchygreentensor}}{\rightcauchygreentensor}\\
    &=3\fnof{J^{2}}{\rightcauchygreentensor}\bulkstrainmodulussymbol\materialsphericaltensorfour^{\pullbacksymbol\sharptensorsymbol}-p\pbrac{3\fnof{J}{\rightcauchygreentensor}\materialsphericaltensorfour^{\pullbacksymbol\sharptensorsymbol}+2\fnof{J}{\rightcauchygreentensor}\materialsymidentitytensorfour^{\pullbacksymbol\sharptensorsymbol}}\\
    &=3\fnof{J}{\rightcauchygreentensor}\pbrac{\fnof{J}{\rightcauchygreentensor}\bulkstrainmodulussymbol-p}\materialsphericaltensorfour^{\pullbacksymbol\sharptensorsymbol}+2\fnof{J}{\rightcauchygreentensor}p\materialsymidentitytensorfour^{\pullbacksymbol\sharptensorsymbol}
  \end{split}
  \label{eqn:MaterialSphericalSecondElasticityTensorDefiniton}
\end{equation}

Finally, combining \eqnrefs{eqn:MaterialSDeviatoricSecondElasticityTensorDefinition}{eqn:MaterialSphericalSecondElasticityTensorDefiniton} gives
\begin{equation}
  \begin{split}
    \fnof{\materialsecondelasticitytensor}{\rightcauchygreentensor}&=
    \fnof{\materialsecondelasticitytensor_{\pullback{\Devop}{}}}{\rightcauchygreentensor}
    +\fnof{\materialsecondelasticitytensor_{\pullback{\Sphop}{}}}{\rightcauchygreentensor}\\
    &=\doubledotprod{
      \doubledotprod{
        \fnof{\pullback{\materialdeviatorictensorfour}{}}{\rightcauchygreentensor}
      }{
        \fnof{\bar{\materialsecondelasticitytensor}}{\fnof{\bar{\rightcauchygreentensor}}{\rightcauchygreentensor}}
      }
    }{
      \transpose{
        \fnof{\pullback{\materialdeviatorictensorfour}{}}{\rightcauchygreentensor}
      }
    }
    -\dfrac{2}{3}\fnof{J^{-\frac{2}{3}}}{\rightcauchygreentensor}
    \pullback{\Trop}{
      \fnof{\bar{\secondpiolakirchoffstresstensor}}{\fnof{\bar{\rightcauchygreentensor}}{\rightcauchygreentensor}}
    }
    \doubledotprod{
      \materialsymidentitytensorfour^{\pullbacksymbol\sharptensorsymbol}
    }{
      \transpose{
        \fnof{\materialdeviatorictensorfour^{\pullbacksymbol}}{\rightcauchygreentensor}
      }
    }\\
    &\qquad-\dfrac{2}{3}\pbrac{
      \tensorprod{
        \fnof{\secondpiolakirchoffstresstensor_{\pullback{\Devop}{}}}{\rightcauchygreentensor}
      }{
        \pioladeformationtensor
      }
      +\tensorprod{
        \pioladeformationtensor
      }{
        \fnof{\secondpiolakirchoffstresstensor_{\pullback{\Devop}{}}}{\rightcauchygreentensor}
      } 
    }
    +3\fnof{J}{\rightcauchygreentensor}\pbrac{\fnof{J}{\rightcauchygreentensor}\bulkstrainmodulussymbol-p}\materialsphericaltensorfour^{\pullbacksymbol\sharptensorsymbol}+2\fnof{J}{\rightcauchygreentensor}p\materialsymidentitytensorfour^{\pullbacksymbol\sharptensorsymbol}
  \end{split}
  \label{eqn:MaterialSecondElasticityTensorDefiniton}
\end{equation}

Now using the inverse Piola transforms which push forward we have
$\bar{\spatialsecondelasticitytensor}=\inverse{J}\pushforward{\chi}{\bar{\materialsecondelasticitytensor}}$,
$\bar{\tensortwo{\sigma}}=\inverse{J}\pushforward{\chi}{\bar{\secondpiolakirchoffstresstensor}}$,
$\tensortwo{\sigma}_{\devop}=\inverse{J}\pushforward{\chi}{\secondpiolakirchoffstresstensor_{\pullback{\Devop}{}}}$,
$\tensor{y}=\inverse{J}\pushforward{\chi}{\tensortwo{Y}}$ and
$\trop\bar{\tensortwo{\sigma}}=\pullback{\Trop}{}\bar{\secondpiolakirchoffstresstensor}$ where
$\bar{\spatialsecondelasticitytensor}$ is the \emph{spatial second psuedo elasticity
  tensor}\symbolat{$\bar{\spatialsecondelasticitytensor}$}{spatial second psuedo elasticity
  tensor}, we can derive the second spatial elasticity tensor \ie
\begin{equation}
  \begin{split}
    \fnof{\spatialsecondelasticitytensor}{\spatialmetrictensor}&=
    \fnof{\spatialsecondelasticitytensor_{\devop}}{\spatialmetrictensor}
    +\fnof{\spatialsecondelasticitytensor_{\sphop}}{\spatialmetrictensor}\\
    &=\doubledotprod{
      \doubledotprod{
        \fnof{\spatialdeviatorictensorfour}{\spatialmetrictensor}
      }{
        \fnof{\bar{\spatialsecondelasticitytensor}}{\fnof{\bar{\leftcauchygreentensor}}{\spatialmetrictensor}}
      }
    }{
      \transpose{
        \fnof{\spatialdeviatorictensorfour}{\spatialmetrictensor}
      }
    }
    -\dfrac{2}{3}\fnof{J^{-\frac{2}{3}}}{\rightcauchygreentensor}
    \trace{}{
      \fnof{\bar{\cauchystresstensor}}{\fnof{\bar{\leftcauchygreentensor}}{\spatialmetrictensor}}
    }
    \doubledotprod{
      \sharptensor{\spatialsymidentitytensorfour}
    }{
      \transpose{
        \fnof{\spatialdeviatorictensorfour}{\spatialmetrictensor}
      }
    }\\
    &\qquad-\dfrac{2}{3}\pbrac{
      \tensorprod{
        \fnof{\cauchystresstensor_{\devop}}{\spatialmetrictensor}
      }{
        \sharptensor{\spatialmetrictensor}
      }
      +\tensorprod{
        \sharptensor{\spatialmetrictensor}
      }{
        \fnof{\cauchystresstensor_{\devop}}{\spatialmetrictensor}
      } 
    }
    +3\fnof{J}{\rightcauchygreentensor}\pbrac{
      \fnof{J}{\rightcauchygreentensor}\bulkstrainmodulussymbol-p
    }\fnof{\sharptensor{\spatialsphericaltensorfour}}{\spatialmetrictensor}
    +2\fnof{J}{\rightcauchygreentensor}p\sharptensor{\spatialsymidentitytensorfour}
  \end{split}
  \label{eqn:MaterialSecondElasticityTensorDefiniton}
\end{equation}


Now from \eqnref{eqn:DerivInverseSymmetricTensorSecondOrder} we have
\begin{equation}
  \delby{\inverse{\rightcauchygreentensor}}{\rightcauchygreentensor}=\symtensorprod{-\inverse{\rightcauchygreentensor}}{\inverse{\rightcauchygreentensor}}
\end{equation}
and so we have
\begin{equation}
  \begin{split}
    \fnof{\materialsecondelasticitytensor_{\pullback{\Sphop}{}}}{\rightcauchygreentensor}
    &=\pbrac{p\fnof{J}{\rightcauchygreentensor}-\fnof{J^{2}}\bulkstrainmodulussymbol{\rightcauchygreentensor}}\tensorprod{\inverse{\rightcauchygreentensor}}{\inverse{\rightcauchygreentensor}}-2p\fnof{J}{\rightcauchygreentensor}\delby{\inverse{\rightcauchygreentensor}}{\rightcauchygreentensor}\\
    &=\pbrac{p\fnof{J}{\rightcauchygreentensor}-\fnof{J^{2}\bulkstrainmodulussymbol}{\rightcauchygreentensor}}\tensorprod{\inverse{\rightcauchygreentensor}}{\inverse{\rightcauchygreentensor}}+2p\fnof{J}{\rightcauchygreentensor}\symtensorprod{\inverse{\rightcauchygreentensor}}{\inverse{\rightcauchygreentensor}}
  \end{split}
\end{equation}

\begin{equation}
  \begin{split}
    \fnof{\materialsecondelasticitytensor}{\rightcauchygreentensor}&=2\delby{\fnof{\secondpiolakirchoffstresstensor}{\rightcauchygreentensor}}{\rightcauchygreentensor}\\
    &=2\delby{}{\rightcauchygreentensor}\pbrac{2\doubledotprod{\delby{\fnof{\bar{W}}{\fnof{\bar{\rightcauchygreentensor}}{\rightcauchygreentensor},\fnof{J}{\rightcauchygreentensor}}}{\bar{\rightcauchygreentensor}}}{\delby{\fnof{\bar{\rightcauchygreentensor}}{\rightcauchygreentensor}}{\rightcauchygreentensor}}+2\doubledotprod{\delby{\fnof{\bar{W}}{\fnof{\bar{\rightcauchygreentensor}}{\rightcauchygreentensor},\fnof{J}{\rightcauchygreentensor}}}{J}}{\delby{\fnof{J}{\rightcauchygreentensor}}{\rightcauchygreentensor}}}\\
    &=2\doubledotprod{\delby{}{\bar{\rightcauchygreentensor}}\pbrac{2\doubledotprod{\delby{\fnof{\bar{W}}{\fnof{\bar{\rightcauchygreentensor}}{\rightcauchygreentensor},\fnof{J}{\rightcauchygreentensor}}}{\bar{\rightcauchygreentensor}}}{\delby{\fnof{\bar{\rightcauchygreentensor}}{\rightcauchygreentensor}}{\rightcauchygreentensor}}+2\doubledotprod{\delby{\fnof{\bar{W}}{\fnof{\bar{\rightcauchygreentensor}}{\rightcauchygreentensor},\fnof{J}{\rightcauchygreentensor}}}{J}}{\delby{\fnof{J}{\rightcauchygreentensor}}{\rightcauchygreentensor}}}}{\delby{\fnof{\bar{\rightcauchygreentensor}}{\rightcauchygreentensor}}{\rightcauchygreentensor}}\\
    &+2\doubledotprod{\delby{}{J}\pbrac{2\doubledotprod{\delby{\fnof{\bar{W}}{\fnof{\bar{\rightcauchygreentensor}}{\rightcauchygreentensor},\fnof{J}{\rightcauchygreentensor}}}{\bar{\rightcauchygreentensor}}}{\delby{\fnof{\bar{\rightcauchygreentensor}}{\rightcauchygreentensor}}{\rightcauchygreentensor}}+2\doubledotprod{\delby{\fnof{\bar{W}}{\fnof{\bar{\rightcauchygreentensor}}{\rightcauchygreentensor},\fnof{J}{\rightcauchygreentensor}}}{J}}{\delby{\fnof{J}{\rightcauchygreentensor}}{\rightcauchygreentensor}}}}{\delby{\fnof{J}{\rightcauchygreentensor}}{\rightcauchygreentensor}}\\
    &=4\doubledotprod{\pbrac{\doubledotprod{\deltwosqby{\fnof{\bar{W}}{\fnof{\bar{\rightcauchygreentensor}}{\rightcauchygreentensor},\fnof{J}{\rightcauchygreentensor}}}{\bar{\rightcauchygreentensor}}}{\delby{\fnof{\bar{\rightcauchygreentensor}}{\rightcauchygreentensor}}{\rightcauchygreentensor}}
        +\doubledotprod{\delby{\fnof{\bar{W}}{\fnof{\bar{\rightcauchygreentensor}}{\rightcauchygreentensor},\fnof{J}{\rightcauchygreentensor}}}{\bar{\rightcauchygreentensor}}}{\deltwosqby{\fnof{\bar{\rightcauchygreentensor}}{\rightcauchygreentensor}}{\rightcauchygreentensor}}
     }}{\delby{\fnof{\bar{\rightcauchygreentensor}}{\rightcauchygreentensor}}{\rightcauchygreentensor}}\\
    &+4\doubledotprod{
      \pbrac{
        \doubledotprod{
          \deltwoby{
            \fnof{\bar{W}}{\fnof{\bar{\rightcauchygreentensor}}{\rightcauchygreentensor},\fnof{J}{\rightcauchygreentensor}}
          }{
            \bar{\rightcauchygreentensor}
          }{
            J
          }
        }{
          \delby{
            \fnof{\bar{\rightcauchygreentensor}}{\rightcauchygreentensor}
          }{
            \rightcauchygreentensor
          }
        }
        +\doubledotprod{
          \delby{
            \fnof{\bar{W}}{\fnof{\bar{\rightcauchygreentensor}}{\rightcauchygreentensor},\fnof{J}{\rightcauchygreentensor}}
          }{
            \bar{\rightcauchygreentensor}
          }
        }{
          \deltwosqby{\fnof{\bar{\rightcauchygreentensor}}{\rightcauchygreentensor}}{\rightcauchygreentensor}
        }
      }
            {
              \delby{\fnof{\bar{\rightcauchygreentensor}}{\rightcauchygreentensor}}{\rightcauchygreentensor}
            }
    }
  \end{split}
\end{equation}

Now considering $\fnof{\materialsecondelasticitytensor_{\Sphop}}{\rightcauchygreentensor}$ we have
\begin{equation}
  \begin{split}
    \fnof{\materialsecondelasticitytensor_{\Sphop}}{\rightcauchygreentensor}&=2\delby{}{\rightcauchygreentensor}\pbrac{2\delby{\fnof{\bar{W}_{\Sphop}}{\fnof{J}{\rightcauchygreentensor}}}{\rightcauchygreentensor}}\\
    &=2\delby{}{\rightcauchygreentensor}\pbrac{2\delby{\fnof{\bar{W}_{\Sphop}}{\fnof{J}{\rightcauchygreentensor}}}{J}\delby{\fnof{J}{\rightcauchygreentensor}}{\rightcauchygreentensor}}\\
    &=4\deltwoby{\fnof{\bar{W}_{\Sphop}}{\fnof{J}{\rightcauchygreentensor}}}{\rightcauchygreentensor}{J}\delby{\fnof{J}{\rightcauchygreentensor}}{\rightcauchygreentensor}+4\delby{\fnof{\bar{W}_{\Sphop}}{\fnof{J}{\rightcauchygreentensor}}}{J}\deltwosqby{\fnof{J}{\rightcauchygreentensor}}{\rightcauchygreentensor}
  \end{split}
\end{equation}

Now if we decompose the strain energy function into volumetric/spherical and
isochoric/deviatoric components we have
\begin{multline}
  \fnof{\materialsecondelasticitytensor}{\rightcauchygreentensor}=2\doubledotprodthree{\transpose{\pbrac{\delby{\bar{\rightcauchygreentensor}}{\rightcauchygreentensor}}}}{\deltwosqby{\fnof{\bar{W}}{\bar{\rightcauchygreentensor},J}}{\bar{\rightcauchygreentensor}}}{\delby{\bar{\rightcauchygreentensor}}{\rightcauchygreentensor}}+2\doubledotprod{\delby{\fnof{\bar{W}}{\bar{\rightcauchygreentensor},J}}{\bar{\rightcauchygreentensor}}}{\deltwosqby{\bar{\rightcauchygreentensor}}{\rightcauchygreentensor}}\\
  +2\deltwosqby{\fnof{\bar{W}}{\bar{\rightcauchygreentensor},J}}{J}\tensorprod{\delby{J}{\rightcauchygreentensor}}{\delby{J}{\rightcauchygreentensor}}+2\delby{\fnof{\bar{W}}{\bar{\rightcauchygreentensor},J}}{J}\deltwosqby{J}{\rightcauchygreentensor}\\
  +2\tensorprod{\delby{J}{\rightcauchygreentensor}}{\pbrac{\doubledotprod{\deltwoby{\fnof{\bar{W}}{\bar{\rightcauchygreentensor},J}}{J}{\bar{\rightcauchygreentensor}}}{\delby{\bar{\rightcauchygreentensor}}{\rightcauchygreentensor}}}}+2\tensorprod{\pbrac{\doubledotprod{\deltwoby{\fnof{\bar{W}}{\bar{\rightcauchygreentensor},J}}{J}{\bar{\rightcauchygreentensor}}}{\delby{\bar{\rightcauchygreentensor}}{\rightcauchygreentensor}}}}{\delby{J}{\rightcauchygreentensor}}
\end{multline}
or
\begin{multline}
  \fnof{\materialsecondelasticitytensor}{\greenlagrangestraintensor}=\doubledotprodthree{\transpose{\pbrac{\delby{\bar{\greenlagrangestraintensor}}{\greenlagrangestraintensor}}}}{\deltwosqby{\fnof{\bar{W}}{\bar{\greenlagrangestraintensor},J}}{\bar{\greenlagrangestraintensor}}}{\delby{\bar{\greenlagrangestraintensor}}{\greenlagrangestraintensor}}+\doubledotprod{\delby{\fnof{\bar{W}}{\bar{\greenlagrangestraintensor},J}}{\bar{\greenlagrangestraintensor}}}{\deltwosqby{\bar{\greenlagrangestraintensor}}{\greenlagrangestraintensor}}\\
  +\deltwosqby{\fnof{\bar{W}}{\bar{\greenlagrangestraintensor},J}}{J}\tensorprod{\delby{J}{\greenlagrangestraintensor}}{\delby{J}{\greenlagrangestraintensor}}+\delby{\fnof{\bar{W}}{\bar{\greenlagrangestraintensor},J}}{J}\deltwosqby{J}{\greenlagrangestraintensor}\\
  +\tensorprod{\delby{J}{\greenlagrangestraintensor}}{\pbrac{\doubledotprod{\deltwoby{\fnof{\bar{W}}{\bar{\greenlagrangestraintensor},J}}{J}{\bar{\greenlagrangestraintensor}}}{\delby{\bar{\greenlagrangestraintensor}}{\greenlagrangestraintensor}}}}+\tensorprod{\pbrac{\doubledotprod{\deltwoby{\fnof{\bar{W}}{\bar{\greenlagrangestraintensor},J}}{J}{\bar{\greenlagrangestraintensor}}}{\delby{\bar{\greenlagrangestraintensor}}{\greenlagrangestraintensor}}}}{\delby{J}{\greenlagrangestraintensor}}
\end{multline}

If we now define
\begin{equation}
  \bar{\materialsecondelasticitytensor}=2\deltwosqby{\fnof{\bar{W}}{\bar{\rightcauchygreentensor},J}}{\bar{\rightcauchygreentensor}}=\deltwosqby{\fnof{\bar{W}}{\bar{\greenlagrangestraintensor},J}}{\bar{\greenlagrangestraintensor}}
\end{equation}
as the \emph{material psuedo second elasticity
  tensor}\symbolat{$\bar{\materialsecondelasticitytensor}$}{Material pseudo second elasticity tensor} and
\begin{equation}
  \tensor{Y}=2\deltwoby{\fnof{\bar{W}}{\bar{\rightcauchygreentensor},J}}{J}{\bar{\rightcauchygreentensor}}=\deltwoby{\fnof{\bar{W}}{\bar{\greenlagrangestraintensor},J}}{J}{\bar{\greenlagrangestraintensor}}
\end{equation}
as the \emph{material coupling tensor}\symbolat{$\tensortwo{Y}$}{Material coupling tensor} (\ie representing any coupling between
deviatoric and spherical parts of the stress) and
\begin{equation}
  K=\deltwosqby{\fnof{\bar{W}}{\bar{\rightcauchygreentensor},J}}{J}=\deltwosqby{\fnof{\bar{W}}{\bar{\greenlagrangestraintensor},J}}{J}
\end{equation}
as the \emph{large strain bulk modulus}\symbolat{$K$}{Large strain bulk modulus}, then we obtain
\begin{multline}
  \fnof{\materialsecondelasticitytensor}{\rightcauchygreentensor}=\doubledotprodthree{\transpose{\pbrac{\delby{\bar{\rightcauchygreentensor}}{\rightcauchygreentensor}}}}{\bar{\materialsecondelasticitytensor}}{\delby{\bar{\rightcauchygreentensor}}{\rightcauchygreentensor}}+\doubledotprod{\bar{\secondpiolakirchoffstresstensor}}{\deltwosqby{\bar{\rightcauchygreentensor}}{\rightcauchygreentensor}}\\
  +2K\tensorprod{\delby{J}{\rightcauchygreentensor}}{\delby{J}{\rightcauchygreentensor}}-2p\deltwosqby{J}{\rightcauchygreentensor}\\
  +\tensorprod{\delby{J}{\rightcauchygreentensor}}{\pbrac{\doubledotprod{\tensor{Y}}{\delby{\bar{\rightcauchygreentensor}}{\rightcauchygreentensor}}}}+\tensorprod{\pbrac{\doubledotprod{\tensor{Y}}{\delby{\bar{\rightcauchygreentensor}}{\rightcauchygreentensor}}}}{\delby{J}{\rightcauchygreentensor}}
\end{multline}
or
\begin{multline}
  \fnof{\materialsecondelasticitytensor}{\greenlagrangestraintensor}=\doubledotprodthree{\transpose{\pbrac{\delby{\bar{\greenlagrangestraintensor}}{\greenlagrangestraintensor}}}}{\bar{\materialsecondelasticitytensor}}{\delby{\bar{\greenlagrangestraintensor}}{\greenlagrangestraintensor}}+\doubledotprod{\bar{\secondpiolakirchoffstresstensor}}{\deltwosqby{\bar{\greenlagrangestraintensor}}{\greenlagrangestraintensor}}\\
  +K\tensorprod{\delby{J}{\greenlagrangestraintensor}}{\delby{J}{\greenlagrangestraintensor}}-p\deltwosqby{J}{\greenlagrangestraintensor}\\
  +\tensorprod{\delby{J}{\greenlagrangestraintensor}}{\pbrac{\doubledotprod{\tensor{Y}}{\delby{\bar{\greenlagrangestraintensor}}{\greenlagrangestraintensor}}}}+\tensorprod{\pbrac{\doubledotprod{\tensor{Y}}{\delby{\bar{\greenlagrangestraintensor}}{\greenlagrangestraintensor}}}}{\delby{J}{\greenlagrangestraintensor}}
\end{multline}

Using
\begin{align}
  2\delby{J}{\rightcauchygreentensor}&=\delby{J}{\greenlagrangestraintensor}=J\pioladeformationtensor \\
  4\tensorprod{\delby{J}{\rightcauchygreentensor}}{\delby{J}{\rightcauchygreentensor}}&=\tensorprod{\delby{J}{\greenlagrangestraintensor}}{\delby{J}{\greenlagrangestraintensor}}=\tensorprod{\pbrac{J\pioladeformationtensor}}{\pbrac{J\pioladeformationtensor}}=3J^{2}\materialsphericaltensorfour^{\sharp
    *} \\
  \delby{\bar{\rightcauchygreentensor}}{\rightcauchygreentensor}&=\delby{\bar{\greenlagrangestraintensor}}{\greenlagrangestraintensor}=J^{-\frac{2}{3}}\pbrac{\transpose{\materialsymidentitytensorfour}-\dfrac{\tensorprod{\rightcauchygreentensor}{\pioladeformationtensor}}{3}}=J^{-\frac{2}{3}}\materialdeviatorictensorfour^{*T} \\
  4\deltwosqby{J}{\rightcauchygreentensor}&=\deltwosqby{J}{\greenlagrangestraintensor}=\tensorprod{\pioladeformationtensor}{\pbrac{J\pioladeformationtensor}}+2J\delby{\pioladeformationtensor}{\rightcauchygreentensor}=3J\materialdeviatorictensorfour^{\sharp
    *}-2J\materialsymidentitytensorfour^{\sharp
    *} \\
  \doubledotprod{\tensor{Q}}{\deltwosqby{\bar{\rightcauchygreentensor}}{\rightcauchygreentensor}}&=\doubledotprod{\tensor{Q}}{\deltwosqby{\bar{\greenlagrangestraintensor}}{\greenlagrangestraintensor}}=\dfrac{2J^{-\frac{2}{3}}\pullback{\Trop}{\tensor{Q}}}{3}\materialdeviatorictensorfour^{\sharp
  *}-\dfrac{2J^{-\frac{2}{3}}}{3}\sqbrac{\tensorprod{\pioladeformationtensor}{\pbrac{\doubledotprod{\pullback{\materialdeviatorictensorfour}{}}{\tensor{Q}}}}+\tensorprod{\pbrac{\doubledotprod{\pullback{\materialdeviatorictensorfour}{}}{\tensor{Q}}}}{\pioladeformationtensor}}
\end{align}
we can derive the second material elasticity tensor
\begin{multline}
  \materialsecondelasticitytensor=3J\pbrac{JK-p}\materialsphericaltensorfour^{\sharp
    *}+2Jp\materialsymidentitytensorfour^{\sharp
    *}\\
  +J^{\frac{1}{3}}\sqbrac{\tensorprod{\pioladeformationtensor}{\pbrac{\doubledotprod{\pullback{\materialdeviatorictensorfour}}{\tensortwo{Y}}{}}}+\tensorprod{\pbrac{\doubledotprod{\pullback{\materialdeviatorictensorfour}{}}{\tensortwo{Y}}}}{\pioladeformationtensor}}
  \\
  +J^{-\frac{4}{3}}\doubledotprodthree{\pullback{\materialdeviatorictensorfour}{}}{\bar{\materialsecondelasticitytensor}}{\materialdeviatorictensorfour^{*T}}+\dfrac{2}{3}J^{-\frac{2}{3}}\pullback{\Trop}{}\bar{\secondpiolakirchoffstresstensor}\materialdeviatorictensorfour^{\sharp
    *}\\
  -\dfrac{2}{3}\sqbrac{\tensorprod{\pioladeformationtensor}{\secondpiolakirchoffstresstensor_{\pullback{\Devop}{}}}+\tensorprod{\secondpiolakirchoffstresstensor_{\pullback{\Devop}{}}}{\pioladeformationtensor}}
\end{multline}

Now using the inverse Piola transforms which push forward we have
$\bar{\spatialsecondelasticitytensor}=\inverse{J}\pushforward{\chi}{\bar{\materialsecondelasticitytensor}}$,
$\bar{\tensortwo{\sigma}}=\inverse{J}\pushforward{\chi}{\bar{\secondpiolakirchoffstresstensor}}$,
$\tensortwo{\sigma}_{\devop}=\inverse{J}\pushforward{\chi}{\secondpiolakirchoffstresstensor_{\pullback{\Devop}{}}}$,
$\tensor{y}=\inverse{J}\pushforward{\chi}{\tensortwo{Y}}$ and
$\trop\bar{\tensortwo{\sigma}}=\pullback{\Trop}{}\bar{\secondpiolakirchoffstresstensor}$ where
$\bar{\spatialsecondelasticitytensor}$ is the \emph{spatial second psuedo elasticity
  tensor}\symbolat{$\bar{\spatialsecondelasticitytensor}$}{spatial second psuedo elasticity
  tensor}, we can derive the second spatial elasticity tensor \ie
\begin{multline}
  \spatialsecondelasticitytensor=3\pbrac{JK-p}\spatialsphericaltensorfour^{\sharp}+2p\spatialsymidentitytensorfour^{\sharp}\\
  +J^{\frac{1}{3}}\sqbrac{\tensorprod{\inverse{\tensortwo{g}}}{\pbrac{\doubledotprod{\spatialdeviatorictensorfour}{\tensortwo{y}}}}+\tensorprod{\pbrac{\doubledotprod{\spatialdeviatorictensorfour}{\tensortwo{y}}}}{\inverse{\tensortwo{g}}}}
  \\
  +J^{-\frac{4}{3}}\doubledotprodthree{\spatialdeviatorictensorfour}{\bar{\spatialsecondelasticitytensor}}{\transpose{\spatialdeviatorictensorfour}}+\dfrac{2}{3}J^{-\frac{2}{3}}\trop\bar{\tensortwo{\sigma}}\spatialdeviatorictensorfour^{\sharp}\\
  -\dfrac{2}{3}\sqbrac{\tensorprod{\inverse{\tensortwo{g}}}{\cauchystresstensor_{\devop}}+\tensorprod{\tensortwo{\sigma}_{\devop}}{\inverse{\tensortwo{g}}}}
\end{multline}

Note that if we do not allow for any coupling between the deviatoric and
spherical parts of stress or for any change in pressure with volume then we
can make the following simplicifications
\begin{equation}
  \tensor{Y}=2\deltwoby{\fnof{\bar{W}}{\bar{\rightcauchygreentensor},J}}{J}{\bar{\rightcauchygreentensor}}=\deltwoby{\fnof{\bar{W}}{\bar{\greenlagrangestraintensor},J}}{J}{\bar{\greenlagrangestraintensor}}=\tensor{0}
\end{equation}
and
\begin{equation}
  K=\deltwosqby{\fnof{\bar{W}}{\bar{\rightcauchygreentensor},J}}{J}=\deltwosqby{\fnof{\bar{W}}{\bar{\greenlagrangestraintensor},J}}{J}=0
\end{equation}

We thus have
\begin{equation}
  \fnof{\materialsecondelasticitytensor}{\rightcauchygreentensor}=\doubledotprodthree{\transpose{\pbrac{\delby{\bar{\rightcauchygreentensor}}{\rightcauchygreentensor}}}}{\bar{\materialsecondelasticitytensor}}{\delby{\bar{\rightcauchygreentensor}}{\rightcauchygreentensor}}+\doubledotprod{\bar{\secondpiolakirchoffstresstensor}}{\deltwosqby{\bar{\rightcauchygreentensor}}{\rightcauchygreentensor}}-2p\deltwosqby{J}{\rightcauchygreentensor}
\end{equation}
or
\begin{equation}
  \fnof{\materialsecondelasticitytensor}{\greenlagrangestraintensor}=\doubledotprodthree{\transpose{\pbrac{\delby{\bar{\greenlagrangestraintensor}}{\greenlagrangestraintensor}}}}{\bar{\materialsecondelasticitytensor}}{\delby{\bar{\greenlagrangestraintensor}}{\greenlagrangestraintensor}}+\doubledotprod{\bar{\secondpiolakirchoffstresstensor}}{\deltwosqby{\bar{\greenlagrangestraintensor}}{\greenlagrangestraintensor}}-p\deltwosqby{J}{\greenlagrangestraintensor}
\end{equation}

The second material elasticity tensor is thus
\begin{multline}
  \materialsecondelasticitytensor=-pJ\pbrac{3\materialsphericaltensorfour^{\sharp *}-2\materialsymidentitytensorfour^{\sharp
      *}}
  +J^{-\frac{4}{3}}\doubledotprodthree{\pullback{\materialdeviatorictensorfour}{}}{\bar{\materialsecondelasticitytensor}}{\materialdeviatorictensorfour^{*T}}+\dfrac{2}{3}J^{-\frac{2}{3}}\pullback{\Trop}{}\bar{\secondpiolakirchoffstresstensor}\materialdeviatorictensorfour^{\sharp
    *}\\
  -\dfrac{2}{3}\sqbrac{\tensorprod{\pioladeformationtensor}{\secondpiolakirchoffstresstensor_{\pullback{\Devop}{}}}+\tensorprod{\secondpiolakirchoffstresstensor_{\pullback{\Devop}{}}}{\pioladeformationtensor}}
  \label{eqn:IncompressibleSecondMaterialElasticityTensor}
\end{multline}
and the second spatial elasticity tensor is thus
\begin{multline}
  \spatialsecondelasticitytensor=-p\pbrac{3\spatialsphericaltensorfour^{\sharp}-2\spatialsymidentitytensorfour^{\sharp}}
  +J^{-\frac{4}{3}}\doubledotprodthree{\spatialdeviatorictensorfour}{\bar{\spatialsecondelasticitytensor}}{\transpose{\spatialdeviatorictensorfour}}+\dfrac{2}{3}J^{-\frac{2}{3}}\trop\bar{\tensortwo{\sigma}}\spatialdeviatorictensorfour^{\sharp}\\
  -\dfrac{2}{3}\sqbrac{\tensorprod{\inverse{\tensortwo{g}}}{\cauchystresstensor_{\devop}}+\tensorprod{\tensortwo{\sigma}_{\devop}}{\inverse{\tensortwo{g}}}}
  \label{eqn:IncompressibleSecondSpatialElasticityTensor}
\end{multline}

\subsection{Isotropic materials}

Because $\rightcauchygreentensor$ is symmetric then we can deal with the invariants \ie
We thus have
$\fnof{W}{\fnof{\rightcauchygreentensor}{\vectr{N}}}=\fnof{W}{\fnof{I_{1}}{\rightcauchygreentensor},\fnof{I_{2}}{\rightcauchygreentensor},\fnof{I_{3}}{\rightcauchygreentensor}}$ and
thus
\begin{equation}
  \secondpiolakirchofftensorsymbol^{AB}=2\pbrac{\delby{W}{I_{1}}\delby{I_{1}}{\rightcauchygreentensorsymbol_{AB}}+\delby{W}{I_{2}}\delby{I_{2}}{\rightcauchygreentensorsymbol_{AB}}+\delby{W}{I_{3}}\delby{I_{3}}{\rightcauchygreentensorsymbol_{AB}}}
\end{equation}
or if we have
$\fnof{W}{\fnof{\greenlagrangestraintensor}{\vectr{N}}}=\fnof{W}{\fnof{I_{1}}{\greenlagrangestraintensor},\fnof{I_{2}}{\greenlagrangestraintensor},\fnof{I_{3}}{\greenlagrangestraintensor}}$ and
thus
\begin{equation}
  \secondpiolakirchofftensorsymbol^{AB}=\pbrac{\delby{W}{I_{1}}\delby{I_{1}}{E_{AB}}+\delby{W}{I_{2}}\delby{I_{2}}{E_{AB}}+\delby{W}{I_{3}}\delby{I_{3}}{E_{AB}}}
\end{equation}

The three invariants are given by
\begin{equation}
  \begin{split}
    \fnof{I_{1}}{\rightcauchygreentensor} &= \trop\rightcauchygreentensor \\
    &= \rightcauchygreentensorsymbol_{11} + \rightcauchygreentensorsymbol_{22} + \rightcauchygreentensorsymbol_{33} \\
    \fnof{I_{2}}{\rightcauchygreentensor} &=
    \dfrac{1}{2}\pbrac{\pbrac{\trop\rightcauchygreentensor}^{2}-\trop\rightcauchygreentensor^{2}}
    = \determinant{\rightcauchygreentensor}\trop\inverse{\rightcauchygreentensor}\\
    &=
    \dfrac{1}{2}\left(\pbrac{\rightcauchygreentensorsymbol_{11}+\rightcauchygreentensorsymbol_{22}+\rightcauchygreentensorsymbol_{33}}^{2}\right. \\
      & \quad\left.-\pbrac{\rightcauchygreentensorsymbol_{11}^{2}+\rightcauchygreentensorsymbol_{12}\rightcauchygreentensorsymbol_{21}+\rightcauchygreentensorsymbol_{13}\rightcauchygreentensorsymbol_{31}+
        \rightcauchygreentensorsymbol_{21}\rightcauchygreentensorsymbol_{12}+\rightcauchygreentensorsymbol_{22}^{2}+\rightcauchygreentensorsymbol_{23}\rightcauchygreentensorsymbol_{32}+\rightcauchygreentensorsymbol_{31}\rightcauchygreentensorsymbol_{13}+\rightcauchygreentensorsymbol_{32}\rightcauchygreentensorsymbol_{23}+\rightcauchygreentensorsymbol_{33}^{2}}\right) \\
    \fnof{I_{3}}{\rightcauchygreentensor} &= \determinant{\rightcauchygreentensor} \\
    &=\rightcauchygreentensorsymbol_{11}\rightcauchygreentensorsymbol_{22}\rightcauchygreentensorsymbol_{33}+\rightcauchygreentensorsymbol_{12}\rightcauchygreentensorsymbol_{23}\rightcauchygreentensorsymbol_{31}+\rightcauchygreentensorsymbol_{13}\rightcauchygreentensorsymbol_{21}\rightcauchygreentensorsymbol_{32}\\
    &\quad-\rightcauchygreentensorsymbol_{13}\rightcauchygreentensorsymbol_{22}\rightcauchygreentensorsymbol_{31}-\rightcauchygreentensorsymbol_{12}\rightcauchygreentensorsymbol_{21}\rightcauchygreentensorsymbol_{33}-\rightcauchygreentensorsymbol_{11}\rightcauchygreentensorsymbol_{23}\rightcauchygreentensorsymbol_{32}
  \end{split}
\end{equation}

Note that we also have \citep{adkins:1954},
\begin{equation}
 \begin{split}
    \fnof{I_{1}}{\rightcauchygreentensor} &= \materialmetrictensorsymbol^{ij}\spatialmetrictensorsymbol_{ij} \\
    \fnof{I_{2}}{\rightcauchygreentensor} &= I_{3}\materialmetrictensorsymbol_{ij}\spatialmetrictensorsymbol^{ij}\\
    \fnof{I_{3}}{\rightcauchygreentensor} &= \frac{\abs{\spatialmetrictensorsymbol_{ij}}}{\abs{\materialmetrictensorsymbol_{ij}}}
  \end{split}  
\end{equation}

Now we have I2 NEEDS TO BE CORRECTED HERE
\begin{align}
  \delby{I_{1}}{\rightcauchygreentensor}&=\sharptensor{\materialmetrictensor}\\
  \delby{I_{2}}{\rightcauchygreentensor}&=\delby{\determinant{\rightcauchygreentensor}\trace{\inverse{\rightcauchygreentensor}}}{\rightcauchygreentensor}\\
  &=\determinant{\rightcauchygreentensor}\invtranspose{\rightcauchygreentensor}\trace{\inverse{\rightcauchygreentensor}}+
  \determinant{\rightcauchygreentensor}\sharptensor{\materialmetrictensor}
  &=\inverse{\rightcauchygreentensor}\determinant{\rightcauchygreentensor}\trop\inverse{\rightcauchygreentensor}-\determinant{\rightcauchygreentensor}\trop\delby{\inverse{\rightcauchygreentensor}}{\rightcauchygreentensor}\\
  &=I_{2}\inverse{\rightcauchygreentensor}-I_{3}\invsquared{\rightcauchygreentensor} \\
  \delby{I_{3}}{\rightcauchygreentensor}&=\determinant{\rightcauchygreentensor}\inverse{\rightcauchygreentensor} \\
  &= I_{3}\inverse{\rightcauchygreentensor} 
\end{align}
and thus
\begin{equation}
  \secondpiolakirchoffstresstensor=2\pbrac{\delby{W}{I_{1}}\sharptensor{\materialmetrictensor}+\pbrac{\delby{W}{I_{2}}I_{2}+\delby{W}{I_{3}}I_{3}}\inverse{\rightcauchygreentensor}+\delby{W}{I_{2}}I_{3}\invsquared{\rightcauchygreentensor}}
\end{equation}
or equivalently
\begin{equation}
  \secondpiolakirchoffstresstensor=2\pbrac{\delby{W}{I_{1}}\sharptensor{\materialmetrictensor}+\pbrac{\delby{W}{I_{2}}I_{2}+\delby{W}{I_{3}}I_{3}}\pioladeformationtensor+\delby{W}{I_{2}}I_{3}\pioladeformationtensor^{2}}
\end{equation}
as $\pioladeformationtensor=\inverse{\rightcauchygreentensor}$.

Note that
\begin{equation}
  \inverse{\rightcauchygreentensor}=\dfrac{1}{\determinant{\rightcauchygreentensor}}\adjop\rightcauchygreentensor=\dfrac{1}{I_{3}}\adjop\rightcauchygreentensor=\dfrac{1}{I_{3}}\delby{I_{3}}{\rightcauchygreentensor}
\end{equation}

Now we have
\begin{equation}
  \delby{I_{1}}{\rightcauchygreentensorsymbol_{AB}}=\begin{bmatrix}
    1 & 0 & 0 \\
    0 & 1 & 0 \\
    0 & 0 & 1
  \end{bmatrix}
\end{equation}
and
\begin{equation}
  \delby{I_{2}}{\rightcauchygreentensorsymbol_{AB}}=\begin{bmatrix}
    \rightcauchygreentensorsymbol_{22}+\rightcauchygreentensorsymbol_{33} & -\rightcauchygreentensorsymbol_{21} & -\rightcauchygreentensorsymbol_{31} \\
    -\rightcauchygreentensorsymbol_{12} & \rightcauchygreentensorsymbol_{11}+\rightcauchygreentensorsymbol_{33} & -\rightcauchygreentensorsymbol_{32} \\
    -\rightcauchygreentensorsymbol_{13} & -\rightcauchygreentensorsymbol_{23} & \rightcauchygreentensorsymbol_{11}+\rightcauchygreentensorsymbol_{22}
  \end{bmatrix}
\end{equation}
and
\begin{equation}
  \delby{I_{3}}{\rightcauchygreentensorsymbol_{AB}}=\begin{bmatrix}
    \rightcauchygreentensorsymbol_{22}\rightcauchygreentensorsymbol_{33}-\rightcauchygreentensorsymbol_{23}\rightcauchygreentensorsymbol_{32} & \rightcauchygreentensorsymbol_{23}\rightcauchygreentensorsymbol_{31}-\rightcauchygreentensorsymbol_{21}\rightcauchygreentensorsymbol_{33} & \rightcauchygreentensorsymbol_{23}\rightcauchygreentensorsymbol_{32}-\rightcauchygreentensorsymbol_{22}\rightcauchygreentensorsymbol_{31} \\
    \rightcauchygreentensorsymbol_{13}\rightcauchygreentensorsymbol_{32}-\rightcauchygreentensorsymbol_{12}\rightcauchygreentensorsymbol_{33} & \rightcauchygreentensorsymbol_{11}\rightcauchygreentensorsymbol_{33}-\rightcauchygreentensorsymbol_{13}\rightcauchygreentensorsymbol_{31} & \rightcauchygreentensorsymbol_{12}\rightcauchygreentensorsymbol_{31}-\rightcauchygreentensorsymbol_{11}\rightcauchygreentensorsymbol_{32} \\
    \rightcauchygreentensorsymbol_{12}\rightcauchygreentensorsymbol_{32}-\rightcauchygreentensorsymbol_{22}\rightcauchygreentensorsymbol_{31} & \rightcauchygreentensorsymbol_{13}\rightcauchygreentensorsymbol_{23}-\rightcauchygreentensorsymbol_{11}\rightcauchygreentensorsymbol_{23} & \rightcauchygreentensorsymbol_{11}\rightcauchygreentensorsymbol_{22}-\rightcauchygreentensorsymbol_{12}\rightcauchygreentensorsymbol_{21}
  \end{bmatrix}
\end{equation}

Now, in terms of the spatial stresses we have
\begin{equation}
  \cauchystresstensor=\inverse{J}\deformationgradienttensor\secondpiolakirchoffstresstensor\transpose{\deformationgradienttensor}
\end{equation}
and so


\subsubsection{Viscoelasticity}

\subsubsection{Plasticity}

\section{Principle of Virtual Work}
\label{sec:ElasticityVirtualWork}

Consider a configuration, defined by a displacement field $\vectr{u}$, of some
deformable body, $\embedmanifold{B}$, subject to some \emph{displacement
  boundary conditions} $\vectr{u}=\bar{\vectr{u}}$ over some part of the
boundary $\boundary{\embedmanifold{B}}_{u}$ and some \emph{traction boundary
  conditions} $\vectr{t}=\dotprod{\cauchystresstensor}{\vectr{n}}=\bar{\vectr{t}}$
over some part of the boundary $\boundary{\embedmanifold{B}}_{t}$. Note that
$\intersection{\boundary{\embedmanifold{B}}_{u}}{\boundary{\embedmanifold{B}}_{t}}=\emptyset$
and
$\union{\boundary{\embedmanifold{B}}_{u}}{\boundary{\embedmanifold{B}}_{t}}\subseteq\boundary{\embedmanifold{B}}$.

We wish to find the displacement field $\vectr{u}$ which satisfies the
deformation boundary conditions where they are applied and the equations of
motion and the traction boundary conditions where they are applied. A
displacement field $\vectr{u}$ which satisfies the displacemnet boundary
conditions where they are applied but whose resulting stress field does not
necessarily satisfy the equations of motion or the traction boundary
conditions where they are applied is known as a \emph{kinematically admissable
  displacement field}. Similarily, a stress field $\cauchystresstensor$ which
satisfies the equations of motion and the traction boundary conditions where
they are applied but whost resulting displacement field does not necessarily
satisfy the displacement boundary conditions where they are applied is known
as a \emph{statically admissable stress field}.

For a statically admissable stress field $\cauchystresstensor$ conservation of
momentum gives us the equations of motion \ie
\begin{equation}
  \gint{\embedmanifold{B}}{}{\densitysymbol\vectr{a}}{v}=\gint{\embedmanifold{B}}{}{\pbrac{\divergence{}{\cauchystresstensor}+\vectr{b}}}{v}
\end{equation}
where $\vectr{a}$ is the acceleration of the body and $\vectr{b}$ are the body
forces. If we now multiply by the equations of motion by a kinematically
admissable displacement field $\vectr{u}$ we obtain
\begin{equation}
  \begin{split}
    \gint{\embedmanifold{B}}{}{\dotprod{\densitysymbol\vectr{a}}{\vectr{u}}}{v} &=
    \gint{\embedmanifold{B}}{}{\dotprod{\pbrac{\divergence{}{\cauchystresstensor}+\vectr{b}}}{\vectr{u}}}{v}\\
    &=\gint{\embedmanifold{B}}{}{\dotprod{\pbrac{\divergence{}{\cauchystresstensor}}}{\vectr{u}}}{v}+
    \gint{\embedmanifold{B}}{}{\dotprod{\vectr{b}}{\vectr{u}}}{v}
  \end{split}
  \label{eqn:equationsofwork}
\end{equation}

Now, by the vector identity
\begin{equation}
  \divergence{}{\pbrac{\dotprod{\tensortwo{\sigma}}{\vectr{u}}}}=\dotprod{\pbrac{\divergence{}{\tensortwo{\sigma}}}}{\vectr{u}}+
  \doubledotprod{\transpose{\tensortwo{\sigma}}}{\gradient{}{\vectr{u}}}
\end{equation}
we have
\begin{equation}
  \dotprod{\pbrac{\divergence{}{\tensortwo{\sigma}}}}{\vectr{u}}=\divergence{}{\pbrac{\dotprod{\tensortwo{\sigma}}{\vectr{u}}}}-
  \doubledotprod{\tensortwo{\sigma}}{\gradient{}{\vectr{u}}}
\end{equation}
as the stress tensor, $\tensortwo{\sigma}$, symmetric and so
$\transpose{\tensortwo{\sigma}}=\tensortwo{\sigma}$. \Eqnref{eqn:equationsofwork} becomes
\begin{equation}
  \begin{split}
    \gint{\embedmanifold{B}}{}{\dotprod{\densitysymbol\vectr{a}}{\vectr{u}}}{v}&=
    \gint{\embedmanifold{B}}{}{\pbrac{\divergence{}{\pbrac{\dotprod{\cauchystresstensor}{\vectr{u}}}}-
        \doubledotprod{\cauchystresstensor}{\gradient{}{\vectr{u}}}}}{v}+
    \gint{\embedmanifold{B}}{}{\dotprod{\vectr{b}}{\vectr{u}}}{v} \\
    &=\gint{\embedmanifold{B}}{}{\divergence{}{\pbrac{\dotprod{\cauchystresstensor}{\vectr{u}}}}}{v}-
    \gint{\embedmanifold{B}}{}{\doubledotprod{\cauchystresstensor}{\gradient{}{\vectr{u}}}}{v}+
    \gint{\embedmanifold{B}}{}{\dotprod{\vectr{b}}{\vectr{u}}}{v}
  \end{split}
\end{equation}

Applying the divergence theorem to the first integral on the right hand side
gives
\begin{equation}
  \begin{split}
    \gint{\embedmanifold{B}}{}{\divergence{}{\pbrac{\dotprod{\cauchystresstensor}{\vectr{u}}}}}{v}
    &=\gint{\boundary{\embedmanifold{B}}}{}{\dotprod{\pbrac{\dotprod{\cauchystresstensor}{\vectr{u}}}}{\vectr{n}}}{a}\\
    &=\gint{\boundary{\embedmanifold{B}}}{}{\dotprod{\pbrac{\dotprod{\cauchystresstensor}{\vectr{n}}}}{\vectr{u}}}{a}
  \end{split}
\end{equation}
or, by Cauchy's law, $\vectr{t}=\dotprod{\cauchystresstensor}{\vectr{n}}$, we have
\begin{equation}
  \gint{\embedmanifold{B}}{}{\divergence{}{\pbrac{\dotprod{\cauchystresstensor}{\vectr{u}}}}}{v}=
  \gint{\boundary{\embedmanifold{B}}}{}{\dotprod{\vectr{t}}{\vectr{u}}}{a}
\end{equation}

\Eqnref{eqn:equationsofwork} thus becomes
\begin{equation}
  \gint{\embedmanifold{B}}{}{\dotprod{\densitysymbol\vectr{a}}{\vectr{u}}}{v}=
  \gint{\boundary{\embedmanifold{B}}}{}{\dotprod{\vectr{t}}{\vectr{u}}}{a}-
  \gint{\embedmanifold{B}}{}{\doubledotprod{\cauchystresstensor}{\gradient{}{\vectr{u}}}}{v}+
  \gint{\embedmanifold{B}}{}{\dotprod{\vectr{b}}{\vectr{u}}}{v}
\end{equation}
or
\begin{equation}
  \gint{\embedmanifold{B}}{}{\dotprod{\densitysymbol\vectr{a}}{\vectr{u}}}{v}+
  \gint{\embedmanifold{B}}{}{\doubledotprod{\cauchystresstensor}{\gradient{}{\vectr{u}}}}{v}=
  \gint{\boundary{\embedmanifold{B}}}{}{\dotprod{\vectr{t}}{\vectr{u}}}{a}+
  \gint{\embedmanifold{B}}{}{\dotprod{\vectr{b}}{\vectr{u}}}{v}
\end{equation}

Taking the boundary conditions into account we have
\begin{equation}
  \gint{\embedmanifold{B}}{}{\dotprod{\densitysymbol\vectr{a}}{\vectr{u}}}{v}+
  \gint{\embedmanifold{B}}{}{\doubledotprod{\cauchystresstensor}{\gradient{}{\vectr{u}}}}{v}=
  \gint{\boundary{\embedmanifold{B}}_{u}}{}{\dotprod{\vectr{t}}{\bar{\vectr{u}}}}{a}+
  \gint{\boundary{\embedmanifold{B}}_{t}}{}{\dotprod{\bar{\vectr{t}}}{\vectr{u}}}{a}+
  \gint{\embedmanifold{B}}{}{\dotprod{\vectr{b}}{\vectr{u}}}{v}
  \label{eqn:virtualworkdisp1}
\end{equation}

If we now consider a second kinematically admissable displacement field
$\vectr{u}^{*}$ then the above equation will also hold \ie
\begin{equation}
  \gint{\embedmanifold{B}}{}{\dotprod{\densitysymbol\vectr{a}}{\vectr{u}^{*}}}{v}+
  \gint{\embedmanifold{B}}{}{\doubledotprod{\cauchystresstensor}{\gradient{}{\vectr{u}^{*}}}}{v}=
  \gint{\boundary{\embedmanifold{B}}_{u}}{}{\dotprod{\vectr{t}}{\bar{\vectr{u}}}}{a}+
  \gint{\boundary{\embedmanifold{B}}_{t}}{}{\dotprod{\bar{\vectr{t}}}{\vectr{u}^{*}}}{a}+
  \gint{\embedmanifold{B}}{}{\dotprod{\vectr{b}}{\vectr{u}^{*}}}{v}
  \label{eqn:virtualworkdisp2}
\end{equation}

Now, subtracting \Eqnref{eqn:virtualworkdisp2} from
\Eqnref{eqn:virtualworkdisp1} gives us
\begin{equation}
  \gint{\embedmanifold{B}}{}{\dotprod{\densitysymbol\vectr{a}}{\pbrac{\vectr{u}-\vectr{u}^{*}}}}{v}+
  \gint{\embedmanifold{B}}{}{\doubledotprod{\cauchystresstensor}{\gradient{}{\pbrac{\vectr{u}-\vectr{u}^{*}}}}}{v}=
  \gint{\boundary{\embedmanifold{B}}_{u}}{}{\dotprod{\vectr{t}}{\bar{\vectr{u}}}}{a}+
  \gint{\boundary{\embedmanifold{B}}_{t}}{}{\dotprod{\bar{\vectr{t}}}{\pbrac{\vectr{u}-\vectr{u}^{*}}}}{a}+
  \gint{\embedmanifold{B}}{}{\dotprod{\vectr{b}}{\pbrac{\vectr{u}-\vectr{u}^{*}}}}{v}
  \label{eqn:virtualworkdispdiff}
\end{equation}

If we now define the \emph{virtual displacements} as
$\delta\vectr{u}=\vectr{u}-\vectr{u}^{*}$ and note that
$\delta\vectr{u}=\vectr{u}-\vectr{u}^{*}=\bar{\vectr{u}}-\bar{\vectr{u}}=\vectr{0}$
on $\boundary{\embedmanifold{B}}_{u}$ then we have
\begin{equation}
  \gint{\embedmanifold{B}}{}{\dotprod{\densitysymbol\vectr{a}}{\delta\vectr{u}}}{v}+
  \gint{\embedmanifold{B}}{}{\doubledotprod{\cauchystresstensor}{\gradient{}{\delta\vectr{u}}}}{v}=
  \gint{\boundary{\embedmanifold{B}}_{t}}{}{\dotprod{\bar{\vectr{t}}}{\delta\vectr{u}}}{a}+
  \gint{\embedmanifold{B}}{}{\dotprod{\vectr{b}}{\delta\vectr{u}}}{v}
  \label{eqn:virtualworkstatment}
\end{equation}

This is known as the \emph{Principle of Virtual Work} \ie when a deformable
body undergoes some virtual displacement, $\delta\vectr{u}$, the \emph{kinetic
  work}, $\fnof{W_{kin}}{\delta\vectr{u}}$, plus the \emph{internal work},
$\fnof{W_{int}}{\delta\vectr{u}}$, is balanced by the \emph{external work},
$\fnof{W_{ext}}{\delta\vectr{u}}$, where
\begin{equation}
  \begin{split}
    \fnof{W_{kin}}{\delta\vectr{u}}&=
    \gint{\embedmanifold{B}}{}{\densitysymbol\dotprod{\vectr{a}}{\delta\vectr{u}}}{v}\\
    \fnof{W_{int}}{\delta\vectr{u}}&=
    \gint{\embedmanifold{B}}{}{\doubledotprod{\cauchystresstensor}{\gradient{}{\delta\vectr{u}}}}{v}\\
    \fnof{W_{ext}}{\delta\vectr{u}}&=\fnof{W_{surf}}{\delta\vectr{u}}+\fnof{W_{body}}{\delta\vectr{u}}\\
    &=\gint{\boundary{\embedmanifold{B}}_{t}}{}{\dotprod{\bar{\vectr{t}}}{\delta\vectr{u}}}{a}+
    \gint{\boundary{\embedmanifold{B}}_{p}}{}{\dotprod{\bar{\vectr{t}}}{\delta\vectr{u}}}{a}+
    \gint{\embedmanifold{B}}{}{\dotprod{\fnof{\vectr{b}}{\vectr{u}}}{\delta\vectr{u}}}{v}
  \end{split}
\end{equation}
where $\fnof{W_{surf}}{\vectr{u},\delta\vectr{u}}$ is the external work due to surface
forces and $\fnof{W_{body}}{\vectr{u},\delta\vectr{u}}$ is the external work due to
body forces. 

Note that the internal work is often writen in terms of \emph{virtual strain}
instead of virtual displacement. Consider
\begin{equation}
  \begin{split}
    \gradient{}{\delta\vectr{u}} &= 
    \dfrac{1}{2}\pbrac{\gradient{}{\delta\vectr{u}}+\transpose{\pbrac{\gradient{}{\delta\vectr{u}}}}}+
    \dfrac{1}{2}\pbrac{\gradient{}{\delta\vectr{u}}-\transpose{\pbrac{\gradient{}{\delta\vectr{u}}}}}\\
    &=\delta\sqbrac{\dfrac{1}{2}\pbrac{\gradient{}{\vectr{u}}+\transpose{\pbrac{\gradient{}{\vectr{u}}}}}}+
    \delta\sqbrac{\dfrac{1}{2}\pbrac{\gradient{}{\vectr{u}}-\transpose{\pbrac{\gradient{}{\vectr{u}}}}}} \\
    &=\delta\smallstraintensor +\delta\smallrotationtensor
  \end{split}
\end{equation}
where $\smallstraintensor$ is the \emph{small strain tensor} and
$\smallrotationtensor$ is the \emph{small rotation tensor}. The internal work is now
given by
\begin{equation}
  \begin{split}
    \fnof{W_{int}}{\vectr{u},\delta\vectr{u}}&=
    \gint{\embedmanifold{B}}{}{\doubledotprod{\fnof{\cauchystresstensor}{\vectr{u}}}{\gradient{}{\delta\vectr{u}}}}{v}\\
    &=\gint{\embedmanifold{B}}{}{\doubledotprod{\fnof{\cauchystresstensor}{\vectr{u}}}{
        \pbrac{\fnof{\delta\smallstraintensor}{\delta\vectr{u}}+\fnof{\delta\smallrotationtensor}{\delta\vectr{u}}}}}{v} \\
    &=\gint{\embedmanifold{B}}{}{\doubledotprod{\fnof{\cauchystresstensor}{\vectr{u}}}{\fnof{\delta\smallstraintensor}{\delta\vectr{u}}}}{v}+
    \gint{\embedmanifold{B}}{}{\doubledotprod{\fnof{\cauchystresstensor}{\vectr{u}}}{\fnof{\delta\smallrotationtensor}{\delta\vectr{u}}}}{v}
  \end{split}
\end{equation}
Therefore
\begin{equation}
  \fnof{W_{int}}{\vectr{u},\delta\vectr{u}}=
  \gint{\embedmanifold{B}}{}{\doubledotprod{\fnof{\cauchystresstensor}{\vectr{u}}}{\fnof{\delta\smallstraintensor}{\delta\vectr{u}}}}{v}
\end{equation}
as the stress tensor is a symmetric tensor and the small rotation tensor is a
skew-symmetric tensor and the double dot product between a symmetric and
skew-symmetric tensor is always zero.

In terms of strain the principle of virtual work can thus be stated as
\begin{equation}
  \gint{\embedmanifold{B}}{}{\dotprod{\densitysymbol\vectr{a}}{\delta\vectr{u}}}{v}+
  \gint{\embedmanifold{B}}{}{\doubledotprod{\cauchystresstensor}{\fnof{\delta\smallstraintensor}{\delta\vectr{u}}}}{v}=
  \gint{\boundary{\embedmanifold{B}}_{t}}{}{\dotprod{\bar{\vectr{t}}}{\delta\vectr{u}}}{a}+
  \gint{\embedmanifold{B}}{}{\dotprod{\vectr{b}}{\delta\vectr{u}}}{v}
  \label{eqn:ElasticityVirtualWorkStrainStatement}
\end{equation}

TALK ABOUT WORK CONJUGACY AND SHOW OTHER CONJUGATE FORMS

The internal work can also be expressed in the undeformed manifold in terms of
the $2^{nd}$ Piola-Kirchoff stress, $\secondpiolakirchoffstresstensor$ and the Green-Lagrange strain
tensor, $\greenlagrangestraintensor$, \ie
\begin{equation}
  \fnof{W_{int}}{\vectr{u},\delta\vectr{u}}=\gint{\embedmanifold{B}_{0}}{}{\doubledotprod{
      \fnof{\secondpiolakirchoffstresstensor}{\vectr{u}}}{\fnof{\delta\greenlagrangestraintensor}{\delta\vectr{u}}}}{V}
\end{equation}

\section{Variational principles}

The principle of virtual work is basically a statement regarding the energy of
a system. In order to solve problems of mechanics we can use the principle of
energy minimisation in order to determine the deformed shape of a body \ie we
can use the principle that a body will adopt a deformed shape that minimises
the energy of the body. Put another way, if we could quatify the total energy
in a body with a function of some quantity (or quantities) that describes the
body we could determine the deformed shape by finding the value of that
quantity that minimises (or rather more strictly that finds the extremum of)
the function. Such a problem is known as a \emph{variational principle}.

There are a number of variational principles that are used in mechanics. The
differences in the principles arise primarily from the types and numbers of
quantities that are allowed to vary. The simplest variation principle is the
\emph{potential energy variational principle} in which just displacement is
allowed to vary. This principle is the basis of \emph{displacement
  methods}. Displacement methods, whilst simple, suffer from a number of
problems included volume locking and are not so good for nearly incompressible
and compressible materials. In order to handle material (in)compressibility
\emph{mixed methods} are used in which additional quantities are allowed in
the variational principle. The \emph{Hellinger-Reissner variational principle}
is allows for both displacement and stress fields which are allowed to
independently vary. The \emph{Hu-Washizu variational principle} allows for
displacement, stress and strain fields to independently vary. 

\subsection{Potential Energy}

The potential energy variational principle can be stated as the difference
between the internal and external energies \ie
\begin{equation}
  \fnof{\Pi_{PE}}{\vectr{u}}=\fnof{\Pi_{int}}{\vectr{u}}-\fnof{\Pi_{ext}}{\vectr{u}}
\end{equation}
where
\begin{equation}
  \fnof{\Pi_{int}}{\vectr{u}}=\gint{\embedmanifold{B}_{0}}{}{\fnof{W}{\vectr{u}}}{V}
\end{equation}
with $\fnof{W}{\vectr{u}}$ is the potential energy in the deformation due to
the displacement $\vectr{u}$ and $\fnof{\Pi_{ext}}{\vectr{u}}$ will be
explained in the next section.

The variation is that the potential energy is stationary \ie
\begin{equation}
  \variation{\fnof{\Pi_{PE}}{\vectr{u}}}{\vectr{u}}=\variation{\fnof{\Pi_{int}}{\vectr{u}}}{\vectr{u}}-\variation{\fnof{\Pi_{int}}{\vectr{u}}}{\vectr{u}}=0
\end{equation}

The variations are given by 
\begin{equation}
  \begin{split}
    \variation{\fnof{\Pi_{int}}{\vectr{u}}}{\vectr{u}}&=\gint{\embedmanifold{B}_{0}}{}{\variation{\fnof{W}{\vectr{u}}}{\vectr{u}}}{V}
    \\
    &=\gint{\embedmanifold{B}_{0}}{}{\doubledotprod{\fnof{\secondpiolakirchoffstresstensor}{\fnof{\greenlagrangestraintensor}{\vectr{u}}}}{\variation{\fnof{\greenlagrangestraintensor}{\vectr{u}}}{\vectr{u}}}}{V}
  \end{split}
\end{equation}
via the virtual work theorem and
\begin{equation}
  \variation{\fnof{\Pi_{ext}}{\vectr{u}}}{\vectr{u}}=\gint{\embedmanifold{B}_{0}}{}{\dotprod{\vectr{b}}{\variationdir{\vectr{u}}}}{v}+\gint{\boundary{\embedmanifold{B}}_{0_{t}}}{}{\dotprod{\bar{\vectr{t}}}{\variationdir{\vectr{u}}}}{\covectr{A}}
\end{equation}

This variation is a nonlinear equation in $\variationdir{\vectr{u}}$. In order to
solve the nonlinear system of equations a Newton scheme can be used which
requires a Jacobian. The Jacobian is given by the derivative of the
variational statement in the direction of a change in $\vectr{u}$,
$\linearisationdir{\vectr{u}}$ \ie the Lie derivative of the variational statement in the
direction $\linearisationdir{\vectr{u}}$.

This requires a linerization. A linearisation of a function
$\fnof{f}{\vectr{x}}$ in the direction of $\linearisationdir{\vectr{x}}$ is
\begin{equation}
  \linearisation{f}{\vectr{x}}{\vectr{x}}=\dby{}{\epsilon}\evalat{\fnof{f}{\vectr{x}+\epsilon\linearisationdir{\vectr{x}}}}{\epsilon=0}=\fnof{f}{\vectr{x}}+\directionalderiv{\vectr{x}}{\fnof{f}{\vectr{x}}}{\linearisationdir{\vectr{x}}}
\end{equation}

The linearization of the variational statement is given by 
\begin{equation}
  \directionalderiv{}{\variation{\fnof{\Pi_{PE}}{\vectr{u}}}{\vectr{u}}}{\linearisationdir{\vectr{u}}}=
  \directionalderiv{}{\variation{\fnof{\Pi_{int}}{\vectr{u}}}{\vectr{u}}}{\linearisationdir{\vectr{u}}}
  -\directionalderiv{}{\variation{\fnof{\Pi_{ext}}{\vectr{u}}}{\vectr{u}}}{\linearisationdir{\vectr{u}}}
\end{equation}

For the directional derivative of the internal work variation (first term on
the right hand side) it is useful to consider the internal work in terms of
second Piola-Kirchoff stress and the Green-Lagrange strain \ie
\begin{equation}
  \begin{split}
    \directionalderiv{}{\delta\fnof{\Pi_{int}}{\vectr{u},\delta\vectr{u}}}{\linearisationdir{\vectr{u}}}
    &=\directionalderiv{}{\gint{\embedmanifold{B}_{0}}{}{\doubledotprod{\fnof{\secondpiolakirchoffstresstensor}{
            \fnof{\greenlagrangestraintensor}{\vectr{u}}}}{\variation{\fnof{\greenlagrangestraintensor}{\vectr{u}}}{\vectr{u}}}}{V}}{\linearisationdir{\vectr{u}}}\\ &=\gint{\embedmanifold{B}_{0}}{}{\left(\doubledotprod{\fnof{\secondpiolakirchoffstresstensor}{
          \fnof{\greenlagrangestraintensor}{\vectr{u}}}}{\directionalderiv{}{\variation{\fnof{\greenlagrangestraintensor}{\vectr{u}}}{\vectr{u}}}{\linearisationdir{\vectr{u}}}}\right.\\ &\qquad
      \qquad + \left.\doubledotprod{\directionalderiv{}{\fnof{\secondpiolakirchoffstresstensor}{
            \fnof{\greenlagrangestraintensor}{\vectr{u}}}}{\linearisationdir{\vectr{u}}}}{\variation{\fnof{\greenlagrangestraintensor}{\vectr{u}}}{\vectr{u}}}\right)}{V}
  \end{split}
\end{equation}

Now, $\variation{\fnof{\greenlagrangestraintensor}{\vectr{u}}}{\vectr{u}}$ can be found by pulling
back $\variation{\fnof{\smallstraintensor}{\vectr{u}}}{\vectr{u}}$ \ie
\begin{equation}
  \begin{split}
    \variation{\fnof{\greenlagrangestraintensor}{\vectr{u}}}{\vectr{u}}&=\pullback{\chi}{\pbrac{\variation{\fnof{\smallstraintensor}{\vectr{u}}}{\vectr{u}}}}\\
    &=\transpose{\deformationgradienttensor}\variation{\fnof{\smallstraintensor}{\vectr{u}}}{\vectr{u}}\deformationgradienttensor\\
    &=\transpose{\deformationgradienttensor}\frac{1}{2}\pbrac{\transpose{\pbrac{\gradient{\vectr{x}}{\variationdir{\vectr{u}}}}}
      +\gradient{\vectr{x}}{\variationdir{\vectr{u}}}}\deformationgradienttensor\\
    &=\frac{1}{2}\pbrac{\transpose{\deformationgradienttensor}\transpose{\pbrac{\gradient{\vectr{x}}{\variationdir{\vectr{u}}}}}\deformationgradienttensor+
      \transpose{\deformationgradienttensor}\gradient{\vectr{x}}{\variationdir{\vectr{u}}}\deformationgradienttensor}\\
    &=\frac{1}{2}\pbrac{\transpose{\pbrac{\gradient{\vectr{X}}{\variationdir{\vectr{u}}}}}\deformationgradienttensor+
      \transpose{\deformationgradienttensor}\gradient{\vectr{X}}{\variationdir{\vectr{u}}}} \\
    &= \symop\pbrac{\transpose{\deformationgradienttensor}\gradient{\vectr{X}}{\variationdir{\vectr{u}}}}
  \end{split}
\end{equation}

To find the directional derivative of the deformation gradient consider
\begin{equation}
  \begin{split}
    \directionalderiv{\vectr{X}}{\fnof{\deformationgradienttensor}{\vectr{X}}}{\linearisationdir{\vectr{U}}}&=\evalat{\dby{}{\epsilon}\fnof{\deformationgradienttensor}{\vectr{X}+\epsilon\linearisationdir{\vectr{U}}}}{\epsilon=0}\\    
    &=\evalat{\dby{}{\epsilon}\delby{\fnof{\vectr{x}}{\vectr{X}+\epsilon\linearisationdir{\vectr{U}}}}{\vectr{X}}}{\epsilon=0}\\
    &=\evalat{\dby{}{\epsilon}\delby{\pbrac{\vectr{x}+\deformationgradienttensor\epsilon\linearisationdir{\vectr{U}}}}{\vectr{X}}}{\epsilon=0}\\
    &=\gradient{\vectr{X}}{\pbrac{\deformationgradienttensor\linearisationdir{\vectr{U}}}}\\
    &=\gradient{\vectr{X}}{\linearisationdir{\vectr{u}}}
  \end{split}
\end{equation}
where $\linearisationdir{\vectr{U}}$ is a direction relative to the position in the
reference configuration, $\vectr{X}$, and $\linearisationdir{\vectr{u}}=\deformationgradienttensor\linearisationdir{\vectr{U}}$ is a direction
relative to the position in the current configuration, $\vectr{x}$. The
directional derivative with respect to the current configuation is given by
\begin{equation}
  \directionalderiv{\vectr{x}}{\fnof{\deformationgradienttensor}{\vectr{x}}}{\linearisationdir{\vectr{u}}}=\pbrac{\gradient{\vectr{x}}{\linearisationdir{\vectr{u}}}}\deformationgradienttensor
\end{equation}

We thus have
\begin{equation}
  \begin{split}
    \directionalderiv{}{\variation{\fnof{\greenlagrangestraintensor}{\vectr{u}}}{\vectr{u}}}{\linearisationdir{\vectr{u}}}&=
    \directionalderiv{}{\symop\pbrac{\transpose{\deformationgradienttensor}\gradient{\vectr{X}}{\variationdir{\vectr{u}}}}}{\linearisationdir{\vectr{u}}}\\
    &=\symop\pbrac{\transpose{\deformationgradienttensor}\directionalderiv{}{\gradient{\vectr{X}}{\variationdir{\vectr{u}}}}{\linearisationdir{\vectr{u}}}+
      \transpose{\pbrac{\directionalderiv{}{\deformationgradienttensor}{\linearisationdir{\vectr{u}}}}}\gradient{\vectr{X}}{\variationdir{\vectr{u}}}}\\
    &=\symop\pbrac{0+\transpose{\pbrac{\gradient{\vectr{X}}{\linearisationdir{\vectr{u}}}}}\gradient{\vectr{X}}{\variationdir{\vectr{u}}}}\\
    &=\symop\pbrac{\transpose{\pbrac{\gradient{\vectr{X}}{\linearisationdir{\vectr{u}}}}}\gradient{\vectr{X}}{\variationdir{\vectr{u}}}}
  \end{split}
\end{equation}
as $\variationdir{\vectr{u}}$ is independent of $\vectr{u}$ and is this unaffected by
the directional derivative (CHECK).

We also have
\begin{equation}
  \directionalderiv{}{\fnof{\greenlagrangestraintensor}{\variationdir{\vectr{u}}}}{\linearisationdir{\vectr{u}}}=
  \symop\pbrac{\transpose{\pbrac{\gradient{\vectr{X}}{\linearisationdir{\vectr{u}}}}}\deformationgradienttensor}
\end{equation}
(DERIVE).

Now
\begin{equation}
  \directionalderiv{}{\fnof{\secondpiolakirchoffstresstensor}{\fnof{\greenlagrangestraintensor}{\vectr{u}}}}{\linearisationdir{\vectr{u}}}=
  \doubledotprod{\delby{\fnof{\secondpiolakirchoffstresstensor}{\fnof{\greenlagrangestraintensor}{\vectr{u}}}}{\fnof{\greenlagrangestraintensor}{\vectr{u}}}}{
    \directionalderiv{}{\variation{\fnof{\greenlagrangestraintensor}{\vectr{u}}}{\vectr{u}}}{\linearisationdir{\vectr{u}}}}
\end{equation}

The derivative of the stress tensor with respect to the strain tensor is the
fourth order \emph{elasticity tensor} (sometimes called the \emph{stiffness
  tensor}) \ie
\begin{equation}
  \materialsecondelasticitytensor=\delby{\fnof{\secondpiolakirchoffstresstensor}{\greenlagrangestraintensor}}{\greenlagrangestraintensor}
\end{equation}
or in component form
\begin{equation}
  C^{ABCD}=\delby{S^{AB}}{E_{CD}}
\end{equation}

Thus we have
\begin{equation}
  \begin{split}
    \doubledotprod{\fnof{\secondpiolakirchoffstresstensor}{\fnof{\greenlagrangestraintensor}{\vectr{u}}}}{\directionalderiv{}{\variation{\fnof{\greenlagrangestraintensor}{\vectr{u}}}{\vectr{u}}}{\linearisationdir{\vectr{u}}}}&=
    \doubledotprod{\fnof{\secondpiolakirchoffstresstensor}{\fnof{\greenlagrangestraintensor}{\vectr{u}}}}{\symop\pbrac{\transpose{\pbrac{\gradient{\vectr{X}}{\linearisationdir{\vectr{u}}}}}\gradient{\vectr{X}}{\variationdir{\vectr{u}}}}}\\
    &=\doubledotprod{\fnof{\secondpiolakirchoffstresstensor}{\fnof{\greenlagrangestraintensor}{\vectr{u}}}}{\transpose{\pbrac{\gradient{\vectr{X}}{\linearisationdir{\vectr{u}}}}}\gradient{\vectr{X}}{\variationdir{\vectr{u}}}}\\
    &=\doubledotprod{\gradient{\vectr{X}}{\variationdir{\vectr{u}}}}{\gradient{\vectr{X}}{\linearisationdir{\vectr{u}}}\fnof{\secondpiolakirchoffstresstensor}{\fnof{\greenlagrangestraintensor}{\vectr{u}}}}
  \end{split}
\end{equation}
as
$\doubledotprod{\tensor{A}}{\transpose{\pioladeformationtensor}\rightcauchygreentensor}=\doubledotprod{\rightcauchygreentensor}{\tensor{B}\tensor{A}}$, and
\begin{equation}
  \begin{split}
    \doubledotprod{\directionalderiv{}{\fnof{\secondpiolakirchoffstresstensor}{\fnof{\greenlagrangestraintensor}{\vectr{u}}}}{\linearisationdir{\vectr{u}}}}{\variation{\fnof{\greenlagrangestraintensor}{\vectr{u}}}{\vectr{u}}}
    &=\doubledotprod{\doubledotprod{\delby{\fnof{\secondpiolakirchoffstresstensor}{\fnof{\greenlagrangestraintensor}{\vectr{u}}}}{\fnof{\greenlagrangestraintensor}{\vectr{u}}}}{\directionalderiv{}{\variation{\fnof{\greenlagrangestraintensor}{\vectr{u}}}{\vectr{u}}}{\linearisationdir{\vectr{u}}}}}{\variation{\fnof{\greenlagrangestraintensor}{\vectr{u}}}{\vectr{u}}}\\
    &=\doubledotprod{\doubledotprod{\materialsecondelasticitytensor}{\symop\pbrac{\transpose{\deformationgradienttensor}\gradient{\vectr{X}}{\linearisationdir{\vectr{u}}}}}}{\symop\pbrac{\transpose{\deformationgradienttensor}\gradient{\vectr{X}}{\variationdir{\vectr{u}}}}}\\
    &=\doubledotprod{\transpose{\deformationgradienttensor}\gradient{\vectr{X}}{\variationdir{\vectr{u}}}}{\doubledotprod{\materialsecondelasticitytensor}{\transpose{\deformationgradienttensor}\gradient{\vectr{X}}{\linearisationdir{\vectr{u}}}}}
  \end{split}
\end{equation}
as
$\doubledotprod{\tensor{A}}{\tensor{B}}=\doubledotprod{\tensor{B}}{\tensor{A}}$.

Putting this together gives
\begin{multline}
  \directionalderiv{}{\variation{\fnof{\Pi_{int}}{\vectr{u}}}{\vectr{u}}}{\linearisationdir{\vectr{u}}}=\\
  \gint{\embedmanifold{B}_{0}}{}{\pbrac{\doubledotprod{\gradient{\vectr{X}}{\variationdir{\vectr{u}}}}{\gradient{\vectr{X}}{\linearisationdir{\vectr{u}}}\fnof{\secondpiolakirchoffstresstensor}{\fnof{\greenlagrangestraintensor}{\vectr{u}}}}+\doubledotprod{\transpose{\deformationgradienttensor}\gradient{\vectr{X}}{\variationdir{\vectr{u}}}}{\doubledotprod{\materialsecondelasticitytensor}{\transpose{\deformationgradienttensor}\gradient{\vectr{X}}{\linearisationdir{\vectr{u}}}}}}}{V}
\end{multline}
or, in component form,
\begin{equation}
  \begin{split}
    \directionalderiv{}{\variation{\fnof{\Pi_{int}}{\vectr{u}}}{u^{i}}}{\linearisationdir{u^{j}}}
    &=\gint{\embedmanifold{B}_{0}}{}{\pbrac{\delby{\variationdir{
            u^{i}}}{X^{B}}\delby{\linearisationdir{u^{j}}}{X^{D}}\contrakronecker{i}{j}S^{BD}+F^{i}_{A}\delby{\variationdir{
            u^{i}}}{X^{B}}C^{ABCD}F^{j}_{C}\delby{\linearisationdir{u^{j}}}{X^{D}}}}{V}\\
    &=\gint{\embedmanifold{B}_{0}}{}{\pbrac{\delby{\variationdir{u^{i}}}{X^{B}}\sqbrac{\contrakronecker{i}{j}S^{BD}+F^{i}_{A}F^{j}_{C}C^{ABCD}}\delby{\linearisationdir{u^{j}}}{X^{D}}}}{V}
  \end{split}
\end{equation}

Now, consider the directional derivative of the internal work variation in the current
configuration \ie with respect to cauchy stress and ??? strain
\begin{equation}
  \begin{split}
    \directionalderiv{}{\variation{\fnof{\Pi_{int}}{\vectr{u}}}{\vectr{u}}}{\linearisationdir{\vectr{u}}}
    &=\directionalderiv{}{\gint{\embedmanifold{B}}{}{\doubledotprod{\fnof{\cauchystresstensor}{
            \fnof{\tensor{e}}{\vectr{u}}}}{\variation{\fnof{\smallstraintensor}{\vectr{u}}}{\vectr{u}}}}{v}}{\linearisationdir{\vectr{u}}}\\
    &=\gint{\embedmanifold{B}}{}{\left(\doubledotprod{\fnof{\cauchystresstensor}{
          \fnof{\tensor{e}}{\vectr{u}}}}{\directionalderiv{}{\variation{\fnof{\smallstraintensor}{\vectr{u}}}{\vectr{u}}}{\linearisationdir{\vectr{u}}}}\right.\\
     & \qquad\qquad + \left.\doubledotprod{\directionalderiv{}{\fnof{\cauchystresstensor}{
            \fnof{\tensor{e}}{\vectr{u}}}}{\linearisationdir{\vectr{u}}}}{\variation{\fnof{\smallstraintensor}{\vectr{u}}}{\vectr{u}}}\right)}{v}
  \end{split}
\end{equation}

We can now use a push forward operation to find the directional derivative of the
Cauchy stress in terms of the results from the directional derivative of the second
Piola-Kirchoff stress \ie
\begin{equation}
  \begin{split}
    \fnof{\cauchystresstensor}{\fnof{\tensor{e}}{\vectr{u}}}&=
    \inverse{J}\pushforward{\chi}{\pbrac{\fnof{\secondpiolakirchoffstresstensor}{\fnof{\greenlagrangestraintensor}{\vectr{u}}}}}\\
    \directionalderiv{}{\fnof{\cauchystresstensor}{\fnof{\tensor{e}}{\vectr{u}}}}{\linearisationdir{\vectr{u}}}&=
    \inverse{J}\pushforward{\chi}{\pbrac{\directionalderiv{}{\fnof{\secondpiolakirchoffstresstensor}{\fnof{\greenlagrangestraintensor}{\vectr{u}}}}{\linearisationdir{\vectr{u}}}}}\\
    \variation{\fnof{\smallstraintensor}{\vectr{u}}}{\vectr{u}}&=\pushforward{\chi}{\pbrac{\variation{\fnof{\greenlagrangestraintensor}{\vectr{u}}}{\vectr{u}}}} \\
    \directionalderiv{}{\variation{\fnof{\smallstraintensor}{\vectr{u}}}{\vectr{u}}}{\linearisationdir{\vectr{u}}}&=
    \pushforward{\chi}{\pbrac{\directionalderiv{}{\variation{\fnof{\greenlagrangestraintensor}{\vectr{u}}}{\vectr{u}}}{\linearisationdir{\vectr{u}}}}}
  \end{split}
\end{equation}

Now,
\begin{equation}
  \begin{split}
    \directionalderiv{}{\fnof{\cauchystresstensor}{\fnof{\tensor{e}}{\vectr{u}}}}{\linearisationdir{\vectr{u}}}&=
    \inverse{J}\pushforward{\chi}{\pbrac{\directionalderiv{}{\fnof{\secondpiolakirchoffstresstensor}{\fnof{\greenlagrangestraintensor}{\vectr{u}}}}{\linearisationdir{\vectr{u}}}}}\\
    &=\inverse{J}\pushforward{\chi}{\pbrac{\doubledotprod{\materialsecondelasticitytensor}{\transpose{\deformationgradienttensor}\gradient{\vectr{X}}{\linearisationdir{\vectr{u}}}}}}\\
    &=\inverse{J}\deformationgradienttensor\pbrac{\doubledotprod{\materialsecondelasticitytensor}{\transpose{\deformationgradienttensor}\gradient{\vectr{X}}{\linearisationdir{\vectr{u}}}}}\transpose{\deformationgradienttensor}
  \end{split}
\end{equation}

The push forward of the fourth order elasticity tensor is
\begin{equation}
  \spatialsecondelasticitytensor=\inverse{J}\pushforward{\chi}{\pbrac{\materialsecondelasticitytensor}}
\end{equation}

In component form this is
\begin{equation}
  c^{ijkl}=\inverse{J}F^{i}_{A}F^{j}_{B}F^{k}_{C}F^{l}_{D}C^{ABCD}
\end{equation}

Putting this together gives
\begin{equation}
  \directionalderiv{}{\variation{\fnof{\Pi_{int}}{\vectr{u}}}{\vectr{u}}}{\linearisationdir{\vectr{u}}}=
  \gint{\embedmanifold{B}}{}{\pbrac{\doubledotprod{\gradient{\vectr{x}}{\variationdir{\vectr{u}}}}{\gradient{\vectr{x}}{\linearisationdir{\vectr{u}}}\fnof{\cauchystresstensor}{\fnof{\tensor{e}}{\vectr{u}}}}+
            \doubledotprod{\gradient{\vectr{x}}{\variationdir{\vectr{u}}}}{\doubledotprod{\spatialsecondelasticitytensor}{\gradient{\vectr{x}}{\linearisationdir{\vectr{u}}}}}}}{v}
\end{equation}
or, in component form,
\begin{equation}
  \begin{split}
    \directionalderiv{}{\variation{\fnof{\Pi_{int}}{\vectr{u}}}{u^{i}}}{\linearisationdir{u^{j}}}
    &=\gint{\embedmanifold{B}}{}{\pbrac{\delby{\variationdir{u^{i}}}{x^{k}}\delby{\linearisationdir{u^{j}}}{x^{l}}\contrakronecker{i}{j}\sigma^{kl}+
        \delby{\variationdir{u^{i}}}{x^{k}}c^{ikjl}\delby{\linearisationdir{u^{j}}}{x^{l}}}}{v}\\
    &=\gint{\embedmanifold{B}}{}{\pbrac{\delby{\variationdir{u^{i}}}{x^{k}}\sqbrac{\contrakronecker{i}{j}\sigma^{kl}+c^{ikjl}}\delby{\linearisationdir{u^{j}}}{x^{l}}}}{v}
  \end{split}
\end{equation}

\subsection{External work}

The external work potential energy is given by
\begin{equation}
  \fnof{\Pi_{ext}}{\vectr{u},\vect{b},\bar{\vect{t}},P_{ext}}=\fnof{\Pi_{body}}{\vectr{u},\vectr{b}}+\fnof{\Pi_{press}}{\vectr{u},P_{ext}}+\fnof{\Pi_{surf}}{\vectr{u},\bar{\vectr{t}}}\\
\end{equation}
where
\begin{equation}
  \fnof{\Pi_{body}}{\vectr{u},\vectr{b}}=\gint{\embedmanifold{B}_{0}}{}{\dotprod{\vect{b}}{\vectr{u}}}{V}
\end{equation}
and 
\begin{equation}
  \fnof{\Pi_{press}}{\vectr{u},P_{ext}}=\gint{\embedmanifold{B}_{0_{P}}}{}{\dotprod{\fnof{\bar{\vect{t}}_{P}}{\vectr{u},P_{ext}}}{\vectr{u}}}{V}
\end{equation}
and
\begin{equation}
  \fnof{\Pi_{surf}}{\vectr{u},\bar{\vectr{t}}}=\gint{\embedmanifold{B}_{0_{t}}}{}{\dotprod{\bar{\vect{t}}}{\vectr{u}}}{V}
\end{equation}

The variations are
\begin{equation}
  \variation{\fnof{\Pi_{body}}{\vectr{u},\vectr{b}}}{\vectr{u}}=\gint{\embedmanifold{B}_{0}}{}{\dotprod{\vect{b}}{\variationdir{\vectr{u}}}}{V}
\end{equation}
and
\begin{equation}
  \variation{\fnof{\Pi_{press}}{\vectr{u},P_{ext}}}{\vectr{u}}=\gint{\embedmanifold{B}_{0_{P}}}{}{\dotprod{\fnof{\bar{\vect{t}}_{P}}{\vectr{u},P_{ext}}}{\variationdir{\vectr{u}}}}{V}
\end{equation}
and
\begin{equation}
  \variation{\fnof{\Pi_{surf}}{\vectr{u},\bar{\vectr{t}}}}{\vectr{u}}=\gint{\embedmanifold{B}_{0_{t}}}{}{\dotprod{\bar{\vect{t}}}{\variationdir{\vectr{u}}}}{V}
\end{equation}

For the body force variation it is often the case that The body force is due to the effect of gravity on the body \ie
\begin{equation}
  \fnof{\vectr{b}}{\vectr{u}}=\densitysymbol_{0}\vectr{a}_{g}
\end{equation}
where $\densitysymbol_{0}$ is the density of the reference configuration and $\vectr{a}_{g}$ is
the acceleration vector due to gravity.

The external body work variation is thus given by
\begin{equation}
  \begin{split}
    \variation{\fnof{\Pi_{body}}{\vectr{u},\vectr{b}}}{\vectr{u}}
    &=\gint{\embedmanifold{B}_{0}}{}{\dotprod{\densitysymbol_{0}\vectr{a}_{g}}{\variationdir{\vectr{u}}}}{V} \\
    &=\gint{\embedmanifold{B}_{0}}{}{\dotprod{\fnof{J}{\vectr{u}}\fnof{\inverse{J}}{\vectr{u}}\densitysymbol_{0}\vectr{a}_{g}}{\variationdir{\vectr{u}}}}{V} \\
    &=\gint{\embedmanifold{B}}{}{\dotprod{\densitysymbol_{0}\fnof{\inverse{J}}{\vectr{u}}\vectr{a}_{g}}{\variationdir{\vectr{u}}}}{v} \\
    &=\gint{\embedmanifold{B}}{}{\dotprod{\densitysymbol\vectr{a}_{g}}{\variationdir{\vectr{u}}}}{v}
  \end{split}
\end{equation}
where $\densitysymbol=\densitysymbol_{0}\fnof{\inverse{J}}{\vectr{u}}$ is the density in the current configuration.

Now for the pressure potential energy variation consider where the surface
tracktion is given by an external pressure \ie
\begin{equation}
  \fnof{\bar{\vectr{t}}_{P}}{\vectr{u},P_{ext}}=P_{ext}\fnof{\hat{\vectr{n}}}{\vectr{u}}
\end{equation}
where $P_{ext}$ is the applied external pressure and $\fnof{\hat{\vectr{n}}}{\vectr{u}}$ is the
unit normal vector to the surface element $\exteriorderiv{\covectr{a}}$. Now
if the surface is parameterised by the coordinates $\xi$ and $\eta$ then
\begin{equation}
  \exteriorderiv{\covectr{a}}=
  \norm{\crossprod{\delby{\vectr{x}}{\xi}}{\delby{\vectr{x}}{\eta}}}
  \wedgeprod{\exteriorderiv{\xi}}{\exteriorderiv{\eta}}
\end{equation}
and the unit normal is given by
\begin{equation}
  \fnof{\hat{\vectr{n}}}{\vectr{u}}=\dfrac{\crossprod{\delby{\vectr{x}}{\xi}}{\delby{\vectr{x}}{\eta}}}{
    \norm{\crossprod{\delby{\vectr{x}}{\xi}}{\delby{\vectr{x}}{\eta}}}}
\end{equation}
where $\vectr{x}$ are the coordinates in the current configuration \ie $\vectr{x}=\vectr{X}+\vectr{u}$.

The external pressure variation is thus given by
\begin{equation}
  \begin{split}
    \variation{\fnof{\Pi_{press}}{\vectr{u},P_{ext}}}{\vectr{u}}&=
    \gint{\boundary{\embedmanifold{B}}_{t}}{}{\dotprod{P_{ext}\fnof{\hat{\vectr{n}}}{\vectr{u}}}{
        \delta\vectr{u}}}{\covectr{a}}\\
    &=\giint{\xi}{}{\eta}{}{\dotprod{P_{ext}\dfrac{\crossprod{\delby{\vectr{x}}{\xi}}{\delby{\vectr{x}}{\eta}}}{\norm{\crossprod{\delby{\vectr{x}}{\xi}}{\delby{\vectr{x}}{\eta}}}}}{\delta\vectr{u}}\norm{\crossprod{\delby{\vectr{x}}{\xi}}{\delby{\vectr{x}}{\eta}}}}{\xi}{\eta} \\
    &=\giint{\xi}{}{\eta}{}{\dotprod{P_{ext}\pbrac{\crossprod{\delby{\vectr{x}}{\xi}}{\delby{\vectr{x}}{\eta}}}}{\delta\vectr{u}}}{\xi}{\eta}\\
    &=\giint{\xi}{}{\eta}{}{\dotprod{P_{ext}\fnof{\vectr{n}}{\vectr{u}}}{\delta\vectr{u}}}{\xi}{\eta}
  \end{split}
\end{equation}
where $\fnof{\vectr{n}}{\vectr{u}}$ is the normal direction (non-normalised) \ie
\begin{equation}
  \fnof{\vectr{n}}{\vectr{u}}=\crossprod{\delby{\vectr{x}}{\xi}}{\delby{\vectr{x}}{\eta}}
\end{equation}


If we now just consider the external traction part. For the case where the
external traction is given by a fixed external pressure we have
\begin{equation}
  \gint{\boundary{\embedmanifold{B}}_{t}}{}{\dotprod{\bar{\vectr{t}}}{\delta\vectr{u}}}{\covectr{a}}=\gint{\boundary{\embedmanifold{B}}_{t}}{}{\dotprod{p_{ext}\hat{\vectr{n}}}{\delta\vectr{u}}}{\covectr{a}}
\end{equation}
where $\hat{\vectr{n}}$ is the unit normal to the surface element
$\exteriorderiv{\covectr{a}}$. Now if the surface is parameterised by the
coordinates $\xi$ and $\eta$ then
\begin{equation}
  \exteriorderiv{\covectr{a}}=\norm{\crossprod{\delby{\vectr{x}}{\xi}}{\delby{\vectr{x}}{\eta}}}\wedgeprod{\exteriorderiv{\xi}}{\exteriorderiv{\eta}}
\end{equation}
and the unit normal is given by
\begin{equation}
  \fnof{\hat{\vectr{n}}}{\vectr{u}}=\dfrac{\crossprod{\delby{\vectr{x}}{\xi}}{\delby{\vectr{x}}{\eta}}}{\norm{\crossprod{\delby{\vectr{x}}{\xi}}{\delby{\vectr{x}}{\eta}}}}
\end{equation}

The directional derivatives of $\variation{\fnof{\Pi_{press}}{\vectr{u},P_{ext}}}{\vectr{u}}$ are given by
\begin{equation}
  \directionalderiv{}{\variation{\fnof{\Pi_{press}}{\vectr{u},P_{ext}}}{\vectr{u}}}{\linearisationdir{\vectr{u}}}=
  \giint{\xi}{}{\eta}{}{\dotprod{P_{ext}\linearisationdir{\fnof{\vectr{n}}{\vectr{u}}}}{\variationdir{\vectr{u}}}}{\xi}{\eta}
\end{equation}
and
\begin{equation}
  \directionalderiv{}{\variation{\fnof{\Pi_{press}}{\vectr{u},P_{ext}}}{\vectr{u}}}{\linearisationdir{P_{ext}}}=
  \giint{\xi}{}{\eta}{}{\dotprod{\linearisationdir{P_{ext}}\fnof{\vectr{n}}{\vectr{u}}}{\variationdir{\vectr{u}}}}{\xi}{\eta}
\end{equation}
  
For problems where $P_{ext}$ is not part of the solution procedure then
$\linearisationdir{P_{ext}}=0$.

The directional derivative of the normal vector is given by
\begin{equation}
  \directionalderiv{}{\variation{\fnof{\Pi_{press}}{\vectr{u},P_{ext}}}{\vectr{u}}}{\linearisationdir{\vectr{u}}}=
  \crossprod{\delby{\Delta\vectr{u}}{\xi}}{\delby{\vectr{x}}{\eta}}-
  \crossprod{\delby{\Delta\vectr{u}}{\eta}}{\delby{\vectr{x}}{\xi}}
\end{equation}
and thus we have
\begin{equation}
  \directionalderiv{}{\variation{\fnof{\Pi_{press}}{\vectr{u},P_{ext}}}{\vectr{u}}}{\linearisationdir{\vectr{u}}}=
  \giint{\xi}{}{\eta}{}{\dotprod{P_{ext}\pbrac{\crossprod{\delby{\linearisationdir{\vectr{u}}}{\xi}}{\delby{\vectr{x}}{\eta}}-
  \crossprod{\delby{\linearisationdir{\vectr{u}}}{\eta}}{\delby{\vectr{x}}{\xi}}}}{\delta\vectr{u}}}{\xi}{\eta}
\end{equation}

The transformation rule for a triple product is
\begin{equation}
  \dotprod{\vectr{a}}{\pbrac{\crossprod{\vectr{b}}{\vectr{c}}}}=
  \dotprod{\vectr{b}}{\pbrac{\crossprod{\vectr{c}}{\vectr{a}}}}=
  \dotprod{\vectr{c}}{\pbrac{\crossprod{\vectr{a}}{\vectr{b}}}}
\end{equation}
and so we have
\begin{equation}
  \begin{split}
  \directionalderiv{}{\variation{\fnof{\Pi_{press}}{\vectr{u},P_{ext}}}{\vectr{u}}}{\linearisationdir{\vectr{u}}}
   &=\giint{\xi}{}{\eta}{}{\dotprod{P_{ext}\pbrac{\crossprod{\delby{\Delta\vectr{u}}{\xi}}{\delby{\vectr{x}}{\eta}}-
            \crossprod{\delby{\Delta\vectr{u}}{\eta}}{\delby{\vectr{x}}{\xi}}}}{\delta\vectr{u}}}{\xi}{\eta}\\
    &=P_{ext}\giint{\xi}{}{\eta}{}{\pbrac{
        \dotprod{\pbrac{\crossprod{\delby{\vectr{x}}{\eta}}{\variationdir{\vectr{u}}}}}{\delby{\linearisationdir{\vectr{u}}}{\xi}}-
        \dotprod{\pbrac{\crossprod{\delby{\vectr{x}}{\xi}}{\variationdir{\vectr{u}}}}}{\delby{\linearisationdir{\vectr{u}}}{\eta}}}}{\xi}{\eta}
  \end{split}
  \label{eqn:nonsympressuredirectderiv}
\end{equation}

Now \eqnref{eqn:nonsympressuredirectderiv} is, in general, non-symmetric in terms of
$\linearisationdir{\vectr{u}}$ and $\variationdir{\vectr{u}}$. This will equate to a
non-conservative force.

If $P_{ext}$ is part of the solution process then
\begin{equation}
  \directionalderiv{}{\variation{\fnof{\Pi_{press}}{\vectr{u},P_{ext}}}{\vectr{u}}}{\linearisationdir{P_{ext}}}=
  \giint{\xi}{}{\eta}{}{\pbrac{\dotprod{\linearisationdir{P_{ext}}\crossprod{\delby{\vectr{x}}{\xi}}{\delby{\vectr{x}}{\eta}}}{\variationdir{\vectr{u}}}}}{\xi}{\eta}
\end{equation}

\subsection{Hellinger-Reissner}

The function is extended by the addition of a pressure constraint
\begin{equation}
  \fnof{\Pi_{HR}}{\vectr{u},p}=\fnof{\bar{\Pi}_{PE}}{\vectr{u}}+\fnof{\Pi_{P}}{\vectr{u},p}
\end{equation}
where
\begin{equation}
  \fnof{\bar{\Pi}_{PE}}{\vectr{u}}=\fnof{\bar{\Pi}_{int}}{\vectr{u}}-\fnof{\Pi_{ext}}{\vectr{u}}
\end{equation}
and
\begin{equation}
  \fnof{\bar{\Pi}_{int}}{\vectr{u}}=\gint{\embedmanifold{B}_{0}}{}{\fnof{\bar{W}}{\vectr{u}}}{V}
\end{equation}
and
\begin{equation}
  \fnof{\Pi_{P}}{\vectr{u},p}=\gint{\embedmanifold{B}_{0}}{}{-p\pbrac{\fnof{J}{\vectr{u}}-1}}{V}
\end{equation}

Note that $-p$ is used in order to establish the sign convention that a
pressure is compressive.

The variations with respect to $\variationdir{\vectr{u}}$ are
\begin{equation}
  \variation{\fnof{\bar{\Pi}_{int}}{\vectr{u}}}{\vectr{u}}=\gint{\embedmanifold{B}_{0}}{}{\variation{\fnof{\bar{W}}{\vectr{u}}}{\vectr{u}}}{V}
\end{equation}
and
\begin{equation}
  \variation{\fnof{\Pi_{P}}{\vectr{u},p}}{\vectr{u}}=\gint{\embedmanifold{B}_{0}}{}{-p\variation{\fnof{J}{\vect{u}}}{\vectr{u}}}{V}
\end{equation}

Now
\begin{equation}
  \begin{split}
    \variation{\fnof{J}{\vectr{u}}}{\vectr{u}}&=\variation{\determinant{\fnof{\deformationgradienttensor}{\vectr{u}}}}{\vectr{u}}
    \\
    &=\directionalderiv{\vectr{u}}{\determinant{\fnof{\deformationgradienttensor}{\vectr{u}}}}{\variationdir{\vectr{u}}}\\
    &=\directionalderiv{\deformationgradienttensor}{\determinant{\fnof{\deformationgradienttensor}{\vectr{u}}}}{\directionalderiv{}{\fnof{\deformationgradienttensor}{\vectr{u}}}{\variationdir{\vectr{u}}}}
    \\
    &=\directionalderiv{\deformationgradienttensor}{\determinant{\fnof{\deformationgradienttensor}{\vectr{u}}}}{\gradient{\vectr{X}}{\variationdir{\vectr{u}}}}
    \\
    &=
    \determinant{\fnof{\deformationgradienttensor}{\vectr{u}}}\doubledotprod{\invtranspose{\deformationgradienttensor}}{\gradient{\vectr{X}}{\variationdir{\vectr{u}}}}\\
    &=\fnof{J}{\vectr{u}}\trop\pbrac{\gradient{\vectr{X}}{\variationdir{\vectr{u}}}\inverse{\deformationgradienttensor}}
    \\
    &= \fnof{J}{\vectr{u}}\trop\pbrac{\gradient{\vectr{x}}{\variationdir{\vectr{u}}}}\\
    &=\fnof{J}{\vectr{u}}\divergence{\vectr{x}}{\variationdir{\vectr{u}}}
  \end{split}
\end{equation}

And so
\begin{equation}
  \begin{split}
    \variation{\fnof{\Pi_{P}}{\vectr{u},p}}{\vectr{u}}&=\gint{\embedmanifold{B}_{0}}{}{-p\fnof{J}{\vectr{u}}\divergence{\vectr{x}}{\variationdir{\vectr{u}}}}{V}\\
    &=\gint{\embedmanifold{B}}{}{-p\divergence{\vectr{x}}{\variationdir{\vectr{u}}}}{v}\\
  \end{split}
\end{equation}

Now
\begin{equation}
  \divergence{\vectr{x}}{\vectr{a}}=\doubledotprod{\sharptensor{\spatialmetrictensor}}{\gradient{\vectr{x}}{\vectr{a}}}=\doubledotprod{\inverse{\spatialmetrictensor}}{\gradient{\vectr{x}}{\covectr{a}}}
\end{equation}
for a vector $\vectr{a}$ with
$\sharptensor{\spatialmetrictensor}=\inverse{\spatialmetrictensor}$ and so we obtain
\begin{equation}
  \begin{split}
    \variation{\fnof{\Pi_{P}}{\vectr{u},p}}{\vectr{u}}&=\gint{\embedmanifold{B}}{}{\doubledotprod{-p\inverse{\spatialmetrictensor}}{\gradient{\vectr{x}}{\variationdir{\vectr{u}}}}}{v}\\
    &=\gint{\embedmanifold{B}}{}{\doubledotprod{\fnof{\cauchystresstensor_{\sphop}}{p}}{\gradient{\vectr{x}}{\variationdir{\vectr{u}}}}}{v} \\
    &=\gint{\embedmanifold{B}}{}{\doubledotprod{\fnof{\cauchystresstensor_{\sphop}}{p}}{\variation{\fnof{\smallstraintensor}{\vectr{u}}}{\vectr{u}}}}{v}
  \end{split}
\end{equation}
as the spherical stress tensor due to pressure is symmetric. Note that we have
\begin{equation}
  \fnof{\cauchystresstensor}{\vectr{u},p}=\fnof{\cauchystresstensor_{\devop}}{\vectr{u}}+\fnof{\cauchystresstensor_{\sphop}}{p}
\end{equation}
where from \eqnrefs{eqn:DeviatoricCauchyStress}{eqn:SphericalCauchyStress}
\begin{equation}
  \begin{split}
    \fnof{\cauchystresstensor_{\devop}}{\vectr{u}}&=\doubledotprod{\spatialdeviatorictensorfour}{\fnof{\tensortwo{\sigma}}{\vectr{u},p}}=J^{-\frac{2}{3}}\doubledotprod{\spatialdeviatorictensorfour}{\fnof{\bar{\tensortwo{\sigma}}}{\vectr{u}}}\\
    \fnof{\cauchystresstensor_{\sphop}}{p}&=\doubledotprod{\spatialsphericaltensorfour}{\fnof{\tensortwo{\sigma}}{\vectr{u},p}}=-p\inverse{\spatialmetrictensor}
  \end{split}
\end{equation}

Note that this variation is an expression in the current configuration. We can
obtain an equivalent expression in the reference configuration via a pull back
\ie
\begin{equation}
  \begin{split}
    \variation{\fnof{\Pi_{P}}{\vectr{u},p}}{\vectr{u}}&=\gint{\embedmanifold{B}_{0}}{}{\doubledotprod{-p\fnof{J}{\vectr{u}}\fnof{\inverse{\rightcauchygreentensor}}{\vectr{u}}}{\variation{\fnof{\greenlagrangestraintensor}{\vectr{u}}}{\vectr{u}}}}{V}\\
    &=\gint{\embedmanifold{B}_{0}}{}{\doubledotprod{\fnof{\secondpiolakirchoffstresstensor_{sph}}{\vectr{u},p}}{\variation{\fnof{\greenlagrangestraintensor}{\vectr{u}}}{\vectr{u}}}}{V}\\
  \end{split}
\end{equation}
as $\exteriorderiv{v}=J\exteriorderiv{V}$,
$\greenlagrangestraintensor=\pullback{\chi}{\smallstraintensor}$, $\secondpiolakirchoffstresstensor_{sph}=J\pullback{\chi}{\cauchystresstensor_{sph}}$,
$\rightcauchygreentensor=\pullback{\chi}{\spatialmetrictensor}$ and so
$\inverse{\rightcauchygreentensor}=\pullback{\chi}{\inverse{\spatialmetrictensor}}$. Thus
\begin{equation}
  \begin{split}
    \variation{\fnof{\bar{\Pi}_{int}}{\vectr{u}}}{\vectr{u}}+\variation{\fnof{\Pi_{P}}{\vectr{u},p}}{\vectr{u}}&=\gint{\embedmanifold{B}_{0}}{}{\variation{\fnof{\bar{W}}{\vectr{u}}}{\vectr{u}}}{V}+\gint{\embedmanifold{B}_{0}}{}{\doubledotprod{\fnof{\secondpiolakirchoffstresstensor_{sph}}{\vectr{u},p}}{\variation{\fnof{\greenlagrangestraintensor}{\vectr{u}}}{\vectr{u}}}}{v}\\
    &=\gint{\embedmanifold{B}_{0}}{}{\doubledotprod{\fnof{\secondpiolakirchoffstresstensor_{dev}}{\vectr{u}}}{\variation{\fnof{\greenlagrangestraintensor}{\vectr{u}}}{\vectr{u}}}}{V}+\gint{\embedmanifold{B}_{0}}{}{\doubledotprod{\fnof{\secondpiolakirchoffstresstensor_{sph}}{\vectr{u},p}}{\variation{\fnof{\greenlagrangestraintensor}{\vectr{u}}}{\vectr{u}}}}{V}\\
    &=\gint{\embedmanifold{B}_{0}}{}{\doubledotprod{\fnof{\secondpiolakirchoffstresstensor}{\vectr{u},p}}{\variation{\fnof{\greenlagrangestraintensor}{\vectr{u}}}{\vectr{u}}}}{V}\\
  \end{split}
\end{equation}
where
\begin{equation}
  \begin{split}
    \secondpiolakirchoffstresstensor&=\secondpiolakirchoffstresstensor_{dev}+\secondpiolakirchoffstresstensor_{sph}\\
    &= \bar{\secondpiolakirchoffstresstensor}-pJ\inverse{\rightcauchygreentensor}
  \end{split}
\end{equation}

The variations with respect to $\variationdir{p}$ are
\begin{equation}
  \variation{\fnof{\bar{\Pi}_{int}}{\vectr{u}}}{p}=0
\end{equation}
and
\begin{equation}
  \begin{split}
    \variation{\fnof{\Pi_{P}}{\vectr{u},p}}{p}&=\gint{\embedmanifold{B}_{0}}{}{-\variationdir{p}\pbrac{\fnof{J}{\vectr{u}}-1}}{V}\\
    &=\gint{\embedmanifold{B}_{0}}{}{-\variationdir{p}\dfrac{\fnof{J}{\vectr{u}}}{\fnof{J}{\vectr{u}}}\pbrac{\fnof{J}{\vectr{u}}-1}}{V}\\
    &=\gint{\embedmanifold{B}}{}{-\variationdir{p}\fnof{\inverse{J}}{\vectr{u}}\pbrac{\fnof{J}{\vectr{u}}-1}}{v}\\
    &=\gint{\embedmanifold{B}}{}{-\variationdir{p}\pbrac{1-\fnof{\inverse{J}}{\vectr{u}}}}{v}
  \end{split}
\end{equation}

The variation statement is the same as the variation statement for the
potential energy case with the exception of the addition of a hydrostatic
pressure term. The linearisation of this term in the reference configuration is
\begin{equation}
  \directionalderiv{}{\variation{\fnof{\bar{\Pi}_{int}}{\vectr{u}}}{\vectr{u}}+\variation{\fnof{\Pi_{P}}{\vectr{u},p}}{\vectr{u}}}{\linearisationdir{\vectr{u}}}=\directionalderiv{}{\gint{\embedmanifold{B}_{0}}{}{\doubledotprod{\fnof{\secondpiolakirchoffstresstensor}{
            \fnof{\greenlagrangestraintensor}{\vectr{u}}}}{\variation{\fnof{\greenlagrangestraintensor}{\vectr{u}}}{\vectr{u}}}}{V}}{\linearisationdir{\vectr{u}}}\\
\end{equation}

Following the same procedure as for the potential energy case we obtain
\begin{multline}
  \directionalderiv{}{\variation{\fnof{\bar{\Pi}_{int}}{\vectr{u}}}{\vectr{u}}+\variation{\fnof{\Pi_{P}}{\vectr{u},p}}{\vectr{u}}}{\linearisationdir{\vectr{u}}}=\\
  \gint{\embedmanifold{B}_{0}}{}{\pbrac{\doubledotprod{\gradient{\vectr{X}}{\variationdir{\vectr{u}}}}{\gradient{\vectr{X}}{\linearisationdir{\vectr{u}}}\fnof{\secondpiolakirchoffstresstensor}{\fnof{\greenlagrangestraintensor}{\vectr{u}}}}+\doubledotprod{\transpose{\deformationgradienttensor}\gradient{\vectr{X}}{\variationdir{\vectr{u}}}}{\doubledotprod{\materialsecondelasticitytensor}{\transpose{\deformationgradienttensor}\gradient{\vectr{X}}{\linearisationdir{\vectr{u}}}}}}}{V}
\end{multline}
where from \eqnref{eqn:IncompressibleSecondMaterialElasticityTensor}
\begin{equation}
  \materialsecondelasticitytensor=\materialsecondelasticitytensor_{e}+\materialsecondelasticitytensor_{p}
\end{equation}
and
\begin{equation}
  \begin{split}
    \materialsecondelasticitytensor_{e}&=J^{-\frac{4}{3}}\doubledotprodthree{\pullback{\materialdeviatorictensorfour}{}}{\bar{\materialsecondelasticitytensor}}{\materialdeviatorictensorfour^{*T}}+\dfrac{2}{3}J^{-\frac{2}{3}}\pullback{\Trop}{}\bar{\secondpiolakirchoffstresstensor}\materialdeviatorictensorfour^{\sharp *}
    -\dfrac{2}{3}\sqbrac{\tensorprod{\pioladeformationtensor}{\secondpiolakirchoffstresstensor_{\pullback{\Devop}{}}}+\tensorprod{\secondpiolakirchoffstresstensor_{\pullback{\Devop}{}}}{\pioladeformationtensor}}\\
    \materialsecondelasticitytensor_{p}&=-pJ\pbrac{3\materialsphericaltensorfour^{\sharp *}-2\materialsymidentitytensorfour^{\sharp *}}\\
    \bar{\materialsecondelasticitytensor}&=2\delby{\bar{\secondpiolakirchoffstresstensor}}{\rightcauchygreentensor}=\delby{\bar{\secondpiolakirchoffstresstensor}}{\greenlagrangestraintensor}
  \end{split}
\end{equation}

In the current configuration we have
\begin{equation}
  \begin{split}
    \directionalderiv{}{\variation{\fnof{\bar{\Pi}_{int}}{\vectr{u}}}{\vectr{u}}+\variation{\fnof{\Pi_{P}}{\vectr{u},p}}{\vectr{u}}}{\linearisationdir{\vectr{u}}}&= \\
  &\gint{\embedmanifold{B}}{}{\pbrac{\doubledotprod{\gradient{\vectr{x}}{\variationdir{\vectr{u}}}}{\gradient{\vectr{x}}{\linearisationdir{\vectr{u}}}\fnof{\cauchystresstensor}{\fnof{\tensor{e}}{\vectr{u}}}}+
            \doubledotprod{\gradient{\vectr{x}}{\variationdir{\vectr{u}}}}{\doubledotprod{\spatialsecondelasticitytensor}{\gradient{\vectr{x}}{\linearisationdir{\vectr{u}}}}}}}{v}
  \end{split}
\end{equation}
where from \eqnref{eqn:IncompressibleSecondSpatialElasticityTensor}
\begin{equation}
  \spatialsecondelasticitytensor=\spatialsecondelasticitytensor_{e}+\spatialsecondelasticitytensor_{p}
\end{equation}
and
\begin{equation}
  \begin{split}
    \spatialsecondelasticitytensor_{e}&=\inverse{J}\pushforward{\chi}{\pbrac{\materialsecondelasticitytensor_{e}}}\\
    &=\spatialsecondelasticitytensor_{e_{1}}+\spatialsecondelasticitytensor_{e_{2}}\\
    \spatialsecondelasticitytensor_{e_{1}}&=J^{-\frac{4}{3}}\doubledotprodthree{\spatialdeviatorictensorfour}{\bar{\spatialsecondelasticitytensor}}{\transpose{\spatialdeviatorictensorfour}}\\
    \spatialsecondelasticitytensor_{e_{2}}&=\dfrac{2}{3}J^{-\frac{2}{3}}\trop\bar{\tensortwo{\sigma}}\spatialdeviatorictensorfour^{\sharp}
    -\dfrac{2}{3}\sqbrac{\tensorprod{\inverse{\tensortwo{g}}}{\cauchystresstensor_{\devop}}+\tensorprod{\tensortwo{\sigma}_{\devop}}{\inverse{\tensortwo{g}}}}\\
    \spatialsecondelasticitytensor_{p}&=\inverse{J}\pushforward{\chi}{\pbrac{\materialsecondelasticitytensor_{p}}}\\
    &=-p\pbrac{3\spatialsphericaltensorfour^{\sharp}-2\spatialsymidentitytensorfour^{\sharp}}\\
  \bar{\spatialsecondelasticitytensor}&=\inverse{J}\pushforward{\chi}{\pbrac{\bar{\materialsecondelasticitytensor}}}
  \end{split}
\end{equation}

For 3D rectangular Cartesian coordinates we have (in Voigt form)
\begin{equation}
  \spatialsymidentitytensorfour^{V}=\begin{bmatrix}
    1 & 0 & 0 & 0 & 0 & 0 \\
    0 & 1 & 0 & 0 & 0 & 0 \\
    0 & 0 & 1 & 0 & 0 & 0 \\
    0 & 0 & 0 & \frac{1}{2} & 0 & 0 \\
    0 & 0 & 0 & 0 & \frac{1}{2} & 0 \\
    0 & 0 & 0 & 0 & 0 & \frac{1}{2}
  \end{bmatrix}
\end{equation}
\begin{equation}
  \spatialsphericaltensorfour^{V}=\begin{bmatrix}
    \frac{1}{3} & \frac{1}{3} & \frac{1}{3} & 0 & 0 & 0 \\
    \frac{1}{3} & \frac{1}{3} & \frac{1}{3} & 0 & 0 & 0 \\
    \frac{1}{3} & \frac{1}{3} & \frac{1}{3} & 0 & 0 & 0 \\
    0 & 0 & 0 & 0 & 0 & 0 \\
    0 & 0 & 0 & 0 & 0 & 0 \\
    0 & 0 & 0 & 0 & 0 & 0 
  \end{bmatrix}
\end{equation}
\begin{equation}
  \spatialdeviatorictensorfour^{V}=\begin{bmatrix}
    \frac{2}{3} & -\frac{1}{3} & -\frac{1}{3} & 0 & 0 & 0 \\
    -\frac{1}{3} & \frac{2}{3} & -\frac{1}{3} & 0 & 0 & 0 \\
    -\frac{1}{3} & -\frac{1}{3} & \frac{2}{3} & 0 & 0 & 0 \\
    0 & 0 & 0 & \frac{1}{2} & 0 & 0 \\
    0 & 0 & 0 & 0 & \frac{1}{2} & 0 \\
    0 & 0 & 0 & 0 & 0 & \frac{1}{2}
  \end{bmatrix}
\end{equation}

The deviatoric stress tensor is given by
\begin{equation}
  \begin{split}
    \cauchystresstensor_{\devop}&=\bar{\cauchystresstensor}-\dfrac{1}{3}\trace{}{\bar{\cauchystresstensor}}\tensor{i}
    \\
    &=\begin{bmatrix}
    \bar{\sigma}^{11}-\frac{1}{3}\trace{}{\bar{\cauchystresstensor}} &
    \bar{\sigma}^{12} & \bar{\sigma}^{13} \\
    \bar{\sigma}^{12} &
    \bar{\sigma}^{22}-\frac{1}{3}\trace{}{\bar{\cauchystresstensor}} &
    \bar{\sigma}^{23} \\ 
    \bar{\sigma}^{13} & \bar{\sigma}^{23} &
    \bar{\sigma}^{33}-\frac{1}{3}\trace{}{\bar{\cauchystresstensor}}
    \end{bmatrix}
  \end{split}
\end{equation}

The elastic part of the elasticity tensor (in Voigt form) is given by
\begin{equation}
  \spatialsecondelasticitytensor_{e_{2}}^{V}=\frac{2}{3}J^{-\frac{2}{3}}\begin{bmatrix}
    -2\bar{\sigma}^{11} & -\bar{\sigma}^{11}-\bar{\sigma}^{22}+\frac{1}{3}\trace{}{\bar{\cauchystresstensor}}
    & -\bar{\sigma}^{11}-\bar{\sigma}^{33}+\frac{1}{3}\trace{}{\bar{\cauchystresstensor}}
    & -\bar{\sigma}^{12} & -\bar{\sigma}^{13} & -\bar{\sigma}^{23} \\
    -\bar{\sigma}^{11}-\bar{\sigma}^{22}+\frac{1}{3}\trace{}{\bar{\cauchystresstensor}}
    & -2\bar{\sigma}^{22}
    & -\bar{\sigma}^{22}-\bar{\sigma}^{33}+\frac{1}{3}\trace{}{\bar{\cauchystresstensor}}
    & -\bar{\sigma}^{12} & -\bar{\sigma}^{13} & -\bar{\sigma}^{23} \\
    -\bar{\sigma}^{11}-\bar{\sigma}^{33}+\frac{1}{3}\trace{}{\bar{\tensor{\sigma}}}
    & -\bar{\sigma}^{22}-\bar{\sigma}^{33}+\frac{1}{3}\trace{}{\bar{\tensor{\sigma}}}
    & -2\bar{\sigma}^{33} 
    & -\bar{\sigma}^{12} & -\bar{\sigma}^{13} & -\bar{\sigma}^{23} \\
    -\bar{\sigma}^{12} & -\bar{\sigma}^{12} & -\bar{\sigma}^{12}
    & \frac{1}{2}\trace{}{\bar{\tensor{\sigma}}} & 0 & 0 \\
    -\bar{\sigma}^{13} & -\bar{\sigma}^{13} & -\bar{\sigma}^{13}
    & 0 & \frac{1}{2}\trace{}{\bar{\tensor{\sigma}}} & 0 \\
    -\bar{\sigma}^{23} & -\bar{\sigma}^{23} & -\bar{\sigma}^{23}
    & 0 & 0 & \frac{1}{2}\trace{}{\bar{\tensor{\sigma}}}
  \end{bmatrix}
\end{equation}

The pressure part of the elasticity tensor (in Voigt form) is given by
\begin{equation}
  \spatialsecondelasticitytensor_{p}^{V}=\begin{bmatrix}
   p & -p & -p & 0 & 0 & 0 \\
  -p &  p & -p & 0 & 0 & 0 \\
  -p & -p &  p & 0 & 0 & 0 \\
   0 &  0 &  0 & p & 0 & 0 \\
   0 &  0 &  0 & 0 & p & 0 \\
   0 &  0 &  0 & 0 & 0 & p 
  \end{bmatrix}
\end{equation}

The linearisation of the other terms are
\begin{equation}
  \directionalderiv{}{\variation{\fnof{\bar{\Pi}_{int}}{\vectr{u}}}{\vectr{u}}+\variation{\fnof{\Pi_{P}}{\vectr{u},p}}{\vectr{u}}}{\linearisationdir{p}}=\gint{\embedmanifold{B}_{0}}{}{-\linearisationdir{p}J\inverse{\rightcauchygreentensor}}{V}
\end{equation}
or
\begin{equation}
  \directionalderiv{}{\variation{\fnof{\bar{\Pi}_{int}}{\vectr{u}}}{\vectr{u}}+\variation{\fnof{\Pi_{P}}{\vectr{u},p}}{\vectr{u}}}{\linearisationdir{p}}=\gint{\embedmanifold{B}}{}{-\linearisationdir{p}\divergence{\vectr{x}}{\variationdir{\vectr{u}}}}{v}
\end{equation}
WHERE IS THE DELTA U TERM IN THE ABOVE???

Now, the linearisation of the pressure variation with respect to pressure is
given by
\begin{equation}
  \begin{split}
    \directionalderiv{}{\variation{\fnof{\Pi_{P}}{\vectr{u},p}}{p}}{\linearisationdir{\vectr{u}}}&=\directionalderiv{}{\gint{\embedmanifold{B}_{0}}{}{-\variationdir{p}\pbrac{\fnof{J}{\vectr{u}}-1}}{V}}{\linearisationdir{\vectr{u}}}\\
    &=\gint{\embedmanifold{B}_{0}}{}{-\variationdir{p}\fnof{J}{\vectr{u}}\divergence{\vectr{x}}{\linearisationdir{\vectr{u}}}}{V} \\
    &=\gint{\embedmanifold{B}}{}{-\variationdir{p}\divergence{\vectr{x}}{\linearisationdir{\vectr{u}}}}{v}
  \end{split}
\end{equation}
and
\begin{equation}
  \begin{split}
    \directionalderiv{}{\variation{\fnof{\Pi_{P}}{\vectr{u},p}}{p}}{\linearisationdir{p}}&=\directionalderiv{}{\gint{\embedmanifold{B}_{0}}{}{-\variationdir{p}\pbrac{\fnof{J}{\vectr{u}}-1}}{V}}{\linearisationdir{p}}\\
    &=\directionalderiv{}{\gint{\embedmanifold{B}}{}{-\variationdir{p}\pbrac{1-\fnof{\inverse{J}}{\vectr{u}}}}{v}}{\linearisationdir{p}}\\
    &=0
  \end{split}
\end{equation}

\subsection{Hu-Washizu}

ADD IN THREE FIELD FORMULATION FOR COMPRESSIBLE ELASTICITY

\section{Finite Element Formulation}

\subsection{Residual}

The residual statement is given by
\begin{equation}
  \vectr{r}=\variation{\fnof{\Pi_{PE}}{\vectr{u}}}{\vectr{u}}=\vectr{0}
\end{equation}
for the potential energy variation and by 
\begin{equation}  
  \vectr{r}=\begin{bmatrix}
  \variation{\fnof{\Pi_{HR}}{\vectr{u},p}}{\vectr{u}} \\
    \variation{\fnof{\Pi_{HR}}{\vectr{u},p}}{p}
  \end{bmatrix}=\vectr{0}
\end{equation}
for the Hellinger-Reissner variation.

Firstly, consider the displacement, $\vectr{u}$, between the
reference/material coordinates, $\vectr{X}$, and the current/spatial
coordinates, $\vectr{z}$. The displacement is defined by
\begin{equation}
  \vectr{u}=\vectr{z}-\vectr{X}
\end{equation}
and thus
\begin{equation}
  \begin{split}
    \variationdir{\vectr{u}} &=\variationdir{\pbrac{\vectr{z} -\vectr{X}}} \\
    &=\variationdir{\vectr{z}}-\variationdir{\vectr{X}} \\
    &=\variationdir{\vectr{z}}
  \end{split}
\end{equation}
as there is no variation in the original reference configuration $\vectr{X}$. 

In component form, we have
\begin{equation}
  \gint{\embedmanifold{B}}{}{\sigma^{ij}\covarderiv{\variationdir{u_{j}}}{i}}{v}=
  \gint{\embedmanifold{B}}{}{b^{j}\variationdir{u_{j}}}{v}+
  \gint{\boundary{\embedmanifold{B}_{P}}}{}{P_{ext}n^{j}\variationdir{u_{j}}}{\covectr{a}}
  \gint{\boundary{\embedmanifold{B}_{t}}}{}{t^{j}\variationdir{u_{j}}}{\covectr{a}}
\end{equation}
and, if there an incompressibility constraint
\begin{equation}
  \gint{\embedmanifold{B}}{}{-\variationdir{p}\pbrac{J-1}}{v}=0
\end{equation}
where $\sigma^{ij}=\sigma_{\devop}^{ij}+\sigma_{\sphop}^{ij}$.

If we now substitute $\delta\vectr{u}=\delta\vectr{z}$ and convert the left
hand side of the virtual work statement from an integral with respect to
spatial coordinates to an integral with respect to $\vectr{\xi}$ coordinates we obtain

\begin{equation}
  \begin{split}
    \gint{\embedmanifold{B}}{}{\sigma^{ij}\pbrac{\delby{\delta
          u_{j}}{x^{i}}-\christoffel{k}{j}{i}\delta u_{k}}}{v}
    &= \gint{\embedmanifold{B}}{}{\sigma^{ij}\pbrac{\delby{\delta
          z_{j}}{x^{i}}-\christoffel{k}{j}{i}\delta z_{k}}}{v} \\
    &= \gint{\vectr{0}}{\vectr{1}}{\fnof{\sigma^{ij}}{\vectr{\xi}}\pbrac{\delby{\xi_{l}}{x^{i}}\delby{\delta
          \fnof{z_{j}}{\vectr{\xi}}}{\xi^{l}}-\christoffel{k}{j}{i}\delta\fnof{z_{k}}{\vect{\xi}}}\fnof{J_{\embedmanifold{B}}}{\vectr{\xi}}}{\vectr{\xi}}
  \end{split}
\end{equation}

Note that in rectangular cartesian coordinates $\christoffel{k}{j}{i}=0$ 
$\forall i,j,k$. In addition it is not necessary to transform either the
Cauchy stress tensor or gradient of the virtual displacements so that the
components are with respect to $\vectr{\xi}$ coordinates. What is important is that
the stress and displacement are with respect to the same coordinate
system. Because the gradient of $\delta \vectr{z}$ is with respect to $\vectr{x}$ coordinates
then $\tensor{\sigma}$ needs to be with respect to $\vectr{x}$ coordinates. As there
is no coordinate transformations the Christoffel symbols are all zero and can
be dropped.

The integral on the left hand side of the virtual work statement is
\begin{equation}
  \begin{split}
    \gint{\embedmanifold{B}}{}{\sigma^{ij}\covarderiv{\variationdir{u_{j}}}{i}}{v}
    &=
    \gint{\embedmanifold{B}}{}{\sigma^{ij}\pbrac{\partialderiv{\variationdir{u_{j}}}{i}
        -\christoffel{k}{j}{i}\variationdir{u_{k}}}}{v} \\
    &=
    \gint{\embedmanifold{B}}{}{\sigma^{ij}\delby{\variationdir{u_{j}}}{x^{i}}}{v} \\
    &=\gint{\vectr{0}}{\vectr{1}}{\fnof{\sigma^{ij}}{\vectr{\xi}}
    \delby{\xi_{l}}{x^{i}}\delby{\variationdir{\fnof{z_{j}}{\vectr{\xi}}}}{\xi^{l}}
    \fnof{J_{\embedmanifold{B}}}{\vectr{\xi}}}{\vectr{\xi}}
  \end{split}
\end{equation}

The right hand side of the virtual work statement is
\begin{equation}
  \begin{split}
    \gint{\embedmanifold{B}}{}{b^{j}\variationdir{u_{j}}}{v}+
    \gint{\boundary{\embedmanifold{B}_{P}}}{}{P_{ext}n^{j}\variationdir{u_{j}}}{\covectr{a}}
    \gint{\boundary{\embedmanifold{B}_{t}}}{}{t^{j}\variationdir{u_{j}}}{\covectr{a}}
    &= \gint{\embedmanifold{B}}{}{b^{j}\variationdir{z_{j}}}{v}+
    \gint{\boundary{\embedmanifold{B}_{P}}}{}{P_{ext}n^{j}\variationdir{z_{j}}}{\covectr{a}} +
    \gint{\boundary{\embedmanifold{B}_{t}}}{}{t^{j}\variationdir{z_{j}}}{\covectr{a}} \\
    &= \gint{\vectr{0}}{\vectr{1}}{\fnof{b^{j}}{\vectr{\xi}}\variationdir{
      \fnof{z_{j}}{\vectr{\xi}}}\fnof{J_{\embedmanifold{B}}}{\vectr{\xi}}}{\vectr{\xi}}\\
    &\quad+\gint{\vectr{0}}{\vectr{1}}{\fnof{P_{ext}}{\vectr{\xi}}\fnof{n^{j}}{\vectr{\xi}}\variationdir{
      \fnof{z_{j}}{\vectr{\xi}}}\fnof{J_{\embedmanifold{B}}}{\vectr{\xi}}}{\vectr{\xi}}\\
    &\quad+\gint{\vectr{0}}{\vectr{1}}{\fnof{t^{j}}{\vectr{\xi}}\variationdir{
      \fnof{z_{j}}{\vectr{\xi}}}\fnof{J_{\embedmanifold{B}}}{\vectr{\xi}}}{\vectr{\xi}} 
    \end{split}
\end{equation}
where $P_{ext}$ is the applied surface pressure.

NEED TO CHECK POSITION OF INDICIES. SHOULD J BE AN UPPER INDEX? NORMAL SHOULD
BE LOWER I.E., A COVECTOR. 

If we have an incompressibility constraint we also have
\begin{equation}
  \begin{split}
    \gint{\embedmanifold{B}}{}{-\variationdir{p}\pbrac{J-1}}{v}&=
    \gint{\vectr{0}}{\vectr{1}}{\variationdir{\fnof{p}{\vectr{\xi}}}\pbrac{
        \fnof{J}{\vectr{\xi}}-1}\fnof{J_{\embedmanifold{B}}}{\vectr{\xi}}}{\vectr{\xi}}\\
    &=0
\  \end{split}
\end{equation}

If we now use basis functions to interpolate the virtual displacements
\begin{equation}
  \variationdir{\fnof{z_{j}}{\vectr{\xi}}} = \idxgbfn{\pbrac{j}}{m}{\alpha}{\vectr{\xi}}\variationdir{z_{j,\alpha}^{m}}\gsf{m}{\alpha}
\end{equation}
and hydrostatic pressure
\begin{equation}
  \variationdir{\fnof{p}{\vectr{\xi}}} = \idxgbfn{\pbrac{N+1}}{m}{\alpha}{\vectr{\xi}}\variationdir{p_{,\alpha}^{m}}\gsf{m}{\alpha}
\end{equation}
which, assuming rectangular cartesian coordinates, gives for the left hand side integral
\begin{equation}
  \begin{split}
    \gint{\vectr{0}}{\vectr{1}}{\fnof{\sigma^{ij}}{\vectr{\xi}}\delby{\xi_{l}}{x^{i}}\delby{\variationdir{
          \fnof{z_{j}}{\vectr{\xi}}}}{\xi^{l}}\fnof{J_{\embedmanifold{B}}}{\vectr{\xi}}}{\vectr{\xi}}
    &= \gint{\vectr{0}}{\vectr{1}}{\fnof{\sigma^{ij}}{\vectr{\xi}}\delby{\xi_{l}}{x^{i}}\delby{
          \pbrac{\idxgbfn{\pbrac{j}}{m}{\alpha}{\vectr{\xi}}\variationdir{z_{j,\alpha}^{m}}\gsf{m}{\alpha}}}{\xi^{l}}
      \fnof{J_{\embedmanifold{B}}}{\vectr{\xi}}}{\vectr{\xi}} \\
    &= \gint{\vectr{0}}{\vectr{1}}{\fnof{\sigma^{ij}}{\vectr{\xi}}\delby{\xi_{l}}{x^{i}}\delby{
          \idxgbfn{\pbrac{j}}{m}{\alpha}{\vectr{\xi}}}{\xi^{l}}\variationdir{z_{j,\alpha}^{m}}\gsf{m}{\alpha}
      \fnof{J_{\embedmanifold{B}}}{\vectr{\xi}}}{\vectr{\xi}} \\
    &= \variationdir{z_{j,\alpha}^{m}}\gsf{m}{\alpha}
    \gint{\vectr{0}}{\vectr{1}}{\fnof{\sigma^{ij}}{\vectr{\xi}}\delby{\xi_{l}}{x^{i}}\delby{
          \idxgbfn{\pbrac{j}}{m}{\alpha}{\vectr{\xi}}}{\xi^{l}}
      \fnof{J_{\embedmanifold{B}}}{\vectr{\xi}}}{\vectr{\xi}} 
  \end{split}
\end{equation}
and for the first integral on the right hand side integral we have
\begin{equation}
  \begin{split}
    \gint{\vectr{0}}{\vectr{1}}{\fnof{b^{j}}{\vectr{\xi}}\variationdir{
        \fnof{z_{j}}{\vectr{\xi}}}\fnof{J_{\embedmanifold{B}}}{\vectr{\xi}}}{\vectr{\xi}}
    &= \gint{\vectr{0}}{\vectr{1}}{\fnof{b^{j}}{\vectr{\xi}}
      \idxgbfn{\pbrac{j}}{m}{\alpha}{\vectr{\xi}}\variationdir{z_{j,\alpha}^{m}}\gsf{m}{\alpha}
      \fnof{J_{\embedmanifold{B}}}{\vectr{\xi}}}{\vectr{\xi}} \\ &=
    \variationdir{z_{j,\alpha}^{m}}\gsf{m}{\alpha}\gint{\vectr{0}}{\vectr{1}}{\fnof{b^{j}}{\vectr{\xi}}
      \idxgbfn{\pbrac{j}}{m}{\alpha}{\vectr{\xi}}
      \fnof{J_{\embedmanifold{B}}}{\vectr{\xi}}}{\vectr{\xi}}
  \end{split}
\end{equation}
and for the second integral on the right hand side we have
\begin{equation}
  \begin{split}
    \gint{\vectr{0}}{\vectr{1}}{\fnof{P_{ext}}{\vectr{\xi}}\fnof{n^{j}}{\vectr{\xi}}\variationdir{
      \fnof{z_{j}}{\vectr{\xi}}}\fnof{J_{\embedmanifold{B}}}{\vectr{\xi}}}{\vectr{\xi}}
    &= \gint{\vectr{0}}{\vectr{1}}{\fnof{P_{ext}}{\vectr{\xi}}\fnof{n^{j}}{\vectr{\xi}}
      \idxgbfn{\pbrac{j}}{m}{\alpha}{\vectr{\xi}}\variationdir{z_{j,\alpha}^{m}}\gsf{m}{\alpha}
      \fnof{J_{\embedmanifold{B}}}{\vectr{\xi}}}{\vectr{\xi}}
    \\
    &= \variationdir{z_{j,\alpha}^{m}}\gsf{m}{\alpha}\gint{\vectr{0}}{\vectr{1}}{\fnof{P}{\vectr{\xi}}\fnof{n^{j}}{\vectr{\xi}}
      \idxgbfn{\pbrac{j}}{m}{\alpha}{\vectr{\xi}}
      \fnof{J_{\embedmanifold{B}}}{\vectr{\xi}}}{\vectr{\xi}}
  \end{split}
\end{equation}
and for the third integral on the right hand side we have
\begin{equation}
  \begin{split}
    \gint{\vectr{0}}{\vectr{1}}{\fnof{t^{j}}{\vectr{\xi}}\variationdir{
      \fnof{z_{j}}{\vectr{\xi}}}\fnof{J_{\embedmanifold{B}}}{\vectr{\xi}}}{\vectr{\xi}}
    &= \gint{\vectr{0}}{\vectr{1}}{\fnof{t^{j}}{\vectr{\xi}}
      \idxgbfn{\pbrac{j}}{m}{\alpha}{\vectr{\xi}}\variationdir{
      z_{j,\alpha}^{m}}\gsf{m}{\alpha}
      \fnof{J_{\embedmanifold{B}}}{\vectr{\xi}}}{\vectr{\xi}} \\
    &=
    \variationdir{z_{j,\alpha}^{m}}\gsf{m}{\alpha}\gint{\vectr{0}}{\vectr{1}}{\fnof{t^{j}}{\vectr{\xi}}
      \idxgbfn{\pbrac{j}}{m}{\alpha}{\vectr{\xi}}
      \fnof{J_{\embedmanifold{B}}}{\vectr{\xi}}}{\vectr{\xi}}
  \end{split}
\end{equation}

This can be formulated as
\begin{equation}
  \variationdir{z_{j,\alpha}^{m}}r_{m}^{j\alpha}=0
\end{equation}
where the residual vector is thus given by
\begin{multline}
  r_{m}^{j\alpha}=\gsf{m}{\alpha}\left(
    \gint{\vectr{0}}{\vectr{1}}{\pbrac{\fnof{\sigma^{ij}}{\vectr{\xi}}\delby{\xi_{l}}{x^{i}}\delby{
          \idxgbfn{\pbrac{j}}{m}{\alpha}{\vectr{\xi}}}{\xi^{l}}+\densitysymbol\pbrac{
        \fnof{a^{j}}{\vectr{\xi}}-\fnof{b^{j}}{\vectr{\xi}}}\idxgbfn{\pbrac{j}}{m}{\alpha}{\vectr{\xi}}}
      \fnof{J_{\embedmanifold{B}}}{\vectr{\xi}}}{\vectr{\xi}}\right. \\
    \left.-\gint{\vectr{0}}{\vectr{1}}{\fnof{P_{ext}}{\vectr{\xi}}\fnof{n^{j}}{\vectr{\xi}}
      \idxgbfn{j}{m}{\alpha}{\vectr{\xi}}
      \fnof{J_{\embedmanifold{B}}}{\vectr{\xi}}}{\vectr{\xi}}\right)
\end{multline}

Now, as the virtual displacements are arbitrary we have the residual statement
\begin{equation}
  r_{m}^{j\alpha}=0
\end{equation}

If we have an incompressible material we also have the additional residual
equation
\begin{equation}
  \variationdir{p_{\pbrac{N+1}\alpha}^{m}}r_{m}^{\pbrac{N+1}\alpha}=0
\end{equation}
where 
\begin{equation}
  \begin{split}
    r_{m}^{\pbrac{N+1}\alpha}&=\gint{\vectr{0}}{\vectr{1}}{\pbrac{\fnof{J}{\vectr{\xi}}-1}
        \idxgbfn{\pbrac{N+1}}{m}{\alpha}{\vectr{\xi}}\gsf{m}{\alpha}\fnof{J_{\embedmanifold{B}}}{\vectr{\xi}}}{\vectr{\xi}}
    \\
    &=\gsf{m}{\alpha}\gint{\vectr{0}}{\vectr{1}}{\pbrac{\fnof{J}{\vectr{\xi}} -
        1}\idxgbfn{\pbrac{N+1}}{m}{\alpha}{\vectr{\xi}}\fnof{J_{\embedmanifold{B}}}{\vectr{\xi}}}{\vectr{\xi}}
  \end{split}
\end{equation}
where $N$ is the number of dimensions.

\subsection{Jacobian}

From the linearisation statements we can form a number of sets of linear
relations. For a potential energy variation we have
\begin{equation}
  \directionalderiv{}{\variation{\fnof{\Pi_{PE}}{\vectr{u}}}{\vectr{u}}}{\linearisationdir{\vectr{u}}}=\transpose{
    \begin{bmatrix}\variationdir{\vectr{u}}\end{bmatrix}}\begin{bmatrix}\matr{K}^{PE}_{\vectr{u}\vectr{u}}\end{bmatrix}
  \begin{bmatrix}\linearisationdir{\vectr{u}}\end{bmatrix}
\end{equation}
where
\begin{equation}
  \matr{K}^{PE}_{\vectr{u}\vectr{u}}=K^{ij\alpha\beta}_{mn}=\gsf{m}{\alpha}\gsf{n}{\beta}\gint{\vectr{0}}{\vectr{1}}{\pbrac{\delby{\idxgbfn{\pbrac{i}}{m}{\alpha}{\vectr{\xi}}}{\xi^{r}}\delby{\xi^{r}}{x^{k}}\sqbrac{\contrakronecker{i}{j}\fnof{\sigma^{kl}}{\vect{\xi}}+\fnof{c^{ikjl}}{\vect{\xi}}}\delby{\idxgbfn{\pbrac{j}}{n}{\beta}{\vect{\xi}}}{\xi^{s}}\delby{\xi^{s}}{x^{l}}}\fnof{J_{\embedmanifold{B}}}{\vectr{\xi}}}{\vect{\xi}}
\end{equation}

ADD IN THE STIFFNESS MATRICES FROM BODY FORCE, EXTERNAL PRESSURE ETC. 

For a Hellinger-Reisner variation we have
\begin{multline}
  \begin{bmatrix}
    \directionalderiv{}{\variation{\fnof{\Pi_{HR}}{\vectr{u},p}}{\vectr{u}}}{\linearisationdir{\vectr{u}}}
    + \directionalderiv{}{\variation{\fnof{\Pi_{HR}}{\vectr{u},p}}{\vectr{u}}}{\linearisationdir{p}} \\
    \directionalderiv{}{\variation{\fnof{\Pi_{HR}}{\vectr{u},p}}{p}}{\linearisationdir{\vectr{u}}}
    + \directionalderiv{}{\variation{\fnof{\Pi_{HR}}{\vectr{u},p}}{p}}{\linearisationdir{p}}
  \end{bmatrix} = \\
  \transpose{\begin{bmatrix}
    \variationdir{\vectr{u}} & \variationdir{p}
  \end{bmatrix}}
  \begin{bmatrix}
    \matr{K}^{HR}_{\vectr{u}\vectr{u}} & \matr{K}^{HR}_{\vectr{u}p} \\
    \matr{K}^{HR}_{p\vectr{u}} & \matr{K}^{HR}_{pp}
  \end{bmatrix}
  \begin{bmatrix}
    \linearisationdir{\vectr{u}} \\
    \linearisationdir{p}
  \end{bmatrix}
\end{multline}

Here
\begin{equation}
  \matr{K}^{HR}_{\vectr{u}p}=K^{i\pbrac{N+1}\alpha\beta}_{mn}=-\gsf{m}{\alpha}\gsf{n}{\beta}\gint{\vectr{0}}{\vectr{1}}{\idxgbfn{\pbrac{N+1}}{n}{\beta}{\vectr{\xi}}\spatialmetrictensorsymbol^{ij}\delby{\idxgbfn{\pbrac{i}}{m}{\alpha}{\vectr{\xi}}}{\xi^{k}}\delby{\xi^{k}}{x^{j}}\fnof{J_{\embedmanifold{B}}}{\vectr{\xi}}}{\vectr{\xi}}
\end{equation}
and
\begin{equation}
  \matr{K}^{HR}_{p\vectr{u}}=K^{\pbrac{N+1}j\alpha\beta}_{mn}=-\gsf{m}{\alpha}\gsf{n}{\beta}\gint{\vectr{0}}{\vectr{1}}{\idxgbfn{\pbrac{N+1}}{m}{\alpha}{\vectr{\xi}}\spatialmetrictensorsymbol^{ij}\delby{\idxgbfn{\pbrac{j}}{n}{\beta}{\vectr{\xi}}}{\xi^{k}}\delby{\xi^{k}}{x^{i}}\fnof{J_{\embedmanifold{B}}}{\vectr{\xi}}}{\vectr{\xi}}
\end{equation}

Note that $\matr{K}^{HR}_{\vectr{u}p}=\transpose{\matr{K}^{HR}_{p\vectr{u}}}$.

For incompressible problems
\begin{equation}
  \matr{K}^{HR}_{pp}=K^{\pbrac{N+1}\pbrac{N+1}\alpha\beta}_{mn}=\matr{0}
\end{equation}

\section{Lower Dimensional Elasticity}
\label{sec:FiniteMechLowerDimensions}

\section{Constitutive Laws}
\label{sec:FiniteMechConstitutiveLaws}

\subsection{St Venant-Kirchoff}

The relationships between the elastic constants are given in \Tabref{tab:RelationshipBetweenElasticConstants}.

TODO: Fill in table more.

\begin{table}[htb] \centering
  \begin{tabular}{|c|c|c|c|c|c|} \hline
    & $\lambda$ & $\mu/G$ & $E$ & $\nu$ & $K$ \\ \hline \hline
    $\pbrac{\lambda,\mu}$ & - & - &
    $\dfrac{\mu\pbrac{3\lambda+2\mu}}{\lambda+\mu}$ & $\dfrac{\lambda}{2\pbrac{\lambda+\mu}}$ & $\dfrac{3\lambda+2\mu}{3}$ \\ \hline
    $\pbrac{E,\nu}$ & $\dfrac{\nu E}{\pbrac{1+\nu}\pbrac{1-2\nu}}$ & $\dfrac{E}{2\pbrac{1-\nu}}$ &
    - & - & $\dfrac{E}{3\pbrac{1-2\nu}}$ \\ \hline
    $\pbrac{\mu/G,K}$ & $\dfrac{3K-2\mu}{3}$ & - & $\dfrac{9K\mu}{3K+\mu}$ &
    $\dfrac{3K-2\mu}{2\pbrac{3K+\mu}}$ & - \\ \hline
    $\pbrac{\lambda,E}$ & - & $\dfrac{E-3\lambda+c}{4}$ & - &
    $\dfrac{2\lambda}{E+\lambda+c}$ &
    $\dfrac{E+3\lambda+c}{6}$ \\ \hline
    $\pbrac{\lambda,\nu}$ & - & $\dfrac{\lambda\pbrac{1-2\nu}}{2\nu}$ & $\dfrac{\lambda\pbrac{1+\nu}\pbrac{1-2\nu}}{\nu}$ & - & $\dfrac{\lambda\pbrac{1+\nu}}{3\nu}$ \\ \hline
  \end{tabular}
  \caption{Reltionships between elastic constants. $\lambda$ is the first
    Lam\'e constant, $\mu$ is the second Lam\'e constant, $E$ is Young's
    modulus, $\nu$ is Poisson's ratio, $G$ is the shear modulus and $K$ is the
    bulk modulus. $c=\sqrt{E^{2}+9\lambda^{2}+2E\lambda}$}.
  \label{tab:RelationshipBetweenElasticConstants}
\end{table}


\subsection{Generalised}

A number of phenomelogical hyperelastic models are off the form
\begin{equation}
  \fnof{\bar{W}}{\bar{I}_{1},\bar{I}_{2},J}=\gsum{i+j=1}{N}{c_{ij}\pbrac{\bar{I}_{1}-n}^{i}\pbrac{\bar{I}_{2}-2}^{j}}+\gsum{k=1}{N}{\dfrac{1}{d_{k}}\pbrac{J-1}^{2k}}
\end{equation}
where $N$ is the order of the model, $n$ is the number of dimensions, and $c_{ij}$ and $d_{k}$ are material constants that can be determined by fitting to experimental data.

\subsection{Neo-Hookean}

The simplest constitutive non-linear model is the Neo-Hookean model. It corresponds to a $N=1$ order model and is given by
\begin{equation}
   \fnof{\bar{W}}{\bar{I}_{1},J}=c_{10}\pbrac{\bar{I}_{1}-n}+\dfrac{1}{d_{1}}\pbrac{J-1}^{2}
\end{equation}

Here the equivalent/initial shear modulus is given by $\frac{G_{0}}{2}=c_{10}$ and the equivalent bulk modulus is given by $\frac{K_{0}}{2}=\frac{1}{d_{1}}$.

Note that an equivalent Poisson's ratio can be calculated from the initial shear and bulk modulii by
\begin{equation}
  \nu=\frac{3K_{0}-2G_{0}}{6K_{0}+2G_{0}}
\end{equation}

This means that
\begin{equation}
  d_{1}=\dfrac{2}{K_{0}}=\dfrac{3\pbrac{1-2\nu}}{G_{0}\pbrac{1+\nu}}
\end{equation}


The second Piola Kirchoff tensor is thus
\begin{equation}
  \begin{split}
    \bar{\secondpiolakirchoffstresstensor}&=2\delby{\fnof{\bar{W}}{\bar{I}_{1}}}{\bar{\rightcauchygreentensor}}\\
      &=2\delby{\fnof{\bar{W}}{\bar{I}_{1}}}{\bar{I}_{1}}\delby{\bar{I}_{1}}{\bar{\rightcauchygreentensor}}\\
      &=2c_{1}\begin{bmatrix}
        1 & 0 \\
        0 & 1
      \end{bmatrix} \\
      &=\begin{bmatrix}
      2c_{1} & 0 \\
      0 & 2c_{1}
      \end{bmatrix}
  \end{split}
\end{equation}
or, in terms of the Lagrange strain tensor
\begin{equation}
  \begin{split}
    \bar{\secondpiolakirchoffstresstensor}&=\delby{\fnof{\bar{W}}{\bar{I}_{1}}}{\bar{\greenlagrangestraintensor}}\\
    &=\begin{bmatrix}
    c_{1} & 0 \\
    0 & c_{1}
    \end{bmatrix}
  \end{split}
\end{equation}

The second material elasticity tensor is given by
\begin{equation}
  \bar{\materialsecondelasticitytensor}=2\delby{\bar{\secondpiolakirchoffstresstensor}}{\bar{\rightcauchygreentensor}}=4\deltwosqby{\fnof{\bar{W}}{\bar{I}_{1}}}{\bar{\rightcauchygreentensor}}\\
\end{equation}
and thus
\begin{equation}
  \begin{split}
    \bar{C}^{1111}&=\delby{\bar{S}^{11}}{\bar{C}^{11}}=0\\
    \bar{C}^{1122}&=\delby{\bar{S}^{11}}{\bar{C}^{22}}=4c_{2}\\
    \bar{C}^{1112}&=\delby{\bar{S}^{11}}{\bar{C}^{22}}=0\\
    \bar{C}^{2211}&=\delby{\bar{S}^{22}}{\bar{C}^{11}}=4c_{2}\\
    \bar{C}^{2222}&=\delby{\bar{S}^{22}}{\bar{C}^{22}}=0\\
    \bar{C}^{2212}&=\delby{\bar{S}^{22}}{\bar{C}^{22}}=0\\
    \bar{C}^{1211}&=\delby{\bar{S}^{22}}{\bar{C}^{11}}=4c_{2}\\
    \bar{C}^{1222}&=\delby{\bar{S}^{22}}{\bar{C}^{22}}=0\\
    \bar{C}^{1212}&=\delby{\bar{S}^{22}}{\bar{C}^{22}}=-4c_{2}\\
  \end{split}
\end{equation}
or, in Voigt form
\begin{equation}
  \bar{\materialsecondelasticitytensor}^{V}=\begin{bmatrix}
  0 & 4c_{2} & 0 \\
  4c_{2} & 0 & 0 \\
  0 & 0 & -4c_{2}
  \end{bmatrix}
\end{equation}

For three dimensions the strain energy function is given by
\begin{equation}
  \fnof{\bar{W}}{\bar{I}_{1},\bar{I}_{2}}=c_{1}\pbrac{\bar{I}_{1}-3}+c_{2}\pbrac{\bar{I}_{2}-3}
\end{equation}
where $\bar{I}_{1}$ and $\bar{I}_{2}$ are the invariants of either $\bar{\rightcauchygreentensor}$ or $\bar{\greenlagrangestraintensor}$.

The second Piola Kirchoff tensor is thus
\begin{equation}
  \begin{split}
    \bar{\secondpiolakirchoffstresstensor}&=2\delby{\fnof{\bar{W}}{\bar{I}_{1},\bar{I}_{2}}}{\bar{\rightcauchygreentensor}}\\
    &=2\pbrac{\delby{\fnof{\bar{W}}{\bar{I}_{1},\bar{I}_{2}}}{\bar{I}_{1}}\delby{\bar{I}_{1}}{\bar{\rightcauchygreentensor}}+\delby{\fnof{\bar{W}}{\bar{I}_{1},\bar{I}_{2}}}{\bar{I}_{2}}\delby{\bar{I}_{2}}{\bar{\rightcauchygreentensor}}}\\
    &=2\pbrac{c_{1}\begin{bmatrix}
        1 & 0 & 0 \\
        0 & 1 & 0 \\
        0 & 0 & 1
      \end{bmatrix}+c_{2}\begin{bmatrix}
        \bar{C}_{22}+\bar{C}_{33} & -\bar{C}_{21} & \bar{C}_{31} \\
        -\bar{C}_{12} & \bar{C}_{11}+\bar{C}_{33} & -\bar{C}_{32} \\
        -\bar{C}_{13} & -\bar{C}_{23} & \bar{C}_{11}+\bar{C}_{22}
    \end{bmatrix}} \\
    &=\begin{bmatrix}
    2c_{1}+2c_{2}\pbrac{\bar{C}_{22}+\bar{C}_{33}} & -2c_{2}\bar{C}_{21} &
    -2c_{2}\bar{C}_{31} \\
    -2c_{2}\bar{C}_{12} & 2c_{1}+2c_{2}\pbrac{\bar{C}_{11}+\bar{C}_{33}} & -2c_{2}\bar{C}_{32} \\
    -2c_{2}\bar{C}_{13} & -2c_{2}\bar{C}_{23} & 2c_{1}+2c_{2}\pbrac{\bar{C}_{11}+\bar{C}_{22}} 
    \end{bmatrix}
  \end{split}
\end{equation}
or, in terms of the Lagrange strain tensor
\begin{equation}
  \begin{split}
    \bar{\secondpiolakirchoffstresstensor}&=\delby{\fnof{\bar{W}}{\bar{I}_{1},\bar{I}_{2}}}{\bar{\greenlagrangestraintensor}}\\
    &=\begin{bmatrix}
    c_{1}+c_{2}\pbrac{\bar{E}_{22}+\bar{E}_{33}} & -c_{2}\bar{E}_{21} & -c_{2}\bar{E}_{31} \\
    -c_{2}\bar{E}_{12} & c_{1}+c_{2}\pbrac{\bar{E}_{11}+\bar{E}_{33}} & -c_{2}\bar{E}_{32} \\
    -c_{2}\bar{E}_{13} & -c_{2}\bar{E}_{23} & c_{1}+c_{2}\pbrac{\bar{E}_{11}+\bar{E}_{22}}
    \end{bmatrix}
  \end{split}
\end{equation}

The second material elasticity tensor is given by
\begin{equation}
  \bar{\materialsecondelasticitytensor}=2\delby{\bar{\secondpiolakirchoffstresstensor}}{\bar{\rightcauchygreentensor}}=4\deltwosqby{\fnof{\bar{W}}{\bar{I}_{1},\bar{I}_{2}}}{\bar{\rightcauchygreentensor}}\\
\end{equation}
and thus
\begin{equation}
  \begin{split}
    \bar{C}^{1111}&=\delby{\bar{S}^{11}}{\bar{C}^{11}}=0\\
    \bar{C}^{1122}&=\delby{\bar{S}^{11}}{\bar{C}^{22}}=4c_{2}\\
    \bar{C}^{1133}&=\delby{\bar{S}^{11}}{\bar{C}^{33}}=4c_{2}\\
    \bar{C}^{1123}&=\delby{\bar{S}^{11}}{\bar{C}^{23}}=0\\
    \bar{C}^{1113}&=\delby{\bar{S}^{11}}{\bar{C}^{13}}=0\\
    \bar{C}^{1112}&=\delby{\bar{S}^{11}}{\bar{C}^{12}}=0\\
    \bar{C}^{2211}&=\delby{\bar{S}^{22}}{\bar{C}^{11}}=4c_{2}\\
    \bar{C}^{2222}&=\delby{\bar{S}^{22}}{\bar{C}^{22}}=0\\
    \bar{C}^{2233}&=\delby{\bar{S}^{22}}{\bar{C}^{33}}=4c_{2}\\
    \bar{C}^{2223}&=\delby{\bar{S}^{22}}{\bar{C}^{23}}=0\\
    \bar{C}^{2213}&=\delby{\bar{S}^{22}}{\bar{C}^{13}}=0\\
    \bar{C}^{2212}&=\delby{\bar{S}^{22}}{\bar{C}^{12}}=0\\
    \bar{C}^{3311}&=\delby{\bar{S}^{33}}{\bar{C}^{11}}=4c_{2}\\
    \bar{C}^{3322}&=\delby{\bar{S}^{33}}{\bar{C}^{22}}=4c_{2}\\
    \bar{C}^{3333}&=\delby{\bar{S}^{33}}{\bar{C}^{33}}=0\\
    \bar{C}^{3323}&=\delby{\bar{S}^{33}}{\bar{C}^{23}}=0\\
    \bar{C}^{3313}&=\delby{\bar{S}^{33}}{\bar{C}^{13}}=0\\
    \bar{C}^{3312}&=\delby{\bar{S}^{33}}{\bar{C}^{12}}=0\\
    \bar{C}^{2311}&=\delby{\bar{S}^{23}}{\bar{C}^{11}}=0\\
    \bar{C}^{2322}&=\delby{\bar{S}^{23}}{\bar{C}^{22}}=0\\
    \bar{C}^{2333}&=\delby{\bar{S}^{23}}{\bar{C}^{33}}=0\\
    \bar{C}^{2323}&=\delby{\bar{S}^{23}}{\bar{C}^{23}}=-4c_{2}\\
    \bar{C}^{2313}&=\delby{\bar{S}^{23}}{\bar{C}^{13}}=0\\
    \bar{C}^{2312}&=\delby{\bar{S}^{23}}{\bar{C}^{12}}=0\\
    \bar{C}^{1311}&=\delby{\bar{S}^{13}}{\bar{C}^{11}}=0\\
    \bar{C}^{1322}&=\delby{\bar{S}^{13}}{\bar{C}^{22}}=0\\
    \bar{C}^{1333}&=\delby{\bar{S}^{13}}{\bar{C}^{33}}=0\\
    \bar{C}^{1323}&=\delby{\bar{S}^{13}}{\bar{C}^{23}}=0\\
    \bar{C}^{1313}&=\delby{\bar{S}^{13}}{\bar{C}^{13}}=-4c_{2}\\
    \bar{C}^{1312}&=\delby{\bar{S}^{13}}{\bar{C}^{12}}=0\\
    \bar{C}^{1211}&=\delby{\bar{S}^{12}}{\bar{C}^{11}}=0\\
    \bar{C}^{1222}&=\delby{\bar{S}^{12}}{\bar{C}^{22}}=0\\
    \bar{C}^{1233}&=\delby{\bar{S}^{12}}{\bar{C}^{33}}=0\\
    \bar{C}^{1223}&=\delby{\bar{S}^{12}}{\bar{C}^{23}}=0\\
    \bar{C}^{1213}&=\delby{\bar{S}^{12}}{\bar{C}^{13}}=0\\
    \bar{C}^{1212}&=\delby{\bar{S}^{12}}{\bar{C}^{12}}=-4c_{2}\\
  \end{split}
\end{equation}
or, in Voigt form
\begin{equation}
  \bar{\materialsecondelasticitytensor}^{V}=\begin{bmatrix}
  0 & 4c_{2} & 4c_{2} & 0 & 0 & 0 \\
  4c_{2} & 0 & 4c_{2} & 0 & 0 & 0 \\
  4c_{2} & 4c_{2} & 0 & 0 & 0 & 0 \\
  0     & 0      & 0 & -4c_{2} & 0 & 0 \\
  0     & 0      & 0 & 0      & -4c_{2} & 0 \\
  0     & 0      & 0 & 0      & 0      & -4c_{2} 
  \end{bmatrix}
\end{equation}


  
\subsection{Mooney-Rivlin}

As an example consider a Mooney-Rivlin material.

For two dimensions the strain energy function is given by
\begin{equation}
  \fnof{\bar{W}}{\bar{I}_{1},\bar{I}_{2}}=c_{1}\pbrac{\bar{I}_{1}-2}+c_{2}\pbrac{\bar{I}_{2}-2}
\end{equation}
where $\bar{I}_{1}$ and $\bar{I}_{2}$ are the invariants of either $\bar{\rightcauchygreentensor}$ or $\bar{\greenlagrangestraintensor}$.

The equivalent/initial shear modulus for a Mooney-Rivlin material is given by $\frac{G_{0}}{2}=\pbrac{c_{10}+c_{01}}$.

The second Piola Kirchoff tensor is thus
\begin{equation}
  \begin{split}
    \bar{\secondpiolakirchoffstresstensor}&=2\delby{\fnof{\bar{W}}{\bar{I}_{1},\bar{I}_{2}}}{\bar{\rightcauchygreentensor}}\\
    &=2\pbrac{\delby{\fnof{\bar{W}}{\bar{I}_{1},\bar{I}_{2}}}{\bar{I}_{1}}\delby{\bar{I}_{1}}{\bar{\rightcauchygreentensor}}+\delby{\fnof{\bar{W}}{\bar{I}_{1},\bar{I}_{2}}}{\bar{I}_{2}}\delby{\bar{I}_{2}}{\bar{\rightcauchygreentensor}}}\\
    &=2\pbrac{c_{1}\begin{bmatrix}
        1 & 0 \\
        0 & 1
    \end{bmatrix}+c_{2}\begin{bmatrix}
        \bar{C}_{22} & -\bar{C}_{21} \\
        -\bar{C}_{12} & \bar{C}_{11}
    \end{bmatrix}} \\
    &=\begin{bmatrix}
        2c_{1}+2c_{2}\bar{C}_{22} & -2c_{2}\bar{C}_{21} \\
        -2c_{2}\bar{C}_{12} & 2c_{1}+2c_{2}\bar{C}_{11}
    \end{bmatrix}
  \end{split}
\end{equation}
or, in terms of the Lagrange strain tensor
\begin{equation}
  \begin{split}
    \bar{\secondpiolakirchoffstresstensor}&=\delby{\fnof{\bar{W}}{\bar{I}_{1},\bar{I}_{2}}}{\bar{\greenlagrangestraintensor}}\\
    &=\begin{bmatrix}
    c_{1}+c_{2}\bar{E}_{22} & -c_{2}\bar{E}_{21} \\
    -c_{2}\bar{E}_{12} & c_{1}+c_{2}\bar{E}_{11}
    \end{bmatrix}
  \end{split}
\end{equation}

The second material elasticity tensor is given by
\begin{equation}
  \bar{\materialsecondelasticitytensor}=2\delby{\bar{\secondpiolakirchoffstresstensor}}{\bar{\rightcauchygreentensor}}=4\deltwosqby{\fnof{\bar{W}}{\bar{I}_{1},\bar{I}_{2}}}{\bar{\rightcauchygreentensor}}\\
\end{equation}
and thus
\begin{equation}
  \begin{split}
    \bar{C}^{1111}&=\delby{\bar{S}^{11}}{\bar{C}^{11}}=0\\
    \bar{C}^{1122}&=\delby{\bar{S}^{11}}{\bar{C}^{22}}=4c_{2}\\
    \bar{C}^{1112}&=\delby{\bar{S}^{11}}{\bar{C}^{22}}=0\\
    \bar{C}^{2211}&=\delby{\bar{S}^{22}}{\bar{C}^{11}}=4c_{2}\\
    \bar{C}^{2222}&=\delby{\bar{S}^{22}}{\bar{C}^{22}}=0\\
    \bar{C}^{2212}&=\delby{\bar{S}^{22}}{\bar{C}^{22}}=0\\
    \bar{C}^{1211}&=\delby{\bar{S}^{22}}{\bar{C}^{11}}=4c_{2}\\
    \bar{C}^{1222}&=\delby{\bar{S}^{22}}{\bar{C}^{22}}=0\\
    \bar{C}^{1212}&=\delby{\bar{S}^{22}}{\bar{C}^{22}}=-4c_{2}\\
  \end{split}
\end{equation}
or, in Voigt form
\begin{equation}
  \bar{\materialsecondelasticitytensor}^{V}=\begin{bmatrix}
  0 & 4c_{2} & 0 \\
  4c_{2} & 0 & 0 \\
  0 & 0 & -4c_{2}
  \end{bmatrix}
\end{equation}

For three dimensions the strain energy function is given by
\begin{equation}
  \fnof{\bar{W}}{\bar{I}_{1},\bar{I}_{2}}=c_{1}\pbrac{\bar{I}_{1}-3}+c_{2}\pbrac{\bar{I}_{2}-3}
\end{equation}
where $\bar{I}_{1}$ and $\bar{I}_{2}$ are the invariants of either $\bar{\rightcauchygreentensor}$ or $\bar{\greenlagrangestraintensor}$.

The second Piola Kirchoff tensor is thus
\begin{equation}
  \begin{split}
    \bar{\secondpiolakirchoffstresstensor}&=2\delby{\fnof{\bar{W}}{\bar{I}_{1},\bar{I}_{2}}}{\bar{\rightcauchygreentensor}}\\
    &=2\pbrac{\delby{\fnof{\bar{W}}{\bar{I}_{1},\bar{I}_{2}}}{\bar{I}_{1}}\delby{\bar{I}_{1}}{\bar{\rightcauchygreentensor}}+\delby{\fnof{\bar{W}}{\bar{I}_{1},\bar{I}_{2}}}{\bar{I}_{2}}\delby{\bar{I}_{2}}{\bar{\rightcauchygreentensor}}}\\
    &=2\pbrac{c_{1}\begin{bmatrix}
        1 & 0 & 0 \\
        0 & 1 & 0 \\
        0 & 0 & 1
      \end{bmatrix}+c_{2}\begin{bmatrix}
        \bar{C}_{22}+\bar{C}_{33} & -\bar{C}_{21} & \bar{C}_{31} \\
        -\bar{C}_{12} & \bar{C}_{11}+\bar{C}_{33} & -\bar{C}_{32} \\
        -\bar{C}_{13} & -\bar{C}_{23} & \bar{C}_{11}+\bar{C}_{22}
    \end{bmatrix}} \\
    &=\begin{bmatrix}
    2c_{1}+2c_{2}\pbrac{\bar{C}_{22}+\bar{C}_{33}} & -2c_{2}\bar{C}_{21} &
    -2c_{2}\bar{C}_{31} \\
    -2c_{2}\bar{C}_{12} & 2c_{1}+2c_{2}\pbrac{\bar{C}_{11}+\bar{C}_{33}} & -2c_{2}\bar{C}_{32} \\
    -2c_{2}\bar{C}_{13} & -2c_{2}\bar{C}_{23} & 2c_{1}+2c_{2}\pbrac{\bar{C}_{11}+\bar{C}_{22}} 
    \end{bmatrix}
  \end{split}
\end{equation}
or, in terms of the Lagrange strain tensor
\begin{equation}
  \begin{split}
    \bar{\secondpiolakirchoffstresstensor}&=\delby{\fnof{\bar{W}}{\bar{I}_{1},\bar{I}_{2}}}{\bar{\greenlagrangestraintensor}}\\
    &=\begin{bmatrix}
    c_{1}+c_{2}\pbrac{\bar{E}_{22}+\bar{E}_{33}} & -c_{2}\bar{E}_{21} & -c_{2}\bar{E}_{31} \\
    -c_{2}\bar{E}_{12} & c_{1}+c_{2}\pbrac{\bar{E}_{11}+\bar{E}_{33}} & -c_{2}\bar{E}_{32} \\
    -c_{2}\bar{E}_{13} & -c_{2}\bar{E}_{23} & c_{1}+c_{2}\pbrac{\bar{E}_{11}+\bar{E}_{22}}
    \end{bmatrix}
  \end{split}
\end{equation}

The second material elasticity tensor is given by
\begin{equation}
  \bar{\materialsecondelasticitytensor}=2\delby{\bar{\secondpiolakirchoffstresstensor}}{\bar{\rightcauchygreentensor}}=4\deltwosqby{\fnof{\bar{W}}{\bar{I}_{1},\bar{I}_{2}}}{\bar{\rightcauchygreentensor}}\\
\end{equation}
and thus
\begin{equation}
  \begin{split}
    \bar{C}^{1111}&=\delby{\bar{S}^{11}}{\bar{C}^{11}}=0\\
    \bar{C}^{1122}&=\delby{\bar{S}^{11}}{\bar{C}^{22}}=4c_{2}\\
    \bar{C}^{1133}&=\delby{\bar{S}^{11}}{\bar{C}^{33}}=4c_{2}\\
    \bar{C}^{1123}&=\delby{\bar{S}^{11}}{\bar{C}^{23}}=0\\
    \bar{C}^{1113}&=\delby{\bar{S}^{11}}{\bar{C}^{13}}=0\\
    \bar{C}^{1112}&=\delby{\bar{S}^{11}}{\bar{C}^{12}}=0\\
    \bar{C}^{2211}&=\delby{\bar{S}^{22}}{\bar{C}^{11}}=4c_{2}\\
    \bar{C}^{2222}&=\delby{\bar{S}^{22}}{\bar{C}^{22}}=0\\
    \bar{C}^{2233}&=\delby{\bar{S}^{22}}{\bar{C}^{33}}=4c_{2}\\
    \bar{C}^{2223}&=\delby{\bar{S}^{22}}{\bar{C}^{23}}=0\\
    \bar{C}^{2213}&=\delby{\bar{S}^{22}}{\bar{C}^{13}}=0\\
    \bar{C}^{2212}&=\delby{\bar{S}^{22}}{\bar{C}^{12}}=0\\
    \bar{C}^{3311}&=\delby{\bar{S}^{33}}{\bar{C}^{11}}=4c_{2}\\
    \bar{C}^{3322}&=\delby{\bar{S}^{33}}{\bar{C}^{22}}=4c_{2}\\
    \bar{C}^{3333}&=\delby{\bar{S}^{33}}{\bar{C}^{33}}=0\\
    \bar{C}^{3323}&=\delby{\bar{S}^{33}}{\bar{C}^{23}}=0\\
    \bar{C}^{3313}&=\delby{\bar{S}^{33}}{\bar{C}^{13}}=0\\
    \bar{C}^{3312}&=\delby{\bar{S}^{33}}{\bar{C}^{12}}=0\\
    \bar{C}^{2311}&=\delby{\bar{S}^{23}}{\bar{C}^{11}}=0\\
    \bar{C}^{2322}&=\delby{\bar{S}^{23}}{\bar{C}^{22}}=0\\
    \bar{C}^{2333}&=\delby{\bar{S}^{23}}{\bar{C}^{33}}=0\\
    \bar{C}^{2323}&=\delby{\bar{S}^{23}}{\bar{C}^{23}}=-4c_{2}\\
    \bar{C}^{2313}&=\delby{\bar{S}^{23}}{\bar{C}^{13}}=0\\
    \bar{C}^{2312}&=\delby{\bar{S}^{23}}{\bar{C}^{12}}=0\\
    \bar{C}^{1311}&=\delby{\bar{S}^{13}}{\bar{C}^{11}}=0\\
    \bar{C}^{1322}&=\delby{\bar{S}^{13}}{\bar{C}^{22}}=0\\
    \bar{C}^{1333}&=\delby{\bar{S}^{13}}{\bar{C}^{33}}=0\\
    \bar{C}^{1323}&=\delby{\bar{S}^{13}}{\bar{C}^{23}}=0\\
    \bar{C}^{1313}&=\delby{\bar{S}^{13}}{\bar{C}^{13}}=-4c_{2}\\
    \bar{C}^{1312}&=\delby{\bar{S}^{13}}{\bar{C}^{12}}=0\\
    \bar{C}^{1211}&=\delby{\bar{S}^{12}}{\bar{C}^{11}}=0\\
    \bar{C}^{1222}&=\delby{\bar{S}^{12}}{\bar{C}^{22}}=0\\
    \bar{C}^{1233}&=\delby{\bar{S}^{12}}{\bar{C}^{33}}=0\\
    \bar{C}^{1223}&=\delby{\bar{S}^{12}}{\bar{C}^{23}}=0\\
    \bar{C}^{1213}&=\delby{\bar{S}^{12}}{\bar{C}^{13}}=0\\
    \bar{C}^{1212}&=\delby{\bar{S}^{12}}{\bar{C}^{12}}=-4c_{2}\\
  \end{split}
\end{equation}
or, in Voigt form
\begin{equation}
  \bar{\materialsecondelasticitytensor}^{V}=\begin{bmatrix}
  0 & 4c_{2} & 4c_{2} & 0 & 0 & 0 \\
  4c_{2} & 0 & 4c_{2} & 0 & 0 & 0 \\
  4c_{2} & 4c_{2} & 0 & 0 & 0 & 0 \\
  0     & 0      & 0 & -4c_{2} & 0 & 0 \\
  0     & 0      & 0 & 0      & -4c_{2} & 0 \\
  0     & 0      & 0 & 0      & 0      & -4c_{2} 
  \end{bmatrix}
\end{equation}

\subsection{Humpfrey Model}

The Humpfrey strain energy function for an incompressible material can be
stated as
\begin{equation}
  \fnof{W}{\bar{I}_{1},J}=\dfrac{c_{1}}{c_{2}}\pbrac{e^{c_{2}\pbrac{\bar{I}_{1}-3}}-1}+\dfrac{p}{2}\pbrac{J-1}^{2}
\end{equation}
where
\begin{equation}
  \bar{I}_{1}=J^{-\frac{1}{3}}I_{1}
\end{equation}

\clearemptydoublepage

\section{Uniaxial extension of an incompressible unit square}

Consider a uniaxial stretch of $\alpha$ of an incompressible unit
square in the $x$ direction and $-\beta$ in the $y$ direction as shown
in \figref{fig:TwoDUniaxialExtension}. The resulting deformed
configuration will be a square of dimensions $1+\alpha$ by $1-\beta$.

\epstexfigure{SolidMechanics/svgs/TwoDUniaxialExtension.eps_tex}{2D
  uniaxial extension.}{Two dimensional uniaxial extension of a unit
  square. The unit square deforms by $\alpha$ in the $x$ direction and
  $\beta$ in the $y$ direction.}{fig:TwoDUniaxialExtension}{0.66}

The deformation gradient tensor will be
\begin{equation}
  \deformationgradienttensor= \begin{bmatrix}
    1+\alpha & 0  \\
    0 & 1-\beta
  \end{bmatrix}
\end{equation}
everywhere, and the Jacobian of the deformation will be 
\begin{equation}
  J = \determinant{\deformationgradienttensor}=\pbrac{1+\alpha}\pbrac{1-\beta}
\end{equation}

The modified deformation gradient tensor will be
\begin{equation}
  \bar{\deformationgradienttensor}=J^{-\frac{1}{2}}\deformationgradienttensor=\frac{1}{\sqrt{\pbrac{1+\alpha}\pbrac{1+\beta}}}\begin{bmatrix}
    1+\alpha & 0  \\
    0 & 1-\beta
  \end{bmatrix}=\begin{bmatrix}
    \frac{\sqrt{1+\alpha}}{\sqrt{1-\beta}} & 0  \\
    0 & \frac{\sqrt{1-\beta}}{\sqrt{1+\alpha}}
  \end{bmatrix}
\end{equation}

The modified right Cauchy-Green tensor is
\begin{equation}
  \bar{\rightcauchygreentensor} = \transpose{\bar{\deformationgradienttensor}}\bar{\deformationgradienttensor} =
  J^{-\frac{1}{2}}J^{-\frac{1}{2}}\transpose{\deformationgradienttensor}\deformationgradienttensor =
  \begin{bmatrix}
    \frac{\sqrt{1+\alpha}}{\sqrt{1-\beta}} & 0  \\
    0 & \frac{\sqrt{1-\beta}}{\sqrt{1+\alpha}}
  \end{bmatrix} \begin{bmatrix}
    \frac{\sqrt{1+\alpha}}{\sqrt{1-\beta}} & 0  \\
    0 & \frac{\sqrt{1-\beta}}{\sqrt{1+\alpha}}
  \end{bmatrix}=\begin{bmatrix}
    \frac{1+\alpha}{1-\beta} & 0  \\
    0 & \frac{1-\beta}{1+\alpha}
  \end{bmatrix}
\end{equation}

The modified Lagrange strain tensor is
\begin{equation}
  \bar{\greenlagrangestraintensor} = \frac{1}{2}\pbrac{\bar{\rightcauchygreentensor}-\tensor{I}} = \frac{1}{2}\pbrac{\begin{bmatrix}
    \frac{1+\alpha}{1-\beta} & 0  \\
    0 & \frac{1-\beta}{1+\alpha}
  \end{bmatrix} - \begin{bmatrix}
    1 & 0 \\
    0 & 1
  \end{bmatrix}} = \begin{bmatrix}
    \frac{\alpha+\beta}{2\pbrac{1-\beta}} & 0 \\
    0 & \frac{-\pbrac{\alpha+\beta}}{2\pbrac{1+\alpha}}
  \end{bmatrix}
\end{equation}

Now consider a Mooney-Rivlin material. The strain energy function is given by
\begin{equation}
  \fnof{\bar{W}}{\bar{I}_{1},\bar{I}_{2}}=c_{1}\pbrac{\bar{I}_{1}-2}+c_{2}\pbrac{\bar{I}_{2}-2}
\end{equation}
where
\begin{equation}
  \begin{split}
    \bar{I}_{1}&=\trace{}{\bar{\rightcauchygreentensor}}=\bar{C}_{11}+\bar{C}_{22}\\
    \bar{I}_{2}&=\frac{1}{2}\pbrac{\pbrac{\trace{}{\bar{\rightcauchygreentensor}}}^{2}-\trace{}{\bar{\rightcauchygreentensor}^{2}}}=\bar{C}_{11}\bar{C}_{22}-\bar{C}_{12}\bar{C}_{21}
  \end{split}
\end{equation}

Now
\begin{equation}
  \begin{split}
    \bar{\secondpiolakirchoffstresstensor}&=2\delby{\fnof{\bar{W}}{\bar{I}_{1},\bar{I}_{2}}}{\bar{\rightcauchygreentensor}}\\
    &=2\pbrac{\delby{\fnof{\bar{W}}{\bar{I}_{1},\bar{I}_{2}}}{\bar{I}_{1}}\delby{\bar{I}_{1}}{\bar{\rightcauchygreentensor}}+\delby{\fnof{\bar{W}}{\bar{I}_{1},\bar{I}_{2}}}{\bar{I}_{2}}\delby{\bar{I}_{2}}{\bar{\rightcauchygreentensor}}}\\
    &=2\pbrac{c_{1}\begin{bmatrix}
        1 & 0 \\
        0 & 1
    \end{bmatrix}+c_{2}\begin{bmatrix}
        \bar{C}_{22} & -\bar{C}_{21} \\
        -\bar{C}_{12} & \bar{C}_{11}
    \end{bmatrix}} \\
    &=\begin{bmatrix}
        2c_{1}+2c_{2}\bar{C}_{22} & -2c_{2}\bar{C}_{21} \\
        -2c_{2}\bar{C}_{12} & 2c_{1}+2c_{2}\bar{C}_{11}
    \end{bmatrix}
  \end{split}
\end{equation}

And so we have
\begin{equation}
  \bar{\secondpiolakirchoffstresstensor}=\begin{bmatrix}
  2c_{1}+2c_{2}\frac{1-\beta}{1+\alpha} & 0 \\
  0 & 2c_{1}+2c_{2}\frac{1+\alpha}{1-\beta}
  \end{bmatrix} 
\end{equation}

Now
\begin{equation}
  \begin{split}
    \bar{\tensor{\sigma}}&=\inverse{J}\pushforward{\chi}{\bar{\secondpiolakirchoffstresstensor}}=\inverse{J}\bar{\deformationgradienttensor}\bar{\secondpiolakirchoffstresstensor}\transpose{\bar{\deformationgradienttensor}}=\inverse{J}\doubledotprod{\pbrac{\tensorprod{\bar{\deformationgradienttensor}}{\bar{\deformationgradienttensor}}}}{\bar{\secondpiolakirchoffstresstensor}}\\
    &=\dfrac{1}{\pbrac{1+\alpha}\pbrac{1-\beta}}\begin{bmatrix}
      \frac{\sqrt{1+\alpha}}{\sqrt{1-\beta}} & 0  \\
      0 & \frac{\sqrt{1-\beta}}{\sqrt{1+\alpha}}
    \end{bmatrix}\begin{bmatrix}
      2c_{1}+2c_{2}\frac{1-\beta}{1+\alpha} & 0 \\
      0 & 2c_{1}+2c_{2}\frac{1+\alpha}{1-\beta}
    \end{bmatrix}\begin{bmatrix}
      \frac{\sqrt{1+\alpha}}{\sqrt{1-\beta}} & 0  \\
      0 & \frac{\sqrt{1-\beta}}{\sqrt{1+\alpha}}
    \end{bmatrix} \\
    &=\dfrac{1}{\pbrac{1+\alpha}\pbrac{1-\beta}}\begin{bmatrix}
      2c_{1}\pbrac{\frac{1+\alpha}{1-\beta}}+2c_{2} & 0 \\
      0 & 2c_{1}\pbrac{\frac{1-\beta}{1+\alpha}}+2c_{2}
    \end{bmatrix} \\
    &=\begin{bmatrix}
      \frac{2c_{1}}{\pbrac{1-\beta}^{2}}+\frac{2c_{2}}{\pbrac{1+\alpha}\pbrac{1-\beta}} & 0 \\
      0 & \frac{2c_{1}}{\pbrac{1+\alpha}^{2}}+\frac{2c_{2}}{\pbrac{1+\alpha}\pbrac{1-\beta}}
    \end{bmatrix}
  \end{split}
\end{equation}

Note that the psuedo Cauchy stress can also be calculated from
\begin{equation}
  \bar{\tensor{\sigma}}=\inverse{J}\doubledotprod{\pbrac{\tensorprod{\bar{\deformationgradienttensor}}{\bar{\deformationgradienttensor}}}}{\bar{\secondpiolakirchoffstresstensor}}\\
\end{equation}

In Voigt form this is given by
\begin{equation}
  \bar{\tensor{\sigma}}^{V}=\inverse{J}\begin{bmatrix}
  \bar{F}_{1}^{1}\bar{F}_{1}^{1} & \bar{F}_{2}^{1}\bar{F}_{2}^{1} &
  \bar{F}_{1}^{1}\bar{F}_{2}^{1} + \bar{F}_{2}^{1}\bar{F}_{1}^{1} \\
  \bar{F}_{1}^{2}\bar{F}_{1}^{2} & \bar{F}_{2}^{2}\bar{F}_{2}^{2} &
  \bar{F}_{1}^{2}\bar{F}_{2}^{2} + \bar{F}_{2}^{2}\bar{F}_{1}^{2} \\
  \bar{F}_{1}^{1}\bar{F}_{1}^{2} & \bar{F}_{2}^{1}\bar{F}_{2}^{2} &
  \bar{F}_{1}^{1}\bar{F}_{2}^{2} + \bar{F}_{2}^{1}\bar{F}_{1}^{2}
  \end{bmatrix}\begin{bmatrix}
    \bar{S}^{11} \\
    \bar{S}^{22} \\
    \bar{S}^{12}
  \end{bmatrix}
\end{equation}

For the unit square case this is
\begin{equation}
  \begin{split}
    \bar{\tensor{\sigma}}^{V}&=\dfrac{1}{\pbrac{1+\alpha}\pbrac{1-\beta}}\begin{bmatrix}
      \frac{1+\alpha}{1-\beta} & 0 & 0 \\
      0 & \frac{1-\beta}{1+\alpha} & 0 \\
      0 & 0 & 1
    \end{bmatrix}\begin{bmatrix}
      2c_{1}+2c_{2}\frac{1-\beta}{1+\alpha} \\
      2c_{1}+2c_{2}\frac{1+\alpha}{1-\beta} \\
      0
    \end{bmatrix} \\
    &=\dfrac{1}{\pbrac{1+\alpha}\pbrac{1-\beta}}\begin{bmatrix}
      \frac{1+\alpha}{1-\beta}\pbrac{2c_{1}+2c_{2}\frac{1-\beta}{1+\alpha}} \\
      \frac{1-\beta}{1+\alpha}\pbrac{2c_{1}+2c_{2}\frac{1+\alpha}{1-\beta}} \\
      0
    \end{bmatrix} \\
    &=\dfrac{1}{\pbrac{1+\alpha}\pbrac{1-\beta}}\begin{bmatrix}
      2c_{1}\pbrac{\frac{1+\alpha}{1-\beta}}+2c_{2} \\
      2c_{1}\pbrac{\frac{1-\beta}{1+\alpha}}+2c_{2} \\
      0
    \end{bmatrix}
  \end{split}
\end{equation}

The deviatoric Cauchy stress tensor is
\begin{equation}
  \tensor{\sigma}_{\devop}=J^{-\frac{1}{2}}\doubledotprod{\spatialdeviatorictensorfour}{\bar{\tensor{\sigma}}}
\end{equation}

In Voigt form this is given by
\begin{equation}
  \begin{split}
    \tensor{\sigma}_{\devop}^{V}&=J^{-\frac{1}{2}}\begin{bmatrix}
      \frac{1}{2} & -\frac{1}{2} & 0 \\
      -\frac{1}{2} & \frac{1}{2} & 0 \\
      0 & 0 & 1
    \end{bmatrix}\begin{bmatrix}
      \bar{\sigma}^{11} \\
      \bar{\sigma}^{22} \\
      \bar{\sigma}^{12}
    \end{bmatrix} \\
    &=J^{-\frac{1}{2}}\begin{bmatrix}
    \frac{1}{2}\bar{\sigma}^{11}-\frac{1}{2}\bar{\sigma}^{22} \\
    -\frac{1}{2}\bar{\sigma}^{11}+\frac{1}{2}\bar{\sigma}^{22} \\
    \bar{\sigma}^{12}
    \end{bmatrix}\\
    &=J^{-\frac{1}{2}}\begin{bmatrix}
    \bar{\sigma}^{11}-\frac{1}{2}\bar{\sigma}^{11}-\frac{1}{2}\bar{\sigma}^{22} \\
    \bar{\sigma}^{22}-\frac{1}{2}\bar{\sigma}^{11}-\frac{1}{2}\bar{\sigma}^{22} \\
    \bar{\sigma}^{12}
    \end{bmatrix}\\
    &=J^{-\frac{1}{2}}\begin{bmatrix}
    \bar{\sigma}^{11}-\frac{1}{2}\trace{}{\bar{\tensor{\sigma}}} \\
    \bar{\sigma}^{22}-\frac{1}{2}\trace{}{\bar{\tensor{\sigma}}}\\
    \bar{\sigma}^{12}
    \end{bmatrix}
  \end{split}
\end{equation}
or
\begin{equation}
  \tensor{\sigma}_{\devop}=J^{-\frac{1}{2}}\pbrac{\bar{\tensor{\sigma}}-\frac{1}{2}\trace{}{\bar{\tensor{\sigma}}}\tensor{i}}
\end{equation}

We thus need
\begin{equation}
  \begin{split}
    \frac{1}{2}\trace{}{\bar{\tensor{\sigma}}}&=\frac{1}{2}\pbrac{\frac{2c_{1}}{\pbrac{1-\beta}^{2}}+\frac{2c_{2}}{\pbrac{1+\alpha}\pbrac{1-\beta}}+\frac{2c_{1}}{\pbrac{1+\alpha}^{2}}+\frac{2c_{2}}{\pbrac{1+\alpha}\pbrac{1-\beta}}}\\
    &=\dfrac{1}{\pbrac{1+\alpha}\pbrac{1-\beta}}\pbrac{c_{1}\pbrac{\frac{1+\alpha}{1-\beta}}+c_{1}\pbrac{\frac{1-\beta}{1+\alpha}}+2c_{2}}
  \end{split}
\end{equation}
which leads to
\begin{equation}
  \begin{split}
    \bar{\tensor{\sigma}}-\frac{1}{2}\trace{}{\bar{\tensor{\sigma}}}\tensor{i}&=\dfrac{1}{\pbrac{1+\alpha}\pbrac{1-\beta}}\begin{bmatrix}
      2c_{1}\pbrac{\frac{1+\alpha}{1-\beta}}+2c_{2} & 0 \\
      0 & 2c_{1}\pbrac{\frac{1-\beta}{1+\alpha}}+2c_{2}
    \end{bmatrix}\\
    &\qquad\qquad-\dfrac{1}{\pbrac{1+\alpha}\pbrac{1-\beta}}\pbrac{c_{1}\pbrac{\frac{1+\alpha}{1-\beta}}+c_{1}\pbrac{\frac{1-\beta}{1+\alpha}}+2c_{2}}\begin{bmatrix}
      1 & 0 \\
      0 & 1
    \end{bmatrix} \\
    &=\dfrac{1}{\pbrac{1+\alpha}\pbrac{1-\beta}}\begin{bmatrix}
      c_{1}\pbrac{\frac{1+\alpha}{1-\beta}}-c_{1}\pbrac{\frac{1-\beta}{1+\alpha}} & 0 \\
      0 & c_{1}\pbrac{\frac{1-\beta}{1+\alpha}}-c_{1}\pbrac{\frac{1+\alpha}{1-\beta}}
    \end{bmatrix} \\
    &=\dfrac{c_{1}}{\pbrac{1+\alpha}\pbrac{1-\beta}}\begin{bmatrix}
      \frac{\pbrac{1+\alpha}^{2}-\pbrac{1-\beta}^{2}}{\pbrac{1+\alpha}\pbrac{1-\beta}} & 0 \\
      0 & \frac{\pbrac{1-\beta}^{2}-\pbrac{1+\alpha}^{2}}{\pbrac{1+\alpha}\pbrac{1-\beta}}
    \end{bmatrix}\\
    &=\dfrac{c_{1}}{\pbrac{1+\alpha}^{2}\pbrac{1-\beta}^{2}}\begin{bmatrix}
      \alpha\pbrac{2+\alpha}-\beta\pbrac{2-\beta} & 0 \\
      0 & -\alpha\pbrac{2+\alpha}+\beta\pbrac{2-\beta}
    \end{bmatrix} \\
    &=c_{1}\pbrac{\alpha\pbrac{2+\alpha}-\beta\pbrac{2-\beta}}J^{-2}\begin{bmatrix}
      1 & 0 \\
      0 & -1
    \end{bmatrix}
  \end{split}
\end{equation}
and therefore
\begin{equation}
  \begin{split}
    \tensor{\sigma}_{\devop}&=J^{-\frac{1}{2}}\doubledotprod{\spatialdeviatorictensorfour}{\bar{\tensor{\sigma}}}=J^{-\frac{1}{2}}\pbrac{\bar{\tensor{\sigma}}-\frac{1}{2}\trace{}{\bar{\tensor{\sigma}}}\tensor{i}}\\
    &=c_{1}\pbrac{\alpha\pbrac{2+\alpha}-\beta\pbrac{2-\beta}}J^{-\frac{5}{2}}\begin{bmatrix}
      1 & 0 \\
      0 & -1
    \end{bmatrix}
  \end{split}
\end{equation}

The spherical cauchy tensor is
\begin{equation}
  \tensor{\sigma}_{\sphop}=-p\tensor{i}=\begin{bmatrix}
  -p & 0 \\
  0 & -p
  \end{bmatrix}
\end{equation}
and so
\begin{equation}
  \tensor{\sigma}=\tensor{\sigma}_{\devop}+\tensor{\sigma}_{\sphop}=c_{1}\pbrac{\alpha\pbrac{2+\alpha}-\beta\pbrac{2-\beta}}J^{-\frac{5}{2}}\begin{bmatrix}
      1-p & 0 \\
      0 & -1-p
    \end{bmatrix}
\end{equation}

Now the second material elasticity tensor is given by
\begin{equation}
  \bar{\materialsecondelasticitytensor}=2\delby{\bar{\secondpiolakirchoffstresstensor}}{\bar{\rightcauchygreentensor}}=4\deltwosqby{\fnof{\bar{W}}{\bar{I}_{1},\bar{I}_{2}}}{\bar{\rightcauchygreentensor}}
\end{equation}

In Voigt notation we have
\begin{equation}
  \bar{\materialsecondelasticitytensor}^{V}=\begin{bmatrix}
  0 & 4c_{2} & 0 \\
  4c_{2} & 0 & 0 \\
  0 & 0 & -4c_{2}
  \end{bmatrix}
\end{equation}

The second spatial elasticity tensor is given by
\begin{equation}
  \bar{\spatialsecondelasticitytensor}=\inverse{J}\pushforward{\chi}{\bar{\materialsecondelasticitytensor}}=\inverse{J}\doubledotprodthree{\pbrac{\tensorprod{\bar{\deformationgradienttensor}}{\bar{\deformationgradienttensor}}}}{\bar{\materialsecondelasticitytensor}}{\transpose{\pbrac{\tensorprod{\bar{\deformationgradienttensor}}{\bar{\deformationgradienttensor}}}}}
\end{equation}

In Voigt notation for the uniaxial square case we have
\begin{equation}
  \begin{split}
    \bar{\spatialsecondelasticitytensor}^{V}&=\dfrac{1}{\pbrac{1+\alpha}\pbrac{1-\beta}}\begin{bmatrix}
      \frac{1+\alpha}{1-\beta} & 0 & 0 \\
      0 & \frac{1-\beta}{1+\alpha} & 0 \\
      0 & 0 & 1
    \end{bmatrix}\begin{bmatrix}
      0 & 4c_{2} & 0 \\
      4c_{2} & 0 & 0 \\
      0 & 0 & -4c_{2}
    \end{bmatrix}\begin{bmatrix}
      \frac{1+\alpha}{1-\beta} & 0 & 0 \\
      0 & \frac{1-\beta}{1+\alpha} & 0 \\
      0 & 0 & 1
    \end{bmatrix}\\
    &=\dfrac{1}{\pbrac{1+\alpha}\pbrac{1-\beta}}\begin{bmatrix}
      0 & 4c_{2}\pbrac{\frac{1+\alpha}{1-\beta}} & 0 \\
      4c_{2}\pbrac{\frac{1-\beta}{1+\alpha}} & 0 & 0 \\
      0 & 0 & -4c_{2}
    \end{bmatrix}\begin{bmatrix}
      \frac{1+\alpha}{1-\beta} & 0 & 0 \\
      0 & \frac{1-\beta}{1+\alpha} & 0 \\
      0 & 0 & 1
    \end{bmatrix}\\
    &=\dfrac{1}{\pbrac{1+\alpha}\pbrac{1-\beta}}\begin{bmatrix}
      0 & 4c_{2} & 0 \\
      4c_{2} & 0 & 0 \\
      0 & 0 & -4c_{2}
    \end{bmatrix}\\
    &=\inverse{J}\bar{\materialsecondelasticitytensor}^{V}
  \end{split}
\end{equation}

Now
\begin{equation}
  \spatialsecondelasticitytensor_{e_{1}}=J^{-\frac{3}{2}}\doubledotprodthree{\spatialdeviatorictensorfour}{\bar{\spatialsecondelasticitytensor}}{\transpose{\spatialdeviatorictensorfour}}
\end{equation}
or in Voigt notation
\begin{equation}
  \begin{split}
    \spatialsecondelasticitytensor_{e_{1}}^{V}&=J^{-\frac{3}{2}}\inverse{J}\begin{bmatrix}
      \frac{1}{2} & -\frac{1}{2} & 0 \\
      -\frac{1}{2} & \frac{1}{2} & 0 \\
      0 & 0 & 1
    \end{bmatrix}\begin{bmatrix}
      0 & 4c_{2} & 0 \\
      4c_{2} & 0 & 0 \\
      0 & 0 & -4c_{2}
    \end{bmatrix}\begin{bmatrix}
      \frac{1}{2} & -\frac{1}{2} & 0 \\
      -\frac{1}{2} & \frac{1}{2} & 0 \\
      0 & 0 & 1
    \end{bmatrix}\\
    &=J^{-\frac{5}{2}}\begin{bmatrix}
      -2c_{2} & 2c_{2} & 0 \\
      2c_{2} & -2c_{2} & 0 \\
      0 & 0 & -4c_{2}
    \end{bmatrix}\begin{bmatrix}
      \frac{1}{2} & -\frac{1}{2} & 0 \\
      -\frac{1}{2} & \frac{1}{2} & 0 \\
      0 & 0 & 1
    \end{bmatrix}\\
    &=J^{-\frac{5}{2}}\begin{bmatrix}
      -2c_{2} & 2c_{2} & 0 \\
      2c_{2} & -2c_{2} & 0 \\
      0 & 0 & -4c_{2}
    \end{bmatrix}
  \end{split}
\end{equation}

We also have
\begin{equation}
  \spatialsecondelasticitytensor_{e_{2}}=\dfrac{1}{2}J^{-\frac{1}{2}}\trop\bar{\tensortwo{\sigma}}\spatialdeviatorictensorfour^{\sharp}
  -\dfrac{1}{2}\sqbrac{\tensorprod{\inverse{\tensortwo{g}}}{\tensor{\sigma}_{\devop}}+\tensorprod{\tensortwo{\sigma}_{\devop}}{\inverse{\tensortwo{g}}}}
\end{equation}

Working in Voigt notation we have
\begin{equation}
  \begin{split}
    \pbrac{\dfrac{1}{2}J^{-\frac{1}{2}}\trop\bar{\tensortwo{\sigma}}\spatialdeviatorictensorfour^{\sharp}}^{V}&=\dfrac{1}{2}J^{-\frac{1}{2}}\trop\bar{\tensortwo{\sigma}}\begin{bmatrix}
      \frac{1}{2} & -\frac{1}{2} & 0 \\
      -\frac{1}{2} & \frac{1}{2} & 0 \\
      0 & 0 & \frac{1}{2}
    \end{bmatrix} \\
    &=\dfrac{1}{2}J^{-\frac{1}{2}}\begin{bmatrix}
     \frac{1}{2}\trace{}{\bar{\tensor{\sigma}}} & -\frac{1}{2}\trace{}{\bar{\tensor{\sigma}}} & 0 \\
      -\frac{1}{2}\trace{}{\bar{\tensor{\sigma}}} & \frac{1}{2}\trace{}{\bar{\tensor{\sigma}}} & 0 \\
      0 & 0 & \frac{1}{2}\trace{}{\bar{\tensor{\sigma}}}
    \end{bmatrix} 
  \end{split}
\end{equation}
and
\begin{equation}
  \pbrac{\tensorprod{\inverse{\tensortwo{g}}}{\tensor{\sigma}_{\devop}}}^{V}=\begin{bmatrix}
  1 \\
  1 \\
  0
  \end{bmatrix}\begin{bmatrix}
    \sigma_{\devop}^{11} & \sigma_{\devop}^{22} & \sigma_{\devop}^{12}
  \end{bmatrix}=\begin{bmatrix}
  \sigma_{\devop}^{11} & \sigma_{\devop}^{22} & \sigma_{\devop}^{12} \\
  \sigma_{\devop}^{11} & \sigma_{\devop}^{22} & \sigma_{\devop}^{12} \\
  0 & 0 & 0
  \end{bmatrix}
\end{equation}
and
\begin{equation}
  \pbrac{\tensorprod{\tensor{\sigma}_{\devop}}{\inverse{\tensortwo{g}}}}^{V}=\begin{bmatrix}
  \sigma_{\devop}^{11} \\
  \sigma_{\devop}^{22} \\
  \sigma_{\devop}^{12}
  \end{bmatrix}\begin{bmatrix}
  1 & 1 & 0
  \end{bmatrix}=\begin{bmatrix}
  \sigma_{\devop}^{11} & \sigma_{\devop}^{11} & 0 \\
  \sigma_{\devop}^{22} & \sigma_{\devop}^{22} & 0 \\
  \sigma_{\devop}^{12} & \sigma_{\devop}^{12} & 0
  \end{bmatrix}
\end{equation}
and so
\begin{equation}
  \begin{split}
    \pbrac{\dfrac{1}{2}\sqbrac{\tensorprod{\inverse{\tensortwo{g}}}{\tensor{\sigma}_{\devop}}+\tensorprod{\tensortwo{\sigma}_{\devop}}{\inverse{\tensortwo{g}}}}}^{V}&=\dfrac{1}{2}\pbrac{\begin{bmatrix}
        \sigma_{\devop}^{11} & \sigma_{\devop}^{22} & \sigma_{\devop}^{12} \\
        \sigma_{\devop}^{11} & \sigma_{\devop}^{22} & \sigma_{\devop}^{12} \\
        0 & 0 & 0
      \end{bmatrix}+\begin{bmatrix}
      \sigma_{\devop}^{11} & \sigma_{\devop}^{11} & 0 \\
      \sigma_{\devop}^{22} & \sigma_{\devop}^{22} & 0 \\
      \sigma_{\devop}^{12} & \sigma_{\devop}^{12} & 0
    \end{bmatrix}}\\
    &=\dfrac{1}{2}\begin{bmatrix}
      2\sigma_{\devop}^{11} & \sigma_{\devop}^{11}+\sigma_{\devop}^{22} & \sigma_{\devop}^{12} \\
      \sigma_{\devop}^{11}+\sigma_{\devop}^{22} & 2\sigma_{\devop}^{22} & \sigma_{\devop}^{12} \\
      \sigma_{\devop}^{12} & \sigma_{\devop}^{12} & 0
    \end{bmatrix} \\
    &=\dfrac{1}{2}J^{-\frac{1}{2}}\begin{bmatrix}
      2\bar{\sigma}^{11}-\trace{}{\bar{\tensor{\sigma}}} & 0 & \bar{\sigma}^{12}
      \\
      0 & 2\bar{\sigma}^{11}-\trace{}{\bar{\tensor{\sigma}}} & \bar{\sigma}^{12}
      \\
      \bar{\sigma}^{12} & \bar{\sigma}^{12} & 0
    \end{bmatrix}
  \end{split}
\end{equation}

We thus have
\begin{equation}
  \begin{split}
    \spatialsecondelasticitytensor_{e_{2}}^{V}&=\dfrac{1}{2}J^{-\frac{1}{2}}\pbrac{\begin{bmatrix}
        \frac{1}{2}\trace{}{\bar{\tensor{\sigma}}} & -\frac{1}{2}\trace{}{\bar{\tensor{\sigma}}} & 0 \\
        -\frac{1}{2}\trace{}{\bar{\tensor{\sigma}}} & \frac{1}{2}\trace{}{\bar{\tensor{\sigma}}} & 0 \\
        0 & 0 & \frac{1}{2}\trace{}{\bar{\tensor{\sigma}}}
      \end{bmatrix}-\begin{bmatrix}
      2\bar{\sigma}^{11}-\trace{}{\bar{\tensor{\sigma}}} & 0 & \bar{\sigma}^{12}
      \\
      0 & 2\bar{\sigma}^{11}-\trace{}{\bar{\tensor{\sigma}}} & \bar{\sigma}^{12}
      \\
      \bar{\sigma}^{12} & \bar{\sigma}^{12} & 0
    \end{bmatrix}}\\
    &=\dfrac{1}{2}J^{-\frac{1}{2}}\begin{bmatrix}
      \frac{3}{2}\trace{}{\bar{\tensor{\sigma}}}-2\bar{\sigma}^{11} &
      -\frac{1}{2}\trace{}{\bar{\tensor{\sigma}}} & -\bar{\sigma}^{12} \\
      -\frac{1}{2}\trace{}{\bar{\tensor{\sigma}}} &
      \frac{3}{2}\trace{}{\bar{\tensor{\sigma}}}-2\bar{\sigma}^{22} &
      -\bar{\sigma}^{12} \\
      -\bar{\sigma}^{12} & -\bar{\sigma}^{12} &
      \frac{1}{2}\trace{}{\bar{\tensor{\sigma}}}
    \end{bmatrix}
  \end{split}
\end{equation}

\section{Uniaxial extension of an incompressible unit cube}

Consider a uniaxial stretch of $\alpha$ of an incompressible unit
square in the $x$ direction and $-\beta$ in the $y$ direction and
$-\gamma$ in the $z$ direction as shown in
\figref{fig:ThreeDUniaxialExtension}. The resulting deformed
configuration will be a cube of dimensions $1+\alpha$ by $1-\beta$ by
$1-\gamma$.

\epstexfigure{SolidMechanics/svgs/ThreeDUniaxialExtension.eps_tex}{3D
  uniaxial extension.}{Three dimensional uniaxial extension of an
  incompressible unit cube. The unit cube deforms by $\alpha$ in the
  $x$ direction, and $\beta$ in the $y$ direction and $\gamma$ in the
  $z$ direction.}{fig:ThreeDUniaxialExtension}{0.66}

The deformation gradient tensor will be
\begin{equation}
  \deformationgradienttensor= \begin{bmatrix}
    1+\alpha & 0 & 0 \\
    0 & 1-\beta & 0 \\
    0 & 0 & 1-\gamma
  \end{bmatrix}
  \label{eqn:FiniteElasticityUniaxialCubeDeformationGradient}
\end{equation}
everywhere, and the Jacobian of the deformation will be 
\begin{equation}
  J = \determinant{\deformationgradienttensor}=\pbrac{1+\alpha}\pbrac{1-\beta}\pbrac{1-\gamma}
  \label{eqn:FiniteElasticityUniaxialCubeJacobian}
\end{equation}

The modified deformation gradient tensor will be
\begin{equation}
  \begin{split}
    \bar{\deformationgradienttensor}&=J^{-\frac{1}{3}}\deformationgradienttensor=\frac{1}{\cuberoot{\pbrac{1+\alpha}\pbrac{1+\beta}\pbrac{1+\gamma}}}\begin{bmatrix}
      1+\alpha & 0 & 0 \\
      0 & 1-\beta & 0 \\
      0 & 0 & 1-\gamma
    \end{bmatrix}\\
    &=\begin{bmatrix}
    \frac{\cuberoot{\pbrac{1+\alpha}^{2}}}{\cuberoot{\pbrac{1-\beta}\pbrac{1-\gamma}}}
    & 0 & 0 \\
    0 &
    \frac{\cuberoot{\pbrac{1-\beta}^{2}}}{\cuberoot{\pbrac{1+\alpha}\pbrac{1-\gamma}}}
    & 0 \\
    0 & 0 & \frac{\cuberoot{\pbrac{1-\gamma}^{2}}}{\cuberoot{\pbrac{1+\alpha}\pbrac{1-\beta}}}
    \end{bmatrix}
  \end{split}
  \label{eqn:FiniteElasticityUniaxialCubeModDeformationGradient}
\end{equation}

The right Cauchy-Green tensor is
\begin{equation}
  \begin{split}
    \rightcauchygreentensor &= \transpose{\deformationgradienttensor}\deformationgradienttensor\\
    &= \begin{bmatrix}
      1+\alpha & 0 & 0 \\
      0 & 1-\beta & 0 \\
      0 & 0 & 1-\gamma
    \end{bmatrix}\begin{bmatrix}
      1+\alpha & 0 & 0 \\
      0 & 1-\beta & 0 \\
      0 & 0 & 1-\gamma
    \end{bmatrix}\\
    &=\begin{bmatrix}
      \pbrac{1+\alpha}^{2} & 0 & 0 \\
      0 & \pbrac{1-\beta}^{2} & 0 \\
      0 & 0 & \pbrac{1-\gamma}^{2}
    \end{bmatrix} 
  \end{split}
  \label{eqn:FiniteElasticityUniaxialCubeRightCachyGreen}
\end{equation}
and the Piola deformation tensor is thus
\begin{equation}
  \begin{split}
    \pioladeformationtensor &= \inverse{\rightcauchygreentensor}\\
    &=\inverse{\begin{bmatrix}
      \pbrac{1+\alpha}^{2} & 0 & 0 \\
      0 & \pbrac{1-\beta}^{2} & 0 \\
      0 & 0 & \pbrac{1-\gamma}^{2}
    \end{bmatrix}}\\
    &=\begin{bmatrix}
      \dfrac{1}{\pbrac{1+\alpha}^{2}} & 0 & 0 \\
      0 & \dfrac{1}{\pbrac{1-\beta}^{2}} & 0 \\
      0 & 0 & \dfrac{1}{\pbrac{1-\gamma}^{2}}
    \end{bmatrix}    
  \end{split}
  \label{eqn:FiniteElasticityUniaxialCubePiola}
\end{equation}

The modified right Cauchy-Green tensor is
\begin{equation}
  \begin{split}
    \bar{\rightcauchygreentensor} &= \transpose{\bar{\deformationgradienttensor}}\bar{\deformationgradienttensor}=
    \transpose{\pbrac{J^{-\frac{1}{3}}\deformationgradienttensor}}\pbrac{J^{-\frac{1}{3}}\deformationgradienttensor}\\
    &= J^{-\frac{2}{3}}\begin{bmatrix}
      1+\alpha & 0 & 0 \\
      0 & 1-\beta & 0 \\
      0 & 0 & 1-\gamma
    \end{bmatrix}\begin{bmatrix}
      1+\alpha & 0 & 0 \\
      0 & 1-\beta & 0 \\
      0 & 0 & 1-\gamma
    \end{bmatrix}\\
    &=J^{-\frac{2}{3}}\begin{bmatrix}
      \pbrac{1+\alpha}^{2} & 0 & 0 \\
      0 & \pbrac{1-\beta}^{2} & 0 \\
      0 & 0 & \pbrac{1-\gamma}^{2}
    \end{bmatrix} \\
     &=\begin{bmatrix}
    \frac{\cuberoot{\pbrac{1+\alpha}^{4}}}{\cuberoot{\pbrac{1-\beta}^{2}\pbrac{1-\gamma}^{2}}}
    & 0 & 0 \\
    0 & \frac{\cuberoot{\pbrac{1-\beta}^{4}}}{\cuberoot{\pbrac{1+\alpha}^{2}\pbrac{1-\gamma}^{2}}}
    & 0  \\
    0 & 0 & \frac{\cuberoot{\pbrac{1-\gamma}^{4}}}{\cuberoot{\pbrac{1+\alpha}^{2}\pbrac{1-\beta}^{2}}}
    \end{bmatrix}
  \end{split}
  \label{eqn:FiniteElasticityUniaxialCubeModRightCachyGreen}
\end{equation}

We now consider a hyperelastic constitutive law as specified by a
strain energy function. From
\eqnref{eqn:FiniteElasticityDecomposedStrainEnergy} we have
\begin{equation}
  \begin{split}
    \fnof{W}{\rightcauchygreentensor} &=
    \fnof{W_{\Devop}}{\rightcauchygreentensor}+\fnof{W_{\Sphop}}{\rightcauchygreentensor} \\
    &=\fnof{\bar{W}_{\Devop}}{\fnof{\bar{\rightcauchygreentensor}}{\rightcauchygreentensor}}+\fnof{\bar{W}_{\Sphop}}{\fnof{J}{\rightcauchygreentensor}}\\
    &=\fnof{\bar{W}}{\fnof{\bar{\rightcauchygreentensor}}{\rightcauchygreentensor},\fnof{J}{\rightcauchygreentensor}}
  \end{split}
\end{equation}
where
$\fnof{\bar{W}}{\fnof{\bar{\rightcauchygreentensor}}{\rightcauchygreentensor},\fnof{J}{\rightcauchygreentensor}}$
is the \emph{modified strain energy function}, $\fnof{J}{\rightcauchygreentensor}$
is the Jacobian which quantifies volume change and
$\fnof{\bar{\rightcauchygreentensor}}{\rightcauchygreentensor}$, is the modified right
Cauchy-Green deformation tensor.

From \eqnref{eqn:FiniteElasticityDecomposedSecondPKinC} we can determine the second Piola-Kirchoff stress tensor, $\secondpiolakirchoffstresstensor$, as a function of the right Cauchy-Green deformation tensor, $\rightcauchygreentensor$, \ie
\begin{equation}
  \begin{split}
    \fnof{\secondpiolakirchoffstresstensor}{\rightcauchygreentensor}
    &=\doubledotprod{\fnof{\bar{\secondpiolakirchoffstresstensor}}{\fnof{\bar{\rightcauchygreentensor}}{\rightcauchygreentensor}}}{\delby{\fnof{\bar{\rightcauchygreentensor}}{\rightcauchygreentensor}}{\rightcauchygreentensor}}-2p\delby{\fnof{J}{\rightcauchygreentensor}}{\rightcauchygreentensor}\\
    &=\doubledotprod{\fnof{\bar{\secondpiolakirchoffstresstensor}}{\fnof{\bar{\rightcauchygreentensor}}{\rightcauchygreentensor}}}{\fnof{J^{-\frac{2}{3}}}{\rightcauchygreentensor}\pbrac{\symidentitytensorfour-\dfrac{\tensorprod{\rightcauchygreentensor}{\tensor{B}}}{3}}}-pJ\tensor{B}\\
    &=\fnof{\secondpiolakirchoffstresstensor_{\pullback{\Devop}{}}}{\fnof{\bar{\rightcauchygreentensor}}{\rightcauchygreentensor}}
    +\fnof{\secondpiolakirchoffstresstensor_{\pullback{\Sphop}{}}}{\fnof{J}{\rightcauchygreentensor}}
  \end{split}
  \label{eqn:FiniteElasticityDecomposedSecondPKinC}
\end{equation}
where $\bar{\secondpiolakirchoffstresstensor}$ is the second Piola-Kirchoff psuedo stress
tensor, $\secondpiolakirchoffstresstensor_{\pullback{\Devop}{}}$ the pulled back
deviatoric second Piola-Kirchoff stress tensor,
$\secondpiolakirchoffstresstensor_{\pullback{\Sphop}{}}$ the pulled back spherical second
Piola-Kirchoff stress tensor, $\pioladeformationtensor=\inverse{\rightcauchygreentensor}$
is the Piola deformation tensor, and $p$ the hydrostatic pressure
which resists volume change.

The second Piola-Kirchoff psuedo stress tensor can be found from the modified strain engergy function \ie
\begin{equation}
  \bar{\secondpiolakirchoffstresstensor}=2\delby{\fnof{\bar{W}_{\Devop}}{\fnof{\bar{\rightcauchygreentensor}}{\rightcauchygreentensor}}}{\bar{\rightcauchygreentensor}}
\end{equation}

For a Mooney-Rivlin material the modified deviatoric strain energy function is given by
\begin{equation}
  \fnof{\bar{W}_{\Devop}}{\bar{\rightcauchygreentensor}}=\fnof{\bar{W}_{\Devop}}{\bar{I}_{1},\bar{I}_{2}}=c_{10}\pbrac{\bar{I}_{1}-3}+c_{01}\pbrac{\bar{I}_{2}-3}
\end{equation}
where $c_{10}$ and $c_{01}$ are the Mooney-Rivlin material constants and
\begin{equation}
  \begin{split}
    \bar{I}_{1}&=\trace{}{\bar{\rightcauchygreentensor}}=\bar{C}_{11}+\bar{C}_{22}+\bar{C}_{33}\\
    \bar{I}_{2}&=\frac{1}{2}\pbrac{\pbrac{\trace{}{\bar{\rightcauchygreentensor}}}^{2}-\trace{}{\bar{\rightcauchygreentensor}^{2}}}
    \\ &=\dfrac{1}{2}\left(\pbrac{\bar{C}_{11}+\bar{C}_{22}+\bar{C}_{33}}^{2}\right. \\ &
    \quad\left.-\pbrac{\bar{C}_{11}^{2}+\bar{C}_{12}\bar{C}_{21}+\bar{C}_{13}\bar{C}_{31}+
      \bar{C}_{21}\bar{C}_{12}+\bar{C}_{22}^{2}+\bar{C}_{23}\bar{C}_{32}+\bar{C}_{31}\bar{C}_{13}+\bar{C}_{32}\bar{C}_{23}+\bar{C}_{33}^{2}}\right)
  \end{split}
\end{equation}

Note that for an incompressible material $\bar{I}_{3}=\determinant{\bar{\rightcauchygreentensor}}=1$.

Now
\begin{equation}
  \begin{split}
    \bar{\secondpiolakirchoffstresstensor}&=2\delby{\fnof{\bar{W}_{\Devop}}{\bar{\rightcauchygreentensor}}}{\bar{\rightcauchygreentensor}} \\
    &=2\pbrac{\delby{\fnof{\bar{W}_{\Devop}}{\bar{I}_{1},\bar{I}_{2}}}{\bar{I}_{1}}\delby{\bar{I}_{1}}{\bar{\rightcauchygreentensor}}+\delby{\fnof{\bar{W}_{\Devop}}{\bar{I}_{1},\bar{I}_{2}}}{\bar{I}_{2}}\delby{\bar{I}_{2}}{\bar{\rightcauchygreentensor}}}\\
    &=2\pbrac{c_{10}\begin{bmatrix}
        1 & 0 & 0 \\
        0 & 1 & 0 \\
        0 & 0 & 1
      \end{bmatrix}+c_{01}\begin{bmatrix}
        \bar{C}_{22}+\bar{C}_{33} & -\bar{C}_{21} & \bar{C}_{31} \\
        -\bar{C}_{12} & \bar{C}_{11}+\bar{C}_{33} & -\bar{C}_{32} \\
        -\bar{C}_{13} & -\bar{C}_{23} & \bar{C}_{11}+\bar{C}_{22}
    \end{bmatrix}} \\
    &=\begin{bmatrix}
    2c_{10}+2c_{01}\pbrac{\bar{C}_{22}+\bar{C}_{33}} & -2c_{01}\bar{C}_{21} & -2c_{01}\bar{C}_{31} \\
    -2c_{01}\bar{C}_{12} & 2c_{10}+2c_{01}\pbrac{\bar{C}_{11}+\bar{C}_{33}} & -2c_{01}\bar{C}_{32} \\
    -2c_{01}\bar{C}_{13} & -2c_{01}\bar{C}_{23} & 2c_{10}+2c_{01}\pbrac{\bar{C}_{11}+\bar{C}_{22}} 
    \end{bmatrix}
  \end{split}
\end{equation}

And so we have
\begin{equation}
  \Voigt{\bar{\secondpiolakirchoffstresstensor}}=2\begin{bmatrix}
  c_{10}+c_{01}J^{\frac{-2}{3}}\pbrac{\pbrac{1-\beta}^{2}+\pbrac{1-\gamma}^{2}} \\
  c_{10}+c_{01}J^{\frac{-2}{3}}\pbrac{\pbrac{1+\alpha}^{2}+\pbrac{1-\gamma}^{2}} \\
  c_{10}+c_{01}J^{\frac{-2}{3}}\pbrac{\pbrac{1+\alpha}^{2}+\pbrac{1-\beta}^{2}} \\
  0 \\
  0 \\
  0
  \end{bmatrix} 
\end{equation}

Now, to calculate the deviatoric projection tensor we have from \eqnref{eqn:SymmetricIdentityTensorFourDefinition}
\begin{equation}
  \symidentitytensorfour=\symtensorprod{\identitytensortwo}{\identitytensortwo}=\standardidentitytensorsymbol^{..kl.}_{ij..}\tensorprodfour{\generalbasevector^{i}}{\generalbasevector^{j}}{\generalbasevector_{k}}{\generalbasevector_{l}}=\dfrac{1}{2}\pbrac{\mixedkronecker{i}{.k}\mixedkronecker{j}{.l}+\mixedkronecker{i}{.l}\mixedkronecker{j}{.k}}\tensorprodfour{\generalbasevector_{i}}{\generalbasevector^{j}}{\generalbasevector_{k}}{\generalbasevector^{l}}
\end{equation}
or, in Voigt form,
\begin{equation}
  \Voigt{\sharptensor{\symidentitytensorfour}}=\begin{bmatrix}
  1 & 0 & 0 & 0 & 0 & 0 \\
  0 & 1 & 0 & 0 & 0 & 0 \\
  0 & 0 & 1 & 0 & 0 & 0 \\
  0 & 0 & 0 & \frac12 & 0 & 0 \\
  0 & 0 & 0 & 0 & \frac12 & 0 \\
  0 & 0 & 0 & 0 & 0 & \frac12
  \end{bmatrix}
\end{equation}

Now
\begin{equation}
  \tensorprod{\rightcauchygreentensor}{\pioladeformationtensor}=\tensorprod{\begin{bmatrix}
      \pbrac{1+\alpha}^{2} & 0 & 0 \\
      0 & \pbrac{1-\beta}^{2} & 0 \\
      0 & 0 & \pbrac{1-\gamma}^{2}
  \end{bmatrix}}{\begin{bmatrix}
      \dfrac{1}{\pbrac{1+\alpha}^{2}} & 0 & 0 \\
      0 & \dfrac{1}{\pbrac{1-\beta}^{2}} & 0 \\
      0 & 0 & \dfrac{1}{\pbrac{1-\gamma}^{2}}
  \end{bmatrix}}
\end{equation}
or
\begin{equation}
  \Voigt{\sharptensor{\pbrac{\tensorprod{\rightcauchygreentensor}{\pioladeformationtensor}}}}=\begin{bmatrix}
  1 & \dfrac{\pbrac{1+\alpha}^{2}}{\pbrac{1-\beta}^{2}} & \dfrac{\pbrac{1+\alpha}^{2}}{\pbrac{1-\gamma}^{2}} & 0 & 0 & 0 \\
  \dfrac{\pbrac{1-\beta}^{2}}{\pbrac{1+\alpha}^{2}} & 1 & \dfrac{\pbrac{1-\beta}^{2}}{\pbrac{1-\gamma}^{2}} & 0 & 0 & 0 \\
  \dfrac{\pbrac{1-\gamma}^{2}}{\pbrac{1+\alpha}^{2}} & \dfrac{\pbrac{1-\gamma}^{2}}{\pbrac{1-\beta}^{2}} & 1 & 0 & 0 & 0 \\
  0 & 0 & 0 & 0 & 0 & 0 \\
  0 & 0 & 0 & 0 & 0 & 0 \\
  0 & 0 & 0 & 0 & 0 & 0 
  \end{bmatrix}
\end{equation}
and so
\begin{equation}
  J^{-\frac{2}{3}}\Voigt{\sharptensor{\pbrac{\symidentitytensorfour-\dfrac{\tensorprod{\rightcauchygreentensor}{\pioladeformationtensor}}{3}}}}=J^{-\frac{2}{3}}\begin{bmatrix}
  \dfrac{2}{3} & -\dfrac{\pbrac{1+\alpha}^{2}}{3\pbrac{1-\beta}^{2}} & -\dfrac{\pbrac{1+\alpha}^{2}}{3\pbrac{1-\gamma}^{2}} & 0 & 0 & 0 \\
  -\dfrac{\pbrac{1-\beta}^{2}}{3\pbrac{1+\alpha}^{2}} & \dfrac{2}{3} & -\dfrac{\pbrac{1-\beta}^{2}}{3\pbrac{1-\gamma}^{2}} & 0 & 0 & 0 \\
  -\dfrac{\pbrac{1-\gamma}^{2}}{3\pbrac{1+\alpha}^{2}} & -\dfrac{\pbrac{1-\gamma}^{2}}{3\pbrac{1-\beta}^{2}} & \dfrac{2}{3} & 0 & 0 & 0 \\
  0 & 0 & 0 & \dfrac{1}{2} & 0 & 0 \\
  0 & 0 & 0 & 0 & \dfrac{1}{2} & 0 \\
  0 & 0 & 0 & 0 & 0 & \dfrac{1}{2} 
  \end{bmatrix}
\end{equation}

Now the psuedo Cauchy stress can also be calculated from
\begin{equation}
  \bar{\tensor{\sigma}}=\inverse{J}\doubledotprod{\pbrac{\tensorprod{\bar{\deformationgradienttensor}}{\bar{\deformationgradienttensor}}}}{\bar{\secondpiolakirchoffstresstensor}}\\
\end{equation}

In Voigt form this is given by
\begin{equation}
  \bar{\tensor{\sigma}}^{V}=\inverse{J}\begin{bmatrix}
  \bar{F}_{1}^{1}\bar{F}_{1}^{1} & \bar{F}_{2}^{1}\bar{F}_{2}^{1} & \bar{F}_{3}^{1}\bar{F}_{3}^{1} &
  \bar{F}_{1}^{1}\bar{F}_{2}^{1} + \bar{F}_{2}^{1}\bar{F}_{1}^{1} &
  \bar{F}_{1}^{1}\bar{F}_{3}^{1} + \bar{F}_{3}^{1}\bar{F}_{1}^{1} &
  \bar{F}_{2}^{1}\bar{F}_{3}^{1} + \bar{F}_{3}^{1}\bar{F}_{2}^{1} \\
  \bar{F}_{1}^{2}\bar{F}_{1}^{2} & \bar{F}_{2}^{2}\bar{F}_{2}^{2} & \bar{F}_{3}^{2}\bar{F}_{3}^{2} &
  \bar{F}_{1}^{2}\bar{F}_{2}^{2} + \bar{F}_{2}^{2}\bar{F}_{1}^{2} &
  \bar{F}_{1}^{2}\bar{F}_{3}^{2} + \bar{F}_{3}^{2}\bar{F}_{1}^{2} &
  \bar{F}_{2}^{2}\bar{F}_{3}^{2} + \bar{F}_{3}^{2}\bar{F}_{2}^{2} \\
  \bar{F}_{1}^{3}\bar{F}_{1}^{3} & \bar{F}_{2}^{3}\bar{F}_{2}^{3} & \bar{F}_{3}^{3}\bar{F}_{3}^{3} &
  \bar{F}_{1}^{3}\bar{F}_{2}^{3} + \bar{F}_{2}^{3}\bar{F}_{1}^{3} &
  \bar{F}_{1}^{3}\bar{F}_{3}^{3} + \bar{F}_{3}^{3}\bar{F}_{1}^{3} &
  \bar{F}_{2}^{3}\bar{F}_{3}^{3} + \bar{F}_{3}^{3}\bar{F}_{2}^{3} \\
  \bar{F}_{1}^{1}\bar{F}_{1}^{2} & \bar{F}_{2}^{1}\bar{F}_{2}^{2} & \bar{F}_{3}^{1}\bar{F}_{3}^{2} &
  \bar{F}_{1}^{1}\bar{F}_{2}^{2} + \bar{F}_{2}^{1}\bar{F}_{1}^{2} &
  \bar{F}_{1}^{1}\bar{F}_{3}^{2} + \bar{F}_{3}^{1}\bar{F}_{1}^{2} &
  \bar{F}_{2}^{1}\bar{F}_{3}^{2} + \bar{F}_{3}^{1}\bar{F}_{2}^{2} \\
  \bar{F}_{1}^{1}\bar{F}_{1}^{3} & \bar{F}_{2}^{1}\bar{F}_{2}^{3} & \bar{F}_{3}^{1}\bar{F}_{3}^{3} &
  \bar{F}_{1}^{1}\bar{F}_{2}^{3} + \bar{F}_{2}^{1}\bar{F}_{1}^{3} &
  \bar{F}_{1}^{1}\bar{F}_{3}^{3} + \bar{F}_{3}^{1}\bar{F}_{1}^{3} &
  \bar{F}_{2}^{1}\bar{F}_{3}^{3} + \bar{F}_{3}^{1}\bar{F}_{2}^{3} \\
  \bar{F}_{1}^{2}\bar{F}_{1}^{3} & \bar{F}_{2}^{2}\bar{F}_{2}^{3} & \bar{F}_{3}^{2}\bar{F}_{3}^{3} &
  \bar{F}_{1}^{2}\bar{F}_{2}^{3} + \bar{F}_{2}^{2}\bar{F}_{1}^{3} &
  \bar{F}_{1}^{2}\bar{F}_{3}^{3} + \bar{F}_{3}^{2}\bar{F}_{1}^{3} &
  \bar{F}_{2}^{2}\bar{F}_{3}^{3} + \bar{F}_{3}^{2}\bar{F}_{2}^{3} 
  \end{bmatrix}\begin{bmatrix}
    \bar{S}^{11} \\
    \bar{S}^{22} \\
    \bar{S}^{33} \\
    \bar{S}^{12} \\
    \bar{S}^{13} \\
    \bar{S}^{23}
  \end{bmatrix}
\end{equation}

For the unit cube case this is
\begin{equation}
  \begin{split}
    \bar{\tensor{\sigma}}^{V}&=\inverse{J}\begin{bmatrix}
    \frac{\cuberoot{\pbrac{1+\alpha}^{4}}}{\cuberoot{\pbrac{1-\beta}^{2}\pbrac{1-\gamma}^{2}}}
    & 0 & 0 & 0 & 0 & 0\\
    0 &
    \frac{\cuberoot{\pbrac{1-\beta}^{4}}}{\cuberoot{\pbrac{1+\alpha}^{2}\pbrac{1-\gamma}^{2}}}
    & 0 & 0 & 0 & 0\\
    0 & 0 & 
    \frac{\cuberoot{\pbrac{1-\gamma}^{4}}}{\cuberoot{\pbrac{1+\alpha}^{2}\pbrac{1-\beta}^{2}}}
    & 0 & 0 & 0\\
    0 & 0 & 0 & 1 & 0 & 0 \\
    0 & 0 & 0 & 0 & 1 & 0 \\
    0 & 0 & 0 & 0 & 0 & 1 
    \end{bmatrix}\begin{bmatrix}
      2c_{1}+2c_{2}\pbrac{\frac{\pbrac{1-\beta}^{2}+\pbrac{1-\gamma}^{2}}{\cuberoot{\pbrac{1+\alpha}^{2}\pbrac{1-\beta}^{2}\pbrac{1-\gamma}^{2}}}}\\
      2c_{1}+2c_{2}\pbrac{\frac{\pbrac{1+\alpha}^{2}+\pbrac{1-\gamma}^{2}}{\cuberoot{\pbrac{1+\alpha}^{2}\pbrac{1-\beta}^{2}\pbrac{1-\gamma}^{2}}}}\\
      2c_{1}+2c_{2}\pbrac{\frac{\pbrac{1+\alpha}^{2}+\pbrac{1-\beta}^{2}}{\cuberoot{\pbrac{1+\alpha}^{2}\pbrac{1-\beta}^{2}\pbrac{1-\gamma}^{2}}}}\\
      0 \\
      0 \\
      0
    \end{bmatrix}\\
    &=\inverse{J}J^{-\frac{2}{3}}\begin{bmatrix}
    \pbrac{1+\alpha}^{2} & 0 & 0 & 0 & 0 & 0\\
    0 & \pbrac{1-\beta}^{2} & 0 & 0 & 0 & 0\\
    0 & 0 & \pbrac{1-\gamma}^{2} & 0 & 0 & 0\\
    0 & 0 & 0 & 1 & 0 & 0 \\
    0 & 0 & 0 & 0 & 1 & 0 \\
    0 & 0 & 0 & 0 & 0 & 1 
    \end{bmatrix}\begin{bmatrix}
      2c_{1}+2c_{2}J^{-\frac{2}{3}}\pbrac{\pbrac{1-\beta}^{2}+\pbrac{1-\gamma}^{2}}\\
      2c_{1}+2c_{2}J^{-\frac{2}{3}}\pbrac{\pbrac{1+\alpha}^{2}+\pbrac{1-\gamma}^{2}}\\
      2c_{1}+2c_{2}J^{-\frac{2}{3}}\pbrac{\pbrac{1+\alpha}^{2}+\pbrac{1-\beta}^{2}} \\
      0 \\
      0 \\
      0
    \end{bmatrix}\\
    &=\begin{bmatrix}
      \pbrac{1+\alpha}^{2}J^{-\frac{5}{3}}\pbrac{2c_{1}+2c_{2}J^{-\frac{2}{3}}\pbrac{\pbrac{1-\beta}^{2}+\pbrac{1-\gamma}^{2}}} \\
      \pbrac{1-\beta}^{2}J^{-\frac{5}{3}}\pbrac{2c_{1}+2c_{2}J^{-\frac{2}{3}}\pbrac{\pbrac{1+\alpha}^{2}+\pbrac{1-\gamma}^{2}}} \\
      \pbrac{1-\gamma}^{2}J^{-\frac{5}{3}}\pbrac{2c_{1}+2c_{2}J^{-\frac{2}{3}}\pbrac{\pbrac{1+\alpha}^{2}+\pbrac{1-\beta}^{2}}} \\
       0\\
       0\\
      0
    \end{bmatrix}
  \end{split}
\end{equation}

The deviatoric Cauchy stress tensor is
\begin{equation}
  \tensor{\sigma}_{\devop}=J^{-\frac{2}{3}}\doubledotprod{\spatialdeviatorictensorfour}{\bar{\tensor{\sigma}}}
\end{equation}
In Voigt form this is given by
\begin{equation}
  \begin{split}
    \tensor{\sigma}_{\devop}^{V}&=J^{-\frac{4}{3}}\begin{bmatrix}
      \frac{2}{3} & -\frac{1}{3} & -\frac{1}{3} & 0 & 0 & 0\\
      -\frac{1}{3} & \frac{2}{3} & -\frac{1}{3} & 0 & 0 & 0\\
      -\frac{1}{3} & -\frac{1}{3} & \frac{2}{3} & 0 & 0 & 0\\
      0 & 0 & 0 & 1 & 0 & 0 \\
      0 & 0 & 0 & 0 & 1 & 0 \\
      0 & 0 & 0 & 0 & 0 & 1
    \end{bmatrix}\begin{bmatrix}
      \bar{\sigma}^{11} \\
      \bar{\sigma}^{22} \\
      \bar{\sigma}^{33} \\
      \bar{\sigma}^{23} \\
      \bar{\sigma}^{13} \\
      \bar{\sigma}^{12}
    \end{bmatrix} \\
    &=J^{-\frac{2}{3}}\begin{bmatrix}
      \frac{2}{3}\bar{\sigma}^{11}-\frac{1}{3}\bar{\sigma}^{22}-\frac{1}{3}\bar{\sigma}^{33} \\
      -\frac{1}{3}\bar{\sigma}^{11}+\frac{2}{3}\bar{\sigma}^{22}-\frac{1}{3}\bar{\sigma}^{33} \\
      -\frac{1}{3}\bar{\sigma}^{11}-\frac{1}{3}\bar{\sigma}^{22}+\frac{2}{3}\bar{\sigma}^{33} \\
      \bar{\sigma}^{23} \\
      \bar{\sigma}^{13} \\
      \bar{\sigma}^{12}
    \end{bmatrix} \\
    &=J^{-\frac{2}{3}}\begin{bmatrix}
      \bar{\sigma}^{11}-\frac{1}{3}\bar{\sigma}^{11}-\frac{1}{3}\bar{\sigma}^{22}-\frac{1}{3}\bar{\sigma}^{33} \\
      \bar{\sigma}^{22}-\frac{1}{3}\bar{\sigma}^{11}-\frac{1}{3}\bar{\sigma}^{22}-\frac{1}{3}\bar{\sigma}^{33} \\
      \bar{\sigma}^{33}-\frac{1}{3}\bar{\sigma}^{11}-\frac{1}{3}\bar{\sigma}^{22}-\frac{1}{3}\bar{\sigma}^{33} \\
      \bar{\sigma}^{23} \\
      \bar{\sigma}^{13} \\
      \bar{\sigma}^{12} 
    \end{bmatrix} \\
    &=J^{-\frac{2}{3}}\begin{bmatrix}
      \bar{\sigma}^{11}-\frac{1}{3}\trace{}{\bar{\tensor{\sigma}}} \\
      \bar{\sigma}^{22}-\frac{1}{3}\trace{}{\bar{\tensor{\sigma}}} \\
      \bar{\sigma}^{33}-\frac{1}{3}\trace{}{\bar{\tensor{\sigma}}} \\
      \bar{\sigma}^{23} \\
      \bar{\sigma}^{13} \\
      \bar{\sigma}^{12}
    \end{bmatrix}
  \end{split}
\end{equation}
or
\begin{equation}
  \tensor{\sigma}_{\devop}=J^{-\frac{2}{3}}\pbrac{\bar{\tensor{\sigma}}-\frac{1}{3}\trace{}{\bar{\tensor{\sigma}}}\tensor{i}}
\end{equation}
We thus need
\begin{equation}
  \begin{split}
    \frac{1}{3}\trace{}{\bar{\tensor{\sigma}}}&=\frac{1}{3}\left(\pbrac{1+\alpha}^{2}J^{-\frac{5}{3}}\pbrac{2c_{1}+2c_{2}J^{-\frac{2}{3}}\pbrac{\pbrac{1-\beta}^{2}+\pbrac{1-\gamma}^{2}}}\right.\\
    &\quad+\pbrac{1-\beta}^{2}J^{-\frac{5}{3}}\pbrac{2c_{1}+2c_{2}J^{-\frac{2}{3}}\pbrac{\pbrac{1+\alpha}^{2}+\pbrac{1-\gamma}^{2}}}\\
    &\quad\quad+\left.\pbrac{1-\gamma}^{2}J^{-\frac{5}{3}}\pbrac{2c_{1}+2c_{2}J^{-\frac{2}{3}}\pbrac{\pbrac{1+\alpha}^{2}+\pbrac{1-\beta}^{2}}}\right)\\
    &=\dfrac{2}{3}J^{-\frac{5}{3}}\left(\pbrac{1+\alpha}^{2}\pbrac{c_{1}+c_{2}\pbrac{\pbrac{1-\beta}^{2}+\pbrac{1-\gamma}^{2}}}\right.\\
    &\quad+\pbrac{1-\beta}^{2}\pbrac{c_{1}+c_{2}\pbrac{\pbrac{1+\alpha}^{2}+\pbrac{1-\gamma}^{2}}}\\
    &\quad\quad\left.+\pbrac{1-\gamma}^{2}\pbrac{c_{1}+c_{2}\pbrac{\pbrac{1+\alpha}^{2}+\pbrac{1-\beta}^{2}}}\right)
  \end{split}
\end{equation}
which leads to (in Voigt form)
\begin{equation}
  \pbrac{\bar{\tensor{\sigma}}-\frac{1}{3}\trace{}{\bar{\tensor{\sigma}}}\tensor{i}}^{V}=\dfrac{2}{3}J^{-\frac{5}{3}}\begin{bmatrix}
    2\pbrac{1+\alpha}^{2}\pbrac{c_{1}+c_{2}\pbrac{\pbrac{1-\beta}^{2}+\pbrac{1-\gamma}^{2}}}\\
    \hspace{1cm}-\pbrac{1-\beta}^{2}\pbrac{c_{1}+c_{2}\pbrac{\pbrac{1+\alpha}^{2}+\pbrac{1-\gamma}^{2}}}\\
    \hspace{2cm}-\pbrac{1-\gamma}^{2}\pbrac{c_{1}+c_{2}\pbrac{\pbrac{1+\alpha}^{2}+\pbrac{1-\beta}^{2}}} \\
    -\pbrac{1+\alpha}^{2}\pbrac{c_{1}+c_{2}\pbrac{\pbrac{1+\beta}^{2}+\pbrac{1-\gamma}^{2}}}\\
    \hspace{1cm}+2\pbrac{1-\beta}^{2}\pbrac{c_{1}+c_{2}\pbrac{\pbrac{1+\alpha}^{2}+\pbrac{1-\gamma}^{2}}}\\
    \hspace{2cm}-\pbrac{1-\gamma}^{2}\pbrac{c_{1}+c_{2}\pbrac{\pbrac{1+\alpha}^{2}+\pbrac{1-\beta}^{2}}} \\
    -\pbrac{1+\alpha}^{2}\pbrac{c_{1}+c_{2}\pbrac{\pbrac{1+\beta}^{2}+\pbrac{1-\gamma}^{2}}}\\
    \hspace{1cm}-\pbrac{1-\beta}^{2}\pbrac{c_{1}+c_{2}\pbrac{\pbrac{1+\alpha}^{2}+\pbrac{1-\gamma}^{2}}}\\
    \hspace{2cm}+2\pbrac{1-\gamma}^{2}\pbrac{c_{1}+c_{2}\pbrac{\pbrac{1+\alpha}^{2}+\pbrac{1-\beta}^{2}}}\\
    0 \\
    0 \\
    0
  \end{bmatrix}
\end{equation}
and therefore
\begin{equation}
  \begin{split}
    \tensor{\sigma}_{\devop}&=J^{-\frac{4}{3}}\doubledotprod{\spatialdeviatorictensorfour}{\bar{\tensor{\sigma}}}=J^{-\frac{4}{3}}\pbrac{\bar{\tensor{\sigma}}-\frac{1}{3}\trace{}{\bar{\tensor{\sigma}}}\tensor{i}}\\
    &=\dfrac{2}{3}J^{-3}\begin{bmatrix}
      2\pbrac{1+\alpha}^{2}\pbrac{c_{1}+c_{2}\pbrac{\pbrac{1-\beta}^{2}+\pbrac{1-\gamma}^{2}}}\\
      \hspace{1cm}-\pbrac{1-\beta}^{2}\pbrac{c_{1}+c_{2}\pbrac{\pbrac{1+\alpha}^{2}+\pbrac{1-\gamma}^{2}}}\\
      \hspace{2cm}-\pbrac{1-\gamma}^{2}\pbrac{c_{1}+c_{2}\pbrac{\pbrac{1+\alpha}^{2}+\pbrac{1-\beta}^{2}}} \\
      -\pbrac{1+\alpha}^{2}\pbrac{c_{1}+c_{2}\pbrac{\pbrac{1+\beta}^{2}+\pbrac{1-\gamma}^{2}}}\\
      \hspace{1cm}+2\pbrac{1-\beta}^{2}\pbrac{c_{1}+c_{2}\pbrac{\pbrac{1+\alpha}^{2}+\pbrac{1-\gamma}^{2}}}\\
      \hspace{2cm}-\pbrac{1-\gamma}^{2}\pbrac{c_{1}+c_{2}\pbrac{\pbrac{1+\alpha}^{2}+\pbrac{1-\beta}^{2}}} \\
      -\pbrac{1+\alpha}^{2}\pbrac{c_{1}+c_{2}\pbrac{\pbrac{1+\beta}^{2}+\pbrac{1-\gamma}^{2}}}\\
      \hspace{1cm}-\pbrac{1-\beta}^{2}\pbrac{c_{1}+c_{2}\pbrac{\pbrac{1+\alpha}^{2}+\pbrac{1-\gamma}^{2}}}\\
      \hspace{2cm}+2\pbrac{1-\gamma}^{2}\pbrac{c_{1}+c_{2}\pbrac{\pbrac{1+\alpha}^{2}+\pbrac{1-\beta}^{2}}}\\
      0 \\
      0 \\
      0
    \end{bmatrix}
  \end{split}
\end{equation}

The spherical cauchy tensor is
\begin{equation}
  \tensor{\sigma}_{\sphop}=-p\tensor{i}=\begin{bmatrix}
  -p & 0 & 0\\
  0 & -p & 0\\
  0 & 0 & -p
  \end{bmatrix}
\end{equation}

Now the second material elasticity tensor is given by
\begin{equation}
  \bar{\materialsecondelasticitytensor}=2\delby{\bar{\secondpiolakirchoffstresstensor}}{\bar{\rightcauchygreentensor}}=4\deltwosqby{\fnof{\bar{W}}{\bar{I}_{1},\bar{I}_{2}}}{\bar{\rightcauchygreentensor}}
\end{equation}

In Voigt notation we have
\begin{equation}
  \bar{\materialsecondelasticitytensor}^{V}=\begin{bmatrix}
  0 & 4c_{2} & 4c_{2} & 0 & 0 & 0 \\
  4c_{2} & 0 & 4c_{2} & 0 & 0 & 0 \\
  4c_{2} & 4c_{2} & 0 & 0 & 0 & 0 \\
  0     & 0      & 0 & -4c_{2} & 0 & 0 \\
  0     & 0      & 0 & 0      & -4c_{2} & 0 \\
  0     & 0      & 0 & 0      & 0      & -4c_{2} 
  \end{bmatrix}
\end{equation}

The second spatial elasticity tensor is given by
\begin{equation}
  \bar{\spatialsecondelasticitytensor}=\inverse{J}\pushforward{\chi}{\bar{\materialsecondelasticitytensor}}=\inverse{J}\doubledotprodthree{\pbrac{\tensorprod{\bar{\deformationgradienttensor}}{\bar{\deformationgradienttensor}}}}{\bar{\materialsecondelasticitytensor}}{\transpose{\pbrac{\tensorprod{\bar{\deformationgradienttensor}}{\bar{\deformationgradienttensor}}}}}
\end{equation}

In Voigt notation for the uniaxial cube case we have
\begin{multline}
  \bar{\spatialsecondelasticitytensor}^{V}=\inverse{J}J^{-\frac{2}{3}}J^{-\frac{2}{3}}\begin{bmatrix}
    \pbrac{1+\alpha}^{2} & 0 & 0 & 0 & 0 & 0\\
    0 & \pbrac{1-\beta}^{2} & 0 & 0 & 0 & 0\\
    0 & 0 & \pbrac{1-\gamma}^{2} & 0 & 0 & 0\\
    0 & 0 & 0 & 1 & 0 & 0 \\
    0 & 0 & 0 & 0 & 1 & 0 \\
    0 & 0 & 0 & 0 & 0 & 1 
  \end{bmatrix}\\\hspace{2cm}\begin{bmatrix}
    0 & 4c_{2} & 4c_{2} & 0 & 0 & 0 \\
    4c_{2} & 0 & 4c_{2} & 0 & 0 & 0 \\
    4c_{2} & 4c_{2} & 0 & 0 & 0 & 0 \\
    0     & 0      & 0 & -4c_{2} & 0 & 0 \\
    0     & 0      & 0 & 0      & -4c_{2} & 0 \\
    0     & 0      & 0 & 0      & 0      & -4c_{2} 
  \end{bmatrix}\\\hspace{4cm}\begin{bmatrix}
    \pbrac{1+\alpha}^{2} & 0 & 0 & 0 & 0 & 0\\
    0 & \pbrac{1-\beta}^{2} & 0 & 0 & 0 & 0\\
    0 & 0 & \pbrac{1-\gamma}^{2} & 0 & 0 & 0\\
    0 & 0 & 0 & 1 & 0 & 0 \\
    0 & 0 & 0 & 0 & 1 & 0 \\
    0 & 0 & 0 & 0 & 0 & 1 
  \end{bmatrix}\\
\end{multline}
that is,
\begin{multline}
  \bar{\spatialsecondelasticitytensor}^{V}=J^{-\frac{7}{3}}\begin{bmatrix}
    0 & 4c_{2}\pbrac{1+\alpha}^{2} & 4c_{2}\pbrac{1+\alpha}^{2} & 0 & 0 & 0\\
    4c_{2}\pbrac{1-\beta}^{2} & 0 & 4c_{2}\pbrac{1-\beta}^{2} & 0 & 0 & 0\\
    4c_{2}\pbrac{1-\gamma}^{2} & 4c_{2}\pbrac{1-\gamma}^{2} & 0 & 0 & 0 & 0\\
    0 & 0 & 0 & -4c_{2} & 0 & 0 \\
    0 & 0 & 0 & 0 & -4c_{2} & 0 \\
    0 & 0 & 0 & 0 & 0 & -4c_{2} 
  \end{bmatrix}\\
  \begin{bmatrix}
    \pbrac{1+\alpha}^{2} & 0 & 0 & 0 & 0 & 0\\
    0 & \pbrac{1-\beta}^{2} & 0 & 0 & 0 & 0\\
    0 & 0 & \pbrac{1-\gamma}^{2} & 0 & 0 & 0\\
    0 & 0 & 0 & 1 & 0 & 0 \\
    0 & 0 & 0 & 0 & 1 & 0 \\
    0 & 0 & 0 & 0 & 0 & 1 
  \end{bmatrix}
\end{multline}
or
\begin{equation}
  \bar{\spatialsecondelasticitytensor}^{V}=J^{-\frac{7}{3}}\begin{bmatrix}
    0 & 4c_{2}\pbrac{1+\alpha}^{2}\pbrac{1-\beta}^{2} & 4c_{2}\pbrac{1+\alpha}^{2}\pbrac{1-\gamma}^{2} & 0 & 0 & 0\\
    4c_{2}\pbrac{1+\alpha}^{2}\pbrac{1-\beta}^{2} & 0 & 4c_{2}\pbrac{1-\beta}^{2}\pbrac{1-\gamma}^{2} & 0 & 0 & 0\\
    4c_{2}\pbrac{1+\alpha}^{2}\pbrac{1-\gamma}^{2} & 4c_{2}\pbrac{1-\beta}^{2}\pbrac{1-\gamma}^{2} & 0 & 0 & 0 & 0\\
    0 & 0 & 0 & -4c_{2} & 0 & 0 \\
    0 & 0 & 0 & 0 & -4c_{2} & 0 \\
    0 & 0 & 0 & 0 & 0 & -4c_{2} 
  \end{bmatrix}
\end{equation}


We also have
\begin{equation}
  \spatialsecondelasticitytensor_{e_{2}}=\dfrac{2}{3}J^{-\frac{2}{3}}\trop\bar{\tensortwo{\sigma}}\spatialdeviatorictensorfour^{\sharp}
  -\dfrac{2}{3}\sqbrac{\tensorprod{\inverse{\tensortwo{g}}}{\tensor{\sigma}_{\devop}}+\tensorprod{\tensortwo{\sigma}_{\devop}}{\inverse{\tensortwo{g}}}}
\end{equation}

Working in Voigt notation we have
\begin{equation}
  \begin{split}
    \pbrac{\dfrac{2}{3}J^{-\frac{2}{3}}\trop\bar{\tensortwo{\sigma}}\spatialdeviatorictensorfour^{\sharp}}^{V}&=\dfrac{2}{3}J^{-\frac{2}{3}}\trop\bar{\tensortwo{\sigma}}\begin{bmatrix}
      \frac{2}{3} & -\frac{1}{3} & -\frac{1}{3} & 0 & 0 & 0 \\
      -\frac{1}{3} & \frac{2}{3} & -\frac{1}{3} & 0 & 0 & 0 \\
      -\frac{1}{3} & -\frac{1}{3} & \frac{2}{3} & 0 & 0 & 0 \\
      0 & 0 & 0 & \frac{1}{2} & 0 & 0 \\
      0 & 0 & 0 & 0 & \frac{1}{2} & 0 \\
      0 & 0 & 0 & 0 & 0 & \frac{1}{2}
    \end{bmatrix} \\
    &=\dfrac{2}{3}J^{-\frac{2}{3}}\begin{bmatrix}
       \frac{2}{3}\trace{}{\bar{\tensor{\sigma}}} &
      -\frac{1}{3}\trace{}{\bar{\tensor{\sigma}}} &
      -\frac{1}{3}\trace{}{\bar{\tensor{\sigma}}} & 0 & 0 & 0 \\
      -\frac{1}{3}\trace{}{\bar{\tensor{\sigma}}} &
       \frac{2}{3}\trace{}{\bar{\tensor{\sigma}}} &
      -\frac{1}{3}\trace{}{\bar{\tensor{\sigma}}} & 0 & 0 & 0 \\
      -\frac{1}{3}\trace{}{\bar{\tensor{\sigma}}} &
      -\frac{1}{3}\trace{}{\bar{\tensor{\sigma}}} &
       \frac{2}{3}\trace{}{\bar{\tensor{\sigma}}} & 0 & 0 & 0 \\
       0 & 0 & 0 & \frac{1}{2}\trace{}{\bar{\tensor{\sigma}}} & 0 & 0 \\
       0 & 0 & 0 & 0 & \frac{1}{2}\trace{}{\bar{\tensor{\sigma}}} & 0 \\
       0 & 0 & 0 & 0 & 0 & \frac{1}{2}\trace{}{\bar{\tensor{\sigma}}} \\
    \end{bmatrix} 
  \end{split}
\end{equation}
and
\begin{equation}
  \begin{split}
    \pbrac{\tensorprod{\inverse{\tensortwo{g}}}{\tensor{\sigma}_{\devop}}}^{V}&=\begin{bmatrix}
    1 \\
    1 \\
    1 \\
    0 \\
    0 \\
    0
    \end{bmatrix}\begin{bmatrix}
      \sigma_{\devop}^{11} & \sigma_{\devop}^{22} & \sigma_{\devop}^{33} & \sigma_{\devop}^{23} & \sigma_{\devop}^{13} & \sigma_{\devop}^{12}
    \end{bmatrix}\\
    &=\begin{bmatrix}
    \sigma_{\devop}^{11} & \sigma_{\devop}^{22} & \sigma_{\devop}^{33} &
    \sigma_{\devop}^{23} & \sigma_{\devop}^{13} & \sigma_{\devop}^{12} \\
    \sigma_{\devop}^{11} & \sigma_{\devop}^{22} & \sigma_{\devop}^{33} &
    \sigma_{\devop}^{23} & \sigma_{\devop}^{13} & \sigma_{\devop}^{12} \\
    \sigma_{\devop}^{11} & \sigma_{\devop}^{22} & \sigma_{\devop}^{33} &
    \sigma_{\devop}^{23} & \sigma_{\devop}^{13} & \sigma_{\devop}^{12} \\
    0 & 0 & 0 & 0 & 0 & 0 \\
    0 & 0 & 0 & 0 & 0 & 0 \\
    0 & 0 & 0 & 0 & 0 & 0
    \end{bmatrix}
  \end{split} 
\end{equation}
and
\begin{equation}
  \begin{split}
    \pbrac{\tensorprod{\tensor{\sigma}_{\devop}}{\inverse{\tensortwo{g}}}}^{V}&=\begin{bmatrix}
    \sigma_{\devop}^{11} \\
    \sigma_{\devop}^{22} \\
    \sigma_{\devop}^{33} \\
    \sigma_{\devop}^{23} \\
    \sigma_{\devop}^{13} \\
    \sigma_{\devop}^{12}
    \end{bmatrix}\begin{bmatrix}
      1 & 1 & 1 & 0 & 0 & 0
    \end{bmatrix}\\
    &=\begin{bmatrix}
    \sigma_{\devop}^{11} & \sigma_{\devop}^{11} & \sigma_{\devop}^{11} & 0 & 0 & 0\\
    \sigma_{\devop}^{22} & \sigma_{\devop}^{22} & \sigma_{\devop}^{22} & 0 & 0 & 0\\
    \sigma_{\devop}^{33} & \sigma_{\devop}^{33} & \sigma_{\devop}^{33} & 0 & 0 & 0\\
    \sigma_{\devop}^{23} & \sigma_{\devop}^{23} & \sigma_{\devop}^{23} & 0 & 0 & 0\\
    \sigma_{\devop}^{13} & \sigma_{\devop}^{13} & \sigma_{\devop}^{13} & 0 & 0 & 0\\
    \sigma_{\devop}^{12} & \sigma_{\devop}^{12} & \sigma_{\devop}^{12} & 0 & 0 & 0
    \end{bmatrix}
  \end{split}
\end{equation}
and so
\begin{equation}
  \begin{split}
    \left(\dfrac{2}{3}\left[\tensorprod{\inverse{\tensortwo{g}}}{\tensor{\sigma}_{\devop}}\right.\right.&+\left.\left.\tensorprod{\tensortwo{\sigma}_{\devop}}{\inverse{\tensortwo{g}}}\right]\right)^{V}\\
    &=\dfrac{2}{3}\pbrac{\begin{bmatrix}
        \sigma_{\devop}^{11} & \sigma_{\devop}^{22} & \sigma_{\devop}^{33} &
    \sigma_{\devop}^{23} & \sigma_{\devop}^{13} & \sigma_{\devop}^{12} \\
    \sigma_{\devop}^{11} & \sigma_{\devop}^{22} & \sigma_{\devop}^{33} &
    \sigma_{\devop}^{23} & \sigma_{\devop}^{13} & \sigma_{\devop}^{12} \\
    \sigma_{\devop}^{11} & \sigma_{\devop}^{22} & \sigma_{\devop}^{33} &
    \sigma_{\devop}^{23} & \sigma_{\devop}^{13} & \sigma_{\devop}^{12} \\
    0 & 0 & 0 & 0 & 0 & 0 \\
    0 & 0 & 0 & 0 & 0 & 0 \\
    0 & 0 & 0 & 0 & 0 & 0
    \end{bmatrix}+\begin{bmatrix}
    \sigma_{\devop}^{11} & \sigma_{\devop}^{11} & \sigma_{\devop}^{11} & 0 & 0 & 0\\
    \sigma_{\devop}^{22} & \sigma_{\devop}^{22} & \sigma_{\devop}^{22} & 0 & 0 & 0\\
    \sigma_{\devop}^{33} & \sigma_{\devop}^{33} & \sigma_{\devop}^{33} & 0 & 0 & 0\\
    \sigma_{\devop}^{23} & \sigma_{\devop}^{23} & \sigma_{\devop}^{23} & 0 & 0 & 0\\
    \sigma_{\devop}^{13} & \sigma_{\devop}^{13} & \sigma_{\devop}^{13} & 0 & 0 & 0\\
    \sigma_{\devop}^{12} & \sigma_{\devop}^{12} & \sigma_{\devop}^{12} & 0 & 0 & 0
    \end{bmatrix}}\\
    &=\dfrac{2}{3}\begin{bmatrix}
     2\sigma_{\devop}^{11} & \sigma_{\devop}^{11}+\sigma_{\devop}^{22} &
      \sigma_{\devop}^{11}+\sigma_{\devop}^{33} & \sigma_{\devop}^{23} &
      \sigma_{\devop}^{13} & \sigma_{\devop}^{12} \\
      \sigma_{\devop}^{11}+\sigma_{\devop}^{22} & 2\sigma_{\devop}^{22} &
      \sigma_{\devop}^{22}+\sigma_{\devop}^{33} & \sigma_{\devop}^{23} &
      \sigma_{\devop}^{13} & \sigma_{\devop}^{12} \\
      \sigma_{\devop}^{11}+\sigma_{\devop}^{33} & \sigma_{\devop}^{22}+\sigma_{\devop}^{33} &
      2\sigma_{\devop}^{33} & \sigma_{\devop}^{23} &
      \sigma_{\devop}^{13} & \sigma_{\devop}^{12} \\
      \sigma_{\devop}^{23} & \sigma_{\devop}^{23} & \sigma_{\devop}^{23} & 0 &
      0 & 0 \\
      \sigma_{\devop}^{13} & \sigma_{\devop}^{13} & \sigma_{\devop}^{13} & 0 &
      0 & 0 \\
      \sigma_{\devop}^{12} & \sigma_{\devop}^{12} & \sigma_{\devop}^{12} & 0 &
      0 & 0
    \end{bmatrix} \\
    &=\dfrac{2}{3}J^{-\frac{2}{3}}\begin{bmatrix}
      2\bar{\sigma}^{11}-\frac{2}{3}\trace{}{\bar{\tensor{\sigma}}} &
      \bar{\sigma}^{11}+\bar{\sigma}^{22}-\frac{2}{3}\trace{}{\bar{\tensor{\sigma}}}
      &\bar{\sigma}^{11}+\bar{\sigma}^{33}-\frac{2}{3}\trace{}{\bar{\tensor{\sigma}}}
      & \bar{\sigma}^{23}& \bar{\sigma}^{13}& \bar{\sigma}^{12}\\
      \bar{\sigma}^{11}+\bar{\sigma}^{22}-\frac{2}{3}\trace{}{\bar{\tensor{\sigma}}} &
      2\bar{\sigma}^{22}-\frac{2}{3}\trace{}{\bar{\tensor{\sigma}}}
      &\bar{\sigma}^{22}+\bar{\sigma}^{33}-\frac{2}{3}\trace{}{\bar{\tensor{\sigma}}}
      & \bar{\sigma}^{23}& \bar{\sigma}^{13}& \bar{\sigma}^{12}\\
      \bar{\sigma}^{11}+\bar{\sigma}^{22}-\frac{2}{3}\trace{}{\bar{\tensor{\sigma}}} &
      \bar{\sigma}^{22}+\bar{\sigma}^{33}-\frac{2}{3}\trace{}{\bar{\tensor{\sigma}}}
      &2\bar{\sigma}^{33}-\frac{2}{3}\trace{}{\bar{\tensor{\sigma}}}
      & \bar{\sigma}^{23}& \bar{\sigma}^{13}& \bar{\sigma}^{12}\\     
      \bar{\sigma}^{23} & \bar{\sigma}^{23} & \bar{\sigma}^{23} & 0 & 0 & 0\\
      \bar{\sigma}^{13} & \bar{\sigma}^{13} & \bar{\sigma}^{13} & 0 & 0 & 0\\
      \bar{\sigma}^{12} & \bar{\sigma}^{12} & \bar{\sigma}^{12} & 0 & 0 & 0
    \end{bmatrix}
  \end{split}
\end{equation}

We thus have
\begin{equation}
  \begin{split}
    \spatialsecondelasticitytensor_{e_{2}}^{V}&=\dfrac{2}{3}J^{-\frac{2}{3}}\left(\begin{bmatrix}
       \frac{2}{3}\trace{}{\bar{\tensor{\sigma}}} &
      -\frac{1}{3}\trace{}{\bar{\tensor{\sigma}}} &
      -\frac{1}{3}\trace{}{\bar{\tensor{\sigma}}} & 0 & 0 & 0 \\
      -\frac{1}{3}\trace{}{\bar{\tensor{\sigma}}} &
       \frac{2}{3}\trace{}{\bar{\tensor{\sigma}}} &
      -\frac{1}{3}\trace{}{\bar{\tensor{\sigma}}} & 0 & 0 & 0 \\
      -\frac{1}{3}\trace{}{\bar{\tensor{\sigma}}} &
      -\frac{1}{3}\trace{}{\bar{\tensor{\sigma}}} &
       \frac{2}{3}\trace{}{\bar{\tensor{\sigma}}} & 0 & 0 & 0 \\
       0 & 0 & 0 & \frac{1}{2}\trace{}{\bar{\tensor{\sigma}}} & 0 & 0 \\
       0 & 0 & 0 & 0 & \frac{1}{2}\trace{}{\bar{\tensor{\sigma}}} & 0 \\
       0 & 0 & 0 & 0 & 0 & \frac{1}{2}\trace{}{\bar{\tensor{\sigma}}} 
    \end{bmatrix}\right. \\
    &\qquad\qquad\qquad\left.-\begin{bmatrix}
      2\bar{\sigma}^{11}-\frac{2}{3}\trace{}{\bar{\tensor{\sigma}}} &
      \bar{\sigma}^{11}+\bar{\sigma}^{22}-\frac{2}{3}\trace{}{\bar{\tensor{\sigma}}}
      &\bar{\sigma}^{11}+\bar{\sigma}^{33}-\frac{2}{3}\trace{}{\bar{\tensor{\sigma}}}
      & \bar{\sigma}^{23}& \bar{\sigma}^{13}& \bar{\sigma}^{12}\\
      \bar{\sigma}^{11}+\bar{\sigma}^{22}-\frac{2}{3}\trace{}{\bar{\tensor{\sigma}}} &
      2\bar{\sigma}^{22}-\frac{2}{3}\trace{}{\bar{\tensor{\sigma}}}
      &\bar{\sigma}^{22}+\bar{\sigma}^{33}-\frac{2}{3}\trace{}{\bar{\tensor{\sigma}}}
      & \bar{\sigma}^{23}& \bar{\sigma}^{13}& \bar{\sigma}^{12}\\
      \bar{\sigma}^{11}+\bar{\sigma}^{22}-\frac{2}{3}\trace{}{\bar{\tensor{\sigma}}} &
      \bar{\sigma}^{22}+\bar{\sigma}^{33}-\frac{2}{3}\trace{}{\bar{\tensor{\sigma}}}
      &2\bar{\sigma}^{33}-\frac{2}{3}\trace{}{\bar{\tensor{\sigma}}}
      & \bar{\sigma}^{23}& \bar{\sigma}^{13}& \bar{\sigma}^{12}\\     
      \bar{\sigma}^{23} & \bar{\sigma}^{23} & \bar{\sigma}^{23} & 0 & 0 & 0\\
      \bar{\sigma}^{13} & \bar{\sigma}^{13} & \bar{\sigma}^{13} & 0 & 0 & 0\\
      \bar{\sigma}^{12} & \bar{\sigma}^{12} & \bar{\sigma}^{12} & 0 & 0 & 0
    \end{bmatrix}\right)\\
    &=\dfrac{2}{3}J^{-\frac{2}{3}}\begin{bmatrix}
      \frac{4}{3}\trace{}{\bar{\tensor{\sigma}}}-2\bar{\sigma}^{11} &
      \frac{1}{3}\trace{}{\bar{\tensor{\sigma}}}-\bar{\sigma}^{11}-\bar{\sigma}^{22}
      &\frac{1}{3}\trace{}{\bar{\tensor{\sigma}}}-\bar{\sigma}^{11}-\bar{\sigma}^{33}
      & -\bar{\sigma}^{23}& -\bar{\sigma}^{13}& -\bar{\sigma}^{12}\\
      \frac{1}{3}\trace{}{\bar{\tensor{\sigma}}}-\bar{\sigma}^{11}-\bar{\sigma}^{22} &
      \frac{4}{3}\trace{}{\bar{\tensor{\sigma}}}-2\bar{\sigma}^{22}
      &\frac{1}{3}\trace{}{\bar{\tensor{\sigma}}}-\bar{\sigma}^{22}-\bar{\sigma}^{33}
      & -\bar{\sigma}^{23}& -\bar{\sigma}^{13}& -\bar{\sigma}^{12}\\
      \frac{1}{3}\trace{}{\bar{\tensor{\sigma}}}-\bar{\sigma}^{11}-\bar{\sigma}^{22} &
      \frac{1}{3}\trace{}{\bar{\tensor{\sigma}}}-\bar{\sigma}^{22}-\bar{\sigma}^{33}
      &\frac{4}{3}\trace{}{\bar{\tensor{\sigma}}}-2\bar{\sigma}^{33}
      & -\bar{\sigma}^{23}& -\bar{\sigma}^{13}& -\bar{\sigma}^{12}\\     
      -\bar{\sigma}^{23} & -\bar{\sigma}^{23} & -\bar{\sigma}^{23} & \frac{1}{2}\trace{}{\bar{\tensor{\sigma}}} & 0 & 0\\
      -\bar{\sigma}^{13} & -\bar{\sigma}^{13} & \bar{\sigma}^{13} & 0 & \frac{1}{2}\trace{}{\bar{\tensor{\sigma}}} & 0\\
      -\bar{\sigma}^{12} & -\bar{\sigma}^{12} & -\bar{\sigma}^{12} & 0 & 0 & \frac{1}{2}\trace{}{\bar{\tensor{\sigma}}}
    \end{bmatrix}
  \end{split}
\end{equation}

\subsection{Special Two-dimensional Cases}
\label{subsec:FiniteElasticity2DCases}

Just like for linear elasticity we can formulate finite elasticity for
lower dimensions by making some assumptions on the deformation. The two main 
assumptions on allowed deformation include:
\begin{itemize}
\item Plane Stress: Assume that deformation happens such that there is
  no stress in one dimension normal to the other two dimensions. This
  corresponds to an infinitely thin plate.
\item Plane Strain: Assume that defomration happens such that there is
  no strain in one dimension normal to the other two dimesions. This
  corresponds to an infinitely thick plate.
\end{itemize}

\subsubsection{Plane Stress}
\label{subsubsec:FiniteElasticityPlaneStress}

For plane stress elasticity the deformation is such that there is no
stress in a third, normal, dimension. In the limit, this corresponds to deformation
in an infinitely thin plate.

For the biaxial plane stress case the deformation tensors are given by
\begin{equation}
  \deformationgradienttensor=\begin{bmatrix}
  1+\alpha & 0 & 0 \\
  0 & 1-\beta & 0 \\
  0 & 0 & 1-\gamma
  \end{bmatrix}  
\end{equation}
where $\gamma$ is now the deformation in the $z$ direction required to ensure that there is zero stress in that direction.

The Jacobian of the deformation gradient, the modified deformation
gradient tensor, the modified right Cauchy-Green strain tensor and the
modified Lagrange strain tensor are given by
Equations~\ref{eqn:FiniteElasticityUniaxialCubeJacobian},\ref{eqn:FiniteElasticityUniaxialCubeModDeformationGradient},
and \ref{eqn:FiniteElasticityUniaxialCubeModRightCachyGreen} \ie
\begin{equation}
  J=\pbrac{1+\alpha}\pbrac{1-\beta}\pbrac{1-\gamma}
\end{equation}
\begin{equation}
  \bar{\deformationgradienttensor}=J^{-\frac{1}{3}}\deformationgradienttensor=\frac{1}{\cuberoot{\pbrac{1+\alpha}\pbrac{1+\beta}\pbrac{1+\gamma}}}\begin{bmatrix}
      1+\alpha & 0 & 0 \\
      0 & 1-\beta & 0 \\
      0 & 0 & 1-\gamma
    \end{bmatrix}
\end{equation}
\begin{equation}
  \bar{\rightcauchygreentensor}=\transpose{\bar{\deformationgradienttensor}}\bar{\deformationgradienttensor}=J^{-\frac{2}{3}}\begin{bmatrix}
    \pbrac{1+\alpha}^{2} & 0 & 0 \\
    0 & \pbrac{1-\beta}^{2} & 0  \\
    0 & 0 & \pbrac{1-\gamma}^{2}
  \end{bmatrix}
\end{equation}

Now consider an incompressible Mooney-Rivlin material. The strain energy function is given by
\begin{equation}
  \fnof{\bar{W}}{\bar{I}_{1},\bar{I}_{2}}=c_{10}\pbrac{\bar{I}_{1}-3}+c_{01}\pbrac{\bar{I}_{2}-3}
\end{equation}
where
\begin{equation}
  \begin{split}
    \bar{I}_{1}&=\trace{}{\bar{\rightcauchygreentensor}}=\bar{C}_{11}+\bar{C}_{22}+\bar{C}_{33} \\
    \bar{I}_{2}&=\frac{1}{2}\pbrac{\pbrac{\trace{}{\bar{\rightcauchygreentensor}}}^{2}-\trace{}{\bar{\rightcauchygreentensor}^{2}}} \\
    &=\dfrac{1}{2}\left(\pbrac{\bar{C}_{11}+\bar{C}_{22}+\bar{C}_{33}}^{2}\right. \\
    & \quad\left.-\pbrac{\bar{C}_{11}^{2}+\bar{C}_{12}\bar{C}_{21}+\bar{C}_{13}\bar{C}_{31}+
      \bar{C}_{21}\bar{C}_{12}+\bar{C}_{22}^{2}+\bar{C}_{23}\bar{C}_{32}+\bar{C}_{31}\bar{C}_{13}+
      \bar{C}_{32}\bar{C}_{23}+\bar{C}_{33}^{2}}\right)
  \end{split}
\end{equation}

The modified second Piola-Kirchoff stress tensor is then given by
Now
\begin{equation}
  \begin{split}
    \bar{\secondpiolakirchoffstresstensor}&=2\pbrac{\delby{\fnof{\bar{W}}{\bar{I}_{1},\bar{I}_{2}}}{\bar{\rightcauchygreentensor}}}\\
    &=2\pbrac{\delby{\fnof{\bar{W}}{\bar{I}_{1},\bar{I}_{2}}}{\bar{I}_{1}}\delby{\bar{I}_{1}}{\bar{\rightcauchygreentensor}}+\delby{\fnof{\bar{W}}{\bar{I}_{1},\bar{I}_{2}}}{\bar{I}_{2}}\delby{\bar{I}_{2}}{\bar{\rightcauchygreentensor}}}\\
    &=2\pbrac{c_{10}\begin{bmatrix}
        1 & 0 & 0 \\
        0 & 1 & 0 \\
        0 & 0 & 1
      \end{bmatrix}+c_{01}\begin{bmatrix}
        \bar{C}_{22}+\bar{C}_{33} & -\bar{C}_{21} & \bar{C}_{31} \\
        -\bar{C}_{12} & \bar{C}_{11}+\bar{C}_{33} & -\bar{C}_{32} \\
        -\bar{C}_{13} & -\bar{C}_{23} & \bar{C}_{11}+\bar{C}_{22}
    \end{bmatrix}} \\
    &=\begin{bmatrix}
    2c_{1}+2c_{2}\pbrac{\bar{C}_{22}+\bar{C}_{33}} & -2c_{2}\bar{C}_{21} &
    -2c_{2}\bar{C}_{31} \\
    -2c_{2}\bar{C}_{12} & 2c_{1}+2c_{2}\pbrac{\bar{C}_{11}+\bar{C}_{33}} & -2c_{2}\bar{C}_{32} \\
    -2c_{2}\bar{C}_{13} & -2c_{2}\bar{C}_{23} & 2c_{1}+2c_{2}\pbrac{\bar{C}_{11}+\bar{C}_{22}} 
    \end{bmatrix}
  \end{split}
\end{equation}

\subsection{Plane Strain}
\label{subsec:FiniteElasticityPlaneStrain}

For plane strain elasticity the deformation is such that there is no
strain in a third, normal, dimension. This corresponds to deformation
in an infinitely thick plate.

\subsection{Membrane}
\label{subsec:FiniteElasticityMembrane}

\subsection{\Onedal}
\label{subsec:FiniteElasticityOneDimension}

\section{Analytic Solutions}
\label{sec:FiniteElasticityAnalyticSolutions}

\subsection{Inflation, Extension and Torsion of a Tube}
\label{subsec:FiniteElasticityAnalyticSolutionTube}

Consider a cylindrical tube of inner radius $R_{i}$, outer radius
$R_{o}$, and length $L$ undergoing deformation from a uniaxial load
$F$, a twisting moment $M$, and an internal pressure $P$, as shown in
\figref{fig:FiniteElasticityAnalyticTube}.

\epstexfigure{SolidMechanics/svgs/AnalyticTube.eps_tex}{}{}{fig:FiniteElasticityAnalyticTube}{0.50}

Using a cylindrical polar coordinate system these loads a point an
undeformed location of $\materialcoordinatevector:\pbrac{R,\Theta,Z}$
will move to the point $\spatialcoordinatevector:\pbrac{r,\theta,z}$
\ie
\begin{align}
  \fnof{r}{R,\Theta,Z}&=\fnof{f}{R} \\
  \fnof{\theta}{R,\Theta,Z}&=\Theta+\phi\lambda_{z} Z \\
  \fnof{z}{R,\Theta,Z}&=\lambda_{z} Z
\end{align}
where $\lambda_{z}$ is the uniaxial extension ratio and $\phi$ is the twisting angle per unit length.

From \secref{sec:CoordinateSystemsCylindricalPolar} we have
\begin{equation}
  \sharptensor{\generalmetrictensor}=\begin{bmatrix}
  1 & 0 & 0 \\
  0 & \frac{1}{r^{2}} & 0 \\
  0 & 0 & 1
  \end{bmatrix}
\end{equation}
and
\begin{align}
  \christoffelsecond{r}{\theta}{\theta}&=-r \\
  \christoffelsecond{\theta}{r}{\theta}=\christoffelsecond{\theta}{\theta}{r}&=\frac{1}{r}
\end{align}
with all other Christofell symbols zero.

The deformation gradient tensor in this case is given by
\begin{equation}
  \deformationgradienttensor=\begin{bmatrix}
  \delby{r}{R} & \dfrac{1}{R}\delby{r}{\Theta} & \delby{r}{Z} \\
  r\delby{\theta}{R} & \dfrac{r}{R}\delby{\theta}{\Theta} & r \delby{\theta}{Z} \\
  \delby{z}{R} & \dfrac{1}{R}\delby{z}{\Theta} & \delby{z}{Z}
  \end{bmatrix}=\begin{bmatrix}
  \delby{f}{R} & 0 & 0 \\
  0 & \dfrac{r}{R} & r\phi\lambda_{z} \\
  0 & 0 & \lambda_{z}
  \end{bmatrix}
\end{equation}

As the tube is incompressible we have the condition that
\begin{equation}
  \det{\deformationgradienttensor}=1
\end{equation}
or
\begin{equation}
  \delby{r}{R}.\dfrac{r}{R}.\lambda_{z}=1
\end{equation}
or
\begin{equation}
  r\delby{r}{R}=\dfrac{R}{\lambda_{z}}
\end{equation}

Integrating both sides gives
\begin{equation}
  \dfrac{r^{2}}{2}=\dfrac{R^{2}}{2\lambda_{z}}+c
\end{equation}
where $c$ is a constant of integration.

We known that $\fnof{r}{R_{i}}=r_{i}=\mu_{i}R_{i}$ and
$\fnof{r}{R_{o}}=r_{o}=\mu_{o}R_{o}$, where $\mu_{i}$ and $\mu_{o}$
are two constants to be determined. We thus obtain
\begin{equation}
  \begin{split}
    c&=\dfrac{r_{i}^{2}}{2}-\dfrac{R_{i}^{2}}{2\lambda_{z}}=\dfrac{\mu_{i}^{2}R_{i}^{2}}{2}-\dfrac{R_{i}^{2}}{2\lambda_{z}}
    =\dfrac{\pbrac{\mu_{i}^{2}\lambda_{z}-1}R_{i}^{2}}{2\lambda_{z}} \\
    &=\dfrac{r_{o}^{2}}{2}-\dfrac{R_{o}^{2}}{2\lambda_{z}}=\dfrac{\mu_{o}^{2}R_{o}^{2}}{2}-\dfrac{R_{o}^{2}}{2\lambda_{z}}
    =\dfrac{\pbrac{\mu_{o}^{2}\lambda_{z}-1}R_{o}^{2}}{2\lambda_{z}} 
  \end{split}
\end{equation}

The expression for deformed radius is thus
\begin{equation}
  \begin{split}
    \dfrac{r^{2}}{2}&=\dfrac{R^{2}}{2\lambda_{z}}+\dfrac{\pbrac{\mu_{i}^{2}\lambda_{z}-1}R_{i}^{2}}{2\lambda_{z}}\\
    &=\dfrac{R^{2}+\mu_{i}^{2}\lambda_{z}R_{i}^{2}-R_{i}^{2}}{2\lambda_{z}}\\
    &=\dfrac{\mu_{i}^{2}R_{i}^{2}}{2}+\dfrac{R^{2}-R_{i}^{2}}{2\lambda_{z}}
  \end{split}
\end{equation}
or
\begin{equation}
  \fnof{r}{R}=\sqrt{K^{2}+\inverse{\lambda_{z}}\pbrac{R^{2}-R_{i}^2}}
  \label{eqn:AnalyticTubeRadialPosition}
\end{equation}
where $K=\mu_{i}R_{i}=\mu_{o}R_{o}$.

Finally, \eqnref{eqn:AnalyticTubeRadialPosition} is of the more general form \todo{make this $r=\sqrt{\alpha+\beta R}$ and then do with with general terms in $\deformationgradienttensor$???}
\begin{equation}
  \fnof{r}{R}=\sqrt{\dfrac{a+R^{2}}{\lambda_{z}}}
  \label{eqn:AnalyticTubeGeneralRadialPosition}
\end{equation}
where $a=K^{2}\lambda_{z}-\dfrac{R_{i}^{2}}{\lambda_{z}}$.

We can now differentiate \eqnref{eqn:AnalyticTubeGeneralRadialPosition} to
finalise our expression for the deformation gradient tensor. Thus
\begin{equation}
  \begin{split}
    \delby{\fnof{r}{R}}{R}&=\delby{\pbrac{\sqrt{\dfrac{a+R^{2}}{\lambda_{z}}}}}{R}\\
    &=\dfrac{R}{\lambda_{z}\sqrt{\dfrac{a+R^{2}}{\lambda_{z}}}}\\
    &=\dfrac{R}{r\lambda_{z}}
  \end{split}
  \label{eqn:AnalyticTubedfdR}
\end{equation}

The deformation gradient tensor, in physical components, is thus
\begin{equation}
  \deformationgradienttensor=\begin{bmatrix}
  \dfrac{R}{r\lambda_{z}} & 0 & 0 \\
  0 & \dfrac{r}{R} & r\phi\lambda_{z} \\
  0 & 0 & \lambda_{z}
  \end{bmatrix}
  \label{eqn:FEAnalyticTubeDeformationGradientTensor}
\end{equation}

From the deformation gradient tensor we can find the left and right Cauchy-Green deformation tensors \ie
\begin{equation}
  \rightcauchygreentensor=\transpose{\deformationgradienttensor}\deformationgradienttensor=\begin{bmatrix}
    \dfrac{R}{r\lambda_{z}} & 0 & 0 \\
    0 & \dfrac{r}{R} & 0 \\
    0 & r\phi\lambda_{z} & \lambda_{z}
  \end{bmatrix}\begin{bmatrix}
    \dfrac{R}{r\lambda_{z}} & 0 & 0 \\
    0 & \dfrac{r}{R} & r\phi\lambda_{z}\\
    0 & 0 & \lambda_{z}
  \end{bmatrix}=\begin{bmatrix}
  \dfrac{R^{2}}{r^{2}\lambda_{z}^{2}} & 0 & 0 \\
  0 &\dfrac{r^{2}}{R^{2}} & \dfrac{r^{2}\phi\lambda_{z}}{R} \\
  0 & \dfrac{r^{2}\phi\lambda_{z}}{R} & \pbrac{1+r^{2}\phi^{2}}\lambda_{z}^{2}
  \end{bmatrix}
  \label{eqn:FEAnalyticTubeRightCauchyGreenTensor}
\end{equation}
and
\begin{equation}
  \leftcauchygreentensor=\deformationgradienttensor\transpose{\deformationgradienttensor}=\begin{bmatrix}
    \dfrac{R}{r\lambda_{z}} & 0 & 0 \\
    0 & \dfrac{r}{R} & r\phi\lambda_{z}\\
    0 & 0 & \lambda_{z}
  \end{bmatrix}\begin{bmatrix}
    \dfrac{R}{r\lambda_{z}} & 0 & 0 \\
    0 & \dfrac{r}{R} & 0 \\
    0 & r\phi\lambda_{z} & \lambda_{z}
  \end{bmatrix}=\begin{bmatrix}
  \dfrac{R^{2}}{r^{2}\lambda_{z}^2} & 0 & 0 \\
  0 & r^{2}\phi^{2}\lambda_{z}^{2}+\dfrac{r^{2}}{R^{2}} & r\phi\lambda_{z}^{2} \\
  0 & r\phi\lambda_{z}^{2} & \lambda_{z}^{2}
  \end{bmatrix}
  \label{eqn:FEAnalyticTubeLeftCauchyGreenTensor}
\end{equation}

Now for hyperelastic materials that can be described using a strain energy function, $W=\fnof{W}{\leftcauchygreentensor}$. The strain energy function can also be described using invariants \ie
\begin{equation}
  W=\fnof{W}{\leftcauchygreentensor}=\fnof{\tilde{W}}{I_{1},I_{2},I_{3}}
\end{equation}
where $I_{1}$, $I_{2}$, $I_{3}$ are the invariants of $\leftcauchygreentensor$ given by
\begin{align}
  I_{1}&=\trace{}{\leftcauchygreentensor}\\
  I_{2}&=\dfrac{1}{2}\pbrac{\pbrac{\trace{}{\leftcauchygreentensor}}^{2}-\trace{}{\leftcauchygreentensor^{2}}}=\trace{}{\inverse{\leftcauchygreentensor}}\determinant{\leftcauchygreentensor}\\
  I_{3}&=\determinant{\leftcauchygreentensor}
\end{align}

For an incompressible material (\ie $J=1$) the Cauchy stress is given by
\begin{equation}
  \cauchystresstensor=2\pbrac{W_{1}+I_{1}W_{2}}\leftcauchygreentensor-2W_{2}\leftcauchygreentensor^{2}-p\spatialidentitytensortwo
\end{equation}
or
\begin{equation}
  \cauchystresstensor=2W_{1}\leftcauchygreentensor-2W_{2}\inverse{\leftcauchygreentensor}-p\spatialidentitytensortwo
\end{equation}
where $W_{i}=\delby{\tilde{W}}{I_{i}}$ and $p$ is the hydrostatic pressure. \todo{Shouldn't p be multiplied by the metric rather than identity tensor????}

For the case of a Mooney-Rivilin material we have
\begin{equation}
  \fnof{W}{\fnof{I_{1}}{\leftcauchygreentensor},\fnof{I_{2}}{\leftcauchygreentensor}}=c_{1}\pbrac{\fnof{I_{1}}{\leftcauchygreentensor}-3}+c_{2}\pbrac{\fnof{I_{2}}{\leftcauchygreentensor}-3}
\end{equation}
and so we have $W_{1}=c_{1}$ and $W_{2}=c_{2}$. Now
\begin{equation}
  \fnof{I_{1}}{\leftcauchygreentensor}=\dfrac{R^{2}}{r^{2}\lambda_{z}^{2}} + r^{2}\phi^{2}\lambda_{z}^{2}+\dfrac{r^{2}}{R^{2}} + \lambda_{z}^{2}
\end{equation}
and
\begin{equation}
  \begin{split}
    \leftcauchygreentensor^{2} &=
    \begin{bmatrix}
      \dfrac{R^{2}}{r^{2}\lambda_{z}^{2}} & 0 & 0 \\
      0 & r^{2}\phi^{2}\lambda_{z}^{2}+\dfrac{r^{2}}{R^{2}} & r\phi\lambda_{z}^{2} \\
      0 & r\phi\lambda_{z}^{2} & \lambda_{z}^{2}
    \end{bmatrix}
    \begin{bmatrix}
      \dfrac{R^{2}}{r^{2}\lambda_{z}^{2}} & 0 & 0 \\
      0 & r^{2}\phi^{2}\lambda_{z}^{2}+\dfrac{r^{2}}{R^{2}} & r\phi\lambda_{z}^{2} \\
      0 & r\phi\lambda_{z}^{2} & \lambda_{z}^{2}
    \end{bmatrix} \\
    &=\begin{bmatrix}
    \dfrac{R^{4}}{r^{4}\lambda_{z}^{4}} & 0 & 0 \\
    0 & r^{2}\phi^{2}\lambda_{z}^{4}+\dfrac{r^{2}\pbrac{1+R^{2}\phi^{2}\lambda_{z}^{2}}^{2}}{R^{4}} & r\phi\lambda_{z}^{2}\pbrac{1+r^{2}\phi^{2}\lambda_{z}^{2}}+\dfrac{r^{3}\phi\lambda_{z}^{2}}{R^{2}} \\
    0 & r\phi\lambda_{z}^{2}\pbrac{1+r^{2}\phi^{2}\lambda_{z}^{2}}+\dfrac{r^{3}\phi\lambda_{z}^{2}}{R^{2}} & \pbrac{1+r^{2}\phi^{2}}\lambda_{z}^{4}
    \end{bmatrix} 
  \end{split}
\end{equation}
and
\begin{equation}
  \inverse{\leftcauchygreentensor}=\inverse{\begin{bmatrix}
      \dfrac{R^{2}}{r^{2}\lambda_{z}^{2}} & 0 & 0 \\
      0 & r^{2}\phi^{2}\lambda_{z}^{2}+\dfrac{r^{2}}{R^{2}} & r\phi\lambda_{z}^{2} \\
      0 & r\phi\lambda_{z}^{2} & \lambda_{z}^{2}
  \end{bmatrix}}=\begin{bmatrix}
  \dfrac{r^{2}\lambda_{z}^{2}}{R^{2}} & 0 & 0 \\
  0 & \dfrac{R^{2}}{r^{2}} & -\dfrac{R^{2}\phi}{r} \\
  0 & -\dfrac{R^{2}\phi}{r} & R^{2}\phi^{2}+\dfrac{1}{\lambda_{z}^{2}}
  \end{bmatrix}
\end{equation}

The Cauchy stress tensor is thus
\begin{equation}
  \begin{split}
    \cauchystresstensor &= 2c_{1}\begin{bmatrix}
      \dfrac{R^{2}}{r^{2}\lambda_{z}^2} & 0 & 0 \\
      0 & r^{2}\phi^{2}\lambda_{z}^{2}+\dfrac{r^{2}}{R^{2}} & r\phi\lambda_{z}^{2} \\
      0 & r\phi\lambda_{z}^{2} & \lambda_{z}^{2}
    \end{bmatrix}-2c_{2}\begin{bmatrix}
      \dfrac{r^{2}\lambda_{z}^{2}}{R^{2}} & 0 & 0 \\
      0 & \dfrac{R^{2}}{r^{2}} & -\dfrac{R^{2}\phi}{r} \\
      0 & -\dfrac{R^{2}\phi}{r} & R^{2}\phi^{2}+\dfrac{1}{\lambda_{z}^{2}}
    \end{bmatrix}-p\begin{bmatrix}
    1 & 0 & 0 \\
    0 & 1 & 0 \\
    0 & 0 & 1
    \end{bmatrix} \\
    &=\begin{bmatrix}
    2c_{1}\dfrac{R^{2}}{r^{2}\lambda_{z}^{2}}-2c_{2}\dfrac{r^{2}\lambda_{z}^{2}}{R^{2}} - p & 0 & 0 \\
    0 & 2c_{1}\pbrac{r^{2}\phi^{2}\lambda_{z}^{2}+\dfrac{r^{2}}{R^{2}}}-2c_{2}\dfrac{R^{2}}{r^{2}} - p & 2c_{1}r\phi\lambda_{z}^{2}+2c_{2}\dfrac{R^{2}\phi}{r} \\
    0 & 2c_{1}r\phi\lambda_{z}^{2}+2c_{2}\dfrac{R^{2}\phi}{r} & 2c_{1}\lambda_{z}^{2}-2c_{2}\pbrac{R^{2}\phi^{2}+\dfrac{1}{\lambda_{z}^{2}}}-p
    \end{bmatrix}
  \end{split}
\end{equation}

This gives
\begin{align}
  \cauchystresstensorsymbol_{rr}&=2c_{1}\dfrac{R^{2}}{r^{2}\lambda_{z}^{2}}-2c_{2}\dfrac{r^{2}\lambda_{z}^{2}}{R^{2}} -p\\
  \cauchystresstensorsymbol_{\theta\theta}&=2c_{1}\pbrac{r^{2}\phi^{2}\lambda_{z}^{2}+\dfrac{r^{2}}{R^{2}}}-2c_{2}\dfrac{R^{2}}{r^{2}} - p\\
  \cauchystresstensorsymbol_{zz}&=2c_{1}\lambda_{z}^{2}-2c_{2}\pbrac{R^{2}\phi^{2}+\dfrac{1}{\lambda_{z}^{2}}}-p\\
  \cauchystresstensorsymbol_{\theta z}=\cauchystresstensorsymbol_{z \theta}&= 2c_{1}r\phi\lambda_{z}^{2}+2c_{2}\dfrac{R^{2}\phi}{r} 
\end{align}

Now from \eqnref{eqn:AnalyticTubeGeneralRadialPosition} we have $r=\sqrt{\dfrac{a+R^{2}}{\lambda_{z}}}$ or $r^{2}=\dfrac{R^{2}+a}{\lambda_{z}}$ and so we have
\begin{equation}
  \begin{split} 
    \cauchystresstensorsymbol_{rr}&=2c_{1}\dfrac{R^{2}}{\pbrac{R^{2}+a}\lambda_{z}}-2c_{2}\dfrac{\pbrac{R^{2}+a}\lambda_{z}}{R^{2}} - p\\
    \cauchystresstensorsymbol_{\theta\theta}&=2c_{1}\pbrac{\pbrac{R^{2}+a}\phi^{2}\lambda_{z}+\dfrac{R^{2}+a}{R^{2}\lambda_{z}}}
    -2c_{2}\dfrac{R^{2}\lambda_{z}}{R^{2}+a}-p \\
    \cauchystresstensorsymbol_{zz}&=2c_{1}\lambda_{z}^{2}-2c_{2}\pbrac{R^{2}\phi^{2}+\dfrac{1}{\lambda_{z}^{2}}} -p \\
    \cauchystresstensorsymbol_{\theta z}=\cauchystresstensorsymbol_{z \theta}&= 2c_{1}\dfrac{\sqrt{R^{2}+a}\phi\lambda_{z}^{2}}{\sqrt{\lambda_{z}}}+2c_{2}\dfrac{R^{2}\phi\sqrt{\lambda_{z}}}{\sqrt{R^{2}+a}}
  \end{split}
  \label{eqn:FEAnalyticTubeStressComponentsR}
\end{equation}

From \eqnref{eqn:AnalyticTubeGeneralRadialPosition} we also have $R=\sqrt{r^{2}\lambda_{z}-a}$ or $R^{2}=r^{2}\lambda_{z}-a$ which gives
\begin{equation}
  \begin{split} 
    \cauchystresstensorsymbol_{rr}&= 2c_{1}\dfrac{r^{2}-a}{r^{2}\lambda_{z}^{2}}-2c_{2}\dfrac{r^{2}\lambda_{z}}{r^{2}\lambda_{z}-a} - p\\
    \cauchystresstensorsymbol_{\theta\theta} &=2c_{1}\pbrac{r^{2}\phi^{2}\lambda_{z}^{2}+\dfrac{r^{2}}{r^{2}\lambda_{z}-a}}-2c_{2}\dfrac{\pbrac{r^{2}\lambda_{z}-a}^{2}}{r^{2}}-p \\
    \cauchystresstensorsymbol_{zz}&=2c_{1}\lambda_{z}^{2}-2c_{2}\pbrac{\pbrac{r^{2}\lambda_{z}-a}^{2}\phi^{2}+\dfrac{1}{\lambda_{z}^{2}}} -p \\
    \cauchystresstensorsymbol_{\theta z}=\cauchystresstensorsymbol_{z \theta}&= 2c_{1}r\phi\lambda_{z}^{2}+2c_{2}\dfrac{\pbrac{r^{2}\lambda_{z}-a}\phi}{r}
  \end{split}
  \label{eqn:FEAnalyticTubeStressComponentsr}
\end{equation}


In the absense of any body forces the equation for the equillibrium of stress is
\begin{equation}
  \divergence{\spatialcoordinatevector}{\cauchystresstensor}=\vectr{0}
\end{equation}
or, in cylindrical polar coordinates,
\begin{equation}
  \begin{split}
    \delby{\cauchystresstensorsymbol^{rr}}{r}+\dfrac{1}{r}\delby{\cauchystresstensorsymbol^{r\theta}}{\theta}+\delby{\cauchystresstensorsymbol^{rz}}{z}+\dfrac{1}{r}\pbrac{\cauchystresstensorsymbol^{rr}-\cauchystresstensorsymbol^{\theta\theta}}  &= 0\\
    \delby{\cauchystresstensorsymbol^{r\theta}}{r}+\dfrac{1}{r}\delby{\cauchystresstensorsymbol^{\theta\theta}}{\theta}+\delby{\cauchystresstensorsymbol^{\theta z}}{z}+\dfrac{2}{r}\cauchystresstensorsymbol^{r\theta} &= 0 \\
    \delby{\cauchystresstensorsymbol^{rz}}{r}+\dfrac{1}{r}\delby{\cauchystresstensorsymbol^{\theta z}}{\theta}+\delby{\cauchystresstensorsymbol^{zz}}{z}+\dfrac{1}{r}\cauchystresstensorsymbol^{rz}&= 0 
  \end{split}
\end{equation}

Now, given the stress components for this problem the equillibrium equations reduce to
\begin{equation}
  \delby{\cauchystresstensorsymbol^{rr}}{r}+\dfrac{1}{r}\pbrac{\cauchystresstensorsymbol^{rr}-\cauchystresstensorsymbol^{\theta\theta}} = 0
  \label{eqn:FEAnalyticTubeEquillibriumEquation}
\end{equation}
or
\begin{equation}
  \delby{\pbrac{\bar{\cauchystresstensorsymbol}^{rr}-p}}{r}+\dfrac{1}{r}\pbrac{\bar{\cauchystresstensorsymbol}^{rr}-p-\bar{\cauchystresstensorsymbol}^{\theta\theta}+p} = 0
  \label{eqn:FEAnalyticTubeEquillibriumEquationSplit}
\end{equation}
where $\fnof{\cauchystresstensorsymbol^{rr}}{r}=\fnof{\bar{\cauchystresstensorsymbol}^{rr}}{r}-\fnof{p}{r}$ and $\fnof{\cauchystresstensorsymbol^{\theta\theta}}{r}=\fnof{\bar{\cauchystresstensorsymbol}^{\theta\theta}}{r}-\fnof{p}{r}$.

Integrating \eqnref{eqn:FEAnalyticTubeEquillibriumEquationSplit} we can obtain the form of the hydrostatic pressure \ie
\begin{equation}
  \fnof{p}{r}=\fnof{\bar{\cauchystresstensorsymbol}^{rr}}{r}+\gint{r}{r_{o}}{\dfrac{\pbrac{\fnof{\bar{\cauchystresstensorsymbol}^{rr}}{r}-\fnof{\bar{\cauchystresstensorsymbol}^{\theta\theta}}{r}}}{r}}{r}
  \label{eqn:FEAnalyticTubeHydrostaticPressureFormSpatial}
\end{equation}

\Eqnref{eqn:FEAnalyticTubeHydrostaticPressureForm} can be integrated to give the hydrostatic pressure as a function of the spatial coordinate, $r$. To obtain an equivalent expression in terms of the material coordinate, $R$, we can make use \eqnref{eqn:AnalyticTubedfdR} \ie $\infinitesimal{r}=\dfrac{r}{R\lambda_{z}}\infinitesimal{R}$. We now have
\begin{equation}
  \fnof{p}{R}=\fnof{\bar{\cauchystresstensorsymbol}^{rr}}{R}+\gint{R}{R_{o}}{\dfrac{\pbrac{\fnof{\bar{\cauchystresstensorsymbol}^{rr}}{R}-\fnof{\bar{\cauchystresstensorsymbol}^{\theta\theta}}{R}}}{R}\dfrac{1}{\lambda_{z}}}{R}
  \label{eqn:FEAnalyticTubeHydrostaticPressureFormMaterial} 
\end{equation}


Now if we substitute \eqnref{eqn:FEAnalyticTubeStressComponentsR} into \eqnref{eqn:FEAnalyticTubeEquillibriumEquation} we obtain
\begin{multline}
\end{multline}
or
\begin{multline}
\end{multline}

The boundary conditions are at $r=r_{i}$ then
$\cauchystresstensorsymbol_{rr}=-P$ and at $r=r_{o}$ then
$\cauchystresstensorsymbol_{rr}=0$. Integrating the equillibrium
equations we obtain
\begin{equation}
  \evalat{\cauchystresstensorsymbol_{rr}}{r=r_{o}}-\evalat{\cauchystresstensorsymbol_{rr}}{r=r_{i}}=\gint{r_{i}}{r_{o}}{\dfrac{\cauchystresstensorsymbol^{rr}-\cauchystresstensorsymbol^{\theta\theta}}{r}}{r}
\end{equation}
or
\begin{equation}
  P=\gint{r_{i}}{r_{o}}{\dfrac{\cauchystresstensorsymbol^{rr}-\cauchystresstensorsymbol^{\theta\theta}}{r}}{r}
\end{equation}

