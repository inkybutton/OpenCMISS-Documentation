\section{Linear Elasticity}
\label{sec:LinearElasticity}

\subsection{Linearisation of Finite Elasticity}

\subsection{Equations of Linear Elasticity}
\label{subsec:LinearElasticityEquations}

The equations of \emph{linear (classical) elasticity} are given by
\begin{equation}
  \addtolength{\fboxsep}{5pt}
  \boxed{
    \fnof{\densitysymbol}{\coordinatevector}\fnof{\ddot{\vectr{u}}}{\coordinatevector,t}=
    \fnof{\densitysymbol}{\coordinatevector}\fnof{\vectr{b}}{\coordinatevector}+
    \divergence{}{\pbrac{\dotprod{\fnof{\linearelasticitytensor}{\coordinatevector}}{\gradient{}{\fnof{\vectr{u}}{\coordinatevector,t}}}}}
  }
  \label{eqn:LinearElasticityEquation}
\end{equation}
where $\fnof{\densitysymbol}{\coordinatevector}$ is the mass density,
$\fnof{\vectr{b}}{\coordinatevector,t}$ is the body force field,
$\fnof{\linearelasticitytensor}{\coordinatevector}$ is the fourth-order elasticity tensor
field, $\fnof{\ddot{\vectr{u}}}{\coordinatevector,t}$ is the is the acceleration field,
and $\fnof{\vectr{u}}{\coordinatevector,t}$ is the is the displacement field.

In component form \eqnref{eqn:LinearElasticityEquation} can be written as
\begin{equation}
  \rho{\ddot{u}}^{i}=\densitysymbol b^{i} + \covarderiv{\pbrac{\linearelasticitytensorsymbol^{ijkl}\covarderiv{u_{k}}{l}}}{j}
\end{equation}

The elasticity tensor is subject to the following symmetry conditions
\begin{equation}
  \linearelasticitytensorsymbol^{ijkl}=\linearelasticitytensorsymbol^{jikl}=\linearelasticitytensorsymbol^{ijlk}=\linearelasticitytensorsymbol^{klij}
\end{equation}

Note that for \emph{linear elastostatics} the
$\fnof{\densitysymbol}{\coordinatevector}\fnof{\ddot{\vectr{u}}}{\coordinatevector,t}$ term is zero and
is dropped. 
 
The boundary conditions for linear elasticity are Dirichlet conditions on
displacement \ie
\begin{equation}
  \fnof{\vectr{u}}{\coordinatevector,t} = \fnof{\vectr{g}}{\coordinatevector,t} \quad \coordinatevector\in\Gamma_{D},t\in[0,\infty)
  \label{eqn:LinearElasticityDirichletBC} 
\end{equation}
or, in component form,
\begin{equation}
  u^{i}=g^{i} \quad \coordinatevector\in\Gamma_{D},t\in[0,\infty)
\end{equation}
and Neumann conditions on traction \ie
\begin{equation}
  \fnof{\vectr{t}}{\coordinatevector,t} =\dotprod{\pbrac{\dotprod{\fnof{\linearelasticitytensor}{\coordinatevector}}
        {\gradient{}{\fnof{\vectr{u}}{\coordinatevector,t}}}}}{\fnof{\normal}{\coordinatevector}}
  = \fnof{\vectr{h}}{\coordinatevector,t} \quad \coordinatevector\in\Gamma_{N},t\in[0,\infty)
  \label{eqn:LinearElasticityNeumannBC} 
\end{equation}
or, in component form,
\begin{equation}
  t^{i}=\linearelasticitytensor^{ijkl}\covarderiv{u_{k}}{l}\normalsymbol_{j}=h^{i} \quad \coordinatesymbol\in\Gamma_{N},t\in[0,\infty)
\end{equation}
where $\fnof{t}{\coordinatesymbol,t}$ is the boundary traction, $\fnof{\normal}{\coordinatesymbol}$ is the normal
vector to the boundary and $\Gamma = \Gamma_D \cup \Gamma_N$.

The \emph{small strain tensor} is given by a linearisation of the Lagrangian
strain tensor $\greenlagrangestraintensor=\frac{1}{2}\pbrac{\rightcauchygreentensor-\materialmetrictensor}$
and can be written as
\begin{equation}
  \smallstraintensor=\frac{1}{2}\liederiv{\vectr{u}}{\spatialmetrictensor}=\frac{1}{2}\pbrac{\gradient{}{\vectr{u}}+\transpose{\gradient{}{\vectr{u}}}}
  \label{eqn:LinearElasticitySmallStrainTensor}
\end{equation}
or, in component form,
\begin{equation}
  \smallstraintensor_{ij}=\frac{1}{2}\pbrac{\covarderiv{u_{i}}{j}+\covarderiv{u_{j}}{i}}
  \label{eqn:LinearElasticitySmallStrainTensorComponents}
\end{equation}
\todo{WHY IS THE DISPLACEMENT FIELD A COVECTOR FIELD???}

The \emph{stress tensor} is given by
\begin{equation}
  \linearstresstensor=\dotprod{\linearelasticitytensor}{\gradient{}{\vectr{u}}}
  \label{eqn:LinearElasticityLinearStressTensor}
\end{equation}
or, in component form,
\begin{equation}
  \linearstresstensorsymbol^{ij}=\linearelasticitytensorsymbol^{ijkl}\covarderiv{u_{k}}{l}
  \label{eqn:LinearElasticityLinearStressTensorComponents}
\end{equation}

Note that $\linearstresstensor$ is symmetric and so $\linearstresstensorsymbol^{ij}=\linearstresstensorsymbol^{ji}$. The stress
tensor, through the symmetry of the elasticity tensor, also follows the
constitutive relationship
\begin{equation}
  \linearstresstensor=\doubledotprod{\linearelasticitytensor}{\smallstraintensor}
  \label{eqn:LinearElasticityConstitutiveLaw}
\end{equation}

Similarily to hyperelastic materials the stored energy function for linear
elasticity is given by
\begin{equation}
  \linearstrainenergysymbol=\frac{1}{2}\doubledotprodthree{\smallstraintensor}{\linearelasticitytensor}{\smallstraintensor}=\frac{1}{2}\smallstraintensorsymbol_{ij}\linearelasticitytensorsymbol^{ijkl}\smallstraintensorsymbol_{kl}
\end{equation}

The stress tensor is thus given by
\begin{equation}
  \linearstresstensor=\linearstresstensorsymbol^{ij}=\delby{\linearstrainenergysymbol}{\smallstraintensor}=\delby{\linearstrainenergysymbol}{\smallstraintensorsymbol_{ij}}
\end{equation}
and the elasticity tensor is given by
\begin{equation}
  \linearelasticitytensor=\linearelasticitytensorsymbol^{ijkl}=\delby{\linearstresstensor}{\smallstraintensor}=\deltwoby{\linearstrainenergysymbol}{\smallstraintensor}{\smallstraintensor}=\deltwoby{\linearstrainenergysymbol}{\smallstraintensorsymbol_{ij}}{\smallstraintensorsymbol_{kl}}
\end{equation}

\subsection{Weak Forumulation}
\label{subsec:LinearElasticityWeakFormulation}

In order to find the weak formula we follow the principle of virtual work as set out in \secref{sec:ElasticityVirtualWork}. Following this principle leads to \eqnref{eqn:ElasticityVirtualWorkStrainStatement} \ie
\begin{equation}
  \gint{\Omega}{}{\dotprod{\densitysymbol\ddot{\vectr{u}}}{\variationdir{\vectr{u}}}}{\Omega}+
  \gint{\Omega}{}{\doubledotprod{\linearstresstensor}{\fnof{\variationdir{\smallstraintensor}}{\variationdir{\vectr{u}}}}}{\Omega}=
  \gint{\Gamma}{}{\dotprod{\bar{\vectr{t}}}{\variationdir{\vectr{u}}}}{\Gamma}+
  \gint{\Omega}{}{\dotprod{\vectr{b}}{\variationdir{\vectr{u}}}}{\Omega}
\end{equation}
where the term
$\doubledotprod{\linearstresstensor}{\fnof{\variationdir{\smallstraintensor}}{\variationdir{\vectr{u}}}}$,
above, is the elastic virtual work that results from the virtual displacements.

Now from \eqnref{eqn:LinearElasticitySmallStrainTensor} we have
\begin{equation}
  \fnof{\variationdir{\smallstraintensor}}{\variationdir{\vectr{u}}}=\dfrac{1}{2}\pbrac{\gradient{}{\variationdir{\vectr{u}}}+\transpose{\gradient{}{\variationdir{\vectr{u}}}}}
\end{equation}
or, in component form,
\begin{equation}
  \variationdir{\smallstraintensorsymbol_{ij}}=\dfrac{1}{2}\pbrac{\covarderiv{\variationdir{u_{i}}}{j}+\covarderiv{\variationdir{u_{j}}}{i}}
\end{equation}

We therefore have
\begin{equation}
  \begin{aligned}
    \linearstresstensorsymbol^{ij}\variationdir{\smallstraintensorsymbol_{ij}}
    &=\dfrac{\linearstresstensorsymbol^{ij}\pbrac{\covarderiv{\variationdir{u_{i}}}{j}+\covarderiv{\variationdir{u_{j}}}{i}}}{2}\\
    &=\dfrac{\linearstresstensorsymbol^{ij}\covarderiv{\variationdir{u_{i}}}{j}
      +\linearstresstensorsymbol^{ij}\covarderiv{\variationdir{u_{j}}}{i}}{2} \\
    &=\linearstresstensorsymbol^{ij}\covarderiv{\variationdir{u_{i}}}{j}
  \end{aligned}
\end{equation}
as the stress tensor is symmetric \ie $\linearstresstensorsymbol^{ij}=\linearstresstensorsymbol^{ji}$.

Substituing \eqnref{eqn:LinearElasticityLinearStressTensorComponents} into the above equation gives
\begin{equation}
  \begin{aligned}
    \linearstresstensorsymbol^{ij}\variationdir{\smallstraintensorsymbol_{ij}}
    &=\linearstresstensorsymbol^{ij}\covarderiv{\variationdir{u_{i}}}{j}\\
    &=\linearelasticitytensorsymbol^{ijkl}\covarderiv{u_{k}}{l}\covarderiv{\variationdir{u_{i}}}{j}\\
    &=\linearelasticitytensorsymbol^{ijkl}\covarderiv{\variationdir{u_{i}}}{j}\covarderiv{u_{k}}{l}
  \end{aligned}
\end{equation}

The virtual work statement for linear elasticity, in component form, is thus
\begin{equation}
  \gint{\Omega}{}{\densitysymbol \ddot{u}^{i} \variationdir{u_{i}}}{\Omega}+
  \gint{\Omega}{}{\linearelasticitytensorsymbol^{ijkl}\covarderiv{\variationdir{u_{i}}}{j}\covarderiv{u_{k}}{l}}{\Omega}=
  \gint{\Gamma}{}{\bar{t}^{i} \variationdir{u_{i}}}{\Gamma}+
  \gint{\Omega}{}{b^{i}\variationdir{u_{i}}}{\Omega}
  \label{eqn:LinearElasticityVirtualWorkComponent}
\end{equation}


Firstly, consider the displacement, $\vectr{u}$, between the
reference/material coordinates, $\vectr{X}$, and the current/spatial
coordinates, $\vectr{z}$. The displacement is defined by
\begin{equation}
  \vectr{u}=\vectr{z}-\vectr{X}
\end{equation}

We thus have
\begin{equation}
  \begin{aligned}
    \variationdir{\vectr{u}} &=\variationdir{\pbrac{\vectr{z} -\vectr{X}}} \\
    &=\variationdir{\vectr{z}}-\variationdir{\vectr{X}} \\
    &=\variationdir{\vectr{z}}
  \end{aligned}
\end{equation}
as there is no variation in the original reference configuration $\vectr{X}$. We also have
\begin{equation}
  \begin{aligned}
    \delby{\variationdir{u_{i}}}{x^{j}}&=\delby{\variationdir{z_{i}}-\variationdir{X_{i}}}{x^{j}}\\
    &=\delby{\variationdir{z_{i}}}{x^{j}}-\delby{\variationdir{X_{i}}}{x^{j}}\\
    &=\delby{\variationdir{z_{i}}}{x^{j}}
  \end{aligned}
\end{equation}
and \todo{WHY IS $\delby{X_{i}}{x^{j}}=0$???}
\begin{equation}
  \begin{aligned}
    \delby{u_{i}}{x^{j}}&=\delby{z_{i}-X_{i}}{x^{j}}\\
    &=\delby{z_{i}}{x^{j}}-\delby{X_{i}}{x^{j}}\\
    &=\delby{z_{i}}{x^{j}}
  \end{aligned}
\end{equation}
\todo{maybe use $\gradient{}{\vectr{u}}=\dfrac{1}{2}\pbrac{\deformationgradienttensor+\transpose{\deformationgradienttensor}}-\identitytensortwo$???}

The left hand side of \eqnref{eqn:LinearElasticityVirtualWorkComponent} is
\begin{equation}
  \begin{aligned}
    \gint{\Omega}{}{\densitysymbol \ddot{u}^{i} \variationdir{u_{i}}}{\Omega}+
    \gint{\Omega}{}{\linearelasticitytensorsymbol^{ijkl}\covarderiv{\variationdir{u_{i}}}{j}\covarderiv{u_{k}}{l}}{\Omega}&=
    \gint{\Omega}{}{\densitysymbol a^{i} \variationdir{z_{i}}}{\Omega}+
    \gint{\Omega}{}{\linearelasticitytensorsymbol^{ijkl}\covarderiv{\variationdir{z_{i}}}{j}\covarderiv{u_{k}}{l}}{\Omega}\\
    &=\gint{\Omega}{}{\densitysymbol a^{i} \variationdir{z_{i}}}{\Omega}+
    \gint{\Omega}{}{\linearelasticitytensorsymbol^{ijkl}\pbrac{\delby{\variationdir{z_{i}}}{x^{j}}-
        \christoffel{o}{i}{j}\variationdir{z_{o}}}\pbrac{\delby{u_{k}}{x^{l}}-
        \christoffel{p}{k}{l} u_{p}}}{\Omega}\\
  \end{aligned}
\end{equation}


\subsection{Finite Element Formulation}
\label{subsec:LinearElasticityFEMFormulation}

We can now discretise the domain into finite elements \ie $\Omega=
\displaystyle{\bigcup_{e=1}^{E}}\Omega_{e}$ with
$\Gamma=\displaystyle{\bigcup_{f=1}^{F}}\Gamma_{f}$, 
\eqnref{eqn:LinearElasticityVirtualWorkComponent} becomes
\begin{equation}
  \dsum_{e=1}^{E}\gint{\Omega_{e}}{}{\densitysymbol \ddot{u}^{i} \variationdir{u_{i}}}{\Omega}+
  \dsum_{e=1}^{E}\gint{\Omega_{e}}{}{\linearelasticitytensorsymbol^{ijkl}\covarderiv{\variationdir{u_{i}}}{j}\covarderiv{u_{k}}{l}}{\Omega}=
  \dsum_{f=1}^{F}\gint{\Gamma_{f}}{}{\bar{t}^{i} \variationdir{u_{i}}}{\Gamma}+
  \dsum_{e=1}^{E}\gint{\Omega_{e}}{}{b^{i}\variationdir{u_{i}}}{\Omega}
  \label{eqn:LinearElasticityVirtualWorkElements}  
\end{equation}

\subsection{Spatial integration}
\label{subsec:LinearElasticitySpatialIntegration}

Now, changing the coordinates of integration to elemental coordinates we have
\begin{equation}
  \begin{split}
    \dsum_{e=1}^{E}\gint{\vectr{0}}{\vectr{1}}{\fnof{\densitysymbol}{\vectr{\xi}} \fnof{\ddot{u}^{i}}{\vectr{\xi}}
      \fnof{\variationdir{u_{i}}}{\vectr{\xi}}\abs{\jacobian{\Omega_{e}}{\vectr{\xi}}}}{\vectr{\xi}}+
    \dsum_{e=1}^{E}\gint{\vectr{0}}{\vectr{1}}{\fnof{\linearelasticitytensorsymbol^{ijkl}}{\vectr{\xi}}
      \delby{\fnof{\variationdir{u_{i}}}{\vectr{\xi}}}{\xi^{r}}\delby{\xi^{r}}{x^{j}}
      \delby{\fnof{u_{k}}{\vectr{\xi}}}{\xi^{s}}\delby{\xi^{s}}{x^{l}}
      \abs{\jacobian{\Omega_{e}}{\vectr{\xi}}}}{\vectr{\xi}}\\
    =\dsum_{f=1}^{F}\gint{\vectr{0}}{\vectr{1}}{\fnof{\bar{t}^{i}}{\vectr{\zeta}}\fnof{\variationdir{u_{i}}}{\vectr{\zeta}}
      \abs{\jacobian{\Gamma_{f}}{\vectr{\zeta}}}}{\vectr{\zeta}}
    +\dsum_{e=1}^{E}\gint{\vectr{0}}{\vectr{1}}{\fnof{b^{i}}{\vectr{\xi}}\fnof{\variationdir{u_{i}}}{\vectr{\xi}}
      \abs{\jacobian{\Omega}{\vectr{\xi}}}}{\vectr{\xi}}
  \end{split}
  \label{eqn:LinearElasticityVirtualWorkElementsXi}  
\end{equation}

We can now introduce basis functions to interpolate the vitual displacements
\begin{equation}
  \fnof{\variationdir{u_{i}}}{\vectr{\xi}} = \idxgbfn{\pbrac{i}}{m}{\alpha}{\vectr{\xi}}\variationdir{u_{i,\alpha}^{m}}\gsf{m}{\alpha}
\end{equation}
displacements 
\begin{equation}
  \fnof{u_{k}}{\vectr{\xi}} = \idxgbfn{\pbrac{k}}{n}{\beta}{\vectr{\xi}} u_{k,\beta}^{n}\gsf{n}{\beta}
\end{equation}
and accerlations
\begin{equation}
  \fnof{\ddot{u}_{i}}{\vectr{\xi}} = \idxgbfn{\pbrac{i}}{n}{\beta}{\vectr{\xi}} \ddot{u}_{i,\beta}^{n}\gsf{n}{\beta}
\end{equation}
and similarily for $\fnof{\densitysymbol}{\vectr{\xi}}$,
$\fnof{\linearelasticitytensor}{\vectr{\xi}}$,
$\fnof{\bar{\vectr{t}}}{\vectr{\zeta}}$, and
$\fnof{\vectr{b}}{\vectr{\xi}}$.

\Eqnref{eqn:LinearElasticityVirtualWorkElementsXi} now becomes
\begin{equation}
  \begin{split}
    \dsum_{e=1}^{E}\gint{\vectr{0}}{\vectr{1}}{\fnof{\densitysymbol}{\vectr{\xi}}
      \idxgbfn{\pbrac{i}}{n}{\beta}{\vectr{\xi}} \ddot{u}_{i,\beta}^{n}\gsf{n}{\beta}
      \idxgbfn{\pbrac{i}}{m}{\alpha}{\vectr{\xi}}\variationdir{u_{i,\alpha}^{m}}\gsf{m}{\alpha}
      \abs{\jacobian{\Omega_{e}}{\vectr{\xi}}}}{\vectr{\xi}}\\
    +\dsum_{e=1}^{E}\gint{\vectr{0}}{\vectr{1}}{\fnof{\linearelasticitytensorsymbol^{ijkl}}{\vectr{\xi}}
      \delby{\pbrac{\idxgbfn{\pbrac{i}}{m}{\alpha}{\vectr{\xi}}\variationdir{u_{i,\alpha}^{m}}\gsf{m}{\alpha}}}{\xi^{r}}\delby{\xi^{r}}{x^{j}}
      \delby{\pbrac{\idxgbfn{\pbrac{k}}{n}{\beta}{\vectr{\xi}} u_{k,\beta}^{n}\gsf{n}{\beta}}}{\xi^{s}}\delby{\xi^{s}}{x^{l}}
      \abs{\jacobian{\Omega_{e}}{\vectr{\xi}}}}{\vectr{\xi}}\\
    =\dsum_{f=1}^{F}\gint{\vectr{0}}{\vectr{1}}{\fnof{\bar{t}^{i}}{\vectr{\zeta}}
      \idxgbfn{\pbrac{i}}{m}{\alpha}{\vectr{\zeta}}\variationdir{u_{i,\alpha}^{m}}\gsf{m}{\alpha}
      \abs{\jacobian{\Gamma_{f}}{\vectr{\zeta}}}}{\vectr{\zeta}}\\
    +\dsum_{e=1}^{E}\gint{\vectr{0}}{\vectr{1}}{\fnof{b^{i}}{\vectr{\xi}}
      \idxgbfn{\pbrac{i}}{m}{\alpha}{\vectr{\xi}}\variationdir{u_{i,\alpha}^{m}}\gsf{m}{\alpha}
      \abs{\jacobian{\Omega_{e}}{\vectr{\xi}}}}{\vectr{\xi}}
  \end{split}
  \label{eqn:LinearElasticityVirtualWorkElementsXiBasis}  
\end{equation}

Taking the fixed nodal degrees-of-freedom and scale factors outside the integral gives
\begin{equation}
  \begin{split}
    \dsum_{e=1}^{E}\variationdir{u_{i,\alpha}^{m}}\gsf{m}{\alpha}\ddot{u}_{i,\beta}^{n}\gsf{n}{\beta}
    \gint{\vectr{0}}{\vectr{1}}{\fnof{\densitysymbol}{\vectr{\xi}}
      \idxgbfn{\pbrac{i}}{m}{\alpha}{\vectr{\xi}}
      \idxgbfn{\pbrac{i}}{n}{\beta}{\vectr{\xi}} 
      \abs{\jacobian{\Omega_{e}}{\vectr{\xi}}}}{\vectr{\xi}}\\
    +\dsum_{e=1}^{E}\variationdir{u_{i,\alpha}^{m}}\gsf{m}{\alpha} u_{k,\beta}^{n}\gsf{n}{\beta}
    \gint{\vectr{0}}{\vectr{1}}{\fnof{\linearelasticitytensorsymbol^{ijkl}}{\vectr{\xi}}
      \delby{\idxgbfn{\pbrac{i}}{m}{\alpha}{\vectr{\xi}}}{\xi^{r}}
      \delby{\idxgbfn{\pbrac{k}}{n}{\beta}{\vectr{\xi}}}{\xi^{s}}
      \delby{\xi^{r}}{x^{j}}\delby{\xi^{s}}{x^{l}}
      \abs{\jacobian{\Omega_{e}}{\vectr{\xi}}}}{\vectr{\xi}}\\
    =\dsum_{f=1}^{F}\variationdir{u_{i,\alpha}^{m}}\gsf{m}{\alpha}
    \gint{\vectr{0}}{\vectr{1}}{\fnof{\bar{t}^{i}}{\vectr{\zeta}}
      \idxgbfn{\pbrac{i}}{m}{\alpha}{\vectr{\zeta}}
      \abs{\jacobian{\Gamma_{f}}{\vectr{\zeta}}}}{\vectr{\zeta}}\\
    +\dsum_{e=1}^{E}\variationdir{u_{i,\alpha}^{m}}\gsf{m}{\alpha}
    \gint{\vectr{0}}{\vectr{1}}{\fnof{b^{i}}{\vectr{\xi}}
      \idxgbfn{\pbrac{i}}{m}{\alpha}{\vectr{\xi}}
      \abs{\jacobian{\Omega_{e}}{\vectr{\xi}}}}{\vectr{\xi}}
  \end{split}
  \label{eqn:LinearElasticityVirtualWorkElementsXiBasis2}  
\end{equation}

Now, slightly abusing notation, we can rearrange \eqnref{eqn:LinearElasticityVirtualWorkElementsXiBasis2} to give
\begin{equation}
  \begin{split}
    \left(\dsum_{e=1}^{E}\gsf{m}{\alpha}\ddot{u}_{i,\beta}^{n}\gsf{n}{\beta}
    \gint{\vectr{0}}{\vectr{1}}{\fnof{\densitysymbol}{\vectr{\xi}}
      \idxgbfn{\pbrac{i}}{m}{\alpha}{\vectr{\xi}}
      \idxgbfn{\pbrac{i}}{n}{\beta}{\vectr{\xi}} 
      \abs{\jacobian{\Omega_{e}}{\vectr{\xi}}}}{\vectr{\xi}}\right.\\
    +\dsum_{e=1}^{E}\gsf{m}{\alpha} u_{k,\beta}^{n}\gsf{n}{\beta}
    \gint{\vectr{0}}{\vectr{1}}{\fnof{\linearelasticitytensorsymbol^{ijkl}}{\vectr{\xi}}
      \delby{\idxgbfn{\pbrac{i}}{m}{\alpha}{\vectr{\xi}}}{\xi^{r}}
      \delby{\idxgbfn{\pbrac{k}}{n}{\beta}{\vectr{\xi}}}{\xi^{s}}
      \delby{\xi^{r}}{x^{j}}\delby{\xi^{s}}{x^{l}}
      \abs{\jacobian{\Omega_{e}}{\vectr{\xi}}}}{\vectr{\xi}}\\
    -\dsum_{f=1}^{F}\gsf{m}{\alpha}
    \gint{\vectr{0}}{\vectr{1}}{\fnof{\bar{t}^{i}}{\vectr{\zeta}}
      \idxgbfn{\pbrac{i}}{m}{\alpha}{\vectr{\zeta}}
      \abs{\jacobian{\Gamma_{f}}{\vectr{\zeta}}}}{\vectr{\zeta}}\\
    -\left.\dsum_{e=1}^{E}\gsf{m}{\alpha}
    \gint{\vectr{0}}{\vectr{1}}{\fnof{b^{i}}{\vectr{\xi}}
      \idxgbfn{\pbrac{i}}{m}{\alpha}{\vectr{\xi}}
      \abs{\jacobian{\Omega_{e}}{\vectr{\xi}}}}{\vectr{\xi}}\right)\variationdir{u_{i,\alpha}^{m}}=0
  \end{split}
  \label{eqn:LinearElasticityVirtualWorkElementsXiBasis2}  
\end{equation}

Now the vitual work statement must be satisfied for any choice of
$\variationdir{\vectr{u}}$ and thus any choice of
$\variationdir{u_{i,\alpha}^{m}}$ provided that
$\variationdir{u_{i,\alpha}^{m}}=0$ only for those degrees-of-freedom
on the boundary. This implies that
\begin{equation}
  \begin{split}
    \dsum_{e=1}^{E}\gsf{m}{\alpha}\ddot{u}_{i,\beta}^{n}\gsf{n}{\beta}
    \gint{\vectr{0}}{\vectr{1}}{\fnof{\densitysymbol}{\vectr{\xi}}
      \idxgbfn{\pbrac{i}}{m}{\alpha}{\vectr{\xi}}
      \idxgbfn{\pbrac{i}}{n}{\beta}{\vectr{\xi}} 
      \abs{\jacobian{\Omega_{e}}{\vectr{\xi}}}}{\vectr{\xi}}\\
    +\dsum_{e=1}^{E}\gsf{m}{\alpha} u_{k,\beta}^{n}\gsf{n}{\beta}
    \gint{\vectr{0}}{\vectr{1}}{\fnof{\linearelasticitytensorsymbol^{ijkl}}{\vectr{\xi}}
      \delby{\idxgbfn{\pbrac{i}}{m}{\alpha}{\vectr{\xi}}}{\xi^{r}}
      \delby{\idxgbfn{\pbrac{k}}{n}{\beta}{\vectr{\xi}}}{\xi^{s}}
      \delby{\xi^{r}}{x^{j}}\delby{\xi^{s}}{x^{l}}
      \abs{\jacobian{\Omega_{e}}{\vectr{\xi}}}}{\vectr{\xi}}\\
    -\dsum_{f=1}^{F}\gsf{m}{\alpha}
    \gint{\vectr{0}}{\vectr{1}}{\fnof{\bar{t}^{i}}{\vectr{\zeta}}
      \idxgbfn{\pbrac{i}}{m}{\alpha}{\vectr{\zeta}}
      \abs{\jacobian{\Gamma_{f}}{\vectr{\zeta}}}}{\vectr{\zeta}}\\
    -\dsum_{e=1}^{E}\gsf{m}{\alpha}
    \gint{\vectr{0}}{\vectr{1}}{\fnof{b^{i}}{\vectr{\xi}}
      \idxgbfn{\pbrac{i}}{m}{\alpha}{\vectr{\xi}}
      \abs{\jacobian{\Omega_{e}}{\vectr{\xi}}}}{\vectr{\xi}}=0
  \end{split}
  \label{eqn:LinearElasticityVirtualWorkElementsXiBasis3}  
\end{equation}

\subsection{Matrix vector form}
\label{subsec:LinearElasticityMatrixVectorForm}

\Eqnref{eqn:LinearElasticityVirtualWorkElementsXiBasis3} is a matrix-vector equation of the form
\begin{equation}
  \matr{M}\ddot{\vectr{u}}+\matr{K}\vect{u}+\vectr{s}+\vect{f}=\vect{0}
\end{equation}
or
\begin{equation}
  \matr{M}\ddot{\vectr{u}}+\matr{K}\vect{u}+\vectr{s}+\matr{N}\vectr{t}=\vect{0}
\end{equation}

Here the elemental mass matrix is given by
\begin{equation}
  M^{i\alpha\beta}_{mn}=\gsf{m}{\alpha}\gsf{n}{\beta}
    \gint{\vectr{0}}{\vectr{1}}{\fnof{\densitysymbol}{\vectr{\xi}}
      \idxgbfn{i}{m}{\alpha}{\vectr{\xi}}
      \idxgbfn{i}{n}{\beta}{\vectr{\xi}} 
      \abs{\jacobian{\Omega}{\vectr{\xi}}}}{\vectr{\xi}}
  \label{eqn:LinearElasticityElementalMassMatrix}
\end{equation}
and the elemental stiffness matrix is given by
\begin{equation}
  K^{ik\alpha\beta}_{mn}=\gsf{m}{\alpha}\gsf{n}{\beta}
  \gint{\vectr{0}}{\vectr{1}}{\fnof{\linearelasticitytensorsymbol^{ijkl}}{\vectr{\xi}}
    \delby{\idxgbfn{i}{m}{\alpha}{\vectr{\xi}}}{\xi^{r}}
    \delby{\idxgbfn{k}{n}{\beta}{\vectr{\xi}}}{\xi^{s}}
    \delby{\xi^{r}}{x^{j}}\delby{\xi^{s}}{x^{l}}
    \abs{\jacobian{\Omega_{e}}{\vectr{\xi}}}}{\vectr{\xi}}
  \label{eqn:LinearElasticityElementalStiffnessMatrix}
\end{equation}
and the elemental source vector is given by
\begin{equation}
  s^{i\alpha}_{m}=-\gsf{m}{\alpha}
  \gint{\vectr{0}}{\vectr{1}}{\fnof{b^{i}}{\vectr{\xi}}
    \idxgbfn{i}{m}{\alpha}{\vectr{\xi}}
    \abs{\jacobian{\Omega}{\vectr{\xi}}}}{\vectr{\xi}}
  \label{eqn:LinearElasticityElementalSourceVector}
\end{equation}
and the elemental RHS vector is given by
\begin{equation}
  f^{i\alpha}_{m}=-\gsf{m}{\alpha}
    \gint{\vectr{0}}{\vectr{1}}{\fnof{\bar{t}^{i}}{\vectr{\zeta}}
      \idxgbfn{i}{m}{\alpha}{\vectr{\zeta}}
      \abs{\jacobian{\Gamma}{\vectr{\zeta}}}}{\vectr{\zeta}}
  \label{eqn:LinearElasticityElementalRHSVector}
\end{equation}

\subsection{Linear Constitutive Laws}
\label{subsec:LinearElasticityMatrixVectorForm}

The constutive laws relate stress and strain. They are of the following (Voigt) form
\begin{equation}
  \Voigt{\linearstresstensor}=\Voigt{\linearelasticitytensor}\Voigt{\smallstraintensor}
\end{equation}
or, similiarily,
\begin{equation} 
  \Voigt{\smallstraintensor}=\Voigt{\linearcompliancetensor}\Voigt{\linearstresstensor}
\end{equation}
where $\Voigt{\linearelasticitytensor}$ is the linear elasticity tensor in Voigt form, $\Voigt{\linearcompliancetensor}$ is the linear compliance tensor in Voigt form, $\Voigt{\linearstresstensor}$ is the linear stress tensor in Voigt form, and $\Voigt{\smallstraintensor}$ is the linear small strain tensor in Voigt form. In \twods we have
\begin{equation}
  \Voigt{\linearstresstensor}=\begin{bmatrix}
  \linearstresstensorsymbol^{11} \\
  \linearstresstensorsymbol^{22} \\
  \linearstresstensorsymbol^{12}
  \end{bmatrix}\qquad\qquad\Voigt{\smallstraintensor}=\begin{bmatrix}
  \smallstraintensorsymbol_{11} \\
  \smallstraintensorsymbol_{22} \\
  2\smallstraintensorsymbol_{12}
  \end{bmatrix}
\end{equation}
and in \threeds we have
\begin{equation}
  \Voigt{\linearstresstensor}=\begin{bmatrix}
  \linearstresstensorsymbol^{11} \\
  \linearstresstensorsymbol^{22} \\
  \linearstresstensorsymbol^{33} \\
  \linearstresstensorsymbol^{23} \\
  \linearstresstensorsymbol^{13} \\
  \linearstresstensorsymbol^{12}
  \end{bmatrix}\qquad\qquad\Voigt{\smallstraintensor}=\begin{bmatrix}
  \smallstraintensorsymbol_{11} \\
  \smallstraintensorsymbol_{22} \\
  \smallstraintensorsymbol_{33} \\
  2\smallstraintensorsymbol_{23} \\
  2\smallstraintensorsymbol_{13} \\
  2\smallstraintensorsymbol_{12}
  \end{bmatrix}
\end{equation}

Note that sometimes engineering shear strain, $\shearstrainsymbol$, is used in
place of small shear strain, $\smallstraintensorsymbol$,
where
$\shearstrainsymbol_{mn}=\smallstraintensorsymbol_{mn}+\smallstraintensorsymbol_{nm}=2\smallstraintensorsymbol_{mn}$.

\subsubsection{3D Isotropic Material}

For a \threedal isotropic material the elasticity tensor is given by
\begin{equation}
  \fnof{\linearelasticitytensor}{\vectr{\xi}}=
  \fnof{\linearelasticitytensorsymbol^{ijkl}}{\vectr{\xi}}
  \tensorprodfour{\generalbasevector_{i}}{\generalbasevector_{j}}{\generalbasevector_{k}}{\generalbasevector_{l}}=
  \pbrac{\fnof{\firstlamesymbol}{\vectr{\xi}}\generalmetrictensorsymbol^{ij}\generalmetrictensorsymbol^{kl}+
    \fnof{\secondlamesymbol}{\vectr{\xi}}\pbrac{\generalmetrictensorsymbol^{ik}\generalmetrictensorsymbol^{il}+
      \generalmetrictensorsymbol^{ilk}\generalmetrictensorsymbol^{kj}}}
  \tensorprodfour{\generalbasevector_{i}}{\generalbasevector_{j}}{\generalbasevector_{k}}{\generalbasevector_{l}}
  \label{eqn:LinearElasticityIsotropicElasticityTensor}
\end{equation}
where $\firstlamesymbol$ and $\secondlamesymbol$ are the
\link{https://en.wikipedia.org/wiki/Lam\%C3\%A9_parameters}{\emph{Lam\'{e}
  constants}}\footnote{named after
\link{https://en.wikipedia.org/wiki/Gabriel_Lam\%C3\%A9}{Gabriel
  Lam\'{e}} (1795-1870), a French mathematician.}.

In Voigt form in Cartesian coordinates this is
\begin{equation}
  \Voigt{\linearelasticitytensor}=\begin{bmatrix}
  \firstlamesymbol+2\secondlamesymbol & \firstlamesymbol & \firstlamesymbol & 0 & 0 & 0 \\
  \firstlamesymbol & \firstlamesymbol+2\secondlamesymbol & \firstlamesymbol & 0 & 0 & 0 \\
  \firstlamesymbol & \firstlamesymbol & \firstlamesymbol+2\secondlamesymbol & 0 & 0 & 0 \\
  0 & 0 & 0 & \secondlamesymbol & 0 & 0 \\
  0 & 0 & 0 & 0 & \secondlamesymbol & 0 \\
  0 & 0 & 0 & 0 & 0 & \secondlamesymbol
  \end{bmatrix}
  \label{eqn:LinearElasticityIsotropicElasticityTensorVoigt}
\end{equation}

In terms of Young's modulus and Poisson's ratio we have
\begin{equation}
  \Voigt{\linearelasticitytensor}=\dfrac{\youngsmodulussymbol}{\pbrac{1+\poissonsratiosymbol}\pbrac{1-2\poissonsratiosymbol}}\begin{bmatrix}
  1-\poissonsratiosymbol & \poissonsratiosymbol & \poissonsratiosymbol & 0 & 0 & 0 \\
  \poissonsratiosymbol & 1-\poissonsratiosymbol & \poissonsratiosymbol & 0 & 0 & 0 \\
  \poissonsratiosymbol & \poissonsratiosymbol & 1-\poissonsratiosymbol & 0 & 0 & 0 \\
  0 & 0 & 0 & \dfrac{1-2\poissonsratiosymbol}{2} & 0 & 0 \\
  0 & 0 & 0 & 0 & \dfrac{1-2\poissonsratiosymbol}{2} & 0 \\
  0 & 0 & 0 & 0 & 0 & \dfrac{1-2\poissonsratiosymbol}{2}
  \end{bmatrix}
  \label{eqn:LinearElasticityIsotropicElasticityTensorVoigtEnu}
\end{equation}
and
\begin{equation}
  \Voigt{\linearcompliancetensor}=\dfrac{1}{\youngsmodulussymbol}\begin{bmatrix}
  1 & -\poissonsratiosymbol & -\poissonsratiosymbol & 0 & 0 & 0 \\
  -\poissonsratiosymbol & 1 & -\poissonsratiosymbol & 0 & 0 & 0 \\
  -\poissonsratiosymbol & -\poissonsratiosymbol & 1 & 0 & 0 & 0 \\
  0 & 0 & 0 & 2\pbrac{1+\poissonsratiosymbol} & 0 & 0 \\
  0 & 0 & 0 & 0 & 2\pbrac{1+\poissonsratiosymbol} & 0 \\
  0 & 0 & 0 & 0 & 0 & 2\pbrac{1+\poissonsratiosymbol}
  \end{bmatrix}
  \label{eqn:LinearElasticityIsotropicComplianceTensorVoigtEnu}
\end{equation}

\todo{CHECK ELASTICITY AND COMPILANCE TENSORS FOR THE FACTORS OF 2 THAT COMES FROM VOIGT????}

\subsubsection{3D Orthotropic Material}

The elasticity tensor for \threedal orthotropic material in terms of Young's modulus and Poisson's ratio in Voigt form is
\begin{equation}
  \Voigt{\linearelasticitytensor}=\begin{bmatrix}
  \dfrac{\youngsmodulussymbol_{1}\pbrac{1-\poissonsratiosymbol_{23}\poissonsratiosymbol_{32}}}{\delta} &
  \dfrac{\youngsmodulussymbol_{1}\pbrac{\poissonsratiosymbol_{21}+\poissonsratiosymbol_{23}\poissonsratiosymbol_{31}}}{\delta} &
  \dfrac{\youngsmodulussymbol_{1}\pbrac{\poissonsratiosymbol_{31}+\poissonsratiosymbol_{21}\poissonsratiosymbol_{32}}}{\delta} & 0 & 0 & 0 \\
  \dfrac{\youngsmodulussymbol_{2}\pbrac{\poissonsratiosymbol_{12}+\poissonsratiosymbol_{13}\poissonsratiosymbol_{32}}}{\delta} &
  \dfrac{\youngsmodulussymbol_{2}\pbrac{1-\poissonsratiosymbol_{13}\poissonsratiosymbol_{31}}}{\delta} &
  \dfrac{\youngsmodulussymbol_{2}\pbrac{\poissonsratiosymbol_{32}+\poissonsratiosymbol_{12}\poissonsratiosymbol_{31}}}{\delta} & 0 & 0 & 0 \\
  \dfrac{\youngsmodulussymbol_{3}\pbrac{\poissonsratiosymbol_{13}+\poissonsratiosymbol_{12}\poissonsratiosymbol_{23}}}{\delta} &
  \dfrac{\youngsmodulussymbol_{3}\pbrac{\poissonsratiosymbol_{23}+\poissonsratiosymbol_{13}\poissonsratiosymbol_{21}}}{\delta} &
  \dfrac{\youngsmodulussymbol_{3}\pbrac{1-\poissonsratiosymbol_{12}\poissonsratiosymbol_{21}}}{\delta} & 0 & 0 & 0 \\
  0 & 0 & 0 & \shearmodulussymbol_{23} & 0 & 0 \\
  0 & 0 & 0 & 0 & \shearmodulussymbol_{13} & 0 \\
  0 & 0 & 0 & 0 & 0 & \shearmodulussymbol_{12}
  \end{bmatrix}
  \label{eqn:LinearElasticityOrthotropicElasticityTensorVoigtEnu}
\end{equation}
where
\begin{equation}
  \delta=1
  -\poissonsratiosymbol_{12}\poissonsratiosymbol_{21}
  -\poissonsratiosymbol_{23}\poissonsratiosymbol_{32}
  -\poissonsratiosymbol_{13}\poissonsratiosymbol_{31}
  -\poissonsratiosymbol_{12}\poissonsratiosymbol_{23}\poissonsratiosymbol_{31}
  -\poissonsratiosymbol_{21}\poissonsratiosymbol_{32}\poissonsratiosymbol_{13}
\end{equation}
and $\youngsmodulussymbol_{m}=\youngsmodulussymbol_{mm}=\dfrac{\linearstresstensorsymbol_{mm}}{\smallstraintensorsymbol_{mm}}$, $\poissonsratiosymbol_{mn}=-\dfrac{\smallstraintensorsymbol_{nn}}{\smallstraintensorsymbol_{mm}}$, and $\shearmodulussymbol_{mn}=\dfrac{\linearstresstensorsymbol_{mn}}{2\smallstraintensorsymbol_{mn}}$. Here $m$ is the loading direction and $n$ is the direction of strain.

Stress symmetry requires that
$\dfrac{\youngsmodulussymbol_{2}\pbrac{\poissonsratiosymbol_{12}+\poissonsratiosymbol_{13}\poissonsratiosymbol_{32}}}{\delta}=\dfrac{\youngsmodulussymbol_{1}\pbrac{\poissonsratiosymbol_{21}+\poissonsratiosymbol_{23}\poissonsratiosymbol_{31}}}{\delta}$,
$\dfrac{\youngsmodulussymbol_{3}\pbrac{\poissonsratiosymbol_{13}+\poissonsratiosymbol_{12}\poissonsratiosymbol_{23}}}{\delta}=\dfrac{\youngsmodulussymbol_{1}\pbrac{\poissonsratiosymbol_{31}+\poissonsratiosymbol_{21}\poissonsratiosymbol_{32}}}{\delta}$,
and
$\dfrac{\youngsmodulussymbol_{3}\pbrac{\poissonsratiosymbol_{23}+\poissonsratiosymbol_{13}\poissonsratiosymbol_{21}}}{\delta}=\dfrac{\youngsmodulussymbol_{2}\pbrac{\poissonsratiosymbol_{32}+\poissonsratiosymbol_{12}\poissonsratiosymbol_{31}}}{\delta}$.

The compliance tensor for \threedal orthotropic material in terms of
Young's modulus and Poisson's ratio in Voigt form is
\begin{equation}
  \Voigt{\linearcompliancetensor}=\begin{bmatrix}
  \frac{1}{\youngsmodulussymbol_{1}} & -\frac{\poissonsratiosymbol_{12}}{\youngsmodulussymbol_{2}} & -\frac{\poissonsratiosymbol_{13}}{\youngsmodulussymbol_{3}} & 0 & 0 & 0 \\
  -\frac{\poissonsratiosymbol_{21}}{\youngsmodulussymbol_{1}} & \frac{1}{\youngsmodulussymbol_{2}} & -\frac{\poissonsratiosymbol_{23}}{\youngsmodulussymbol_{3}} & 0 & 0 & 0 \\
  -\frac{\poissonsratiosymbol_{31}}{\youngsmodulussymbol_{1}} & -\frac{\poissonsratiosymbol_{32}}{\youngsmodulussymbol_{2}} & \frac{1}{\youngsmodulussymbol_{3}} & 0 & 0 & 0 \\
  0 & 0 & 0 & \frac{1}{\shearmodulussymbol_{23}} & 0 & 0 \\
  0 & 0 & 0 & 0 & \frac{1}{\shearmodulussymbol_{13}} & 0 \\
  0 & 0 & 0 & 0 & 0 & \frac{1}{\shearmodulussymbol_{12}}
  \end{bmatrix}
  \label{eqn:LinearElasticityOrthotropicComplianceTensorVoigtEnu}
\end{equation}

Note that strain comptability requirements mean that $\dfrac{\poissonsratiosymbol_{23}}{\youngsmodulussymbol_{2}}=\dfrac{\poissonsratiosymbol_{32}}{\youngsmodulussymbol_{3}}$, $\dfrac{\poissonsratiosymbol_{13}}{\youngsmodulussymbol_{1}}=\dfrac{\poissonsratiosymbol_{31}}{\youngsmodulussymbol_{3}}$, and $\dfrac{\poissonsratiosymbol_{12}}{\youngsmodulussymbol_{1}}=\dfrac{\poissonsratiosymbol_{21}}{\youngsmodulussymbol_{2}}$. 

\subsubsection{2D Plane Stress Material}

The elasticity tensor for \twodal plane stress materials in terms of $\youngsmodulussymbol$ and $\poissonsratiosymbol$ in Voigt form is
\begin{equation}
  \Voigt{\linearelasticitytensor}=\dfrac{\youngsmodulussymbol}{1-\poissonsratiosymbol^{2}}\begin{bmatrix}
  1 & \poissonsratiosymbol & 0 \\
  \poissonsratiosymbol & 1 & 0 \\
  0 & 0 & \dfrac{1-\poissonsratiosymbol}{2}
  \end{bmatrix}
  \label{eqn:LinearElasticityPlaneStressElasticityTensorVoigtEnu}
\end{equation}
and the equivalent compliance tensor is
\begin{equation}
  \Voigt{\linearcompliancetensor}=\dfrac{1}{\youngsmodulussymbol}\begin{bmatrix}
  1 & -\poissonsratiosymbol & 0 \\
  -\poissonsratiosymbol & 1 & 0 \\
  0 & 0 & 2\pbrac{1+\poissonsratiosymbol}
  \end{bmatrix}
  \label{eqn:LinearElasticityPlaneStressComplianceTensorVoigtEnu}
\end{equation}
plus
\begin{equation}
  \smallstraintensorsymbol_{33}=-\dfrac{\poissonsratiosymbol}{\youngsmodulussymbol}\pbrac{\linearstresstensorsymbol_{11}+\linearstresstensorsymbol_{22}}
\end{equation}

\subsubsection{2D Plane Strain Material}

The elasticity tensor for \twodal plane strain materials in terms of $\youngsmodulussymbol$ and $\poissonsratiosymbol$ in Voigt form is
\begin{equation}
  \Voigt{\linearelasticitytensor}=\dfrac{\youngsmodulussymbol}{\pbrac{1+\poissonsratiosymbol}\pbrac{1-2\poissonsratiosymbol}}\begin{bmatrix}
  1-\poissonsratiosymbol & \poissonsratiosymbol & 0 \\
  \poissonsratiosymbol & 1-\poissonsratiosymbol & 0 \\
  0 & 0 & \dfrac{1-2\poissonsratiosymbol}{2}
  \end{bmatrix}
  \label{eqn:LinearElasticityPlaneStrainElasticityTensorVoigtEnu}
\end{equation}
plus
\begin{equation}
  \smallstraintensorsymbol_{33}=\dfrac{\poissonsratiosymbol\youngsmodulussymbol}{\pbrac{1+\poissonsratiosymbol}\pbrac{1-2\poissonsratiosymbol}}\pbrac{\smallstraintensorsymbol_{11}+\smallstraintensorsymbol_{22}}
\end{equation}
and the equivalent compliance tensor is
\begin{equation}
  \Voigt{\linearcompliancetensor}=\dfrac{1+\poissonsratiosymbol}{\youngsmodulussymbol}\begin{bmatrix}
  1-\poissonsratiosymbol & -\poissonsratiosymbol & 0 \\
  -\poissonsratiosymbol & 1-\poissonsratiosymbol & 0 \\
  0 & 0 & 2
  \end{bmatrix}
  \label{eqn:LinearElasticityPlaneStrainComplianceTensorVoigtEnu}
\end{equation}

\subsection{2D Worked example}

Consider a cantilever as shown in \figref{fig:2DLinearElasticCantileverExample}. 

\epstexfigure{SolidMechanics/svgs/TwoDLinearCantilever.eps_tex}{2D linear elastic cantilever example.}{Two dimensional linear elastic cantilever example. The cantilever is of length $L$ and height $H$ and is built into the wall at the left hand end. A downward force, $f$, is applied at the right hand end. The cantilever is modelled with four simplex elements.}{fig:2DLinearElasticCantileverExample}{0.6}

The static system is governed by the linear system of equations
\begin{equation}
  \matr{K}\vectr{u}=\vectr{f}
\end{equation}
where $\matr{K}$ is given by \eqnref{eqn:LinearElasticityElementalStiffnessMatrix} \ie
\begin{equation}
    K^{ik\alpha\beta}_{mn}=\gsf{m}{\alpha}\gsf{n}{\beta}
  \gint{\vectr{0}}{\vectr{1}}{\fnof{\linearelasticitytensorsymbol^{ijkl}}{\elementcoordinatevector}
    \delby{\idxgbfn{i}{m}{\alpha}{\elementcoordinatevector}}{\elementcoordinate{r}}
    \delby{\idxgbfn{k}{n}{\beta}{\elementcoordinatevector}}{\elementcoordinate{s}}
    \delby{\elementcoordinate{r}}{\coordinate{j}}\delby{\elementcoordinate{s}}{\coordinate{l}}
    \abs{\jacobian{\Omega_{e}}{\elementcoordinatevector}}}{\elementcoordinatevector}
\end{equation}
where the elasticity tensor, $\linearelasticitytensor$, is given by \eqnref{eqn:LinearElasticityPlaneStressElasticityTensorVoigtEnu} for plane stress elements. The only non-zero entries in the elasticity tensor are $\linearelasticitytensorsymbol^{1111}=\linearelasticitytensorsymbol^{2222}=\dfrac{\youngsmodulussymbol}{1-\poissonsratiosymbol^{2}}$, $\linearelasticitytensorsymbol^{1122}=\linearelasticitytensorsymbol^{2211}=\dfrac{\youngsmodulussymbol\poissonsratiosymbol}{1-\poissonsratiosymbol^{2}}$, and $\linearelasticitytensorsymbol^{1212}=\dfrac{\youngsmodulussymbol}{2\pbrac{1+\poissonsratiosymbol}}$.

major ijkl=klij
minor ijkl=jikl=jilk

c1111 -
c2111 mi
c1211 x
c2211 -
c1121 mi
c2121
c1221
c2221 mi
c1112 x
c2112
c1212
c2212 x
c1122 -
c2122 mi
c1222 x
c2222 -

Substituting the non-zero elasticity tensor entries into the formulae for the stiffness matrix gives
\begin{equation}
  \begin{aligned}
    K^{11\alpha\beta}_{mn}&=\dfrac{\youngsmodulussymbol}{1-\poissonsratiosymbol^{2}}\gsf{m}{\alpha}\gsf{n}{\beta}
    \gint{\vectr{0}}{\vectr{1}}{
      \delby{\idxgbfn{1}{m}{\alpha}{\elementcoordinatevector}}{\elementcoordinate{r}}
      \delby{\idxgbfn{1}{n}{\beta}{\elementcoordinatevector}}{\elementcoordinate{s}}
      \delby{\elementcoordinate{r}}{\coordinate{1}}\delby{\elementcoordinate{s}}{\coordinate{1}}
      \abs{\jacobian{\Omega_{e}}{\elementcoordinatevector}}}{\elementcoordinatevector} \\
    K^{22\alpha\beta}_{mn}&=\dfrac{\youngsmodulussymbol}{1-\poissonsratiosymbol^{2}}\gsf{m}{\alpha}\gsf{n}{\beta}
    \gint{\vectr{0}}{\vectr{1}}{
      \delby{\idxgbfn{2}{m}{\alpha}{\elementcoordinatevector}}{\elementcoordinate{r}}
      \delby{\idxgbfn{2}{n}{\beta}{\elementcoordinatevector}}{\elementcoordinate{s}}
      \delby{\elementcoordinate{r}}{\coordinate{2}}\delby{\elementcoordinate{s}}{\coordinate{2}}
      \abs{\jacobian{\Omega_{e}}{\elementcoordinatevector}}}{\elementcoordinatevector} \\
    K^{12\alpha\beta}_{mn}&=\dfrac{\youngsmodulussymbol\poissonsratiosymbol}{1-\poissonsratiosymbol^{2}}\gsf{m}{\alpha}\gsf{n}{\beta}
    \gint{\vectr{0}}{\vectr{1}}{
      \delby{\idxgbfn{1}{m}{\alpha}{\elementcoordinatevector}}{\elementcoordinate{r}}
      \delby{\idxgbfn{2}{n}{\beta}{\elementcoordinatevector}}{\elementcoordinate{s}}
      \delby{\elementcoordinate{r}}{\coordinate{1}}\delby{\elementcoordinate{s}}{\coordinate{2}}
      \abs{\jacobian{\Omega_{e}}{\elementcoordinatevector}}}{\elementcoordinatevector} \\
    K^{21\alpha\beta}_{mn}&=\dfrac{\youngsmodulussymbol\poissonsratiosymbol}{1-\poissonsratiosymbol^{2}}\gsf{m}{\alpha}\gsf{n}{\beta}
    \gint{\vectr{0}}{\vectr{1}}{
      \delby{\idxgbfn{2}{m}{\alpha}{\elementcoordinatevector}}{\elementcoordinate{r}}
      \delby{\idxgbfn{1}{n}{\beta}{\elementcoordinatevector}}{\elementcoordinate{s}}
      \delby{\elementcoordinate{r}}{\coordinate{2}}\delby{\elementcoordinate{s}}{\coordinate{1}}
      \abs{\jacobian{\Omega_{e}}{\elementcoordinatevector}}}{\elementcoordinatevector}\\
    K^{11\alpha\beta}_{mn}&=\dfrac{\youngsmodulussymbol}{2\pbrac{1+\poissonsratiosymbol}}\gsf{m}{\alpha}\gsf{n}{\beta}
    \gint{\vectr{0}}{\vectr{1}}{
      \delby{\idxgbfn{1}{m}{\alpha}{\elementcoordinatevector}}{\elementcoordinate{r}}
      \delby{\idxgbfn{1}{n}{\beta}{\elementcoordinatevector}}{\elementcoordinate{s}}
      \delby{\elementcoordinate{r}}{\coordinate{2}}\delby{\elementcoordinate{s}}{\coordinate{2}}
      \abs{\jacobian{\Omega_{e}}{\elementcoordinatevector}}}{\elementcoordinatevector}\\
    K^{12\alpha\beta}_{mn}&=\dfrac{\youngsmodulussymbol}{2\pbrac{1+\poissonsratiosymbol}}\gsf{m}{\alpha}\gsf{n}{\beta}
    \gint{\vectr{0}}{\vectr{1}}{
      \delby{\idxgbfn{1}{m}{\alpha}{\elementcoordinatevector}}{\elementcoordinate{r}}
      \delby{\idxgbfn{2}{n}{\beta}{\elementcoordinatevector}}{\elementcoordinate{s}}
      \delby{\elementcoordinate{r}}{\coordinate{2}}\delby{\elementcoordinate{s}}{\coordinate{1}}
      \abs{\jacobian{\Omega_{e}}{\elementcoordinatevector}}}{\elementcoordinatevector}\\
    K^{21\alpha\beta}_{mn}&=\dfrac{\youngsmodulussymbol}{2\pbrac{1+\poissonsratiosymbol}}\gsf{m}{\alpha}\gsf{n}{\beta}
    \gint{\vectr{0}}{\vectr{1}}{
      \delby{\idxgbfn{2}{m}{\alpha}{\elementcoordinatevector}}{\elementcoordinate{r}}
      \delby{\idxgbfn{1}{n}{\beta}{\elementcoordinatevector}}{\elementcoordinate{s}}
      \delby{\elementcoordinate{r}}{\coordinate{1}}\delby{\elementcoordinate{s}}{\coordinate{2}}
      \abs{\jacobian{\Omega_{e}}{\elementcoordinatevector}}}{\elementcoordinatevector}\\
    K^{22\alpha\beta}_{mn}&=\dfrac{\youngsmodulussymbol}{2\pbrac{1+\poissonsratiosymbol}}\gsf{m}{\alpha}\gsf{n}{\beta}
    \gint{\vectr{0}}{\vectr{1}}{
      \delby{\idxgbfn{2}{m}{\alpha}{\elementcoordinatevector}}{\elementcoordinate{r}}
      \delby{\idxgbfn{2}{n}{\beta}{\elementcoordinatevector}}{\elementcoordinate{s}}
      \delby{\elementcoordinate{r}}{\coordinate{1}}\delby{\elementcoordinate{s}}{\coordinate{1}}
      \abs{\jacobian{\Omega_{e}}{\elementcoordinatevector}}}{\elementcoordinatevector}
  \end{aligned}  
\end{equation}


\subsection{Analytic solutions}

\subsubsection{Cantilever beam}

Consider a cantilever as shown in \figref{fig:AnalyticCantilever}.

The beam equation is given by
\begin{equation}
  EI\dtwosqby{\fnof{y}{x}}{x}=\fnof{M}{x}
\end{equation}
where $E$ is Young's modulus, $I$ is the second moment of inertia, $x$
is the coordinate along the cantilever, $\fnof{y}{x}$ is the
displacment, and $\fnof{M}{x}$ is the bending moment.

For the loading condition shown in \figref{fig:AnalyticCantilever} we
have $\fnof{M}{x}=F\pbrac{L-x}$. We can thus integrate the beam
equation to obtain the equation for deflection \ie
\begin{equation}
  \begin{aligned}
    EI\dtwosqby{\fnof{y}{x}}{x}&=\fnof{M}{x}\\
    EI\dtwosqby{\fnof{y}{x}}{x}&=F\pbrac{L-x}\\
    EI\dby{\fnof{y}{x}}{x}&=FLx-\frac{1}{2}Fx^{2}+C_{1}\\
    EI\fnof{y}{x}&=\frac{1}{2}FLx^{2}-\frac{1}{6}Fx^{3}+C_{1}x+C_{2}
  \end{aligned}
\end{equation}

For a built in cantilever the boundary conditions are $\fnof{y}{0}=0$
which implies that $C_{2}=0$ and
$\evalat{\dby{\fnof{y}{x}}{x}}{x=0}=0$ which implies that $C_{1}=0$.

The equation for deflection is thus
\begin{equation}
  \fnof{y}{x}=\dfrac{Fx^{2}\pbrac{3L-x}}{6EI}
\end{equation}
and the maximum deflection is
\begin{equation}
  \fnof{y}{L}=\dfrac{FL^{3}}{3EI}
\end{equation}

For a simple rectangular beam of height $H$ and width $W$ the second momemnt of area is given by
\begin{equation}
  I=\dfrac{WH^{3}}{12}
\end{equation}
and thus the equation for deflection is
\begin{equation}
  \fnof{y}{x}=\dfrac{2Fx^{2}\pbrac{3L-x}}{EWH^{3}}
\end{equation}
and the maximum end deflection is given by
\begin{equation}
  \fnof{y}{L}=\dfrac{4FL^{3}}{3EWH^{3}}
\end{equation}


\subsection{\OpenCMISS fields}
\label{subsec:LinearElasticityOpenCMISSFields}

The following fields are used in the \OpenCMISS implementation of linear elasticity equations.

\subsubsection{Equations set field}

An equations set field is not used for linear elasticity equations sets.

\subsubsection{Geometric field}

The geometric field definition for the coordinate vector
$\vectr{x}\in\rntopology{n}$, where $n$ is the number of dimensions,
is given in \tabref{tab:OpenCMISSGeometricFieldLinearElasticityEQS}.

\begin{table}[htb] \centering
  \begin{tabular}{|c|c|c|} \hline
    \emph{Field variable type} & \emph{Geometric field component} & \emph{Interpretation} \\ \hline \hline
    \compcode{FIELD\_U\_VARIABLE\_TYPE} & $1$ & $x^{1}$ \\ 
    & \vdots & \vdots \\ 
    & $n$ & $x^{n}$ \\ \hline
  \end{tabular}
  \caption{\OpenCMISS geometric field components for linear elasticity equation sets.}
  \label{tab:OpenCMISSGeometricFieldLinearElasticityEQS}
\end{table}

\subsubsection{Fibre field}

A fibre field can be optionally used to define the fibre coordinate
system rotation angles $\vectr{\alpha}\in\rntopology{n}$, where $n$ is
the number of dimensions, is given in
\tabref{tab:OpenCMISSFibreFieldLinearElasticityEQS}.

\begin{table}[htb] \centering
  \begin{tabular}{|c|c|c|} \hline
    \emph{Field variable type} & \emph{Fibre field component} & \emph{Interpretation} \\ \hline \hline
    \compcode{FIELD\_U\_VARIABLE\_TYPE} & $1$ & $\alpha^{1}$ \\ 
    & \vdots & \vdots \\ 
    & $n$ & $\alpha^{n}$ \\ \hline
  \end{tabular}
  \caption{\OpenCMISS fibre field components for linear elasiticity equation sets.}
  \label{tab:OpenCMISSFibreFieldLinearElasticityEQS}
\end{table}

\subsubsection{Dependent field}

The dependent field defintion for $\fnof{u}{\vectr{x},t}$ is given in \tabref{tab:OpenCMISSDependentFieldLinearElasticityEQS}.

\begin{table}[htb] \centering
  \begin{tabular}{|c|c|c|} \hline
    \emph{Field variable type} & \emph{Dependent field component} & \emph{Interpretation} \\ \hline \hline
    \compcode{FIELD\_U\_VARIABLE\_TYPE} & $1$ & $u$ \\ \hline
  \end{tabular}
  \caption{\OpenCMISS dependent field components for linear elasticity equation sets.}
  \label{tab:OpenCMISSDependentFieldLinearElasticityEQS}
\end{table}
  
\subsubsection{Materials field}

For an
