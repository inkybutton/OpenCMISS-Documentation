\section{Bidomain Equation}
\label{sec:Bidomain}

\subsection{Governing Equations}
\label{subsec:BidomainGoverningEquations}

The bidomain model \citep{henriquez:1993} can be thought of as two
co-existant intra and extracelluar spaces. The potential in the
intracellular space is denoted as $\intracellularpotential$ and the
potential in the extracellular space is denoted as
$\extracellularpotential$. The intra- and extra-cellular potentials are
related through the transmembrane voltage, $\transmembranevoltage$,
\ie
\begin{equation}
  \transmembranevoltage=\intracellularpotential-\extracellularpotential
  \label{eqn:TransMembraneVoltageDefinition}
\end{equation} 

The intra- and extra- cellular current densities can be found from
\begin{align}
  \intracellularcurrentdensityvector&=
  -\sharptensor{\intracellularconductivitytensor}\gradient{}{\intracellularpotential}
  \label{eqn:IntraCellularCurrentDensityVectorDefinition} \\
  \extracellularcurrentdensityvector&=
  -\sharptensor{\extracellularconductivitytensor}\gradient{}{\extracellularpotential}
  \label{eqn:ExtraCellularCurrentDensityVectorDefinition}  
\end{align}
where $\intracellularcurrentdensityvector$ is the intracellular
current density vector, $\extracellularcurrentdensityvector$ is the
extracellular current density vector,
$\intracellularconductivitytensor$ is the intracellular conductivity
tensor, $\extracellularconductivitytensor$ is the extracellular
conductivity tensor, and the negative signs are to ensure that current
flows from areas of high potential to areas of low potential.

Now, any current which leaves one domain must cross the cell membrane
and flow into the other domain. This means that the change in current
density in one domain must be of equal but opposite magnitude to that
of the other domain. This change in current density must also be
equation to the current density across the membrane \ie
\begin{equation}
  -\divergence{}{\intracellularcurrentdensityvector}=
  \divergence{}{\extracellularcurrentdensityvector}=
  \membraneareavolumeratio\transmembranecurrent
  \label{eqn:BidomainConservationOfCurrent}
\end{equation}
where $\membraneareavolumeratio$ is the surface area of the cell
membrane to the cell volume ratio and $\transmembranecurrent$ is the
transmembrane current per unit area.

\Eqnref{eqn:BidomainConservationOfCurrent} represents the conservation
of current densities in the combined domains. Using
\eqnrefs{eqn:IntraCellularCurrentDensityVectorDefinition}{eqn:ExtraCellularCurrentDensityVectorDefinition}
we can rewrite \eqnref{eqn:BidomainConservationOfCurrent} as
\begin{align}
  \divergence{}{\intracellularconductivitytensor\gradient{}{\intracellularpotential}}&=
  \membraneareavolumeratio\transmembranecurrent \label{eqn:IntraCellularCurrentDensityConservation} \\
  \divergence{}{\extracellularconductivitytensor\gradient{}{\extracellularpotential}}&=
  -\membraneareavolumeratio\transmembranecurrent \label{eqn:ExtraCellularCurrentDensityConservation} 
\end{align}
or
\begin{equation}
  \divergence{}{\intracellularconductivitytensor\gradient{}{\intracellularpotential}}=
  -\divergence{}{\extracellularconductivitytensor\gradient{}{\extracellularpotential}}  
\end{equation}

If we now subtract
$\divergence{}{\intracellularconductivitytensor\gradient{}{\extracellularpotential}}$
from both sides we obtain
\begin{equation}
  \divergence{}{\intracellularconductivitytensor\gradient{}{\intracellularpotential}}-
  \divergence{}{\intracellularconductivitytensor\gradient{}{\extracellularpotential}}=
  -\divergence{}{\extracellularconductivitytensor\gradient{}{\extracellularpotential}}-
  \divergence{}{\intracellularconductivitytensor\gradient{}{\extracellularpotential}}
\end{equation}
or, using \eqnref{eqn:TransMembraneVoltageDefinition}, we have
\begin{equation}
  \divergence{}{\intracellularconductivitytensor\gradient{}{\transmembranevoltage}}=
  -\divergence{}{\pbrac{\intracellularconductivitytensor+\extracellularconductivitytensor}\gradient{}{\extracellularpotential}}
  \label{eqn:FirstBidomainEquation}
\end{equation}

\Eqnref{eqn:FirstBidomainEquation} is known as the first, or elliptic, bidomain equation.

Now, the current flow across the membrane, $\transmembranecurrent$,
can be described by a time dependent capacitive current and an ionic
current \ie
\begin{equation}
  \transmembranecurrent=\membranecapacitance\delby{\transmembranevoltage}{t}+
  \fnof{\ioniccurrent}{\cellstatevariablevector,\transmembranevoltage}
  \label{eqn:TransMembraneCurrentDefinition}
\end{equation}
where $\membranecapacitance$ is the membrane capacitance per unit
area,  $\fnof{\ioniccurrent}{\cellstatevariablevector,\transmembranevoltage}$ is the sum of all ionic currents across the
cell membrane, and $\cellstatevariablevector$ is the vector of cell state variables for a particular cell model.

If we now substitute \eqnref{eqn:TransMembraneCurrentDefinition} into
\eqnref{eqn:IntraCellularCurrentDensityConservation} we obtain
\begin{equation}
  \divergence{}{\intracellularconductivitytensor\gradient{}{\intracellularpotential}}=
  \membraneareavolumeratio\pbrac{\membranecapacitance\delby{\transmembranevoltage}{t}+
    \fnof{\ioniccurrent}{\cellstatevariablevector,\transmembranevoltage}}
  \label{eqn:IntraCellularCurrentDensityConservationTransmembraneVoltage}
\end{equation}

Now, if we add
$\divergence{}{\intracellularconductivitytensor\gradient{}{\extracellularpotential}}$
to both sides of
\eqnref{eqn:IntraCellularCurrentDensityConservationTransmembraneVoltage}
we obtain
\begin{equation}
  \divergence{}{\intracellularconductivitytensor\gradient{}{\intracellularpotential}} +
  \divergence{}{\intracellularconductivitytensor\gradient{}{\extracellularpotential}} =
  \membraneareavolumeratio\pbrac{\membranecapacitance\delby{\transmembranevoltage}{t}+
    \fnof{\ioniccurrent}{\cellstatevariablevector,\transmembranevoltage}} +
  \divergence{}{\intracellularconductivitytensor\gradient{}{\extracellularpotential}}  
\end{equation}
or, rearranging, we obtain
\begin{equation}
  \divergence{}{\intracellularconductivitytensor\gradient{}{\intracellularpotential}} +
  \divergence{}{\intracellularconductivitytensor\gradient{}{\extracellularpotential}} -
  \divergence{}{\intracellularconductivitytensor\gradient{}{\extracellularpotential}} =
  \membraneareavolumeratio\pbrac{\membranecapacitance\delby{\transmembranevoltage}{t}+
    \fnof{\ioniccurrent}{\cellstatevariablevector,\transmembranevoltage}} 
\end{equation}
which, using \eqnref{eqn:TransMembraneVoltageDefinition}, we can rewrite as
\begin{equation}
  \divergence{}{\intracellularconductivitytensor\gradient{}{\transmembranevoltage}} +
  \divergence{}{\intracellularconductivitytensor\gradient{}{\extracellularpotential}} =
  \membraneareavolumeratio\pbrac{\membranecapacitance\delby{\transmembranevoltage}{t}+
    \fnof{\ioniccurrent}{\cellstatevariablevector,\transmembranevoltage}} 
  \label{eqn:SecondBidomainEquation}
\end{equation}

\Eqnref{eqn:SecondBidomainEquation} is known as the second, or
parabolic, bidomain equation.

Now we can combine
\eqnrefs{eqn:FirstBidomainEquation}{eqn:SecondBidomainEquation},
allow for stimulus currents in each domain, and rearrange to obtain the
\emph{bidomain equations}. The two bidomain equations are
\begin{equation}
  \addtolength{\fboxsep}{5pt}
  \boxed{
    \membraneareavolumeratio\membranecapacitance\delby{\transmembranevoltage}{t}-
    \divergence{}{\pbrac{\tensor{\sigma}_{i}\gradient{}{\transmembranevoltage}}}=
    \divergence{}{\pbrac{\tensor{\sigma}_{i}\gradient{}{\extracellularpotential}}}-
    \membraneareavolumeratio\fnof{\ioniccurrent}{\cellstatevariablevector,\transmembranevoltage}+
    \intracellularstimuluscurrent
  } \label{eqn:BidomainEquation1}
\end{equation}
and
\begin{equation}
  \addtolength{\fboxsep}{5pt}
  \boxed{
    \divergence{}{\pbrac{\pbrac{\tensor{\sigma}_{e}+\tensor{\sigma}_{i}}\gradient{}{\extracellularpotential}}}=
    -\divergence{}{\pbrac{\tensor{\sigma}_{i}\gradient{}{\transmembranevoltage}}}+\extracellularstimuluscurrent
  } \label{eqn:BidomainEquation2}
\end{equation}
where $\intracellularstimuluscurrent$ is the intracellular stimulus
current and $\extracellularstimuluscurrent$ is the extracellular
stimulus current.

\subsection{Weak formulation}
\label{subsec:BidomainWeakForm}

The corresponding weak form of \eqnref{eqn:BidomainEquation1} is
\begin{equation}
  \gint{\Omega}{}{\pbrac{\membraneareavolumeratio\membranecapacitance\delby{\transmembranevoltage}{t}-
      \divergence{}{\pbrac{\tensor{\sigma}_{i}\gradient{}{\transmembranevoltage}}}-
      \divergence{}{\pbrac{\tensor{\sigma}_{i}\gradient{}{\extracellularpotential}}}+
      \membraneareavolumeratio\fnof{\ioniccurrent}{\cellstatevariablevector,\transmembranevoltage}-
      \intracellularstimuluscurrent}w}{\Omega}=0
  \label{eqn:Bidomain1WeakForm1}
\end{equation}
and the weak form of \eqnref{eqn:BidomainEquation2} is
\begin{equation}
  \gint{\Omega}{}{\pbrac{\divergence{}{\pbrac{\pbrac{\tensor{\sigma}_{e}+\tensor{\sigma}_{i}}\gradient{}{\extracellularpotential}}}-
      \divergence{}{\pbrac{\tensor{\sigma}_{i}\gradient{}{\transmembranevoltage}}}-\extracellularstimuluscurrent}w}{\Omega}=0
  \label{eqn:Bidomain2WeakForm1}
\end{equation}

Applying the divergence thereom to \eqnref{eqn:Bidomain1WeakForm1} gives
\begin{equation}
  \begin{split}
    \gint{\Omega}{}{\membraneareavolumeratio\membranecapacitance\delby{\transmembranevoltage}{t}}{\Omega}
    +\gint{\Omega}{}{\dotprod{\pbrac{\tensor{\sigma}_{i}\gradient{}{\transmembranevoltage}}}{\gradient{}{w}}}{\Omega}
    -\gint{\Gamma}{}{\dotprod{\pbrac{\tensor{\sigma}_{i}\gradient{}{\transmembranevoltage}}}{\normal}}{\Gamma} \\
    +\gint{\Omega}{}{\dotprod{\pbrac{\tensor{\sigma}_{i}\gradient{}{\extracellularpotential}}}{\gradient{}{w}}}{\Omega}
    -\gint{\Gamma}{}{\dotprod{\pbrac{\tensor{\sigma}_{i}\gradient{}{\extracellularpotential}}}{\normal}}{\Gamma} \\
    +\gint{\Omega}{}{\pbrac{\membraneareavolumeratio\fnof{\ioniccurrent}{\cellstatevariablevector,\transmembranevoltage}}w}{\Omega}
    -\gint{\Omega}{}{\intracellularstimuluscurrent w}{\Omega}=0
  \end{split}
  \label{eqn:Bidomain1WeakForm2}
\end{equation}
and to \eqnref{eqn:Bidomain2WeakForm1} gives
\begin{equation}
  \begin{split}
    -\gint{\Omega}{}{\dotprod{\pbrac{\pbrac{\tensor{\sigma}_{e}+
            \tensor{\sigma}_{i}}\gradient{}{\extracellularpotential}}}{\gradient{}{w}}}{\Omega}
    +\gint{\Gamma}{}{\dotprod{\pbrac{\pbrac{\tensor{\sigma}_{e}+
            \tensor{\sigma}_{i}}\gradient{}{\extracellularpotential}}}{\normal}}{\Gamma} \\
    +\gint{\Omega}{}{\dotprod{\pbrac{\tensor{\sigma}_{i}\gradient{}{\transmembranevoltage}}}{\gradient{}{w}}}{\Omega}
    -\gint{\Gamma}{}{\dotprod{\pbrac{\tensor{\sigma}_{i}\gradient{}{\transmembranevoltage}}}{\normal}}{\Gamma}
    -\gint{\Omega}{}{\extracellularstimuluscurrent w}{\Omega}=0
  \end{split}
\end{equation}

\subsection{Tensor notation}
\label{subsec:BidomainTensorNotation}

\Eqnref{eqn:Bidomain1WeakForm2} in tensor notation is given by
\begin{equation}
  \begin{split}
    \gint{\Omega}{}{\membraneareavolumeratio\membranecapacitance\delby{\transmembranevoltage}{t}}{\Omega}
    +\gint{\Omega}{}{\generalmetrictensorsymbol^{lk}\conductivitytensorsymbol^{j}_{.ki}
      \covarderiv{\transmembranevoltage}{j}\covarderiv{w}{l}}{\Omega}
    -\gint{\Gamma}{}{\generalmetrictensorsymbol^{lk}\conductivitytensorsymbol^{j}_{.ki}
      \covarderiv{\transmembranevoltage}{k}\normalsymbol_{l}w}{\Gamma} \\
    +\gint{\Omega}{}{\generalmetrictensorsymbol^{lk}\conductivitytensorsymbol^{j}_{.ki}
      \covarderiv{\extracellularpotential}{j}\covarderiv{w}{l}}{\Omega}
    -\gint{\Gamma}{}{\generalmetrictensorsymbol^{lk}\conductivitytensorsymbol^{j}_{.ki}
      \covarderiv{\extracellularpotential}{k}\normalsymbol_{l}w}{\Gamma} \\
    +\gint{\Omega}{}{\pbrac{\membraneareavolumeratio\fnof{\ioniccurrent}{\cellstatevariablevector,\transmembranevoltage}}w}{\Omega}
    -\gint{\Omega}{}{\intracellularstimuluscurrent w}{\Omega}=0    
  \end{split}
  \label{eqn:Bidomain1TensorNotation1}
\end{equation}
and \eqnref{eqn:Bidomain2WeakForm2} in tensor notation is given by
\begin{equation}
  \begin{split}
    -\gint{\Omega}{}{\generalmetrictensorsymbol^{lk}\pbrac{\conductivitytensorsymbol^{j}_{.ke}+\conductivitytensorsymbol^{j}_{.ki}}
      \covarderiv{\extracellularpotential}{j}\covarderiv{w}{l}}{\Omega}
    +\gint{\Gamma}{}{\generalmetrictensorsymbol^{lk}\pbrac{\conductivitytensorsymbol^{j}_{.ke}+\conductivitytensorsymbol^{j}_{.ki}}
      \covarderiv{\extracellularpotential}{k}\normalsymbol_{l}w}{\Gamma} \\
    +\gint{\Omega}{}{\generalmetrictensorsymbol^{lk}\conductivitytensorsymbol^{j}_{.ki}
      \covarderiv{\transmembranevoltage}{j}\covarderiv{w}{l}}{\Omega}
    -\gint{\Gamma}{}{\generalmetrictensorsymbol^{lk}\conductivitytensorsymbol^{j}_{.ki}
      \covarderiv{\extracellularpotential}{k}\normalsymbol_{l}w}{\Gamma}
    -\gint{\Omega}{}{\extracellularstimuluscurrent w}{\Omega}=0    
  \end{split}
  \label{eqn:Bidomain2TensorNotation1}
\end{equation}

Rearranging terms gives
\begin{equation}
  \begin{split}
    \gint{\Omega}{}{\membraneareavolumeratio\membranecapacitance\delby{\transmembranevoltage}{t}}{\Omega}
    +\gint{\Omega}{}{\generalmetrictensorsymbol^{lk}\conductivitytensorsymbol^{j}_{.ki}
      \covarderiv{\transmembranevoltage}{j}\covarderiv{w}{l}}{\Omega}
    -\gint{\Gamma}{}{\generalmetrictensorsymbol^{lk}\conductivitytensorsymbol^{j}_{.ki}
      \covarderiv{\transmembranevoltage}{k}\normalsymbol_{l}w}{\Gamma} \\
    +\gint{\Omega}{}{\generalmetrictensorsymbol^{lk}\conductivitytensorsymbol^{j}_{.ki}
      \covarderiv{\extracellularpotential}{j}\covarderiv{w}{l}}{\Omega}
    -\gint{\Gamma}{}{\generalmetrictensorsymbol^{lk}\conductivitytensorsymbol^{j}_{.ki}
      \covarderiv{\extracellularpotential}{k}\normalsymbol_{l}w}{\Gamma} \\
    +\gint{\Omega}{}{\pbrac{\membraneareavolumeratio\fnof{\ioniccurrent}{\cellstatevariablevector,\transmembranevoltage}}w}{\Omega}
    -\gint{\Omega}{}{\intracellularstimuluscurrent w}{\Omega}=0    
  \end{split}
  \label{eqn:Bidomain1TensorNotation2}
\end{equation}
