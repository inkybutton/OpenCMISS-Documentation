\section{Monodomain Equation}
\label{sec:MonodomainEquation}

\subsection{Governing Equations}
\label{subsec:MonodomainGoverningEquations}

Under certain conditions the bidomain equations given in
\eqnref{eqn:BidomainEquation1} and \eqnref{eqn:BidomainEquation2} can
be simplified. If we assume that the intra- and extra-cellular
conductivity tensors have equal anisotropy ratios \ie
\begin{equation}
  \extracellularconductivitytensor=\lambda\intracellularconductivitytensor
  \label{eqn:MonodomainConductivityAssumption}
\end{equation}
where $\lambda$ is the conductivity scaling, we can substitute
\eqnref{eqn:MonodomainConductivityAssumption} into
\eqnref{eqn:FirstBidomainEquation} to obtain
\begin{equation}
  \begin{aligned}
    \divergence{}{\intracellularconductivitytensor\gradient{}{\transmembranevoltage}}&=
    -\divergence{}{\pbrac{\pbrac{\intracellularconductivitytensor+\extracellularconductivitytensor}
      \gradient{}{\extracellularpotential}}}\\&=
    -\divergence{}{\pbrac{\pbrac{\intracellularconductivitytensor+\lambda\intracellularconductivitytensor}
      \gradient{}{\extracellularpotential}}}\\
    &=-\divergence{}{\pbrac{\pbrac{1+\lambda}\intracellularconductivitytensor\gradient{}{\extracellularpotential}}}
  \end{aligned}
\end{equation}
or
\begin{equation}
  \divergence{}{\pbrac{\intracellularconductivitytensor\gradient{}{\extracellularpotential}}}=
  \dfrac{-1}{1+\lambda}\divergence{}{\intracellularconductivitytensor\gradient{}{\transmembranevoltage}}
  \label{eqn:MonodomainFirstBidomainEquation}
\end{equation}

If we now substitute \eqnref{eqn:MonodomainFirstBidomainEquation} into \eqnref{eqn:SecondBidomainEquation} we obtain
\begin{equation}
  \begin{aligned}
    \membraneareavolumeratio\pbrac{\membranecapacitance\delby{\transmembranevoltage}{t}+\ioniccurrent}&=
    \divergence{}{\intracellularconductivitytensor\gradient{}{\transmembranevoltage}} +
    \dfrac{-1}{1+\lambda}\divergence{}{\intracellularconductivitytensor\gradient{}{\transmembranevoltage}} \\
    &=\dfrac{\pbrac{1+\lambda}\divergence{}{\intracellularconductivitytensor\gradient{}{\transmembranevoltage}} -
      \divergence{}{\intracellularconductivitytensor\gradient{}{\transmembranevoltage}}}{1+\lambda} \\
    &=\dfrac{\divergence{}{\intracellularconductivitytensor\gradient{}{\transmembranevoltage}}+
      \lambda\divergence{}{\intracellularconductivitytensor\gradient{}{\transmembranevoltage}} -
      \divergence{}{\intracellularconductivitytensor\gradient{}{\transmembranevoltage}}}{1+\lambda} \\
    &=\dfrac{\lambda}{1+\lambda}\divergence{}{\intracellularconductivitytensor\gradient{}{\transmembranevoltage}}\\
    &=\divergence{}{\pbrac{\dfrac{\lambda}{1+\lambda}\intracellularconductivitytensor\gradient{}{\transmembranevoltage}}}
  \end{aligned}
\end{equation}
We can now define a \emph{monodomain conductivity tensor}\index{Monodomain conductivity tensor}, $\monodomainconductivitytensor$, as
\begin{equation}
  \monodomainconductivitytensor=\dfrac{\lambda\intracellularconductivitytensor}{1+\lambda}=
  \dfrac{\intracellularconductivitytensor\extracellularconductivitytensor}{\intracellularconductivitytensor+
    \extracellularconductivitytensor}
  \label{eqn:MonodomainConductivityTensor}  
\end{equation}

\Eqnref{eqn:MonodomainConductivityTensor} defines the monodomain
conductivity tensor as the arithmetic mean of the intra- and
extra-cellular conductivity tensors.

Subsituting \eqnref{eqn:MonodomainConductivityTensor} into
\eqnref{eqn:SecondBidomainEquation} and adding a stimulus current
gives us
\begin{equation}
  \addtolength{\fboxsep}{5pt}
  \boxed{
    \membraneareavolumeratio\membranecapacitance\delby{\transmembranevoltage}{t}-
    \divergence{}{\pbrac{\intracellularconductivitytensor\gradient{}{\transmembranevoltage}}}=
    -\membraneareavolumeratio\ioniccurrent+\stimuluscurrent
  }
  \label{eqn:MonodomainEquation}
\end{equation}
where $\stimuluscurrent$ is the monodomain stimulus current.

\Eqnref{eqn:MonodomainEquation} is known as the \emph{monodomain equation}\index{Monodomain equation}.

The boundary condition for the monodomain equation is that there is no current flow out of the domain \ie
\begin{equation}
  \dotprod{\intracellularconductivitytensor\gradient{}{\transmembranevoltage}}{\normal}=0
  \label{eqn:MonodomainBC}
\end{equation}

\subsection{Weak formulation}
\label{subsec:MonodomainWeakForm}

The corresponding weak form of \eqnref{eqn:MonodomainEquation} is
\begin{equation}
  \gint{\Omega}{}{\pbrac{\membraneareavolumeratio\membranecapacitance\delby{\transmembranevoltage}{t}-
      \divergence{}{\pbrac{\intracellularconductivitytensor\gradient{}{\transmembranevoltage}}}
      +\membraneareavolumeratio\ioniccurrent-\stimuluscurrent}w}{\Omega}=0
  \label{eqn:MonodomainWeakForm1}
\end{equation}

Applying the divergence thereom to \eqnref{eqn:MonodomainWeakForm1} gives
\begin{equation}
  \gint{\Omega}{}{\membraneareavolumeratio\membranecapacitance\delby{\transmembranevoltage}{t}w}{\Omega}
  +\gint{\Omega}{}{\dotprod{\pbrac{\intracellularconductivitytensor\gradient{}{\transmembranevoltage}}}{\gradient{}{w}}}{\Omega}
  -\gint{\Gamma}{}{\dotprod{\pbrac{\intracellularconductivitytensor\gradient{}{\transmembranevoltage}}}{\normal}}{\Gamma}
  +\gint{\Omega}{}{\membraneareavolumeratio\ioniccurrent w}{\Omega}
  -\gint{\Omega}{}{\stimuluscurrent w}{\Omega}=0
  \label{eqn:MonodomainWeakForm2}
\end{equation}

\subsection{Tensor notation}
\label{subsec:MonodomainTensorNotation}

\Eqnref{eqn:MonodomainWeakForm2} in tensor notation is given by
\begin{equation}
  \gint{\Omega}{}{\membraneareavolumeratio\membranecapacitance\dot{\transmembranevoltage}w}{\Omega}
  +\gint{\Omega}{}{\generalmetrictensorsymbol^{lk}\conductivitytensorsymbol^{j}_{.ki}
      \covarderiv{\transmembranevoltage}{j}\covarderiv{w}{l}}{\Omega}
  -\gint{\Gamma}{}{\generalmetrictensorsymbol^{lk}\conductivitytensorsymbol^{j}_{.ki}
      \covarderiv{\transmembranevoltage}{k}\normalsymbol_{l}w}{\Gamma}
  +\gint{\Omega}{}{\membraneareavolumeratio\ioniccurrent w}{\Omega}
  -\gint{\Omega}{}{\stimuluscurrent w}{\Omega}=0
  \label{eqn:MonodomainTensorNotation1}
\end{equation}

Rearranging \eqnref{eqn:MonodomainTensorNotation1} and dividing by $\membraneareavolumeratio\membranecapacitance$ gives
\begin{equation}
  \gint{\Omega}{}{\dot{\transmembranevoltage}w}{\Omega}=
  \dfrac{-1}{\membraneareavolumeratio\membranecapacitance}\gint{\Omega}{}{\generalmetrictensorsymbol^{lk}
    \conductivitytensorsymbol^{j}_{.ki}\covarderiv{\transmembranevoltage}{j}\covarderiv{w}{l}}{\Omega}
  +\dfrac{1}{\membranecapacitance}\gint{\Omega}{}{\pbrac{\ioniccurrent-
      \dfrac{\stimuluscurrent}{\membraneareavolumeratio}}w}{\Omega}
  +\dfrac{1}{\membraneareavolumeratio\membranecapacitance}\gint{\Gamma}{}{qw}{\Gamma}
  \label{eqn:MonodomainTensorNotation2}
\end{equation}
where $q$ is the surface current flux. In the standard monodomain
formulation the surface current flux is zero by virtue of the boundary
condition (\eqnref{eqn:MonodomainBC}). Although it is possible to
extend the monodomain formulation to allow for a non-zero surface
current flux we will assume that it is zero from now on.

\subsection{Finite element formulation}
\label{subsec:MonodomainFEMFormulation}

We can now discretise the spatial domain into finite elements \ie
$\Omega= \displaystyle{\bigcup_{e=1}^{E}}\Omega_{e}$ with
$\Gamma=\displaystyle{\bigcup_{f=1}^{F}}\Gamma_{f}$.
\Eqnref{eqn:MonodomainTensorNotation2} now becomes
\begin{equation}
  \dsum_{e=1}^{E}\gint{\Omega_{e}}{}{\dot{\transmembranevoltage}w}{\Omega}=
  \dsum_{e=1}^{E}\dfrac{-1}{\membraneareavolumeratio\membranecapacitance}\gint{\Omega_{e}}{}{\generalmetrictensorsymbol^{lk}
    \conductivitytensorsymbol^{j}_{.ki}\covarderiv{\transmembranevoltage}{j}\covarderiv{w}{l}}{\Omega}
  +\dsum_{e=1}^{E}\dfrac{1}{\membranecapacitance}\gint{\Omega_{e}}{}{\pbrac{\ioniccurrent-
      \dfrac{\stimuluscurrent}{\membraneareavolumeratio}}w}{\Omega}
  \label{eqn:MonodomainFEM1}
\end{equation}

If we now assume that the dependent variable $\transmembranevoltage$ can be interpolated
separately in space and in time we can write
\begin{equation}
  \fnof{\transmembranevoltage}{\vectr{x},t}=\gbfn{n}{}{\vectr{x}}\fnof{\nodept{\transmembranevoltage}{n}}{t}
\end{equation}
or, in standard interpolation notation within an element,
\begin{equation}
  \fnof{\transmembranevoltage}{\vectr{\xi},t}=\gbfn{n}{\beta}{\vectr{\xi}}
  \fnof{\nodedof{\transmembranevoltage}{n}{\beta}}{t}\gsf{n}{\beta}
\end{equation}
where $\fnof{\nodedof{\transmembranevoltage}{n}{\beta}}{t}$ are the
time varying nodal degrees-of-freedom for node $n$, global derivative
$\beta$, $\gbfn{n}{\beta}{\vectr{\xi}}$ are the corresponding basis
functions and $\gsf{n}{\beta}$ are the scale factors.
