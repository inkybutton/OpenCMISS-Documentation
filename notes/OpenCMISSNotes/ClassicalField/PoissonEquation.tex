\section{Poisson Equations} 

\subsection{Governing equations}

The form of the \emph{general Poisson's equation} with source on a domain $\Omega$ with boundary
$\Gamma$ in \OpenCMISS can be stated as
\begin{equation}
  \addtolength{\fboxsep}{5pt}
  \boxed{
    \divergence{}{\pbrac{\fnof{\tensor{\sigma}}{\vectr{x}}\gradient{}{\fnof{u}{\vectr{x},t}}}}+
    \fnof{s}{\vectr{x},t}=0
  }
  \label{eqn:GeneralPoissonEquation}
\end{equation}
where $\vectr{x}\in\Omega$, $t\in[0,\infty)$, $\fnof{u}{\vectr{x},t}$ is the dependent variable, 
$\fnof{\tensor{\sigma}}{\vectr{x}}$ is the conductivity tensor throughout 
the domain and $\fnof{s}{\vectr{x},t}$ is a source term.

Additional forms of Poisson's equation in \OpenCMISS include a \emph{linear
  source Poisson's equation} given by
\begin{equation}
  \addtolength{\fboxsep}{5pt}
  \boxed{
    \divergence{}{\pbrac{\fnof{\tensor{\sigma}}{\vectr{x}}\gradient{}{\fnof{u}{\vectr{x},t}}}}+
    \fnof{a}{\vectr{x}}\fnof{u}{\vectr{x},t}+\fnof{s}{\vectr{x},t}=0
  }
  \label{eqn:LinearSourcePoissonEquation}
\end{equation}
a \emph{quadratic source Poisson's equation} given by
\begin{equation}
  \addtolength{\fboxsep}{5pt}
  \boxed{
    \divergence{}{\pbrac{\fnof{\tensor{\sigma}}{\vectr{x}}\gradient{}{\fnof{u}{\vectr{x},t}}}}+
    \fnof{a}{\vectr{x}}\fnof{u}{\vectr{x},t}+\fnof{b}{\vectr{x}}\pbrac{\fnof{u}{\vectr{x},t}}^{2}+\fnof{s}{\vectr{x},t}=0
  }
  \label{eqn:QuadraticSourcePoissonEquation}
\end{equation}
and an \emph{exponential source Poisson's equation} given by
\begin{equation}
  \addtolength{\fboxsep}{5pt}
  \boxed{
    \divergence{}{\pbrac{\fnof{\tensor{\sigma}}{\vectr{x}}\gradient{}{\fnof{u}{\vectr{x},t}}}}+
    \fnof{a}{\vectr{x}}e^{\fnof{b}{\vectr{x}}\fnof{u}{\vectr{x},t}}+\fnof{s}{\vectr{x},t}=0
  }
  \label{eqn:ExponentialSourcePoissonEquation}
\end{equation}
where $\fnof{a}{\vectr{x}}$ and $\fnof{b}{\vectr{x}}$ are material coefficients.

Appropriate boundary conditions conditions for the diffusion
equation are specification of Dirichlet boundary conditions on the solution,
$\fnof{g}{\vectr{x},t}$, \ie
\begin{equation}
  \fnof{u}{\vectr{x},t} = \fnof{g}{\vectr{x},t} \quad \vectr{x}\in\Gamma_{D},t\in[0,\infty)
  \label{eqn:PoissonDirichletBC} 
\end{equation}
and/or Neumann conditions in terms of the solution flux in the normal
direction, $\fnof{h}{\vectr{x},t}$, \ie
\begin{equation}
  \fnof{q}{\vectr{x},t} = \dotprod{\pbrac{\fnof{\tensor{\sigma}}{\vectr{x}}
      \gradient{}{\fnof{u}{\vectr{x},t}}}}{\fnof{\normal}{\vectr{x}}} =
  \fnof{h}{\vectr{x},t} \quad \vectr{x}\in\Gamma_{N},t\in[0,\infty)
  \label{eqn:PoissonNeumannBC} 
\end{equation}
where $\fnof{q}{\vectr{x},t}$ is the flux in the normal direction, $\fnof{\normal}{\vectr{x}}$ is the normal
vector to the boudary and $\Gamma=\union{\Gamma_{D}}{\Gamma_{N}}$.

\subsection{Weak formulation}

The corresponding weak forms of \eqnthrurefs{eqn:GeneralPoissonEquation}{eqn:ExponentialSourcePoissonEquation} can be found by
integrating over the spatial domain with a test function \ie
\begin{equation}
  \gint{\Omega}{}{\pbrac{\divergence{}{\pbrac{\tensor{\sigma}\gradient{}{u}}}+s}w}{\Omega}=0
  \label{eqn:GeneralPoissonWeakForm1}
\end{equation}
\begin{equation}
  \gint{\Omega}{}{\pbrac{\divergence{}{\pbrac{\tensor{\sigma}\gradient{}{u}}}+au+s}w}{\Omega}=0
  \label{eqn:LinearSourcePoissonWeakForm1}
\end{equation}
\begin{equation}
  \gint{\Omega}{}{\pbrac{\divergence{}{\pbrac{\tensor{\sigma}\gradient{}{u}}}+bu^{2}+au+s}w}{\Omega}=0
  \label{eqn:QuadraticSourcePoissonWeakForm1}
\end{equation}
\begin{equation}
  \gint{\Omega}{}{\pbrac{\divergence{}{\pbrac{\tensor{\sigma}\gradient{}{u}}}+ae^{bu}+s}w}{\Omega}=0
  \label{eqn:ExponentialSourcePoissonWeakForm1}
\end{equation}
where $w$ is a suitable spatial test function.

Applying the divergence theorem in \eqnref{eqn:DivergenceTheormScalarVector}
to \eqnthrurefs{eqn:GeneralPoissonWeakForm1}{eqn:ExponentialSourcePoissonWeakForm1} gives
\begin{equation}
  -\gint{\Omega}{}{\dotprod{\tensor{\sigma}
      \gradient{}{u}}{\gradient{}{w}}}{\Omega}
  +\gint{\Gamma}{}{\pbrac{\dotprod{\tensor{\sigma}\gradient{}{u}}{\normal}}w}{\Gamma}+
  \gint{\Omega}{}{sw}{\Omega}=0
  \label{eqn:GeneralPoissonWeakForm2}
\end{equation}
\begin{equation}
  -\gint{\Omega}{}{\dotprod{\tensor{\sigma}
      \gradient{}{u}}{\gradient{}{w}}}{\Omega}
  +\gint{\Gamma}{}{\pbrac{\dotprod{\tensor{\sigma}\gradient{}{u}}{\normal}}w}{\Gamma}+
  \gint{\Omega}{}{auw}{\Omega}+\gint{\Omega}{}{sw}{\Omega}=0
  \label{eqn:LinearSourcePoissonWeakForm2}
\end{equation}
\begin{equation}
  -\gint{\Omega}{}{\dotprod{\tensor{\sigma}
      \gradient{}{u}}{\gradient{}{w}}}{\Omega}
  +\gint{\Gamma}{}{\pbrac{\dotprod{\tensor{\sigma}\gradient{}{u}}{\normal}}w}{\Gamma}+
  \gint{\Omega}{}{bu^{2}w}{\Omega}+\gint{\Omega}{}{auw}{\Omega}+\gint{\Omega}{}{sw}{\Omega}=0
  \label{eqn:QuadraticSourcePoissonWeakForm2}
\end{equation}
\begin{equation}
  -\gint{\Omega}{}{\dotprod{\tensor{\sigma}
      \gradient{}{u}}{\gradient{}{w}}}{\Omega}
  +\gint{\Gamma}{}{\pbrac{\dotprod{\tensor{\sigma}\gradient{}{u}}{\normal}}w}{\Gamma}+
  \gint{\Omega}{}{ae^{bu}w}{\Omega}+\gint{\Omega}{}{sw}{\Omega}=0
  \label{eqn:ExponentialSourcePoissonWeakForm2}
\end{equation}
where $\normal$ is the unit outward normal vector to the boundary $\Gamma$.

The generalised form of
\eqnthrurefs{eqn:GeneralPoissonWeakForm2}{eqn:ExponentialSourcePoissonWeakForm2}
which includes all of the terms is 
\begin{equation}
  -\gint{\Omega}{}{\dotprod{\tensor{\sigma}
      \gradient{}{u}}{\gradient{}{w}}}{\Omega}
  +\gint{\Gamma}{}{\pbrac{\dotprod{\tensor{\sigma}\gradient{}{u}}{\normal}}w}{\Gamma}+
  \gint{\Omega}{}{ae^{bu}w}{\Omega}+\gint{\Omega}{}{bu^{2}w}{\Omega}+\gint{\Omega}{}{auw}{\Omega}+\gint{\Omega}{}{sw}{\Omega}=0
  \label{eqn:PoissonWeakForm2}
\end{equation}
and this form will be used for the remainder of the derivation.

\subsection{Tensor notation}

\Eqnref{eqn:PoissonWeakForm2} can be written in tensor notation as
\begin{equation}
  -\gint{\Omega}{}{G^{jk}\sigma^{i}_{j}\covarderiv{u}{i}\covarderiv{w}{k}}{\Omega}+
  \gint{\Gamma}{}{G^{jk}\sigma^{i}_{j}\covarderiv{u}{i}n_{k}w}{\Gamma} +
  \gint{\Omega}{}{ae^{bu}w}{\Omega}+\gint{\Omega}{}{bu^{2}w}{\Omega}+\gint{\Omega}{}{auw}{\Omega}+\gint{\Omega}{}{sw}{\Omega}=0
  \label{eqn:PoissonTensorForm1}
\end{equation}
or
\begin{equation}
  \gint{\Omega}{}{G^{jk}\sigma^{i}_{j}\covarderiv{u}{i}\covarderiv{w}{k}}{\Omega}
  -\gint{\Omega}{}{ae^{bu}w}{\Omega}-\gint{\Omega}{}{bu^{2}w}{\Omega}-\gint{\Omega}{}{auw}{\Omega}-\gint{\Omega}{}{sw}{\Omega}
  =\gint{\Gamma}{}{qw}{\Gamma}
  \label{eqn:PoissonTensorForm2}
\end{equation}
where $G^{jk}$ is the contravariant metric tensor for the spatial coordinate system.

\subsection{Finite element formulation}

We can now discretise the spatial domain into finite elements \ie $\Omega=
\displaystyle{\bigcup_{e=1}^{E}}\Omega_{e}$ with
$\Gamma=\displaystyle{\bigcup_{f=1}^{F}}\Gamma_{f}$. 
\Eqnref{eqn:PoissonTensorForm2} becomes
\begin{multline}
  \dsum_{e=1}^{E}\gint{\Omega_{e}}{}{G^{jk}\sigma^{i}_{j}
    \covarderiv{u}{i}\covarderiv{w}{k}}{\Omega}
  -\dsum_{e=1}^{E}\gint{\Omega_{e}}{}{ae^{bu}w}{\Omega}
  -\dsum_{e=1}^{E}\gint{\Omega_{e}}{}{bu^{2}w}{\Omega}\\
  -\dsum_{e=1}^{E}\gint{\Omega_{e}}{}{auw}{\Omega}
  -\dsum_{e=1}^{E}\gint{\Omega_{e}}{}{sw}{\Omega}
  =\dsum_{f=1}^{F}\gint{\Gamma_{f}}{}{qw}{\Gamma} 
  \label{eqn:PoissonFemForm}
\end{multline}

We can now interpolate the variables with basis functions \ie
\begin{equation}
  \begin{split}
    \fnof{u}{\vectr{\xi}}&=\gbfn{n}{\beta}{\vectr{\xi}}\nodedof{u}{n}{\beta}\gsf{n}{\beta} \\
    \fnof{q}{\vectr{\xi}}&= \gbfn{o}{\gamma}{\vectr{\xi}}\nodedof{q}{o}{\gamma}\gsf{o}{\gamma} \\
    \fnof{\tilde{\tensor{\sigma}}}{\vectr{\xi}}&=
    \gbfn{l}{\delta}{\vectr{\xi}}\nodedof{\tilde{\tensor{\sigma}}}{l}{\delta}\gsf{l}{\delta} \\
    \fnof{s}{\vectr{\xi}}&=\gbfn{e}{\epsilon}{\vectr{\xi}}\nodedof{s}{e}{\epsilon}\gsf{e}{\epsilon} \\
    \fnof{a}{\vectr{\xi}}&=\gbfn{f}{\zeta}{\vectr{\xi}}\nodedof{a}{\zeta}{\delta}\gsf{f}{\zeta} \\
    \fnof{b}{\vectr{\xi}}&=\gbfn{g}{\eta}{\vectr{\xi}}\nodedof{b}{\eta}{\delta}\gsf{g}{\eta}
  \end{split}
\end{equation}
where $\nodedof{u}{n}{\beta}$, $\nodedof{q}{o}{\gamma}$,
$\nodedof{\tilde{\tensor{\sigma}}}{p}{\delta}$,$\nodedof{s}{e}{\epsilon}$,$\nodedof{a}{f}{\zeta}$
and $\nodedof{b}{g}{\eta}$ are the nodal degrees-of-freedom for the variables.

For a Galerkin finite element formulation we also choose the spatial weighting
function $w$ to be equal to the basis fucntions \ie
\begin{equation}
  \fnof{w}{\vectr{\xi}}=\gbfn{m}{\alpha}{\vectr{\xi}}\gsf{m}{\alpha}
\end{equation}

\subsection{Spatial integration}

Adopting the standard integration notation we can write the spatial
integration term in \eqnref{eqn:PoissonFemForm} as
\begin{multline}
  \dsum_{e=1}^{E}\dintl{\vectr{0}}{\vectr{1}}G^{jk}\delby{x^{i}}{\xi^{d}}
  \delby{\xi^{e}}{x^{j}}\delby{\xi^{d}}{\nu^{s}}
  \delby{\nu^{b}}{\xi^{e}}\fnof{{\tilde{\sigma}}^{a}_{.b}}{\vectr{\xi}}\delby{\xi^{s}}{x^{i}}
  \delby{\pbrac{\gbfn{n}{\beta}{\vectr{\xi}}\nodedof{u}{n}{\beta}\gsf{n}{\beta}}}{\xi^{s}}
  \delby{\xi^{r}}{x^{k}}\delby{\pbrac{\gbfn{m}{\alpha}{\vectr{\xi}}\gsf{m}{\alpha}}}{\xi^{r}}
  \abs{\fnof{\matr{J}}{\vectr{\xi}}}d\vectr{\xi} \\
  -\dsum_{e=1}^{E}\gint{\vectr{0}}{\vectr{1}}{\fnof{a}{\vectr{\xi}}e^{\pbrac{\fnof{b}{\vectr{\xi}}\fnof{u}{\vectr{\xi}}}}
    \gbfn{m}{\alpha}{\vectr{\xi}}\gsf{m}{\alpha}\abs{\fnof{\matr{J}}{\vectr{\xi}}}}{\vectr{\xi}}
  -\dsum_{e=1}^{E}\gint{\vectr{0}}{\vectr{1}}{\fnof{b}{\vectr{\xi}}\pbrac{\fnof{u}{\vectr{\xi}}}^{2}
    \gbfn{m}{\alpha}{\vectr{\xi}}\gsf{m}{\alpha}\abs{\fnof{\matr{J}}{\vectr{\xi}}}}{\vectr{\xi}} \\
  -\dsum_{e=1}^{E}\gint{\vectr{0}}{\vectr{1}}{\fnof{a}{\vectr{\xi}}\gbfn{n}{\beta}{\vectr{\xi}}\nodedof{u}{n}{\beta}\gsf{n}{\beta}
    \gbfn{m}{\alpha}{\vectr{\xi}}\gsf{m}{\alpha}\abs{\fnof{\matr{J}}{\vectr{\xi}}}}{\vectr{\xi}}  
  -\dsum_{e=1}^{E}\gint{\vectr{0}}{\vectr{1}}{\fnof{s}{\vectr{\xi}}\gbfn{m}{\alpha}{\vectr{\xi}}
    \gsf{m}{\alpha}\abs{\fnof{\matr{J}}{\vectr{\xi}}}}{\vectr{\xi}} \\
  =\dsum_{f=1}^{F}\gint{\Gamma_{f}}{}{\gbfn{o}{\gamma}{\vectr{\xi}}\nodedof{q}{o}{\gamma}
    \gsf{o}{\gamma}\gbfn{m}{\alpha}{\vectr{\xi}}\gsf{m}{\alpha}}{\Gamma}\end{multline}
where $\fnof{\matr{J}}{\vectr{\xi}}$ is the \emph{Jacobian} of the
transformation from the integration $\vectr{x}$ to $\vectr{\xi}$ coordinates.

Taking values that are constant over the integration interval outside the
integration gives
\begin{multline}
  \dsum_{e=1}^{E}\nodedof{u}{n}{\beta}\gsf{m}{\alpha}\gsf{n}{\beta}
  \gint{\vectr{0}}{\vectr{1}}{\delby{\gbfn{m}{\alpha}{\vectr{\xi}}}{\xi^{r}}
  \delby{\gbfn{n}{\beta}{\vectr{\xi}}}{\xi^{s}}\fnof{\gamma^{rs}}{\vectr{\xi}}
  \abs{\fnof{\matr{J}}{\vectr{\xi}}}}{\vectr{\xi}} \\ 
  -\dsum_{e=1}^{E}\gsf{m}{\alpha}\gint{\vectr{0}}{\vectr{1}}{\fnof{a}{\vectr{\xi}}
    e^{\pbrac{\fnof{b}{\vectr{\xi}}\fnof{u}{\vectr{\xi}}}}
    \gbfn{m}{\alpha}{\vectr{\xi}}\abs{\fnof{\matr{J}}{\vectr{\xi}}}}{\vectr{\xi}}\\
  -\dsum_{e=1}^{E}\gsf{m}{\alpha}\gint{\vectr{0}}{\vectr{1}}{\fnof{b}{\vectr{\xi}}
    \pbrac{\fnof{u}{\vectr{\xi}}}^{2}
    \gbfn{m}{\alpha}{\vectr{\xi}}\abs{\fnof{\matr{J}}{\vectr{\xi}}}}{\vectr{\xi}}\\
  -\dsum_{e=1}^{E}\nodedof{u}{n}{\beta}\gsf{m}{\alpha}\gsf{n}{\beta}
  \gint{\vectr{0}}{\vectr{1}}{\fnof{a}{\vectr{\xi}}\gbfn{m}{\alpha}{\vectr{\xi}}
    \gbfn{n}{\beta}{\vectr{\xi}}\abs{\fnof{\matr{J}}{\vectr{\xi}}}}{\vectr{\xi}}\\  
  -\dsum_{e=1}^{E}\nodedof{s}{e}{\epsilon}\gsf{m}{\alpha}\gsf{e}{\epsilon}
  \gint{\vectr{0}}{\vectr{1}}{\gbfn{m}{\alpha}{\vectr{\xi}}
    \gbfn{p}{\delta}{\vectr{\xi}}\abs{\fnof{\matr{J}}{\vectr{\xi}}}}{\vectr{\xi}}\\
  =\dsum_{f=1}^{F}\nodedof{q}{o}{\gamma}\gsf{m}{\alpha}\gsf{o}{\gamma}
  \gint{\Gamma_{f}}{}{\gbfn{m}{\alpha}{\vectr{\xi}}\gbfn{o}{\gamma}{\vectr{\xi}}
  }{\Gamma}
  \label{eqn:PoissonFemForm}
\end{multline}
where $\fnof{\gamma^{rs}}{\vectr{\xi}}$ is defined in 
\eqnthrurefs{eqn:diffusiongammadefinition1}{eqn:diffusiongammadefinition3}.

\subsection{Matrix vector form}

\subsubsection{Generalised Poisson's equation}

For a \emph{generalised Poisson's equation} \eqnref{eqn:PoissonFemForm} is an equation of the form
\begin{equation}
  \matr{K}\vect{u}-\matr{R}\vect{s}-\vect{f}=\matr{K}\vect{u}-\matr{R}\vect{s}-\matr{N}\vect{q}=\vect{0}
\end{equation}
where $\vect{u}$ is the vector of $u$ DOFs, $\vect{s}$ is the vector of source
DOFs and $\vectr{q}$ is the vector of flux DOFs.

The elemental stiffness matrix, $K^{\alpha\beta}_{mn}$, is given by
\begin{equation}
  K^{\alpha\beta}_{mn}=\gsf{m}{\alpha}\gsf{m}{\beta}
  \gint{\vectr{0}}{\vectr{1}}{\delby{\gbfn{m}{\alpha}{\vectr{\xi}}}{\xi^{r}}
    \delby{\gbfn{n}{\beta}{\vectr{\xi}}}{\xi^{s}}\fnof{\gamma^{rs}}{\vectr{\xi}}
    \abs{\fnof{\matr{J}}{\vectr{\xi}}}}{\vectr{\xi}}
  \label{eqn:elementalfemlhs2}
\end{equation}

The elemental source matrix, $R^{\alpha\epsilon}_{me}$, 
\begin{equation}
  R^{\alpha\epsilon}_{me}=\gsf{m}{\alpha}\gsf{e}{\epsilon}
  \gint{\vectr{0}}{\vectr{1}}{\gbfn{m}{\alpha}{\vectr{\xi}}\gbfn{e}{\epsilon}{\vectr{\xi}}
    \abs{\fnof{\matr{J}}{\vectr{\xi}}}}{\vectr{\xi}}
\end{equation}
and the elemental flux matrix, $N^{\alpha\gamma}_{mo}$, is given by
\begin{equation}
  N^{\alpha\gamma}_{mo} =\gsf{m}{\alpha}\gsf{o}{\gamma}
  \gint{\vectr{0}}{\vectr{1}}{\gbfn{m}{\alpha}{\vectr{\xi}}\gbfn{o}{\gamma}{\vectr{\xi}}
    \abs{\fnof{\matr{J}}{\vectr{\xi}}}}{\vectr{\xi}}
\end{equation}

\subsubsection{Linear source Poisson's equation}

For a \emph{linear source Poisson's equation}
 \eqnref{eqn:PoissonFemForm} is an equation of the form
\begin{equation}
  \matr{K}\vect{u}-\matr{R}\vect{s}-\vect{f}=\matr{K}\vect{u}-\matr{R}\vect{s}-\matr{N}\vect{q}=\vect{0}
\end{equation}

The elemental stiffness matrix, $K^{\alpha\beta}_{mn}$, is given by
\begin{equation}
  K^{\alpha\beta}_{mn}=\gsf{m}{\alpha}\gsf{m}{\beta}
  \gint{\vectr{0}}{\vectr{1}}{\pbrac{\delby{\gbfn{m}{\alpha}{\vectr{\xi}}}{\xi^{r}}
      \delby{\gbfn{n}{\beta}{\vectr{\xi}}}{\xi^{s}}\fnof{\gamma^{rs}}{\vectr{\xi}}-
      \fnof{a}{\vectr{\xi}}\gbfn{m}{\alpha}{\vectr{\xi}}\gbfn{n}{\beta}{\vectr{\xi}}}
    \abs{\fnof{\matr{J}}{\vectr{\xi}}}}{\vectr{\xi}}
  \label{eqn:elementalfemlhs2}
\end{equation}

The elemental source matrix, $R^{\alpha\epsilon}_{me}$, 
\begin{equation}
  R^{\alpha\epsilon}_{me}=\gsf{m}{\alpha}\gsf{e}{\epsilon}
  \gint{\vectr{0}}{\vectr{1}}{\gbfn{m}{\alpha}{\vectr{\xi}}\gbfn{e}{\epsilon}{\vectr{\xi}}
    \abs{\fnof{\matr{J}}{\vectr{\xi}}}}{\vectr{\xi}}
\end{equation}
and the elemental flux matrix, $N^{\alpha\gamma}_{mo}$, is given by
\begin{equation}
  N^{\alpha\gamma}_{mo} = \gsf{m}{\alpha}\gsf{o}{\gamma}
  \gint{\vectr{0}}{\vectr{1}}{\gbfn{m}{\alpha}{\vectr{\xi}}\gbfn{o}{\gamma}{\vectr{\xi}}
    \abs{\fnof{\matr{J}}{\vectr{\xi}}}}{\vectr{\xi}}
\end{equation}

\subsubsection{Quadratic source Poisson's equation}

For a \emph{Quadratic source Poisson's equation}
\eqnref{eqn:PoissonFemForm} is an equation of the form
\begin{equation}
  \matr{K}\vect{u}-\fnof{\vect{g}}{\vect{u}}-\matr{R}\vect{s}-\vect{f}=\matr{K}\vect{u}-\fnof{\vect{g}}{\vect{u}}-\matr{R}\vect{s}-\matr{N}\vect{q}=\vect{0}
\end{equation}

The elemental stiffness matrix, $K^{\alpha\beta}_{mn}$, is given by
\begin{equation}
  K^{\alpha\beta}_{mn}=\gsf{m}{\alpha}\gsf{m}{\beta}
  \gint{\vectr{0}}{\vectr{1}}{\pbrac{\delby{\gbfn{m}{\alpha}{\vectr{\xi}}}{\xi^{r}}
      \delby{\gbfn{n}{\beta}{\vectr{\xi}}}{\xi^{s}}\fnof{\gamma^{rs}}{\vectr{\xi}}-
      \fnof{a}{\vectr{\xi}}\gbfn{m}{\alpha}{\vectr{\xi}}\gbfn{n}{\beta}{\vectr{\xi}}}
    \abs{\fnof{\matr{J}}{\vectr{\xi}}}}{\vectr{\xi}}
  \label{eqn:elementalfemlhs2}
\end{equation}
and the nonlinear elemental residual vector, $g^{\alpha}_{m}$, is given by 
\begin{equation}
  g^{\alpha}_{m}=\gsf{m}{\alpha}\gint{\vectr{0}}{\vectr{1}}{\fnof{b}{\vectr{\xi}}
    \pbrac{\fnof{u}{\vectr{\xi}}}^{2}\gbfn{m}{\alpha}{\vectr{\xi}}\abs{\fnof{\matr{J}}{\vectr{\xi}}}}{\vectr{\xi}}
\end{equation}
with an elemental Jacobian matrix, $J^{\alpha\beta}_{mn}$, given by
\begin{equation}
  J^{\alpha\beta}_{mn}=\delby{g^{\alpha}_{m}}{u^{\beta}_{n}}=\gsf{m}{\alpha}\gsf{n}{\beta}\gint{\vectr{0}}{\vectr{1}}{
    2\fnof{b}{\vectr{\xi}}\fnof{u}{\vectr{\xi}}\gbfn{m}{\alpha}{\vectr{\xi}}\gbfn{n}{\beta}{\vectr{\xi}}
    x\abs{\fnof{\matr{J}}{\vectr{\xi}}}}{\vectr{\xi}}
\end{equation}

The elemental source matrix, $R^{\alpha\epsilon}_{me}$, 
\begin{equation}
  R^{\alpha\epsilon}_{me}=\gsf{m}{\alpha}\gsf{e}{\epsilon}
  \gint{\vectr{0}}{\vectr{1}}{\gbfn{m}{\alpha}{\vectr{\xi}}\gbfn{e}{\epsilon}{\vectr{\xi}}
    \abs{\fnof{\matr{J}}{\vectr{\xi}}}}{\vectr{\xi}}
\end{equation}
and the elemental flux matrix, $N^{\alpha\gamma}_{mo}$, is given by
\begin{equation}
  N^{\alpha\gamma}_{mo} = \gsf{m}{\alpha}\gsf{o}{\gamma}
  \gint{\vectr{0}}{\vectr{1}}{\gbfn{m}{\alpha}{\vectr{\xi}}\gbfn{o}{\gamma}{\vectr{\xi}}
    \abs{\fnof{\matr{J}}{\vectr{\xi}}}}{\vectr{\xi}}
\end{equation}

\subsubsection{Exponential source Poisson's equation}

For an \emph{Exponential source Poisson's equation}
\eqnref{eqn:PoissonFemForm} is an equation of the form
\begin{equation}
  \matr{K}\vect{u}-\fnof{\vect{g}}{\vect{u}}-\matr{R}\vect{s}-\vect{f}=\matr{K}\vect{u}-\fnof{\vect{g}}{\vect{u}}-\matr{R}\vect{s}-\matr{N}\vect{q}=\vect{0}
\end{equation}

The elemental stiffness matrix, $K^{\alpha\beta}_{mn}$, is given by
\begin{equation}
  K^{\alpha\beta}_{mn}=\gsf{m}{\alpha}\gsf{m}{\beta}
  \gint{\vectr{0}}{\vectr{1}}{\delby{\gbfn{m}{\alpha}{\vectr{\xi}}}{\xi^{r}}
      \delby{\gbfn{n}{\beta}{\vectr{\xi}}}{\xi^{s}}\fnof{\gamma^{rs}}{\vectr{\xi}}
    \abs{\fnof{\matr{J}}{\vectr{\xi}}}}{\vectr{\xi}}
  \label{eqn:elementalfemlhs2}
\end{equation}
and the nonlinear elemental residual vector, $g^{\alpha}_{m}$, is given by
\begin{equation}
  g^{\alpha}_{m}=\gsf{m}{\alpha}\gint{\vectr{0}}{\vectr{1}}{\fnof{a}{\vectr{\xi}}
    e^{\pbrac{\fnof{b}{\vectr{\xi}}\fnof{u}{\vectr{\xi}}}}\gbfn{m}{\alpha}{\vectr{\xi}}
    \abs{\fnof{\matr{J}}{\vectr{\xi}}}}{\vectr{\xi}}
\end{equation}
with an elemental Jacobian matrix, $J^{\alpha\beta}_{mn}$, given by
\begin{equation}
  J^{\alpha\beta}_{mn}=\delby{g^{\alpha}_{m}}{u^{\beta}_{n}}=\gsf{m}{\alpha}\gsf{n}{\beta}\gint{\vectr{0}}{\vectr{1}}{
    \fnof{a}{\vectr{\xi}}\fnof{b}{\vectr{\xi}}e^{\pbrac{\fnof{b}{\vectr{\xi}}\fnof{u}{\vectr{\xi}}}}
    \gbfn{m}{\alpha}{\vectr{\xi}}\gbfn{n}{\beta}{\vectr{\xi}}
    \abs{\fnof{\matr{J}}{\vectr{\xi}}}}{\vectr{\xi}}
\end{equation}

The elemental source matrix, $R^{\alpha\epsilon}_{me}$, 
\begin{equation}
  R^{\alpha\epsilon}_{me}=\gsf{m}{\alpha}\gsf{e}{\epsilon}
  \gint{\vectr{0}}{\vectr{1}}{\gbfn{m}{\alpha}{\vectr{\xi}}\gbfn{e}{\epsilon}{\vectr{\xi}}
    \abs{\fnof{\matr{J}}{\vectr{\xi}}}}{\vectr{\xi}}
\end{equation}
and the elemental flux matrix, $N^{\alpha\gamma}_{mo}$, is given by
\begin{equation}
  N^{\alpha\gamma}_{mo} = \gsf{m}{\alpha}\gsf{o}{\gamma}
  \gint{\vectr{0}}{\vectr{1}}{\gbfn{m}{\alpha}{\vectr{\xi}}\gbfn{o}{\gamma}{\vectr{\xi}}
    \abs{\fnof{\matr{J}}{\vectr{\xi}}}}{\vectr{\xi}}
\end{equation}

\subsection{Analytic solutions}

\subsubsection{Two-dimensional Exponential Poisson Analytic Function 1}

Consider the problem shown in \figref{fig:TwoDimExpSourcePoisson}. 

\epstexfigure{ClassicalField/svgs/twodimexpsourcepoisson.eps_tex}{2D domain for an exponential source Poisson equation.}{2D domain for an exponential source Poisson equation.}{fig:TwoDimExpSourcePoisson}{0.75}

Following from \citet{polyanin:2004} and \citet{crowdy:1997} we can
obtain an number of analytic solutions.

The first analytic solution we consider is from
\urllink{http://eqworld.ipmnet.ru/en/solutions/npde/npde3103.pdf}
\begin{equation}
  \fnof{u}{x,y}=\dfrac{1}{\beta}\ln\sqbrac{\dfrac{\sign{\alpha\beta}2\pbrac{A^{2}+B^{2}}}{\alpha\beta\cos^{2}\pbrac{Ax+By+C}}} 
\end{equation}
where $A$, $B$ and $C$ are constants. Note that $\sign{\alpha\beta}$
is added to ensure that we are taking the natural logarithm of a
positive number if $\alpha\beta<0$.

Differentiating we obtain
\begin{multline}
  \delby{\fnof{u}{x,y}}{x}=\dfrac{1}{\beta}\sqbrac{\dfrac{\alpha\beta\cos^{2}\pbrac{Ax+By+C}}{\sign{\alpha\beta}2\pbrac{A^{2}+B^{2}}}}\\
  \sqbrac{\dfrac{\sign{\alpha\beta}2\pbrac{A^{2}+B^{2}}.-2\alpha\beta\cos\pbrac{Ax+By+C}.-\sin\pbrac{Ax+By+C}.A}{\pbrac{\alpha\beta\cos^{2}\pbrac{Ax+By+C}}^{2}}}
\end{multline}
or
\begin{equation}
  \begin{aligned}
    \delby{\fnof{u}{x,y}}{x}&=\dfrac{1}{\beta}\sqbrac{\dfrac{2A\alpha\beta\cos\pbrac{Ax+By+C}\sin\pbrac{Ax+By+C}}{\alpha\beta\cos^{2}\pbrac{Ax+By+C}}}\\
    &=\dfrac{1}{\beta}\sqbrac{\dfrac{2A\sin\pbrac{Ax+By+C}}{\cos\pbrac{Ax+By+C}}} \\
    &=\dfrac{2A}{\beta}\tan\pbrac{Ax+By+C}
  \end{aligned}
\end{equation}

Similarily
\begin{equation}
  \delby{\fnof{u}{x,y}}{y}=\dfrac{2B}{\beta}\tan\pbrac{Ax+By+C}
\end{equation}

Differentiating again we have
\begin{equation}
  \begin{aligned}
    \deltwosqby{\fnof{u}{x,y}}{x}&=\dfrac{2A.A}{\beta\cos^{2}\pbrac{Ax+By+C}}\\
    &=\dfrac{2A^{2}}{\beta\cos^{2}\pbrac{Ax+By+C}}
  \end{aligned}
\end{equation}

Similarily
\begin{equation}
  \deltwosqby{\fnof{u}{x,y}}{y}=\dfrac{2B^{2}}{\beta\cos^{2}\pbrac{Ax+By+C}}
\end{equation}

Substituting the above equations into the governing equation we thus have
\begin{equation}
  \begin{aligned}
    \laplacian{}{\fnof{u}{x,y}}&=\alpha e^{\beta\fnof{u}{x,y}}\\
    \deltwosqby{\fnof{u}{x,y}}{x}+\deltwosqby{\fnof{u}{x,y}}{y}&=\alpha e^{\beta\fnof{u}{x,y}}\\
    \dfrac{2A^{2}}{\beta\cos^{2}\pbrac{Ax+By+C}}+\dfrac{2B^{2}}{\beta\cos^{2}\pbrac{Ax+By+C}}&=
    \alpha e^{\beta\dfrac{1}{\beta}\ln\sqbrac{\dfrac{\sign{\alpha\beta}2\pbrac{A^{2}+B^{2}}}{\alpha\beta\cos^{2}\pbrac{Ax+By+C}}}}\\
    \dfrac{2\pbrac{A^{2}+B^{2}}}{\alpha\beta\cos^{2}\pbrac{Ax+By+C}}&=
    e^{\ln\sqbrac{\dfrac{\sign{\alpha\beta}2\pbrac{A^{2}+B^{2}}}{\alpha\beta\cos^{2}\pbrac{Ax+By+C}}}} \\
    \dfrac{2\pbrac{A^{2}+B^{2}}}{\alpha\beta\cos^{2}\pbrac{Ax+By+C}}&=
    \dfrac{\sign{\alpha\beta}2\pbrac{A^{2}+B^{2}}}{\alpha\beta\cos^{2}\pbrac{Ax+By+C}} \\
  \end{aligned}
\end{equation}
thus showing that that we have an analytic solution that satisfies the governing equation.

The analytic field component definitions in \OpenCMISS are shown in \tabref{tab:OpenCMISSAnalyticFieldExponentialPoisonEquationTwoDim1}.

\begin{table}[htb] \centering
  \begin{tabular}{|c|c|} \hline
    \emph{Analytic constant} & \emph{Analytic field component} \\ \hline \hline
    $A$ & $1$ \\ 
    $B$ & $2$ \\
    $C$ & $3$ \\
    $\alpha$ & $4$ \\ 
    $\beta$ & $5$ \\ \hline
  \end{tabular}
  \caption{\OpenCMISS analytic field components for the \twod exponential Poisson equation
    analytic function 1.}
  \label{tab:OpenCMISSAnalyticFieldExponentialPoisonEquationTwoDim1}
\end{table}
