\section{Notation}
\label{subsec:MathsNotation}

\subsection{Voigt Notation}
\label{subsec:MathsNotationVoigt}

Voigt\footnote{\link{https://en.wikipedia.org/wiki/Woldemar_Voigt}{Woldermar
  Voigt} (1850-1919), a German physicist.} notation\index{Voigt
  notation} allows for components of tensors to be
written as lower dimensional objects (vectors or lower dimension
tensors). Tensor symmetry \etc can be exploited to reduce the number
of components that need to be recorded. For further information see
\citet{helnwein:2001} and \citet{Brannon:2018}.

Consider expanding a vector $\vectr{a}$ in terms of it's contravariant
components $a^{i}$ and base vectors $\vectr{e}_{i}$ \ie
\begin{equation}
  \vectr{a}=\begin{bmatrix} a^{1} \\ a^{2} \\ a^{3} \end{bmatrix}
  =a^{1}\begin{bmatrix} 1 \\ 0 \\ 0 \end{bmatrix}
  +a^{2}\begin{bmatrix} 0 \\ 1 \\ 0 \end{bmatrix} 
  +a^{3}\begin{bmatrix} 0 \\ 0 \\ 1 \end{bmatrix}
  =a^{1}\vectr{e}_{1}+a^{2}\vectr{e}_{2}+a^{3}\vectr{e}_{3}
\end{equation}

An analogous procedure can be used with a tensor. Consider expanding the second order tensor
\begin{equation}
  \tensortwo{A}=\begin{bmatrix}
  A^{11} & A^{12} & A^{13} \\
  A^{21} & A^{22} & A^{23} \\
  A^{31} & A^{32} & A^{33}
  \end{bmatrix}
  \label{eqn:VoigtNotationSecondOrderTensor}
\end{equation}
in terms of it's components and base tensors \ie
\begin{equation}
  \begin{aligned}  
    \tensortwo{A}&=A^{11}\begin{bmatrix} 1 & 0 & 0 \\ 0 & 0 & 0 \\ 0 & 0 & 0 \end{bmatrix}
    +A^{12}\begin{bmatrix} 0 & 1 & 0 \\ 0 & 0 & 0 \\ 0 & 0 & 0 \end{bmatrix}
    +A^{13}\begin{bmatrix} 0 & 0 & 1 \\ 0 & 0 & 0 \\ 0 & 0 & 0 \end{bmatrix} \\
    &+A^{21}\begin{bmatrix} 0 & 0 & 0 \\ 1 & 0 & 0 \\ 0 & 0 & 0 \end{bmatrix}
    +A^{22}\begin{bmatrix} 0 & 0 & 0 \\ 0 & 1 & 0 \\ 0 & 0 & 0 \end{bmatrix}
    +A^{23}\begin{bmatrix} 0 & 0 & 0 \\ 0 & 0 & 1 \\ 0 & 0 & 0 \end{bmatrix} \\
    &+A^{31}\begin{bmatrix} 0 & 0 & 0 \\ 0 & 0 & 0 \\ 1 & 0 & 0 \end{bmatrix}
    +A^{32}\begin{bmatrix} 0 & 0 & 0 \\ 0 & 0 & 0 \\ 0 & 1 & 0 \end{bmatrix}
    +A^{33}\begin{bmatrix} 0 & 0 & 0 \\ 0 & 0 & 0 \\ 0 & 0 & 1 \end{bmatrix} \\
    &=A^{11}\tensortwo{E}_{11}+A^{12}\tensortwo{E}_{12}+A^{13}\tensortwo{E}_{13} \\
    &+A^{21}\tensortwo{E}_{21}+A^{22}\tensortwo{E}_{22}+A^{23}\tensortwo{E}_{23} \\
    &+A^{31}\tensortwo{E}_{31}+A^{32}\tensortwo{E}_{32}+A^{33}\tensortwo{E}_{33}
  \end{aligned}
  \label{eqn:VoigtNotationSecondOrderTensorStdBasis}
\end{equation}
where $\tensortwo{E}_{ij}$ are the base tensors for the contravariant
tensor components $A^{ij}$. The base tensors are formed by taking the
tensor product of the standard basis vectors \ie
\begin{equation}
  \tensortwo{E}_{ij}=\tensorprod{\vectr{e}_{i}}{\vectr{e}_{j}}
\end{equation}


We note that the tensor/vector components are labelled as
contravariant as they vary \emph{contra} to the base tensors/vectors
under a change of basis. That is if the ``length'' of the base
tensors/vectors was to double then the contravariant components would
need to halve in order that the tensor/vector has the same physical
``length''. For this reason the base tensors/vectors are often
\emph{normalised} so that they have unit ``length''.

All second order tensors can be written as the sum of a symmetric
tensor and a skew-symmetric tensor \ie
\begin{equation}
  \tensortwo{A}=\begin{bmatrix}
  A^{11} & A^{12} & A^{13} \\
  A^{21} & A^{22} & A^{23} \\
  A^{31} & A^{32} & A^{33}
  \end{bmatrix}=\dfrac{1}{2}\pbrac{\symmetric{\tensortwo{A}}+\skewsym{\tensortwo{A}}}
\end{equation}

We can use a change of basis in \eqnref{eqn:VoigtNotationSecondOrderTensorStdBasis} to write any second order tensor as
\begin{equation}
  \begin{aligned}  
    \tensortwo{A}&=A^{\pbrac{11}}\begin{bmatrix} 1 & 0 & 0 \\ 0 & 0 & 0 \\ 0 & 0 & 0 \end{bmatrix}
    +A^{\pbrac{22}}\begin{bmatrix} 0 & 0 & 0 \\ 0 & 1 & 0 \\ 0 & 0 & 0 \end{bmatrix}
    +A^{\pbrac{33}}\begin{bmatrix} 0 & 0 & 0 \\ 0 & 0 & 0 \\ 0 & 0 & 1 \end{bmatrix} \\
    &+A^{\pbrac{23}}\begin{bmatrix} 0 & 0 & 0 \\ 0 & 0 & 1 \\ 0 & 1 & 0 \end{bmatrix}
    +A^{\pbrac{13}}\begin{bmatrix} 0 & 0 & 1 \\ 0 & 0 & 0 \\ 1 & 0 & 0 \end{bmatrix}
    +A^{\pbrac{12}}\begin{bmatrix} 0 & 1 & 0 \\ 1 & 0 & 0 \\ 0 & 0 & 0 \end{bmatrix} \\
    &+A^{\sqbrac{32}}\begin{bmatrix} 0 & 0 & 0 \\ 0 & 0 & -1 \\ 0 & 1 & 0 \end{bmatrix}
    +A^{\sqbrac{13}}\begin{bmatrix} 0 & 0 & 1 \\ 0 & 0 & 0 \\ -1 & 0 & 0 \end{bmatrix}
    +A^{\sqbrac{21}}\begin{bmatrix} 0 & -1 & 0 \\ 1 & 0 & 0 \\ 0 & 0 & 0 \end{bmatrix} \\
    &=A^{\pbrac{11}}\tensortwo{V}_{11}+A^{\pbrac{22}}\tensortwo{V}_{22}+A^{\pbrac{33}}\tensortwo{V}_{33} \\
    &+A^{\pbrac{23}}\tensortwo{V}_{23}+A^{\pbrac{13}}\tensortwo{V}_{13}+A^{\pbrac{12}}\tensortwo{V}_{12} \\
    &+A^{\sqbrac{32}}\tensortwo{V}_{32}+A^{\sqbrac{13}}\tensortwo{V}_{13}+A^{\sqbrac{21}}\tensortwo{V}_{21}
  \end{aligned}
  \label{eqn:VoigtNotationSecondOrderTensorVoigtBasis}
\end{equation}
where
\begin{equation}
  A^{\pbrac{ij}}=\dfrac{1}{2}\pbrac{A^{ij}+A^{ji}}
\end{equation}
are the symmetric components of $\tensortwo{A}$ and
\begin{equation}
  A^{\sqbrac{ij}}=\dfrac{1}{2}\pbrac{A^{ij}-A^{ji}}
\end{equation}
are the skew-symmetric components of $\tensortwo{A}$. The tensors
$\tensortwo{V}_{ij}$ are the \emph{covariant Voigt basis
tensors}. Note that the last six Voigt basis tensors are not of unit
magnitude but have a magnitude of $\sqrt{2}$. They can thus be
normalised by dividing by $\sqrt{2}$.

Using Voigt notation we can write any second order tensor as
\begin{equation}
  \Voigt{\tensortwo{A}}=A^{I}\tensortwo{V}_{I}
\end{equation}
where the $\Voigtsymbol$ indicates a Voigt notation tensor. Here
summation is implied and $I$ ranges from $1$ to $9$, $A^{I}$ are the
contravariant Voigt components and $\tensortwo{V}_{I}$ are the covariant
Voigt basis tensors and they are given by
\begin{equation}
  A^{1}=A^{\pbrac{11}} \qquad
  \tensortwo{V}_{1}=\tensorprod{\vectr{e}_{1}}{\vectr{e}_{1}}=\begin{bmatrix}
  1 & 0 & 0 \\
  0 & 0 & 0 \\
  0 & 0 & 0
  \end{bmatrix}
\end{equation}
\begin{equation}
  A^{2}=A^{\pbrac{22}} \qquad
  \tensortwo{V}_{2}=\tensorprod{\vectr{e}_{2}}{\vectr{e}_{2}}=\begin{bmatrix}
  0 & 0 & 0 \\
  0 & 1 & 0 \\
  0 & 0 & 0
  \end{bmatrix}
\end{equation}
\begin{equation}
  A^{3}=A^{\pbrac{33}} \qquad
  \tensortwo{V}_{3}=\tensorprod{\vectr{e}_{3}}{\vectr{e}_{3}}=\begin{bmatrix}
  0 & 0 & 0 \\
  0 & 0 & 0 \\
  0 & 0 & 1
  \end{bmatrix}
\end{equation}
\begin{equation}
  A^{4}=A^{\pbrac{23}} \qquad
  \tensortwo{V}_{4}=\tensorprod{\vectr{e}_{2}}{\vectr{e}_{3}}+\tensorprod{\vectr{e}_{3}}{\vectr{e}_{2}}=\begin{bmatrix}
  0 & 0 & 0 \\
  0 & 0 & 1 \\
  0 & 1 & 0
  \end{bmatrix}
\end{equation}
\begin{equation}
  A^{5}=A^{\pbrac{13}} \qquad
  \tensortwo{V}_{5}=\tensorprod{\vectr{e}_{1}}{\vectr{e}_{3}}+\tensorprod{\vectr{e}_{3}}{\vectr{e}_{1}}=\begin{bmatrix}
  0 & 0 & 1 \\
  0 & 0 & 0 \\
  1 & 0 & 0
  \end{bmatrix}
\end{equation}
\begin{equation}
  A^{6}=A^{\pbrac{12}} \qquad
  \tensortwo{V}_{6}=\tensorprod{\vectr{e}_{1}}{\vectr{e}_{2}}+\tensorprod{\vectr{e}_{2}}{\vectr{e}_{1}}=\begin{bmatrix}
  0 & 1 & 0 \\
  1 & 0 & 0 \\
  0 & 0 & 0
  \end{bmatrix}
\end{equation}
\begin{equation}
  A^{7}=A^{\sqbrac{32}} \qquad
  \tensortwo{V}_{7}=\tensorprod{\vectr{e}_{3}}{\vectr{e}_{2}}-\tensorprod{\vectr{e}_{2}}{\vectr{e}_{3}}=\begin{bmatrix}
  0 & 0 & 0 \\
  0 & 0 & -1 \\
  0 & 1 & 0
  \end{bmatrix}
\end{equation}
\begin{equation}
  A^{8}=A^{\sqbrac{13}} \qquad
  \tensortwo{V}_{8}=\tensorprod{\vectr{e}_{1}}{\vectr{e}_{3}}-\tensorprod{\vectr{e}_{3}}{\vectr{e}_{1}}=\begin{bmatrix}
  0 & 0 & 1 \\
  0 & 0 & 0 \\
  -1 & 0 & 0
  \end{bmatrix}
\end{equation}
\begin{equation}
  A^{9}=A^{\sqbrac{21}} \qquad
  \tensortwo{V}_{9}=\tensorprod{\vectr{e}_{2}}{\vectr{e}_{1}}-\tensorprod{\vectr{e}_{1}}{\vectr{e}_{2}}=\begin{bmatrix}
  0 & -1 & 0 \\
  1 & 0 & 0 \\
  0 & 0 & 0
  \end{bmatrix}
\end{equation}

In addition if the Voigt array of contravariant components is
$\transpose{\sqbrac{A^{1},A^{2},A^{3},A^{4},A^{5},A^{6},A^{7},A^{8},A^{9}}}$
then the equivalent contravariant tensor is
\begin{equation}
  \begin{bmatrix}
    A^{1} & A^{6}-A^{9} & A^{5}+A^{8} \\
    A^{6}+A^{9} & A^{2} & A^{4}-A^{7} \\
    A^{5}-A^{8} & A^{4}+A^{7} & A^{3}
  \end{bmatrix}
\end{equation}

TODO CHECK UPPER AND LOWER FOR THIS METRIC PART BELOW

The covariant metric tensor for the Voigt basis is 
\begin{equation}
  \Voigt{\flattensor{\tensortwo{G}}}=G_{IJ}\tensorprod{\tensortwo{V}^{I}}{\tensortwo{V}^{J}}=
  \doubledotprod{\tensortwo{V}^{I}}{\tensortwo{V}^{J}}=\begin{bmatrix}
  1 & 0 & 0 & 0 & 0 & 0 & 0 & 0 & 0 \\
  0 & 1 & 0 & 0 & 0 & 0 & 0 & 0 & 0 \\
  0 & 0 & 1 & 0 & 0 & 0 & 0 & 0 & 0 \\
  0 & 0 & 0 & 2 & 0 & 0 & 0 & 0 & 0 \\
  0 & 0 & 0 & 0 & 2 & 0 & 0 & 0 & 0 \\
  0 & 0 & 0 & 0 & 0 & 2 & 0 & 0 & 0 \\
  0 & 0 & 0 & 0 & 0 & 0 & 2 & 0 & 0 \\
  0 & 0 & 0 & 0 & 0 & 0 & 0 & 2 & 0 \\
  0 & 0 & 0 & 0 & 0 & 0 & 0 & 0 & 2
  \end{bmatrix}
\end{equation}

We note that that metric tensor with the Voigt basis is not the
identity tensor. The contravariant metric tensor can thus be found
from the inverse of the covariant metric tensor \ie
\begin{equation}
  \Voigt{\sharptensor{\tensortwo{G}}}=\inverse{\pbrac{\Voigt{\flattensor{\tensortwo{G}}}}}=G^{IJ}\tensorprod{\tensortwo{V}_{I}}{\tensortwo{V}_{J}}=
  \begin{bmatrix}
  1 & 0 & 0 & 0 & 0 & 0 & 0 & 0 & 0 \\
  0 & 1 & 0 & 0 & 0 & 0 & 0 & 0 & 0 \\
  0 & 0 & 1 & 0 & 0 & 0 & 0 & 0 & 0 \\
  0 & 0 & 0 & \frac{1}{2} & 0 & 0 & 0 & 0 & 0 \\
  0 & 0 & 0 & 0 & \frac{1}{2} & 0 & 0 & 0 & 0 \\
  0 & 0 & 0 & 0 & 0 & \frac{1}{2} & 0 & 0 & 0 \\
  0 & 0 & 0 & 0 & 0 & 0 & \frac{1}{2} & 0 & 0 \\
  0 & 0 & 0 & 0 & 0 & 0 & 0 & \frac{1}{2} & 0 \\
  0 & 0 & 0 & 0 & 0 & 0 & 0 & 0 & \frac{1}{2}
  \end{bmatrix}
\end{equation}

We can thus calculate the contravariant Voigt basis using the metric tensor
\begin{equation}
  \tensortwo{V}^{I}=G^{IJ}\tensortwo{V}_{J}
\end{equation}
and thus the contravariant Voigt components are also given by
\begin{equation}
  A^{I}=\doubledotprod{\tensortwo{A}}{\tensortwo{V}^{I}}
\end{equation}

As well as using a covariant Voigt basis we can also expand the tensor in terms of a contravariant Voigt basis \ie
\begin{equation}
  \Voigt{\tensortwo{A}}=A_{I}\tensortwo{V}^{I}
\end{equation}
where
\begin{equation}
  A_{I}=\doubledotprod{\tensortwo{A}}{\tensortwo{V}_{I}}=G_{IJ}A^{J}
\end{equation}
are the ``covariant'' Voigt components. Note that we are abusing
notation here in that all the Voigt components are really
contravariant and it is just the basis to which they are referred to
that changes. This will lead to the requirement for correction factors
(see below).

By substituting the metric tensor we can thus find the relationship between the covariant and contravariant Voigt components \ie
\begin{equation}
  \begin{bmatrix}
    A_{1} \\
    A_{2} \\
    A_{3} \\
    A_{4} \\
    A_{5} \\
    A_{6} \\
    A_{7} \\
    A_{8} \\
    A_{9}
  \end{bmatrix} = \begin{bmatrix}
    A^{1} \\
    A^{2} \\
    A^{3} \\
    2A^{4} \\
    2A^{5} \\
    2A^{6} \\
    2A^{7} \\
    2A^{8} \\
    2A^{9}
  \end{bmatrix} = \begin{bmatrix}
    A^{\pbrac{11}} \\
    A^{\pbrac{22}} \\
    A^{\pbrac{33}} \\
    2A^{\pbrac{23}} \\
    2A^{\pbrac{13}} \\
    2A^{\pbrac{12}} \\
    2A^{\sqbrac{32}} \\
    2A^{\sqbrac{12}} \\
    2A^{\sqbrac{21}}
  \end{bmatrix}
\end{equation}

The factor of $2$ that is applied to covariant Voigt components can
cause mistakes when using them. For a Voigt representation
``off-diagonal'' ``covariant'' components require a correction factor of
$2$. \textbf{Thus care must be taken when using Voigt components in
  calculations!}

Now, it is often the case that the tensors encountered in physical
problems are symmetric or skew-symmetric. A symmetric second order
tensor in three dimensions only has six independent components instead
of the nine shown in
\eqnref{eqn:VoigtNotationSecondOrderTensor}. Voigt notation exploits
this reduction in the number of independent components to store the
tensor more efficiently by only storing the first six components of
the Voigt array for three-dimensions \ie
\begin{equation}
  \tensortwo{A}=A^{ij}\tensorprod{\vectr{e}_{i}}{\vectr{e}_{j}}=\begin{bmatrix}
  A^{11} & A^{12} & A^{13} \\
  A^{21} & A^{22} & A^{23} \\
  A^{31} & A^{23} & A^{33}
  \end{bmatrix} = \Voigt{\tensortwo{A}}=A^{I}\tensortwo{V}_{I}=\begin{bmatrix}
    A^{11} \\
    A^{22} \\
    A^{33} \\
    A^{23} \\
    A^{13} \\
    A^{12}
  \end{bmatrix}    
\end{equation}
and the first three components for two-dimensions \ie
\begin{equation}
  \tensortwo{A}=A^{ij}\tensorprod{\vectr{e}_{i}}{\vectr{e}_{j}}=\begin{bmatrix}
  A^{11} & A^{12} \\
  A^{21} & A^{22}
  \end{bmatrix} = \Voigt{\tensortwo{A}}=A^{I}\tensortwo{V}_{I} = \begin{bmatrix}
    A^{11} \\
    A^{22} \\
    A^{12}
  \end{bmatrix} \\
\end{equation}

Voigt notation can also simplify tensor contraction. For example
consider two second order tensors, $\tensortwo{A}$ and
$\tensortwo{B}$,
\begin{equation}
  \tensortwo{A}=\begin{bmatrix}
  A^{11} & A^{12} & A^{13} \\
  A^{21} & A^{22} & A^{23} \\
  A^{31} & A^{23} & A^{33}
  \end{bmatrix} \qquad \tensortwo{B}=\begin{bmatrix}
  B_{11} & B_{12} & B_{13} \\
  B_{21} & B_{22} & B_{23} \\
  B_{31} & B_{23} & B_{33}
  \end{bmatrix}
\end{equation}

The tensor contraction of $\tensortwo{A}$ and $\tensortwo{B}$ is given by
\begin{align}
  \doubledotprod{\tensortwo{A}}{\tensortwo{B}}&=A^{ij}B_{ij} \\
  &=A^{11}B_{11}+A^{12}B_{21}+A^{13}B_{31}+A^{21}B_{12}+A^{22}B_{22}+A^{23}B_{32}+A^{31}B_{13}+A^{32}B_{23}+A^{33}B_{33}
\end{align}

Now if $\tensortwo{A}$ and $\tensortwo{B}$ are symmetric this reduces to
\begin{align}
  \doubledotprod{\tensortwo{A}}{\tensortwo{B}}&=A^{ij}B_{ij} \\
  &=A^{11}B_{11}+2A^{12}B_{21}+2A^{13}B_{31}+A^{22}B_{22}+2A^{23}B_{32}+A^{33}B_{33}
  \label{eqn:VoigtNotationDoubleContraction}
\end{align}

Using a Voigt notation for the symmetric tensors $\tensortwo{A}$ and
$\tensortwo{B}$ simplifies the double contraction to a simple vector
contraction (or dot product) \ie
\begin{align}
  \doubledotprod{\tensortwo{A}}{\tensortwo{B}}&=\dotprod{\Voigt{\tensortwo{A}}}{\Voigt{\tensortwo{B}}}=A^{I}B_{I}\\ &=\begin{bmatrix}
  A^{1} & A^{2} & A^{3} & A^{4} & A^{5} &
  A^{6} \end{bmatrix} \begin{bmatrix} B_{1} \\ B_{2} \\ B_{3}
    \\ 2B_{4} \\ 2B_{5} \\ 2B_{6}
  \end{bmatrix} \\
  &=A^{1}B_{1} + A^{2}B_{2} + A^{3}B_{3} + 2A^{4}B_{4} + 2A^{5}B_{5} + 2A^{6}B_{6} \\
  &=A^{11}B_{11}+2A^{12}B_{21}+2A^{13}B_{31}+A^{22}B_{22}+2A^{23}B_{32}+A^{33}B_{33}
\end{align}
which gives the same result as
\eqnref{eqn:VoigtNotationDoubleContraction}. Note the factors of two
for the last three ``off-diagonal'' covariant components of the
$\tensortwo{B}$ tensor.

Voigt notation is often used for ``symmetric'' fourth order tensors as
well as symmetric second order tensors. Just as Voigt notation will
reduce a $3 \times 3=9$ component symmetric second order tensor to $6$
Voigt components, the notation will reduce a $3\times 3\times 3\times
3=81$ component ``symmetric fourth order tensor to $6\times 6=36$
Voigt components.

Similar to what was done for a second order tensor, a fourth order tensor can be expanded in terms of a Voigt basis \ie
\begin{align}
  \tensorfour{C}&=C^{ijkl}\tensorprodfour{\vectr{e}_{i}}{\vectr{e}_{j}}{\vectr{e}_{k}}{\vectr{e}_{l}}=
  \Voigt{\tensorfour{C}}=C^{IJ}\tensorprod{\tensortwo{V}_{I}}{\tensortwo{V}_{J}} \\
  \tensorfour{C}&=C_{ijkl}\tensorprodfour{\vectr{e}^{i}}{\vectr{e}^{j}}{\vectr{e}^{k}}{\vectr{e}^{l}}=
  \Voigt{\tensorfour{C}}=C_{IJ}\tensorprod{\tensortwo{V}^{I}}{\tensortwo{V}^{J}} \\
  \tensorfour{C}&=C^{ij..}_{..kl}\tensorprodfour{\vectr{e}_{i}}{\vectr{e}_{j}}{\vectr{e}^{k}}{\vectr{e}^{l}}=
  \Voigt{\tensorfour{C}}=C^{I.}_{.J}\tensorprod{\tensortwo{V}_{I}}{\tensortwo{V}^{J}} \\
  \tensorfour{C}&=C^{..kl}_{ij..}\tensorprodfour{\vectr{e}^{i}}{\vectr{e}^{j}}{\vectr{e}_{k}}{\vectr{e}_{l}}=
  \Voigt{\tensorfour{C}}=C^{.J}_{I.}\tensorprod{\tensortwo{V}^{I}}{\tensortwo{V}_{J}}
\end{align}

A fourth order tensor with minor and major symmetries in three-dimensions can be written as a
matrix \ie
\begin{equation}
  \Voigt{\tensorfour{C}}=C^{IJ}\tensorprodtwo{\vectr{V}_{I}}{\vectr{V}_{J}}=\begin{bmatrix}
    C^{1111} & C^{1122} & C^{1133} & C^{1123} & C^{1113} & C^{1112} \\
    C^{2211} & C^{2222} & C^{2233} & C^{2223} & C^{2213} & C^{2212} \\
    C^{3311} & C^{3322} & C^{3333} & C^{3323} & C^{3313} & C^{3312} \\
    C^{2311} & C^{2322} & C^{2333} & C^{2323} & C^{2313} & C^{2312} \\
    C^{1311} & C^{1322} & C^{1333} & C^{1323} & C^{1313} & C^{1312} \\
    C^{1211} & C^{1222} & C^{1233} & C^{1223} & C^{1213} & C^{1212}     
  \end{bmatrix}
\end{equation}
\begin{equation}
  \Voigt{\tensorfour{C}}=C_{IJ}\tensorprodtwo{\vectr{V}^{I}}{\vectr{V}^{J}}=\begin{bmatrix}
    C_{1111} & C_{1122} & C_{1133} & 2C_{1123} & 2C_{1113} & 2C_{1112} \\
    C_{2211} & C_{2222} & C_{2233} & 2C_{2223} & 2C_{2213} & 2C_{2212} \\
    C_{3311} & C_{3322} & C_{3333} & 2C_{3323} & 2C_{3313} & 2C_{3312} \\
    2C_{2311} & 2C_{2322} & 2C_{2333} & 4C_{2323} & 4C_{2313} & 4C_{2312} \\
    2C_{1311} & 2C_{1322} & 2C_{1333} & 4C_{1323} & 4C_{1313} & 4C_{1312} \\
    2C_{1211} & 2C_{1222} & 2C_{1233} & 4C_{1223} & 4C_{1213} & 4C_{1212}     
  \end{bmatrix}
\end{equation}
\begin{equation}
  \Voigt{\tensorfour{C}}=C^{I.}_{.J}\tensorprodtwo{\vectr{V}_{I}}{\vectr{V}^{J}}=\begin{bmatrix}
    C^{11..}_{..11} & C^{11..}_{..22} & C^{11..}_{..33} & 2C^{11..}_{..23} & 2C^{11..}_{..13} & 2C^{11..}_{..12} \\
    C^{22..}_{..11} & C^{22..}_{..22} & C^{22..}_{..33} & 2C^{22..}_{..23} & 2C^{22..}_{..13} & 2C^{22..}_{..12} \\
    C^{33..}_{..11} & C^{33..}_{..22} & C^{33..}_{..33} & 2C^{33..}_{..23} & 2C^{33..}_{..13} & 2C^{33..}_{..12} \\
    C^{23..}_{..11} & C^{23..}_{..22} & C^{23..}_{..33} & 2C^{23..}_{..23} & 2C^{23..}_{..13} & 2C^{23..}_{..12} \\
    C^{13..}_{..11} & C^{13..}_{..22} & C^{13..}_{..33} & 2C^{13..}_{..23} & 2C^{13..}_{..13} & 2C^{13..}_{..12} \\
    C^{12..}_{..11} & C^{12..}_{..22} & C^{12..}_{..33} & 2C^{12..}_{..23} & 2C^{12..}_{..13} & 2C^{12..}_{..12}     
  \end{bmatrix}
\end{equation}
\begin{equation}
  \Voigt{\tensorfour{C}}=C^{.J}_{I.}\tensorprodtwo{\vectr{V}^{I}}{\vectr{V}_{J}}=\begin{bmatrix}
    C^{..11}_{11..} & C^{..22}_{11..} & C^{..33}_{11..} & C^{..23}_{11..} & C^{..13}_{11..} & C^{..12}_{11..} \\
    C^{..11}_{22..} & C^{..22}_{22..} & C^{..33}_{22..} & C^{..23}_{22..} & C^{..13}_{22..} & C^{..12}_{22..} \\
    C^{..11}_{33..} & C^{..22}_{33..} & C^{..33}_{33..} & C^{..23}_{33..} & C^{..13}_{33..} & C^{..12}_{33..} \\
    2C^{..11}_{23..} & 2C^{..22}_{23..} & 2C^{..33}_{23..} & 2C^{..23}_{23..} & 2C^{..13}_{23..} & 2C^{..12}_{23..} \\
    2C^{..11}_{13..} & 2C^{..22}_{13..} & 2C^{..33}_{13..} & 2C^{..23}_{13..} & 2C^{..13}_{13..} & 2C^{..12}_{13..} \\
    2C^{..11}_{12..} & 2C^{..22}_{12..} & 2C^{..33}_{12..} & 2C^{..23}_{12..} & 2C^{..13}_{12..} & 2C^{..12}_{12..}     
  \end{bmatrix}
\end{equation}

For a fourth order tensor with minor and major symmetries in two-dimensions we have
\begin{equation}
  \Voigt{\tensorfour{C}}=C^{IJ}\tensorprodtwo{\vectr{V}_{I}}{\vectr{V}_{J}}=\begin{bmatrix}
    C^{1111} & C^{1122} & C^{1112} \\
    C^{2211} & C^{2222} & C^{2212} \\
    C^{1211} & C^{1222} & C^{1212}     
  \end{bmatrix}
\end{equation}
\begin{equation}
  \Voigt{\tensorfour{C}}=C_{IJ}\tensorprodtwo{\vectr{V}^{I}}{\vectr{V}^{J}}=\begin{bmatrix}
    C_{1111} & C_{1122} & 2C_{1112} \\
    C_{2211} & C_{2222} & 2C_{2212} \\
    2C_{1211} & 2C_{1222} & 4C_{1212}     
  \end{bmatrix}
\end{equation}
\begin{equation}
  \Voigt{\tensorfour{C}}=C^{I.}_{.J}\tensorprodtwo{\vectr{V}_{I}}{\vectr{V}^{J}}=\begin{bmatrix}
    C^{11..}_{..11} & C^{11..}_{..22} & 2C^{11..}_{..12} \\
    C^{22..}_{..11} & C^{22..}_{..22} & 2C^{22..}_{..12} \\
    C^{12..}_{..11} & C^{12..}_{..22} & 2C^{12..}_{..12}     
  \end{bmatrix}
\end{equation}
\begin{equation}
  \Voigt{\tensorfour{C}}=C^{.J}_{I.}\tensorprodtwo{\vectr{V}^{I}}{\vectr{V}_{J}}=\begin{bmatrix}
    C^{..11}_{11..} & C^{..22}_{11..} & C^{..12}_{11..} \\
    C^{..11}_{22..} & C^{..22}_{22..} & C^{..12}_{22..} \\
    2C^{..11}_{12..} & 2C^{..22}_{12..} & 2C^{..12}_{12..}     
  \end{bmatrix}
\end{equation}

Note that the covariant and contravariant fourth order Voigt arrays are symmetric but the mixed arrays are not.

Voigt notation is often used in elasticity as it transforms the
stress-strain relationship from a double tensor contraction to a
matrix-vector product \ie
\begin{equation}
  \tensortwo{\sigma}=\tensorfour{C}\tensortwo{\varepsilon}=\sigma^{ij}\tensorprodtwo{\vectr{e}_{i}}{\vectr{e}_{j}}=
  \pbrac{C^{ijkl}\tensorprodfour{\vectr{e}_{i}}{\vectr{e}_{j}}{\vectr{e}_{k}}{\vectr{e}_{l}}}
  \pbrac{\varepsilon_{kl}\tensorprodtwo{\vectr{e}^{k}}{\vectr{e}^{l}}}
\end{equation}
transforms into
\begin{equation}
  \begin{bmatrix}
    \sigma^{11} \\
    \sigma^{22} \\
    \sigma^{33} \\
    \sigma^{23} \\
    \sigma^{13} \\
    \sigma^{12}
  \end{bmatrix} = \begin{bmatrix}
    C^{1111} & C^{1122} & C^{1133} & C^{1123} & C^{1113} & C^{1112} \\
    C^{2211} & C^{2222} & C^{2233} & C^{2223} & C^{2213} & C^{2212} \\
    C^{3311} & C^{3322} & C^{3333} & C^{3323} & C^{3313} & C^{3312} \\
    C^{2311} & C^{2322} & C^{2333} & C^{2323} & C^{2313} & C^{2312} \\
    C^{1311} & C^{1322} & C^{1333} & C^{1323} & C^{1313} & C^{1312} \\
    C^{1211} & C^{1222} & C^{1233} & C^{1223} & C^{1213} & C^{1212}     
  \end{bmatrix} \begin{bmatrix}
    \varepsilon_{11} \\
    \varepsilon_{22} \\
    \varepsilon_{33} \\
    2\varepsilon_{23} \\
    2\varepsilon_{13} \\
    2\varepsilon_{12}
  \end{bmatrix}
\end{equation}
or
\begin{equation}
  \begin{bmatrix}
    \sigma^{11} \\
    \sigma^{22} \\
    \sigma^{33} \\
    \sigma^{23} \\
    \sigma^{13} \\
    \sigma^{12}
  \end{bmatrix} = \begin{bmatrix}
    C^{11..}_{..11} & C^{11..}_{..22} & C^{11..}_{..33} & 2C^{11..}_{..23} & 2C^{11..}_{..13} & 2C^{11..}_{..12} \\
    C^{22..}_{..11} & C^{22..}_{..22} & C^{22..}_{..33} & 2C^{22..}_{..23} & 2C^{22..}_{..13} & 2C^{22..}_{..12} \\
    C^{33..}_{..11} & C^{33..}_{..22} & C^{33..}_{..33} & 2C^{33..}_{..23} & 2C^{33..}_{..13} & 2C^{33..}_{..12} \\
    C^{23..}_{..11} & C^{23..}_{..22} & C^{23..}_{..33} & 2C^{23..}_{..23} & 2C^{23..}_{..13} & 2C^{23..}_{..12} \\
    C^{13..}_{..11} & C^{13..}_{..22} & C^{13..}_{..33} & 2C^{13..}_{..23} & 2C^{13..}_{..13} & 2C^{13..}_{..12} \\
    C^{12..}_{..11} & C^{12..}_{..22} & C^{12..}_{..33} & 2C^{12..}_{..23} & 2C^{12..}_{..13} & 2C^{12..}_{..12}     
  \end{bmatrix} \begin{bmatrix}
    \varepsilon^{11} \\
    \varepsilon^{22} \\
    \varepsilon^{33} \\
    \varepsilon^{23} \\
    \varepsilon^{13} \\
    \varepsilon^{12}
  \end{bmatrix}
\end{equation}

\subsection{Mandel Notation}
\label{subsec:MathsNotationMandel}

\citet{Mandel:1965}\footnote{named after Jean
Mandel (1907-1982), a French scientist.} notation\index{Mandel notation} is also known as normalised
Voigt notation and sometimes as Kelvin notation or Kelvin-Mandel
notation. Mandel notation solves the problem of the correction factors
that are required when using Voigt notation.

For a second order tensor, the Mandel basis is the same as the Voigt
basis but with the last six (non-diagonal) components normalised. A second order tensor may thus be expanded as
\begin{equation}
  \Mandel{\tensortwo{A}}=A^{I}\tensortwo{M}_{I}
\end{equation}
where
the $\Mandelsymbol$ indicates a Mandel notation tensor. Here
summation is implied and $I$ ranges from $1$ to $9$, $A^{I}$ are the
contravariant Mandel components and $\tensortwo{M}_{I}$ are the covariant
Mandel basis tensors and they are given by
\begin{equation}
  A^{1}=A^{\pbrac{11}} \qquad
  \tensortwo{M}_{1}=\tensorprod{\vectr{e}_{1}}{\vectr{e}_{1}}=\begin{bmatrix}
  1 & 0 & 0 \\
  0 & 0 & 0 \\
  0 & 0 & 0
  \end{bmatrix}
\end{equation}
\begin{equation}
  A^{2}=A^{\pbrac{22}} \qquad
  \tensortwo{M}_{2}=\tensorprod{\vectr{e}_{2}}{\vectr{e}_{2}}=\begin{bmatrix}
  0 & 0 & 0 \\
  0 & 1 & 0 \\
  0 & 0 & 0
  \end{bmatrix}
\end{equation}
\begin{equation}
  A^{3}=A^{\pbrac{33}} \qquad
  \tensortwo{M}_{3}=\tensorprod{\vectr{e}_{3}}{\vectr{e}_{3}}=\begin{bmatrix}
  0 & 0 & 0 \\
  0 & 0 & 0 \\
  0 & 0 & 1
  \end{bmatrix}
\end{equation}
\begin{equation}
  A^{4}=\sqrt{2}A^{\pbrac{23}} \qquad
  \tensortwo{M}_{4}=\dfrac{\tensorprod{\vectr{e}_{2}}{\vectr{e}_{3}}+\tensorprod{\vectr{e}_{3}}{\vectr{e}_{2}}}{\sqrt{2}}=
  \dfrac{1}{\sqrt{2}}\begin{bmatrix}
    0 & 0 & 0 \\
    0 & 0 & 1 \\
    0 & 1 & 0
  \end{bmatrix}
\end{equation}
\begin{equation}
  A^{5}=\sqrt{2}A^{\pbrac{13}} \qquad
  \tensortwo{M}_{5}=\dfrac{\tensorprod{\vectr{e}_{1}}{\vectr{e}_{3}}+\tensorprod{\vectr{e}_{3}}{\vectr{e}_{1}}}{\sqrt{2}}=
  \dfrac{1}{\sqrt{2}}\begin{bmatrix}
    0 & 0 & 1 \\
    0 & 0 & 0 \\
    1 & 0 & 0
  \end{bmatrix}
\end{equation}
\begin{equation}
  A^{6}=\sqrt{2}A^{\pbrac{12}} \qquad
  \tensortwo{M}_{6}=\dfrac{\tensorprod{\vectr{e}_{1}}{\vectr{e}_{2}}+\tensorprod{\vectr{e}_{2}}{\vectr{e}_{1}}}{\sqrt{2}}=
  \dfrac{1}{\sqrt{2}}\begin{bmatrix}
    0 & 1 & 0 \\
    1 & 0 & 0 \\
    0 & 0 & 0
  \end{bmatrix}
\end{equation}
\begin{equation}
  A^{7}=\sqrt{2}A^{\sqbrac{32}} \qquad
  \tensortwo{M}_{7}=\dfrac{\tensorprod{\vectr{e}_{3}}{\vectr{e}_{2}}-\tensorprod{\vectr{e}_{2}}{\vectr{e}_{3}}}{\sqrt{2}}=
  \dfrac{1}{\sqrt{2}}\begin{bmatrix}
    0 & 0 & 0 \\
    0 & 0 & -1 \\
    0 & 1 & 0
  \end{bmatrix}
\end{equation}
\begin{equation}
  A^{8}=\sqrt{2}A^{\sqbrac{13}} \qquad
  \tensortwo{M}_{8}=\dfrac{\tensorprod{\vectr{e}_{1}}{\vectr{e}_{3}}-\tensorprod{\vectr{e}_{3}}{\vectr{e}_{1}}}{\sqrt{2}}=
  \dfrac{1}{\sqrt{2}}\begin{bmatrix}
    0 & 0 & 1 \\
    0 & 0 & 0 \\
    -1 & 0 & 0
  \end{bmatrix}
\end{equation}
\begin{equation}
  A^{9}=\sqrt{2}A^{\sqbrac{21}} \qquad
  \tensortwo{M}_{9}=\dfrac{\tensorprod{\vectr{e}_{2}}{\vectr{e}_{1}}-\tensorprod{\vectr{e}_{1}}{\vectr{e}_{2}}}{\sqrt{2}}=
  \dfrac{1}{\sqrt{2}}\begin{bmatrix}
    0 & -1 & 0 \\
    1 & 0 & 0 \\
    0 & 0 & 0
  \end{bmatrix}
\end{equation}

Because the bases are normalised no additional correction factors are
required and the differences between covariant and contravariant
components disappear. In this case a double contraction of two second
order tensors simply becomes the dot product of two Mandel vectors.

A fourth order tensor with minor and major symmetries in
three-dimensions can be written as a matrix \ie
\begin{equation}
  \Mandel{\tensorfour{C}}=C^{IJ}\tensorprodtwo{\vectr{M}_{I}}{\vectr{M}_{J}}=\begin{bmatrix}
    C^{1111} & C^{1122} & C^{1133} & \sqrt{2}C^{1123} & \sqrt{2}C^{1113} & \sqrt{2}C^{1112} \\
    C^{2211} & C^{2222} & C^{2233} & \sqrt{2}C^{2223} & \sqrt{2}C^{2213} & \sqrt{2}C^{2212} \\
    C^{3311} & C^{3322} & C^{3333} & \sqrt{2}C^{3323} & \sqrt{2}C^{3313} & \sqrt{2}C^{3312} \\
    \sqrt{2}C^{2311} & \sqrt{2}C^{2322} & \sqrt{2}C^{2333} & 2C^{2323} & 2C^{2313} & 2C^{2312} \\
    \sqrt{2}C^{1311} & \sqrt{2}C^{1322} & \sqrt{2}C^{1333} & 2C^{1323} & 2C^{1313} & 2C^{1312} \\
    \sqrt{2}C^{1211} & \sqrt{2}C^{1222} & \sqrt{2}C^{1233} & 2C^{1223} & 2C^{1213} & 2C^{1212}     
  \end{bmatrix}
\end{equation}
\begin{equation}
  \Mandel{\tensorfour{C}}=C_{IJ}\tensorprodtwo{\vectr{M}^{I}}{\vectr{M}^{J}}=\begin{bmatrix}
    C_{1111} & C_{1122} & C_{1133} & \sqrt{2}C_{1123} & \sqrt{2}C_{1113} & \sqrt{2}C_{1112} \\
    C_{2211} & C_{2222} & C_{2233} & \sqrt{2}C_{2223} & \sqrt{2}C_{2213} & \sqrt{2}C_{2212} \\
    C_{3311} & C_{3322} & C_{3333} & \sqrt{2}C_{3323} & \sqrt{2}C_{3313} & \sqrt{2}C_{3312} \\
    \sqrt{2}C_{2311} & \sqrt{2}C_{2322} & \sqrt{2}C_{2333} & 2C_{2323} & 2C_{2313} & 2C_{2312} \\
    \sqrt{2}C_{1311} & \sqrt{2}C_{1322} & \sqrt{2}C_{1333} & 2C_{1323} & 2C_{1313} & 2C_{1312} \\
    \sqrt{2}C_{1211} & \sqrt{2}C_{1222} & \sqrt{2}C_{1233} & 2C_{1223} & 2C_{1213} & 2C_{1212}     
  \end{bmatrix}
\end{equation}
\begin{equation}
  \Mandel{\tensorfour{C}}=C^{I.}_{.J}\tensorprodtwo{\vectr{M}_{I}}{\vectr{M}^{J}}=\begin{bmatrix}
    C^{11..}_{..11} & C^{11..}_{..22} & C^{11..}_{..33} & \sqrt{2}C^{11..}_{..23} & \sqrt{2}C^{11..}_{..13} & \sqrt{2}C^{11..}_{..12} \\
    C^{22..}_{..11} & C^{22..}_{..22} & C^{22..}_{..33} & \sqrt{2}C^{22..}_{..23} & \sqrt{2}C^{22..}_{..13} & \sqrt{2}C^{22..}_{..12} \\
    C^{33..}_{..11} & C^{33..}_{..22} & C^{33..}_{..33} & \sqrt{2}C^{33..}_{..23} & \sqrt{2}C^{33..}_{..13} & \sqrt{2}C^{33..}_{..12} \\
    \sqrt{2}C^{23..}_{..11} & \sqrt{2}C^{23..}_{..22} & \sqrt{2}C^{23..}_{..33} & 2C^{23..}_{..23} & 2C^{23..}_{..13} & 2C^{23..}_{..12} \\
    \sqrt{2}C^{13..}_{..11} & \sqrt{2}C^{13..}_{..22} & \sqrt{2}C^{13..}_{..33} & 2C^{13..}_{..23} & 2C^{13..}_{..13} & 2C^{13..}_{..12} \\
    \sqrt{2}C^{12..}_{..11} & \sqrt{2}C^{12..}_{..22} & \sqrt{2}C^{12..}_{..33} & 2C^{12..}_{..23} & 2C^{12..}_{..13} & 2C^{12..}_{..12}     
  \end{bmatrix}
\end{equation}
\begin{equation}
  \Mandel{\tensorfour{C}}=C^{.J}_{I.}\tensorprodtwo{\vectr{M}^{I}}{\vectr{M}_{J}}=\begin{bmatrix}
    C^{..11}_{11..} & C^{..22}_{11..} & C^{..33}_{11..} & \sqrt{2}C^{..23}_{11..} & \sqrt{2}C^{..13}_{11..} & \sqrt{2}C^{..12}_{11..} \\
    C^{..11}_{22..} & C^{..22}_{22..} & C^{..33}_{22..} & \sqrt{2}C^{..23}_{22..} & \sqrt{2}C^{..13}_{22..} & \sqrt{2}C^{..12}_{22..} \\
    C^{..11}_{33..} & C^{..22}_{33..} & C^{..33}_{33..} & \sqrt{2}C^{..23}_{33..} & \sqrt{2}C^{..13}_{33..} & \sqrt{2}C^{..12}_{33..} \\
    \sqrt{2}C^{..11}_{23..} & \sqrt{2}C^{..22}_{23..} & \sqrt{2}C^{..33}_{23..} & 2C^{..23}_{23..} & 2C^{..13}_{23..} & 2C^{..12}_{23..} \\
    \sqrt{2}C^{..11}_{13..} & \sqrt{2}C^{..22}_{13..} & \sqrt{2}C^{..33}_{13..} & 2C^{..23}_{13..} & 2C^{..13}_{13..} & 2C^{..12}_{13..} \\
    \sqrt{2}C^{..11}_{12..} & \sqrt{2}C^{..22}_{12..} & \sqrt{2}C^{..33}_{12..} & 2C^{..23}_{12..} & 2C^{..13}_{12..} & 2C^{..12}_{12..}     
  \end{bmatrix}
\end{equation}

For a fourth order tensor with minor and major symmetries in two-dimensions we have
\begin{equation}
  \Mandel{\tensorfour{C}}=C^{IJ}\tensorprodtwo{\vectr{M}_{I}}{\vectr{M}_{J}}=\begin{bmatrix}
    C^{1111} & C^{1122} & \sqrt{2}C^{1112} \\
    C^{2211} & C^{2222} & \sqrt{2}C^{2212} \\
    \sqrt{2}C^{1211} & \sqrt{2}C^{1222} & 2C^{1212}     
  \end{bmatrix}
\end{equation}
\begin{equation}
  \Mandel{\tensorfour{C}}=C_{IJ}\tensorprodtwo{\vectr{M}^{I}}{\vectr{M}^{J}}=\begin{bmatrix}
    C_{1111} & C_{1122} & \sqrt{2}C_{1112} \\
    C_{2211} & C_{2222} & \sqrt{2}C_{2212} \\
    \sqrt{2}C_{1211} & \sqrt{2}C_{1222} & 2C_{1212}     
  \end{bmatrix}
\end{equation}
\begin{equation}
  \Mandel{\tensorfour{C}}=C^{I.}_{.J}\tensorprodtwo{\vectr{M}_{I}}{\vectr{M}^{J}}=\begin{bmatrix}
    C^{11..}_{..11} & C^{11..}_{..22} & \sqrt{2}C^{11..}_{..12} \\
    C^{22..}_{..11} & C^{22..}_{..22} & \sqrt{2}C^{22..}_{..12} \\
    \sqrt{2}C^{12..}_{..11} & \sqrt{2}C^{12..}_{..22} & 2C^{12..}_{..12}     
  \end{bmatrix}
\end{equation}
\begin{equation}
  \Mandel{\tensorfour{C}}=C^{.J}_{I.}\tensorprodtwo{\vectr{M}^{I}}{\vectr{M}_{J}}=\begin{bmatrix}
    C^{..11}_{11..} & C^{..22}_{11..} & \sqrt{2}C^{..12}_{11..} \\
    C^{..11}_{22..} & C^{..22}_{22..} & \sqrt{2}C^{..12}_{22..} \\
    \sqrt{2}C^{..11}_{12..} & \sqrt{2}C^{..22}_{12..} & 2C^{..12}_{12..}     
  \end{bmatrix}
\end{equation}

A general fourth order tensor with $81$ independent components may be written in Mandel notation as a $9 \times 9$ array \ie
\begin{equation}
  \scriptsize\Mandel{\tensorfour{C}}=\left[\begin{array}{ccc|ccc|ccc}
      C^{\pbrac{11}\pbrac{11}} & C^{\pbrac{11}\pbrac{22}} & C^{\pbrac{11}\pbrac{33}} &
      \sqrt{2}C^{\pbrac{11}\pbrac{23}} & \sqrt{2}C^{\pbrac{11}\pbrac{13}} & \sqrt{2}C^{\pbrac{11}\pbrac{12}} &
      \sqrt{2}C^{\pbrac{11}\sqbrac{32}} & \sqrt{2}C^{\pbrac{11}\sqbrac{13}} & \sqrt{2}C^{\pbrac{11}\sqbrac{21}} \\
      C^{\pbrac{22}\pbrac{11}} & C^{\pbrac{22}\pbrac{22}} & C^{\pbrac{22}\pbrac{33}} &
      \sqrt{2}C^{\pbrac{22}\pbrac{23}} & \sqrt{2}C^{\pbrac{22}\pbrac{13}} & \sqrt{2}C^{\pbrac{22}\pbrac{12}} &
      \sqrt{2}C^{\pbrac{22}\sqbrac{32}} & \sqrt{2}C^{\pbrac{22}\sqbrac{13}} & \sqrt{2}C^{\pbrac{22}\sqbrac{21}} \\
      C^{\pbrac{33}\pbrac{11}} & C^{\pbrac{33}\pbrac{22}} & C^{\pbrac{33}\pbrac{33}} &
      \sqrt{2}C^{\pbrac{33}\pbrac{23}} & \sqrt{2}C^{\pbrac{33}\pbrac{13}} & \sqrt{2}C^{\pbrac{33}\pbrac{12}} &
      \sqrt{2}C^{\pbrac{33}\sqbrac{32}} & \sqrt{2}C^{\pbrac{33}\sqbrac{13}} & \sqrt{2}C^{\pbrac{33}\sqbrac{21}} \\ \hline
      \sqrt{2}C^{\pbrac{23}\pbrac{11}} & \sqrt{2}C^{\pbrac{23}\pbrac{22}} & \sqrt{2}C^{\pbrac{23}\pbrac{33}} &
      2C^{\pbrac{23}\pbrac{23}} & 2C^{\pbrac{23}\pbrac{13}} & 2C^{\pbrac{23}\pbrac{12}} &
      2C^{\pbrac{23}\sqbrac{32}} & 2C^{\pbrac{23}\sqbrac{13}} & 2C^{\pbrac{23}\sqbrac{21}} \\
      \sqrt{2}C^{\pbrac{13}\pbrac{11}} & \sqrt{2}C^{\pbrac{13}\pbrac{22}} & \sqrt{2}C^{\pbrac{13}\pbrac{33}} &
      2C^{\pbrac{13}\pbrac{23}} & 2C^{\pbrac{13}\pbrac{13}} & 2C^{\pbrac{13}\pbrac{12}} &
      2C^{\pbrac{13}\sqbrac{32}} & 2C^{\pbrac{13}\sqbrac{13}} & 2C^{\pbrac{13}\sqbrac{21}} \\
      \sqrt{2}C^{\pbrac{12}\pbrac{11}} & \sqrt{2}C^{\pbrac{12}\pbrac{22}} & \sqrt{2}C^{\pbrac{12}\pbrac{33}} &
      2C^{\pbrac{12}\pbrac{23}} & 2C^{\pbrac{12}\pbrac{13}} & 2C^{\pbrac{12}\pbrac{12}} &
      2C^{\pbrac{12}\sqbrac{32}} & 2C^{\pbrac{12}\sqbrac{13}} & 2C^{\pbrac{12}\sqbrac{21}} \\ \hline
      \sqrt{2}^{\sqbrac{32}\pbrac{11}} & \sqrt{2}C^{\sqbrac{32}\pbrac{22}} & \sqrt{2}C^{\sqbrac{32}\pbrac{33}} &
      2C^{\sqbrac{32}\pbrac{23}} & 2C^{\sqbrac{32}\pbrac{13}} & 2C^{\sqbrac{32}\pbrac{12}} &
      2C^{\sqbrac{32}\sqbrac{32}} & 2C^{\sqbrac{32}\sqbrac{13}} & 2C^{\sqbrac{32}\sqbrac{21}} \\
      \sqrt{2}C^{\sqbrac{13}\pbrac{11}} & \sqrt{2}C^{\sqbrac{13}\pbrac{22}} & \sqrt{2}C^{\sqbrac{13}\pbrac{33}} &
      2C^{\sqbrac{13}\pbrac{23}} & 2C^{\sqbrac{13}\pbrac{13}} & 2C^{\sqbrac{13}\pbrac{12}} &
      2C^{\sqbrac{13}\sqbrac{32}} & 2C^{\sqbrac{13}\sqbrac{13}} & 2C^{\sqbrac{13}\sqbrac{21}} \\
      \sqrt{2}C^{\sqbrac{21}\pbrac{11}} & \sqrt{2}C^{\sqbrac{21}\pbrac{22}} & \sqrt{2}C^{\sqbrac{21}\pbrac{33}} &
      2C^{\sqbrac{21}\pbrac{23}} & 2C^{\sqbrac{21}\pbrac{13}} & 2C^{\sqbrac{21}\pbrac{12}} &
      2C^{\sqbrac{21}\sqbrac{32}} & 2C^{\sqbrac{21}\sqbrac{13}} & 2C^{\sqbrac{21}\sqbrac{21}} \\ 
    \end{array}\right]\normalsize
\end{equation}
where
\begin{align}
  C^{\pbrac{ij}\pbrac{kl}} &= \dfrac{1}{4}\pbrac{C^{ijkl}+C^{jikl}+C^{ijlk}+C^{jilk}} \\
  C^{\pbrac{ij}\sqbrac{kl}} &= \dfrac{1}{4}\pbrac{C^{ijkl}+C^{jikl}-C^{ijlk}-C^{jilk}} \\
  C^{\sqbrac{ij}\pbrac{kl}} &= \dfrac{1}{4}\pbrac{C^{ijkl}-C^{jikl}+C^{ijlk}-C^{jilk}} \\
  C^{\sqbrac{ij}\sqbrac{kl}} &= \dfrac{1}{4}\pbrac{C^{ijkl}-C^{jikl}-C^{ijlk}-C^{jilk}}
\end{align}

In Mandel notation the stress-strain relationship from a double tensor
contraction to a matrix-vector product \ie
\begin{equation}
  \tensortwo{\sigma}=\tensorfour{C}\tensortwo{\varepsilon}=\sigma^{ij}\tensorprodtwo{\vectr{e}_{i}}{\vectr{e}_{j}}=
  \pbrac{C^{ijkl}\tensorprodfour{\vectr{e}_{i}}{\vectr{e}_{j}}{\vectr{e}_{k}}{\vectr{e}_{l}}}
  \pbrac{\varepsilon_{kl}\tensorprodtwo{\vectr{e}^{k}}{\vectr{e}^{l}}}
\end{equation}
transforms into
\begin{equation}
  \begin{bmatrix}
    \sigma^{11} \\
    \sigma^{22} \\
    \sigma^{33} \\
    \sqrt{2}\sigma^{23} \\
    \sqrt{2}\sigma^{13} \\
    \sqrt{2}\sigma^{12}
  \end{bmatrix} = \begin{bmatrix}
    C^{1111} & C^{1122} & C^{1133} & \sqrt{2}C^{1123} & \sqrt{2}C^{1113} & \sqrt{2}C^{1112} \\
    C^{2211} & C^{2222} & C^{2233} & \sqrt{2}C^{2223} & \sqrt{2}C^{2213} & \sqrt{2}C^{2212} \\
    C^{3311} & C^{3322} & C^{3333} & \sqrt{2}C^{3323} & \sqrt{2}C^{3313} & \sqrt{2}C^{3312} \\
    \sqrt{2}C^{2311} & \sqrt{2}C^{2322} & \sqrt{2}C^{2333} & \sqrt{2}C^{2323} & \sqrt{2}C^{2313} & \sqrt{2}C^{2312} \\
    \sqrt{2}C^{1311} & \sqrt{2}C^{1322} & \sqrt{2}C^{1333} & \sqrt{2}C^{1323} & \sqrt{2}C^{1313} & \sqrt{2}C^{1312} \\
    \sqrt{2}C^{1211} & \sqrt{2}C^{1222} & \sqrt{2}C^{1233} & \sqrt{2}C^{1223} & \sqrt{2}C^{1213} & \sqrt{2}C^{1212}     
  \end{bmatrix} \begin{bmatrix}
    \varepsilon_{11} \\
    \varepsilon_{22} \\
    \varepsilon_{33} \\
    \sqrt{2}\varepsilon_{23} \\
    \sqrt{2}\varepsilon_{13} \\
    \sqrt{2}\varepsilon_{12}
  \end{bmatrix}
\end{equation}
