\subsubsection{Numbers}
\label{subsubsec:MathsSpaceSetsNumbers}

\subsubsubsection{Natural numbers}

The set of \emph{natural numbers} is denoted $\naturalnums$.

\subsubsubsection{Integer numbers}

The set of \emph{integer numbers} is denoted $\integernums$.

\subsubsubsection{Rational numbers}

The set of \emph{rational numbers} is denoted $\rationalnums$.

\subsubsubsection{Real numbers}

The set of \emph{real numbers} is denoted $\realnums$.

\subsubsubsection{Complex numbers}

The set of \emph{complex numbers} is denoted $\complexnums$.

\subsubsubsection{Quaternion numbers}

\emph{Quaterions}\index{quaterions} are an extension of complex
numbers that can be used to provide a way of rotating
vectors. Quaterions were first described by
Hamilton\footnote{\link{https://en.wikipedia.org/wiki/William_Rowan_Hamilton}{Sir
  William Rown Hamilton} (1805-1865), an Irish mathematician.} in
1843. Hamiliton was looking for a way of extending complex numbers,
which could be represented as a point in a plane, in a way that would
allow them to be represented as a point in three-dimensional space. In
order to represent this point in three-dimensional space Hamilton
initially tried using three numbers. Despite working out how such a
triplet of numbers could be added and subtracted, Hamilton struggled
to work out how such a triplet points could be multipled or
divided. The break-through finally came whilst walking along a canal
towpath in Dublin when Hamilton realised he need four numbers. Excited, Hamilton carved the formula
\begin{equation}
  \quaternion{i}^{2}=\quaternion{j}^{2}=\quaternion{k}^{2}=\quaternion{i}\quaternion{j}\quaternion{k}=-1
\end{equation}
into the stone of Brougham Bridge. Hamilton named these
four-dimensional points \emph{quaterions} after the four numbers used
to define them. The space of quaterions is given the symbol
$\quaternionnums$ after Hamilton.

Quaterions are reprsented in the form
\begin{equation}
  q = a\quaternion{1} + b\quaternion{i} + c\quaternion{j} + d\quaternion{k}
\end{equation}
where $a,b,c,d\in\realnums$ and
$\quaternion{1},\quaternion{i},\quaternion{j},\quaternion{k}$ form the basis for
the quaterion numbers, $\quaternionnums$.


