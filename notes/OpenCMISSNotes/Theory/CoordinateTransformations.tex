\section{Coordinate Transformations}
\label{sec:CoordinateTransformations}

\subsection{Two dimensions}
\label{subsec:CoordinateTransformationsTwoD}

Consider the two dimensional system shown in
\figref{fig:CoordinateTransformationTwoD}. There are three coordinate
systems of interest. The first is the geometric coordinate system,
$\coordinatevector$, and the second is the $\elementcoordinatevector$
coordinate system of an element. The final coordinate system is the
$\fibrecoordinatevector$ coordinate system of the fibre directions.

\epstexfigure{Theory/svgs/TwoDCoordinateTransformation.eps_tex}{Two
  dimensional coordinate transformations.}{Two dimensional coordinate
  transformations.}{fig:CoordinateTransformationTwoD}{0.5}

To work out the transformations between these coordinate systems we
first define the geometric position vector (with respect to the base
vectors $\generalbasevector_{i}$) of a point inside the element using
interpolation \ie
\begin{equation}
  \fnof{\coordinatevector}{\elementcoordinatevector}=\fnof{\coordinate{i}}{\elementcoordinatevector}\generalbasevector_{i}=
  \gsf{n}{\beta}\gbfn{n}{\beta}{\elementcoordinatevector}\nodedof{\coordinatevector}{n}{\beta}
\end{equation}

The transformation matrix between geometric coordinates,
$\coordinatevector$, and element coordinates,
$\elementcoordinatevector$, is thus given by
\begin{equation}
  \delby{\coordinate{i}}{\elementcoordinate{r}}=\begin{bmatrix}
  \delby{\coordinate{1}}{\xione} & \delby{\coordinate{2}}{\xione} \\
  \delby{\coordinate{1}}{\xitwo} & \delby{\coordinate{2}}{\xitwo}
  \end{bmatrix}
  \label{eqn:GeometricToElementTransformationTwoD}
\end{equation}
and the inverse transformation between element coordinates,
$\elementcoordinatevector$, and geometric coordinates,
$\coordinatevector$, is thus given by
\begin{equation}
  \delby{\elementcoordinate{r}}{\coordinate{i}}=\inverse{\sqbrac{\delby{\coordinate{i}}{\elementcoordinate{r}}}}=
  \dfrac{1}{\delby{\coordinate{1}}{\xione}\delby{\coordinate{2}}{\xitwo}-\delby{\coordinate{2}}{\xione}
    \delby{\coordinate{1}}{\xitwo}}\begin{bmatrix}
    \delby{\coordinate{2}}{\xitwo} & -\delby{\coordinate{2}}{\xione} \\
    -\delby{\coordinate{1}}{\xitwo} & \delby{\coordinate{1}}{\xione}    
  \end{bmatrix}=\begin{bmatrix}
  \delby{\elementcoordinate{1}}{\coordinate{1}} & \delby{\elementcoordinate{2}}{\coordinate{1}} \\
  \delby{\elementcoordinate{1}}{\coordinate{2}} & \delby{\elementcoordinate{2}}{\coordinate{2}}
  \end{bmatrix}
  \label{eqn:ElementToGeometricTransformationTwoD}
\end{equation}

The derivative of the geometric interpolation also gives the tangent
direction with respect to $\xione$. We define the normalisation of
this direction as the vector, $\vectr{a}$, \ie
\begin{equation}
  \fnof{\vectr{a}}{\elementcoordinatevector}=\fnof{a^{i}}{\elementcoordinatevector}\generalbasevector_{i}=
  \norm{\delby{\fnof{\coordinatevector}{\elementcoordinatevector}}{\xione}}
\end{equation}

We now define the vector, $\vectr{b}$, to be orthogonal to the vector, $\vectr{a}$, by rotating $\vectr{a}$ counter clockwise
$90\degree$ \ie
\begin{equation}
  \fnof{\vectr{b}}{\elementcoordinatevector}=\fnof{b^{i}}{\elementcoordinatevector}\generalbasevector_{i}=
  \fnof{a^{2}}{\elementcoordinatevector}\generalbasevector_{1}-\fnof{a^{1}}{\elementcoordinatevector}\generalbasevector_{2}
\end{equation}

Now the fibre direction is defined by a rotation angle, $\theta$, from
the $\xione$ direction. This angle is given by interpolation within
the element \ie
\begin{equation}
  \fnof{\theta}{\elementcoordinatevector}=
  \gsf{o}{\gamma}\gbfn{o}{\gamma}{\elementcoordinatevector}\nodedof{\vectr{\theta}}{o}{\gamma}
\end{equation}

We can now find the fibre vector, $\vectr{f}$, and sheet vector,
$\vectr{s}$, by rotating the vectors $\vectr{a}$ and $\vectr{b}$ by
$\theta$. To do this we make use of the rotation matrix
\begin{equation}
  \fnof{\matr{R}}{\fnof{\theta}{\elementcoordinatevector}}=\begin{bmatrix}
  \cosine{\fnof{\theta}{\elementcoordinatevector}} & \sine{\fnof{\theta}{\elementcoordinatevector}} \\
  -\sine{\fnof{\theta}{\elementcoordinatevector}} & \cosine{\fnof{\theta}{\elementcoordinatevector}}
  \end{bmatrix}
\end{equation}
as
\begin{equation}
  \fnof{\vectr{f}}{\elementcoordinatevector}=\fnof{f^{i}}{\elementcoordinatevector}\generalbasevector_{i}=
  \fnof{\matr{R}}{\fnof{\theta}{\elementcoordinatevector}}\fnof{\vectr{a}}{\elementcoordinatevector}
\end{equation}
and
\begin{equation}
  \fnof{\vectr{s}}{\elementcoordinatevector}=\fnof{s^{i}}{\elementcoordinatevector}\generalbasevector_{i}=
  \fnof{\matr{R}}{\fnof{\theta}{\elementcoordinatevector}}\fnof{\vectr{b}}{\elementcoordinatevector}
\end{equation}
  
The rotation matrix thus gives the transformation between element
coordinates, $\elementcoordinatevector$, and fibre coordinates,
$\fibrecoordinatevector$, \ie
\begin{equation}
  \delby{\fibrecoordinate{a}}{\elementcoordinate{r}}=\fnof{\matr{R}}{\fnof{\theta}{\elementcoordinatevector}}\begin{bmatrix}
  \cosine{\fnof{\theta}{\elementcoordinatevector}} & \sine{\fnof{\theta}{\elementcoordinatevector}} \\
  -\sine{\fnof{\theta}{\elementcoordinatevector}} & \cosine{\fnof{\theta}{\elementcoordinatevector}}
  \end{bmatrix}  
  \label{eqn:ElementToMaterialTransformationTwoD}
\end{equation}

The transformation between fibre coordinates,
$\fibrecoordinatevector$, and element coordinates,
$\elementcoordinatevector$, can thus be found from
\begin{equation}
  \delby{\elementcoordinate{r}}{\fibrecoordinate{a}}=
  \inverse{\sqbrac{\delby{\fibrecoordinate{a}}{\elementcoordinate{r}}}}=
  \inverse{\fnof{\matr{R}}{\fnof{\theta}{\elementcoordinatevector}}}=
  \transpose{\fnof{\matr{R}}{\fnof{\theta}{\elementcoordinatevector}}}=\begin{bmatrix}
  \cosine{\fnof{\theta}{\elementcoordinatevector}} & -\sine{\fnof{\theta}{\elementcoordinatevector}} \\
  \sine{\fnof{\theta}{\elementcoordinatevector}} & \cosine{\fnof{\theta}{\elementcoordinatevector}}
  \end{bmatrix}  
  \label{eqn:FibreToElementTransformationTwoD}
\end{equation}
as the rotation matrix is orthogonal.

\subsection{Three dimensions}
\label{subsec:CoordinateTransformationsThreeD}

Unlike in two-dimensions, rotation of coordinates in three-dimensions
is not straightforward and there a number of different ways to define
the coordinate transformations.

\subsubsection{Euler and Tait-Bryan angles}
\label{subsubsec:CoordinateTransformationsThreeDEulerTaitBryan}

\subsubsection{Quaternions}
\label{subsubsec:CoordinateTransformationsThreeDQuaternions}


\subsection{Tensor transformations}
\label{subsec:CoordinateTransformationsTensor}

Having defined the relationships between the various fibre, element and
geometric coordinate systems we can transform fibre tensors.

For example consider transforming a fibre vector, $\vectr{a}$, to a
geometric vector, $\vectr{b}$. The transformations are
\begin{equation}
  b^{i}=\delby{\coordinate{i}}{\elementcoordinate{r}}\delby{\elementcoordinate{r}}{\fibrecoordinate{a}}a^{a}
  \label{eqn:Rank10FibreToElementToCoordinateTransformation}
\end{equation}
for rank(1,0) vectors, and,
\begin{equation}
  b_{i}=\delby{\fibrecoordinate{a}}{\elementcoordinate{r}}\delby{\elementcoordinate{r}}{\coordinate{i}}a_{a}
  \label{eqn:Rank01FibreToElementToCoordinateTransformation}
\end{equation}
for rank(0,1) vectors.

Now, by defining
\begin{equation}
  \delby{\coordinate{i}}{\fibrecoordinate{a}}=\delby{\coordinate{i}}{\elementcoordinate{r}}
  \delby{\elementcoordinate{r}}{\fibrecoordinate{a}}
  \label{eqn:CoordinateToFibreTransformationViaElement}
\end{equation}
and
\begin{equation}
  \delby{\fibrecoordinate{a}}{\coordinate{i}}=\delby{\fibrecoordinate{a}}{\elementcoordinate{r}}
  \delby{\elementcoordinate{r}}{\coordinate{i}}
  \label{eqn:FibreToCoordinateTransformationViaElement}
\end{equation}
we can find that transformations without going through element coordinates is given by
\begin{equation}
  b^{i}=\delby{\coordinate{i}}{\fibrecoordinate{a}}a^{a}
  \label{eqn:Rank10FibreToCoordinateTransformation}
\end{equation}
for rank(1,0) vectors, and,
\begin{equation}
  b_{i}=\delby{\fibrecoordinate{a}}{\coordinate{i}}a_{a}
  \label{eqn:Rank01FibreToCoordinateTransformation}
\end{equation}
for rank(0,1) vectors.

The inverse transformation between a geometric vector, $\vectr{b}$, and a fibre vector, $\vectr{a}$, is given by
\begin{equation}
  a^{a}=\delby{\fibrecoordinate{a}}{\elementcoordinate{r}}\delby{\elementcoordinate{r}}{\coordinate{i}}b^{i}
  \label{eqn:Rank10CoordinateToElementToFibreTransformation}
\end{equation}
for rank(1,0) vectors, and,
\begin{equation}
  a_{a}=\delby{\coordinate{i}}{\elementcoordinate{r}}\delby{\elementcoordinate{r}}{\fibrecoordinate{a}}b_{i}
  \label{eqn:Rank01CoordinateToElementToFibreTransformation}
\end{equation}
for rank(0,1) vectors.

Using
\eqnrefs{eqn:CoordinateToFibreTransformationViaElement}{eqn:FibreToCoordinateTransformationViaElement}
the inverse transformations are
\begin{equation}
  a^{a}=\delby{\fibrecoordinate{a}}{\coordinate{i}}b^{i}
  \label{eqn:Rank10CoordinateToFibreTransformation}
\end{equation}
for rank(1,0) vectors, and,
\begin{equation}
  a_{a}=\delby{\coordinate{i}}{\fibrecoordinate{a}}b_{i}
  \label{eqn:Rank01CoordinateToFibreTransformation}
\end{equation}
for rank(0,1) vectors.

For a second order fibre tensors, $\tensortwo{A}$, to geometric
tensors, $\tensortwo{B}$, the transformations are
\begin{equation}
  B^{ij}=\delby{\coordinate{i}}{\elementcoordinate{r}}\delby{\elementcoordinate{r}}{\fibrecoordinate{a}}
  \delby{\coordinate{j}}{\elementcoordinate{s}}\delby{\elementcoordinate{s}}{\fibrecoordinate{b}}A^{ab}
  \label{eqn:Rank20FibreToElementToCoordinateTransformation}
\end{equation}
for rank(2,0) tensors, and,
\begin{equation}
  B_{ij}=\delby{\fibrecoordinate{a}}{\elementcoordinate{r}}\delby{\elementcoordinate{r}}{\coordinate{i}}
  \delby{\fibrecoordinate{b}}{\elementcoordinate{s}}\delby{\elementcoordinate{s}}{\coordinate{j}}A_{ab}
  \label{eqn:Rank02FibreToElementToCoordinateTransformation}
\end{equation}
for rank(0,2) tensors, and,
\begin{equation}
  B^{i.}_{.j}=\delby{\coordinate{i}}{\elementcoordinate{r}}\delby{\elementcoordinate{r}}{\fibrecoordinate{a}}
  \delby{\fibrecoordinate{b}}{\elementcoordinate{s}}\delby{\elementcoordinate{s}}{\coordinate{j}}A^{a.}_{.b}
  \label{eqn:Rank11FibreToElementToCoordinateTransformation1}
\end{equation}
and
\begin{equation}
  B^{.j}_{i.}=\delby{\fibrecoordinate{a}}{\elementcoordinate{r}}\delby{\elementcoordinate{r}}{\coordinate{i}}
  \delby{\coordinate{j}}{\elementcoordinate{s}}\delby{\elementcoordinate{s}}{\fibrecoordinate{b}}A^{.b}_{a.}
  \label{eqn:Rank11FibreToElementToCoordinateTransformation2}
\end{equation}
for rank(1,1) tensors.

Using \eqnrefs{eqn:CoordinateToFibreTransformationViaElement}{eqn:FibreToCoordinateTransformationViaElement}
the transformations are
\begin{equation}
  B^{ij}=\delby{\coordinate{i}}{\fibrecoordinate{a}}\delby{\coordinate{j}}{\fibrecoordinate{b}}A^{ab}
  \label{eqn:Rank20FibreToCoordinateTransformation}
\end{equation}
for rank(2,0) tensors, and,
\begin{equation}
  B_{ij}=\delby{\fibrecoordinate{a}}{\coordinate{i}}\delby{\fibrecoordinate{b}}{\coordinate{j}}A_{ab}
  \label{eqn:Rank02FibreToCoordinateTransformation}
\end{equation}
for rank(0,2) tensors, and,
\begin{equation}
  B^{i.}_{.j}=\delby{\coordinate{i}}{\fibrecoordinate{a}}\delby{\fibrecoordinate{b}}{\coordinate{j}}A^{a.}_{.b}
  \label{eqn:Rank11FibreToCoordinateTransformation1}
\end{equation}
and
\begin{equation}
  B^{.j}_{i.}=\delby{\fibrecoordinate{a}}{\coordinate{i}}\delby{\coordinate{j}}{\fibrecoordinate{b}}A^{.b}_{a.}
  \label{eqn:Rank11FibreToCoordinateTransformation2}
\end{equation}
for rank(1,1) tensors.

The inverse transformation for second order geometric tensors, $\tensortwo{B}$, to fibre
tensors, $\tensortwo{A}$, are given by
\begin{equation}
  A^{ab}=\delby{\fibrecoordinate{a}}{\elementcoordinate{r}}\delby{\elementcoordinate{r}}{\coordinate{i}}
  \delby{\fibrecoordinate{b}}{\elementcoordinate{s}}\delby{\elementcoordinate{s}}{\coordinate{j}}B^{ij}
  \label{eqn:Rank20CoordinatToElementToFibreTransformation}
\end{equation}
for rank(2,0) tensors, and,
\begin{equation}
  A_{ab}=\delby{\coordinate{i}}{\elementcoordinate{r}}\delby{\elementcoordinate{r}}{\fibrecoordinate{a}}
  \delby{\coordinate{j}}{\elementcoordinate{s}}\delby{\elementcoordinate{s}}{\fibrecoordinate{b}}B_{ij}
  \label{eqn:Rank02CoordinateToElementToFibreTransformation}
\end{equation}
for rank(0,2) tensors, and,
\begin{equation}
  A^{a.}_{.b}=\delby{\fibrecoordinate{a}}{\elementcoordinate{r}}\delby{\elementcoordinate{r}}{\coordinate{i}}
  \delby{\coordinate{j}}{\elementcoordinate{s}}\delby{\elementcoordinate{s}}{\fibrecoordinate{b}}B^{i.}_{.j}
  \label{eqn:Rank11CoordinateToElementToFibreTransformation1}
\end{equation}
and
\begin{equation}
  A^{.b}_{a.}=\delby{\coordinate{i}}{\elementcoordinate{r}}\delby{\elementcoordinate{r}}{\fibrecoordinate{a}}
  \delby{\fibrecoordinate{b}}{\elementcoordinate{s}}\delby{\elementcoordinate{s}}{\coordinate{j}}B^{.j}_{i.}
  \label{eqn:Rank11CoordinateToElementToFibreTransformation2}
\end{equation}
for rank(1,1) tensors.

Using \eqnrefs{eqn:CoordinateToFibreTransformationViaElement}{eqn:FibreToCoordinateTransformationViaElement}
the inverse transformations are
\begin{equation}
  A^{ab}=\delby{\fibrecoordinate{a}}{\coordinate{i}}\delby{\fibrecoordinate{b}}{\coordinate{j}}B^{ij}
  \label{eqn:Rank20CoordinateToFibreTransformation}
\end{equation}
for rank(2,0) tensors, and,
\begin{equation}
  A_{ab}=\delby{\coordinate{i}}{\fibrecoordinate{a}}\delby{\coordinate{j}}{\fibrecoordinate{b}}B_{ij}
  \label{eqn:Rank02CoordinateToFibreTransformation}
\end{equation}
for rank(0,2) tensors, and,
\begin{equation}
  A^{a.}_{.b}=\delby{\fibrecoordinate{a}}{\coordinate{i}}\delby{\coordinate{j}}{\fibrecoordinate{b}}B^{i.}_{.j}
  \label{eqn:Rank11CoordinateToFibreTransformation1}
\end{equation}
and
\begin{equation}
  A^{.b}_{a.}=\delby{\coordinate{i}}{\fibrecoordinate{a}}\delby{\fibrecoordinate{b}}{\coordinate{j}}B^{.j}_{i.}
  \label{eqn:Rank11CoordinateToFibreTransformation2}
\end{equation}
for rank(1,1) tensors.

For fourth order fibre tensors, $\tensorfour{A}$, to geometric
tensors, $\tensorfour{B}$, the transformations are
\begin{equation}
  B^{ijkl}=\delby{\coordinate{i}}{\elementcoordinate{r}}\delby{\elementcoordinate{r}}{\fibrecoordinate{a}}
  \delby{\coordinate{j}}{\elementcoordinate{s}}\delby{\elementcoordinate{s}}{\fibrecoordinate{b}}
  \delby{\coordinate{k}}{\elementcoordinate{t}}\delby{\elementcoordinate{t}}{\fibrecoordinate{c}}
  \delby{\coordinate{l}}{\elementcoordinate{u}}\delby{\elementcoordinate{u}}{\fibrecoordinate{d}}A^{abcd}
  \label{eqn:Rank40FibreToElementToCoordinateTransformation}
\end{equation}
for rank(4,0) tensors, and,
\begin{equation}
  B_{ijkl}=\delby{\fibrecoordinate{a}}{\elementcoordinate{r}}\delby{\elementcoordinate{r}}{\coordinate{i}}
  \delby{\fibrecoordinate{b}}{\elementcoordinate{s}}\delby{\elementcoordinate{s}}{\coordinate{j}}
  \delby{\fibrecoordinate{c}}{\elementcoordinate{t}}\delby{\elementcoordinate{t}}{\coordinate{k}}
  \delby{\fibrecoordinate{d}}{\elementcoordinate{u}}\delby{\elementcoordinate{u}}{\coordinate{l}}A_{abcd}
  \label{eqn:Rank04FibreToElementToCoordinateTransformation}
\end{equation}
for rank(0,4) tensors, and,
\begin{equation}
  B^{ij..}_{..kl}=\delby{\coordinate{i}}{\elementcoordinate{r}}\delby{\elementcoordinate{r}}{\fibrecoordinate{a}}
  \delby{\coordinate{j}}{\elementcoordinate{s}}\delby{\elementcoordinate{s}}{\fibrecoordinate{b}}
  \delby{\fibrecoordinate{c}}{\elementcoordinate{t}}\delby{\elementcoordinate{t}}{\coordinate{k}}
  \delby{\fibrecoordinate{d}}{\elementcoordinate{u}}\delby{\elementcoordinate{u}}{\coordinate{l}}A^{ab..}_{..cd}
  \label{eqn:Rank22FibreToElementToCoordinateTransformation1}
\end{equation}
and
\begin{equation}
  B^{..kl}_{ij..}=\delby{\fibrecoordinate{a}}{\elementcoordinate{r}}\delby{\elementcoordinate{r}}{\coordinate{i}}
  \delby{\fibrecoordinate{b}}{\elementcoordinate{s}}\delby{\elementcoordinate{s}}{\coordinate{j}}
  \delby{\coordinate{k}}{\elementcoordinate{t}}\delby{\elementcoordinate{t}}{\fibrecoordinate{c}}
  \delby{\coordinate{l}}{\elementcoordinate{u}}\delby{\elementcoordinate{u}}{\fibrecoordinate{d}}A^{..cd}_{ab..}
  \label{eqn:Rank22FibreToElementToCoordinateTransformation2}
\end{equation}
\etc, for rank(2,2) tensors.

Using \eqnrefs{eqn:CoordinateToFibreTransformationViaElement}{eqn:FibreToCoordinateTransformationViaElement}
the transformations are
\begin{equation}
  B^{ijkl}=\delby{\coordinate{i}}{\fibrecoordinate{a}}\delby{\coordinate{j}}{\fibrecoordinate{b}}
  \delby{\coordinate{k}}{\fibrecoordinate{c}}\delby{\coordinate{l}}{\fibrecoordinate{d}}A^{abcd}
  \label{eqn:Rank40FibreToCoordinateTransformation}
\end{equation}
for rank(4,0) tensors, and,
\begin{equation}
  B_{ijkl}=\delby{\fibrecoordinate{a}}{\coordinate{i}}\delby{\fibrecoordinate{b}}{\coordinate{j}}
  \delby{\fibrecoordinate{c}}{\coordinate{k}}\delby{\fibrecoordinate{d}}{\coordinate{l}}A_{abcd}
  \label{eqn:Rank04FibreToCoordinateTransformation}
\end{equation}
for rank(0,4) tensors, and,
\begin{equation}
  B^{ij..}_{..kl}=\delby{\coordinate{i}}{\fibrecoordinate{a}}\delby{\coordinate{j}}{\fibrecoordinate{b}}
  \delby{\fibrecoordinate{c}}{\coordinate{k}}\delby{\fibrecoordinate{d}}{\coordinate{l}}A^{ab..}_{..cd}
  \label{eqn:Rank22FibreToCoordianteTransformation1}
\end{equation}
and
\begin{equation}
  B^{..kl}_{ij..}=\delby{\fibrecoordinate{a}}{\coordinate{i}}\delby{\fibrecoordinate{b}}{\coordinate{j}}
  \delby{\coordinate{k}}{\fibrecoordinate{c}}\delby{\coordinate{l}}{\fibrecoordinate{d}}A^{..cd}_{ab..}
  \label{eqn:Rank22FibreToCoordinateTransformation2}
\end{equation}
\etc, for rank(2,2) tensors.

For fourth order geometric tensors, $\tensorfour{B}$, to fibre
tensors, $\tensorfour{A}$, the transformations are
\begin{equation}
  A^{abcd}=\delby{\fibrecoordinate{a}}{\elementcoordinate{r}}\delby{\elementcoordinate{r}}{\coordinate{i}}
  \delby{\fibrecoordinate{b}}{\elementcoordinate{s}}\delby{\elementcoordinate{s}}{\coordinate{j}}
  \delby{\fibrecoordinate{c}}{\elementcoordinate{t}}\delby{\elementcoordinate{t}}{\coordinate{k}}
  \delby{\fibrecoordinate{d}}{\elementcoordinate{u}}\delby{\elementcoordinate{u}}{\coordinate{l}}B^{ijkl}
  \label{eqn:Rank40CoordinateToElementToFibreTransformation}
\end{equation}
for rank(4,0) tensors, and,
\begin{equation}
  A_{abcd}=\delby{\coordinate{i}}{\elementcoordinate{r}}\delby{\elementcoordinate{r}}{\fibrecoordinate{a}}
  \delby{\coordinate{j}}{\elementcoordinate{s}}\delby{\elementcoordinate{s}}{\fibrecoordinate{b}}
  \delby{\coordinate{k}}{\elementcoordinate{t}}\delby{\elementcoordinate{t}}{\fibrecoordinate{c}}
  \delby{\coordinate{l}}{\elementcoordinate{u}}\delby{\elementcoordinate{u}}{\fibrecoordinate{d}}B_{ijkl}
  \label{eqn:Rank04CoordinateToElementToFibreTransformation}
\end{equation}
for rank(0,4) tensors, and,
\begin{equation}
  A^{ab..}_{..cd}=\delby{\fibrecoordinate{a}}{\elementcoordinate{r}}\delby{\elementcoordinate{r}}{\coordinate{i}}
  \delby{\fibrecoordinate{b}}{\elementcoordinate{s}}\delby{\elementcoordinate{s}}{\coordinate{j}}
  \delby{\coordinate{k}}{\elementcoordinate{t}}\delby{\elementcoordinate{t}}{\fibrecoordinate{c}}
  \delby{\coordinate{l}}{\elementcoordinate{u}}\delby{\elementcoordinate{u}}{\fibrecoordinate{d}}B^{ij..}_{..kl}
  \label{eqn:Rank22CoordinateToElementToFibreTransformation1}
\end{equation}
and
\begin{equation}
  A^{..cd}_{ab..}=\delby{\coordinate{i}}{\elementcoordinate{r}}\delby{\elementcoordinate{r}}{\fibrecoordinate{a}}
  \delby{\coordinate{j}}{\elementcoordinate{s}}\delby{\elementcoordinate{s}}{\fibrecoordinate{b}}
  \delby{\fibrecoordinate{c}}{\elementcoordinate{t}}\delby{\elementcoordinate{t}}{\coordinate{k}}
  \delby{\fibrecoordinate{d}}{\elementcoordinate{u}}\delby{\elementcoordinate{u}}{\coordinate{l}}B^{..kl}_{ij..}
  \label{eqn:Rank22CoordinateToElementToFibreTransformation2}
\end{equation}
\etc, for rank(2,2) tensors.


Using \eqnrefs{eqn:CoordinateToFibreTransformationViaElement}{eqn:FibreToCoordinateTransformationViaElement}
the inverse transformations are
\begin{equation}
  A^{abcd}=\delby{\fibrecoordinate{a}}{\coordinate{i}}\delby{\fibrecoordinate{b}}{\coordinate{j}}
  \delby{\fibrecoordinate{c}}{\coordinate{k}}\delby{\fibrecoordinate{d}}{\coordinate{l}}B^{ijkl}
  \label{eqn:Rank40CoordinateToFibreTransformation}
\end{equation}
for rank(4,0) tensors, and,
\begin{equation}
  A_{abcd}=\delby{\coordinate{i}}{\fibrecoordinate{a}}\delby{\coordinate{j}}{\fibrecoordinate{b}}
  \delby{\coordinate{k}}{\fibrecoordinate{c}}\delby{\coordinate{l}}{\fibrecoordinate{d}}B_{ijkl}
  \label{eqn:Rank04CoordinateToFibreTransformation}
\end{equation}
for rank(0,4) tensors, and,
\begin{equation}
  A^{ab..}_{..cd}=\delby{\fibrecoordinate{a}}{\coordinate{i}}\delby{\fibrecoordinate{b}}{\coordinate{j}}
  \delby{\coordinate{k}}{\fibrecoordinate{c}}\delby{\coordinate{l}}{\fibrecoordinate{d}}B^{ij..}_{..kl}
  \label{eqn:Rank22CoordinateToFibreTransformation1}
\end{equation}
and
\begin{equation}
  A^{..cd}_{ab..}=\delby{\coordinate{i}}{\fibrecoordinate{a}}\delby{\coordinate{j}}{\fibrecoordinate{b}}
  \delby{\fibrecoordinate{c}}{\coordinate{k}}\delby{\fibrecoordinate{d}}{\coordinate{l}}B^{..kl}_{ij..}
  \label{eqn:Rank22CoordinateToFibreTransformation2}
\end{equation}
\etc, for rank(2,2) tensors.

Here,
\eqnrefsfour{eqn:GeometricToElementTransformationTwoD}{eqn:ElementToGeometricTransformationTwoD}{eqn:ElementToFibreTransformationTwoD}{eqn:FibreToElementTransformationTwoD}
define the transformations in two-dimensions, and
\eqnrefsfour{eqn:GeometricToElementTransformationThreeD}{eqn:ElementToGeometricTransformationThreeD}{eqn:ElementToFibreTransformationThreeD}{eqn:FibreToElementTransformationThreeD}
define the transformations in three-dimensions. The transformations in
one-dimension are trivial.

