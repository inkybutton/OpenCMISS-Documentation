%
% opencmiss_symbols.tex
%
% Global OpenCMISS symbol definitions i.e., assigning the symbol used to a latexname for a physical quantities
% (i.e., non-mathematical symbols).
%

%
% General
%

\newcommand{\gravitysymbol}{\ensuremath{ %
    a %
}} % gravity symbol e.g., \gravitysymbol => a
\newcommand{\gravityvector}{\ensuremath{ %
    \vectr{\gravitysymbol} %
}} % gravity vector e.g., \gravityvector => a
\newcommand{\densitysymbol}{\ensuremath{ %
    \rho %
}} % density symbol e.g., \densitysymbol => \rho
\newcommand{\voltagesymbol}{\ensuremath{ %
    V %
}} % voltage symbol e.g., \voltagesymbol => V
\newcommand{\currentsymbol}{\ensuremath{ %
    I %
}} % current symbol e.g., \currentsymbol => I
\newcommand{\currentdensitysymbol}{\ensuremath{ %
    J %
}} % current density symbol e.g., \currentdensitysymbol => J
\newcommand{\currentdensityvector}{\ensuremath{ %
    \vectr{J} %
}} % current density vector e.g., \currentdensityvector => J
\newcommand{\potentialsymbol}{\ensuremath{ %
    \phi %
}} % potential symbol e.g., \potentialsymbol => V
\newcommand{\conductivitytensorsymbol}{\ensuremath{ %
    \sigma %
}} % conductivity tensor symbol e.g., \conductivitytensorsymbol => \sigma
\newcommand{\conductivitytensor}{\ensuremath{ %
    \tensortwo{\conductivitytensorsymbol} %
}} % conductivity tensor e.g., \conductivitytensor => \sigma
\newcommand{\diffusivitytensorsymbol}{\ensuremath{ %
    \sigma %
}} % diffusivity tensor symbol e.g., \diffusivitytensorsymbol => \sigma
\newcommand{\diffusivitytensor}{\ensuremath{ %
    \tensortwo{\diffusivitytensorsymbol} %
}} % diffusivity tensor e.g., \diffusivitytensor => \sigma


%
% Interpolation
%

\newcommand{\xione}{\ensuremath{\xi^{1}}\xspace} % xi 1
\newcommand{\xionesq}{\ensuremath{{\xi^{1}}^{2}}\xspace} % (xi 1)^2
\newcommand{\xionecube}{\ensuremath{{\xi^{1}}^{3}}\xspace} % (xi 1)^3
\newcommand{\xitwo}{\ensuremath{\xi^{2}}\xspace} % xi 2
\newcommand{\xitwosq}{\ensuremath{{\xi^{2}}^{2}}\xspace} % (xi 2)^2
\newcommand{\xitwocube}{\ensuremath{{\xi^{2}}^{3}}\xspace} % (xi 2)^3
\newcommand{\xithree}{\ensuremath{\xi^{3}}\xspace} % xi 3
\newcommand{\xithreesq}{\ensuremath{{\xi^{3}}^{2}}\xspace} % (xi 3)^2
\newcommand{\xithreecube}{\ensuremath{{\xi^{3}}^{3}}\xspace} % (xi 3)^3

\newcommand{\Xione}{\ensuremath{\Xi^{1}}\xspace} % Xi 1
\newcommand{\Xionesq}{\ensuremath{{\Xi^{1}}^{2}}\xspace} % (Xi 1)^2
\newcommand{\Xionecube}{\ensuremath{{\Xi^{1}}^{3}}\xspace} % (Xi 1)^3
\newcommand{\Xitwo}{\ensuremath{\Xi^{2}}\xspace} % Xi 2
\newcommand{\Xitwosq}{\ensuremath{{\Xi^{2}}^{2}}\xspace} % (Xi 2)^2
\newcommand{\Xitwocube}{\ensuremath{{\Xi^{2}}^{3}}\xspace} % (Xi 2)^3
\newcommand{\Xithree}{\ensuremath{\Xi^{3}}\xspace} % xi 3
\newcommand{\Xithreesq}{\ensuremath{{\Xi^{3}}^{2}}\xspace} % (Xi 3)^2
\newcommand{\Xithreecube}{\ensuremath{{\Xi^{3}}^{3}}\xspace} % (Xi 3)^3

\newcommand{\zetaone}{\ensuremath{\zeta^{1}}\xspace} % zeta 1
\newcommand{\zetaonesq}{\ensuremath{{\zeta^{1}}^{2}}\xspace} % (zeta 1)^2
\newcommand{\zetaonecube}{\ensuremath{{\zeta^{1}}^{3}}\xspace} % (zeta 1)^3
\newcommand{\zetatwo}{\ensuremath{\zeta^{2}}\xspace} % zeta 2
\newcommand{\zetatwosq}{\ensuremath{{zeta^{2}}^{2}}\xspace} % (zeta 2)^2
\newcommand{\zetatwocube}{\ensuremath{{\zeta^{2}}^{3}}\xspace} % (zeta 2)^3

\newcommand{\Zetaone}{\ensuremath{\Zeta^{1}}\xspace} % Zeta 1
\newcommand{\Zetaonesq}{\ensuremath{{\Zeta^{1}}^{2}}\xspace} % (Zeta 1)^2
\newcommand{\Zetaonecube}{\ensuremath{{\Zeta^{1}}^{3}}\xspace} % (Zeta 1)^3
\newcommand{\Zetatwo}{\ensuremath{\Zeta^{2}}\xspace} % Zeta 2
\newcommand{\Zetatwosq}{\ensuremath{{\Zeta^{2}}^{2}}\xspace} % (Zeta 2)^2
\newcommand{\Zetatwocube}{\ensuremath{{\Zeta^{2}}^{3}}\xspace} % (Zeta 2)^3

%
%
% Coordinate systems etc. (fibre, elemental, world etc.)
%
\newcommand{\coordinatesymbol}{\ensuremath{ %
    x %
}} % coordinate symbol e.g., \coordinatesymbol => x
\newcommand{\coordinatevector}{\ensuremath{ %
    \vectr{\coordinatesymbol} %
}} % coordinate vectr e.g., \coordinatevector => x
\newcommand{\coordinate}[1]{\ensuremath{ %
    \ifthenelse{\equal{#1}{}}{%
      \coordinatesymbol%
    }{
      \coordinatesymbol^{#1}%
}}} % coordinate e.g., \coordinate{i} => x^{i}
\newcommand{\coordone}{\ensuremath{\coordinatesymbol^{1}}\xspace} % x^1
\newcommand{\coordonesq}{\ensuremath{{\coordinatesymbol^{1}}^{2}}\xspace} % (x 1)^2
\newcommand{\coordonecube}{\ensuremath{{\coordinatesymbol^{1}}^{3}}\xspace} % (x 1)^3
\newcommand{\coordtwo}{\ensuremath{\coordinatesymbol^{2}}\xspace} % x 2
\newcommand{\coordtwosq}{\ensuremath{{\coordinatesymbol^{2}}^{2}}\xspace} % (x 2)^2
\newcommand{\coordtwocube}{\ensuremath{{\coordinatesymbol^{2}}^{3}}\xspace} % (x 2)^3
\newcommand{\coordthree}{\ensuremath{\coordinatesymbol^{3}}\xspace} % x 3
\newcommand{\coordthreesq}{\ensuremath{{\coordinatesymbol^{3}}^{2}}\xspace} % (x 3)^2
\newcommand{\coordthreecube}{\ensuremath{{\coordinatesymbol^{3}}^{3}}\xspace} % (x 3)^3

\newcommand{\xicoordinatesymbol}{\ensuremath{ %
    \xi %
}} % xi coordinate symbol e.g., \xicoordinatesymbol => \xi
\newcommand{\xicoordinatevector}{\ensuremath{ %
    \vectr{\xicoordinatesymbol} %
}} % xi coordinate vector e.g., \xicoordinatevector => \xi
\newcommand{\elementcoordinatesymbol}{\ensuremath{ %
    \xi %
}} % element coordinate symbol e.g., \elementcoordymbol => \xi
\newcommand{\elementcoordinatevector}{\ensuremath{ %
    \vectr{\elementcoordinatesymbol} %
}} % element coordinate vector e.g., \elementcoordvector => \xi
\newcommand{\elementcoordinate}[1]{\ensuremath{ %
    \ifthenelse{\equal{#1}{}}{%
      \elementcoordinatesymbol%
    }{
      \elementcoordinatesymbol^{#1}%
}}} % element coordinate e.g., \elementcoordinate{i} => \xi^{i}

\newcommand{\elemcoordone}{\ensuremath{\elementcoordinatesymbol^{1}}\xspace} % xi 1
\newcommand{\elemcoordonesq}{\ensuremath{{\elementcoordinatesymbol^{1}}^{2}}\xspace} % (xi 1)^2
\newcommand{\elemcoordonecube}{\ensuremath{{\elementcoordinatesymbol^{1}}^{3}}\xspace} % (xi 1)^3
\newcommand{\elemcoordtwo}{\ensuremath{\elementcoordinatesymbol^{2}}\xspace} % xi 2
\newcommand{\elemcoordtwosq}{\ensuremath{{\elementcoordinatesymbol^{2}}^{2}}\xspace} % (xi 2)^2
\newcommand{\elemcoordtwocube}{\ensuremath{{\elementcoordinatesymbol^{2}}^{3}}\xspace} % (xi 2)^3
\newcommand{\elemcoordthree}{\ensuremath{\elementcoordinatesymbol^{3}}\xspace} % xi 3
\newcommand{\elemcoordthreesq}{\ensuremath{{\elementcoordinatesymbol^{3}}^{2}}\xspace} % (xi 3)^2
\newcommand{\elemcoordthreecube}{\ensuremath{{\elementcoordinatesymbol^{3}}^{3}}\xspace} % (xi 3)^3
\newcommand{\boundarycoordinatesymbol}{\ensuremath{ %
    \zeta %
}} % boundary (element) coordinate symbol e.g., \boundarycoordinateymbol => \zeta
\newcommand{\boundarycoordinatevector}{\ensuremath{ %
    \vectr{\boundarycoordinatesymbol} %
}} % boundary (element) coordinate vector e.g., \boundarycoordinatevector => \zeta
\newcommand{\boundarycoordinate}[1]{\ensuremath{ %
    \ifthenelse{\equal{#1}{}}{%
      \boundarycoordinatesymbol%
    }{
      \boundarycoordinatesymbol^{#1}%
}}} % boundary (element) coordinate e.g., \boundarycoordinate{i} => \zeta^{i}
\newcommand{\boundarycoordone}{\ensuremath{\boundarycoordinatesymbol^{1}}\xspace} % zeta 1
\newcommand{\boundarycoordonesq}{\ensuremath{{\boundarycoordinatesymbol^{1}}^{2}}\xspace} % (zeta 1)^2
\newcommand{\boundarycoordonecube}{\ensuremath{{\boundarycoordinatesymbol^{1}}^{3}}\xspace} % (zeta 1)^3
\newcommand{\boundarycoordtwo}{\ensuremath{\boundarycoordinatesymbol^{2}}\xspace} % zeta 2
\newcommand{\boundarycoordtwosq}{\ensuremath{{zeta^{2}}^{2}}\xspace} % (zeta 2)^2
\newcommand{\boundarycoordtwocube}{\ensuremath{{\boundarycoordinatesymbol^{2}}^{3}}\xspace} % (zeta 2)^3
\newcommand{\arclengthcoordinatesymbol}{\ensuremath{ %
    s%
}} % arc-length coordinate symbol e.g., \arclengthcoordinateymbol => s
\newcommand{\arclengthcoordinatevector}{\ensuremath{ %
    \vectr{\arclengthcoordinatesymbol} %
}} % arc-length coordinate vector e.g., \arclengthcoordinatevector => s
\newcommand{\arclengthcoordinate}[1]{\ensuremath{ %
    \ifthenelse{\equal{#1}{}}{%
      \arclengthcoordinatesymbol%
    }{
      \arclengthcoordinatesymbol_{#1}%
}}} % arclength coordinate e.g., \arclengthcoordinate{i} => s_{i}
\newcommand{\barycoordinatesymbol}{\ensuremath{ %
    \lambda%
}} % barycentric coordinate symbol e.g., \barycoordinateymbol => \lambda
\newcommand{\barycoordinatevector}{\ensuremath{ %
    \vectr{\barycoordinatesymbol} %
}} % barycentric coordinate vector e.g., \barycoordinatevector => \lambda
\newcommand{\barycoordinate}[1]{\ensuremath{ %
    \ifthenelse{\equal{#1}{}}{%
      \barycoordinatesymbol%
    }{
      \barycoordinatesymbol^{#1}%
}}} % barycentric coordinate e.g., \barycoordinate{i} => \lambda^{i}
\newcommand{\barycoordone}{\ensuremath{\barycoordinatesymbol^{1}}\xspace} % lambda 1
\newcommand{\barycoordonesq}{\ensuremath{{\barycoordinatesymbol^{1}}^{2}}\xspace} % (lambda 1)^2
\newcommand{\barycoordonecube}{\ensuremath{{\barycoordinatesymbol^{1}}^{3}}\xspace} % (lambda 1)^3
\newcommand{\barycoordtwo}{\ensuremath{\barycoordinatesymbol^{2}}\xspace} % lambda 2
\newcommand{\barycoordtwosq}{\ensuremath{{\barycoordinatesymbol^{2}}^{2}}\xspace} % (lambda 2)^2
\newcommand{\barycoordtwocube}{\ensuremath{{\barycoordinatesymbol^{2}}^{3}}\xspace} % (lambda 2)^3
\newcommand{\barycoordthree}{\ensuremath{\barycoordinatesymbol^{3}}\xspace} % lambda 3
\newcommand{\barycoordthreesq}{\ensuremath{{\barycoordinatesymbol^{3}}^{2}}\xspace} % (lambda 3)^2
\newcommand{\barycoordthreecube}{\ensuremath{{\barycoordinatesymbol^{3}}^{3}}\xspace} % (lambda 3)^3
\newcommand{\barycoordfour}{\ensuremath{\barycoordinatesymbol^{4}}\xspace} % lambda 4
\newcommand{\barycoordfoursq}{\ensuremath{{\barycoordinatesymbol^{4}}^{2}}\xspace} % (lambda 4)^2
\newcommand{\barycoordfourcube}{\ensuremath{{\barycoordinatesymbol^{4}}^{3}}\xspace} % (lambda 4)^3
\newcommand{\fibrecoordinatesymbol}{\ensuremath{ %
    \nu %
}} % Fibre coordinate symbol e.g., \fibrecoordinatesymbol => \nu
\newcommand{\fibrecoordinatevector}{\ensuremath{ %
    \vectr{\fibrecoordinatesymbol} %
}} % Fibre coordinate vector e.g., \fibrecoordinatevector => \nu
\newcommand{\fibrecoordinate}[1]{\ensuremath{ %
    \ifthenelse{\equal{#1}{}}{%
      \fibrecoordinatesymbol%
    }{
      \fibrecoordinatesymbol^{#1}%
}}} % fibre coordinate e.g., \fibrecoordinate{i} => \nu^{i}
\newcommand{\spatialcoordinatesymbol}{\ensuremath{ %
    \coordinatesymbol %
}} % spatial coordinate symbol e.g., \spatialcoordinatesymbol => x
\newcommand{\spatialcoordinatevector}{\ensuremath{ %
    \vectr{\spatialcoordinatesymbol} %
}} % spatial coordinate vector e.g., \spatialcoordinatevector => x
\newcommand{\spatialcoordinate}[1]{\ensuremath{ %
    \ifthenelse{\equal{#1}{}}{%
      \spatialcoordinatesymbol%
    }{
      \spatialcoordinatesymbol^{#1}%
}}} % spatial coordinate e.g., \spatialcoordinate{i} => x^{i}
\newcommand{\materialcoordinatesymbol}{\ensuremath{ %
    X %
}} % material coordinate symbol e.g., \materialcoordinatesymbol => X
\newcommand{\materialcoordinatevector}{\ensuremath{ %
    \vectr{\materialcoordinatesymbol} %
}} % material coordinate vector e.g., \materialcoordinatevector => X
\newcommand{\materialcoordinate}[1]{\ensuremath{ %
    \ifthenelse{\equal{#1}{}}{%
      \materialcoordinatesymbol%
    }{
      \materialcoordinatesymbol^{#1}%
}}} % material coordinate e.g., \materialcoordinate{i} => x^{i}
\newcommand{\spatialxicoordinatesymbol}{\ensuremath{ %
    \xicoordinatesymbol %
}} % Spatial xi coordinate symbol e.g., \spatialxicoordinatesymbol => \xi
\newcommand{\spatialxicoordinatevector}{\ensuremath{ %
    \vectr{\spatialxicoordinatesymbol} %
}} % Spatial xi coordinate vector e.g., \spatialxicoordinatevector => \xi
\newcommand{\spatialelemcoordinatesymbol}{\ensuremath{ %
    \elementcoordinatesymbol %
}} % Spatial element coordinate symbol e.g., \spatialelemcoordinatesymbol => \xi
\newcommand{\spatialelemcoordinatevector}{\ensuremath{ %
    \vectr{\spatialelemcoordinatesymbol} %
}} % Spatial elem coordinate vector e.g., \spatialelemcoordinatevector => \xi
\newcommand{\spatialelemcoordinate}[1]{\ensuremath{ %
    \ifthenelse{\equal{#1}{}}{%
      \spatialelemcoordinatesymbol%
    }{
      \spatialelemcoordinatesymbol^{#1}%
}}} % spatial element coordinate e.g., \spatialelemcoordinate{i} => \xi^{i}
\newcommand{\materialxicoordinatesymbol}{\ensuremath{ %
    \Xi %
}} % Material xi coordinate symbol e.g., \materialxicoordinatesymbol => \Xi
\newcommand{\materialxicoordinatevector}{\ensuremath{ %
    \vectr{\materialxicoordinatesymbol} %
}} % Material xi coordinate vector e.g., \materialxicoordinatevector => \Xi
\newcommand{\materialelemcoordinate}[1]{\ensuremath{ %
    \ifthenelse{\equal{#1}{}}{%
      \materialelemcoordinatesymbol%
    }{
      \materialelemcoordinatesymbol^{#1}%
}}} % material element coordinate e.g., \materialelemcoordinate{i} => \Xi^{i}
\newcommand{\spatialfibrecoordinatesymbol}{\ensuremath{ %
    \fibrecoordinatesymbol %
}} % Spatial fibre coordinate symbol e.g., \spatialfibrecoordinatesymbol => \nu
\newcommand{\spatialfibrecoordinatevector}{\ensuremath{ %
    \vectr{\spatialfibrecoordinatesymbol} %
}} % Spatial fibre coordinate vector e.g., \spatialfibrecoordinatevector => \nu
\newcommand{\spatialfibrecoordinate}[1]{\ensuremath{ %
    \ifthenelse{\equal{#1}{}}{%
      \spatialfibrecoordinatesymbol%
    }{
      \spatialfibrecoordinatesymbol^{#1}%
}}} % spatial fibre coordinate e.g., \spatialfibrecoordinate{i} => \nu^{i}
\newcommand{\materialfibrecoordinatesymbol}{\ensuremath{ %
    N %
}} % Material fibre coordinate symbol e.g., \materialfibrecoordinatesymbol => N
\newcommand{\materialfibrecoordinatevector}{\ensuremath{ %
    \vectr{\materialfibrecoordinatesymbol} %
}} % Material fibre coordinate vector e.g., \materialfibrecoordinatevector => N
\newcommand{\materialfibrecoordinate}[1]{\ensuremath{ %
    \ifthenelse{\equal{#1}{}}{%
      \materialfibrecoordinatesymbol%
    }{
      \materialfibrecoordinatesymbol^{#1}%
}}} % material fibre coordinate e.g., \materialfibrecoordinate{i} => \Nu^{i}
\newcommand{\spatialfibrestructuretensorsymbol}{\ensuremath{ %
    s %
}} % Spatial fibre structure tensor symbol e.g., \spatialfibrestructuretensorsymbol => s
\newcommand{\spatialfibrestructuretensor}{\ensuremath{ %
    \tensortwo{\spatialfibrestructuretensorsymbol} %
}} % Spatial fibre structure tensor e.g., \spatialfibrestructuretensor => s

\newcommand{\materialstructuraltensorsymbol}{\ensuremath{ %
    M %
}} % Material structural tensor symbol e.g., \materialtructuretensorsymbol => H
\newcommand{\materialstructuraltensor}{\ensuremath{ %
    \tensortwo{\materialstructuraltensorsymbol} %
}} % Material structural tensor e.g., \materiastructuraltensor => M
\newcommand{\materialstructuraltensorcomponent}[1]{\ensuremath{ %
    \tensortwo{\materialstructuraltensorsymbol}_{#1} %
}} % Material structural tensor component e.g., \materialstructuraltensorcomponent{1} => M_1

\newcommand{\spatialstructuraltensorsymbol}{\ensuremath{ %
    m %
}} % Spatial structural tensor symbol e.g., \spatialtructuretensorsymbol => m
\newcommand{\spatialstructuraltensor}{\ensuremath{ %
    \tensortwo{\spatialstructuraltensorsymbol} %
}} %  Spatial structural tensor e.g., \spatialstructuraltensor => m
\newcommand{\spatialstructuraltensorcomponent}[1]{\ensuremath{ %
    \tensortwo{\spatialstructuraltensorsymbol}_{#1} %
}} % Spatial structural tensor component e.g., \spatialstructuraltensorcomponent{1} => m_1


%
% Solid Mechanics
% 

% Constants

\newcommand{\firstlamesymbol}{\ensuremath{ %
    \lambda%
}} % First Lame constant symbol e.g., \firstlamesymbol => \lambda
\newcommand{\secondlamesymbol}{\ensuremath{ %
    \mu%
}} % Second Lame constant symbol e.g., \secondlamesymbol => \mu
\newcommand{\poissonsratiosymbol}{\ensuremath{ %
    \nu%
}} % Poisson's ratio symbol e.g., \poissonsratiosymbol => \nu
\newcommand{\youngsmodulussymbol}{\ensuremath{ %
    E%
}} % Young's modulus symbol e.g., \youngsmodulussymbol => E
\newcommand{\shearmodulussymbol}{\ensuremath{ %
    G%
}} % Shear modulus symbol e.g., \shearmodulussymbol => G
\newcommand{\bulkstrainmodulussymbol}{\ensuremath{ %
    K %
}} % Bulk strain modulus symbol e.g., \bulkstrainmodulussymbol => K
\newcommand{\soliddensitysymbol}{\ensuremath{ %
    \densitysymbol_{s} %
}} % Solid density symbol e.g., \soliddensitysymbol => \rho_s


% Deformation and strain

\newcommand{\deformationgradienttensorsymbol}{\ensuremath{ %
    F %
}} % Deformation gradient tensor symbol e.g., \deformationgradienttensorsymbol => F
\newcommand{\deformationgradienttensor}{\ensuremath{ %
    \tensortwo{\deformationgradienttensorsymbol} %
}} % Deformation gradient tensor e.g., \deformationgradienttensor => F
\newcommand{\rightstretchtensorsymbol}{\ensuremath{ %
    U %
}} % Right stretch tensor symbol e.g., \rightstretchtensorsymbol => U
\newcommand{\rightstretchtensor}{\ensuremath{ %
    \tensortwo{\rightstretchtensorsymbol} %
}} % Right stretch tensor e.g., \rightstretchtensor => U
\newcommand{\leftstretchtensorsymbol}{\ensuremath{ %
    V %
}} % Left stretch tensor symbol e.g., \leftstretchtensorsymbol => V
\newcommand{\leftstretchtensor}{\ensuremath{ %
    \tensortwo{\leftstretchtensorsymbol} %
}} % Left stretch tensor e.g., \leftstretchtensor => V
\newcommand{\rightcauchygreentensorsymbol}{\ensuremath{ %
    C %
}} % Right Cauchy-Green (or Green) deformation tensor symbol e.g., \rightcauchygreentensorsymbol => C
\newcommand{\rightcauchygreentensor}{\ensuremath{ %
    \tensortwo{\rightcauchygreentensorsymbol} %
}} % Right Cauchy-Green (or Green) deformation tensor e.g., \rightcauchygreentensor => C
\newcommand{\pioladeformationtensorsymbol}{\ensuremath{ %
    B %
}} % Piola deformation tensor symbol e.g., \pioladefomationtensorsymbol => B
\newcommand{\pioladeformationtensor}{\ensuremath{ %
    \tensortwo{\pioladeformationtensorsymbol} %
}} % Piola deformation tensor e.g., \pioladefomationtensor => B
\newcommand{\greenlagrangestraintensorsymbol}{\ensuremath{ %
    E %
}} % Green-Lagrange strain tensor symbol e.g., \greenlagrangestraintensorsymbol => E
\newcommand{\greenlagrangestraintensor}{\ensuremath{ %
    \tensortwo{\greenlagrangestraintensorsymbol} %
}} % Green-Lagrange strain tensor e.g., \greenlagrangestraintensor => B
\newcommand{\leftcauchygreentensorsymbol}{\ensuremath{ %
    b %
}} % Left Cauchy-Green (or Finger) deformation tensor symbol e.g., \leftcauchygreentensorsymbol => b
\newcommand{\leftcauchygreentensor}{\ensuremath{ %
    \tensortwo{\leftcauchygreentensorsymbol} %
}} % Left Cauchy-Green (or Finger) deformation tensor e.g., \leftcauchygreentensor => b
\newcommand{\cauchydeformationtensorsymbol}{\ensuremath{ %
    c %
}} % Cauchy deformation tensor symbol e.g., \cauchydeformationtensorsymbol => c
\newcommand{\cauchydeformationtensor}{\ensuremath{ %
    \tensortwo{\cauchydeformationtensorsymbol} %
}} % Cauchy deformation tensor e.g., \cauchydeformationtensor => c
\newcommand{\euleralmansistraintensorsymbol}{\ensuremath{ %
    e %
}} % Euler-Almansi strain tensor symbol e.g., \euleralmansistraintensorsymbol => e
\newcommand{\euleralmansistraintensor}{\ensuremath{ %
    \tensortwo{\euleralmansistraintensorsymbol} %
}} % Euler-Almansi strain tensor e.g., \euleralmansistraintensor => e

% Stress

\newcommand{\firstpiolakirchofftensorsymbol}{\ensuremath{ %
    P %
}} % First Piola-Kirchoff stress tensor symbol e.g., \firstpiolakirchoffstresstensorsymbol => P
\newcommand{\firstpiolakirchoffstresstensor}{\ensuremath{ %
    \tensortwo{\firstpiolakirchofftensorsymbol} %
}} % First Piola-Kirchoss stress tensor e.g., \firstpiolakirchoffstresstensor => P
\newcommand{\secondpiolakirchofftensorsymbol}{\ensuremath{ %
    S %
}} % Second Piola-Kirchoff stress tensor symbol e.g., \secondpiolakirchoffstresstensorsymbol => S
\newcommand{\secondpiolakirchoffstresstensor}{\ensuremath{ %
    \tensortwo{\secondpiolakirchofftensorsymbol} %
}} % Second Piola-Kirchoss stress tensor e.g., \secondpiolakirchoffstresstensor => S
\newcommand{\cauchystresstensorsymbol}{\ensuremath{ %
    \sigma %
}} % Cauchy stress tensor symbol e.g., \cauchystresstensorsymbol => \sigma
\newcommand{\cauchystresstensor}{\ensuremath{ %
    \tensortwo{\cauchystresstensorsymbol} %
}} % Cauchy stress tensor e.g., \cauchystresstensor => \sigma
\newcommand{\kirchoffstresstensorsymbol}{\ensuremath{ %
    \tau %
}} % Kirchoff stress tensor symbol e.g., \kirchoffstresstensorsymbol => \tau
\newcommand{\kirchoffstresstensor}{\ensuremath{ %
    \tensortwo{\kirchoffstresstensorsymbol} %
}} % Kirchoff stress tensor e.g., \kirchoffstresstensor => \tau

% Elasticity

\newcommand{\materialfirstelasticitytensorsymbol}{\ensuremath{ %
    A %
}} % Material first elasticity tensor symbol e.g., \materialfirstelasticitytensorsymbol => A
\newcommand{\materialfirstelasticitytensor}{\ensuremath{ %
    \tensorfour{\materialfirstelasticitytensorsymbol} %
}} % Material first elasticity tensor e.g., \materialfirstelasticitytensor => A
\newcommand{\materialsecondelasticitytensorsymbol}{\ensuremath{ %
    C %
}} % Material second elasticity tensor symbol e.g., \materialsecondelasticitytensorsymbol => C
\newcommand{\materialsecondelasticitytensor}{\ensuremath{ %
    \tensorfour{\materialsecondelasticitytensorsymbol} %
}} % Material second elasticity tensor e.g., \materialsecondelasticitytensor => C
\newcommand{\materialsecondcompliancetensorsymbol}{\ensuremath{ %
    S %
}} % Material second compliance tensor symbol e.g., \materialsecondcompliancetensorsymbol => S
\newcommand{\materialcompliancetensor}{\ensuremath{ %
    \tensorfour{\materialsecondcompliancetensorsymbol} %
}} % Material second compliance tensor e.g., \materialsecondcompilancetensor => S
\newcommand{\spatialfirstelasticitytensorsymbol}{\ensuremath{ %
    a %
}} % Spatial first elasticity tensor symbol e.g., \spatialfirstelasticitytensorsymbol => a
\newcommand{\spatialfirstelasticitytensor}{\ensuremath{ %
    \tensorfour{\spatialfirstelasticitytensorsymbol} %
}} % Spatial first elasticity tensor e.g., \spatialfirstelasticitytensor => a
\newcommand{\spatialsecondelasticitytensorsymbol}{\ensuremath{ %
    c %
}} % Spatial second elasticity tensor symbol e.g., \spatialsecondelasticitytensorsymbol => c
\newcommand{\spatialsecondelasticitytensor}{\ensuremath{ %
    \tensorfour{\spatialsecondelasticitytensorsymbol} %
}} % Spatial second elasticity tensor e.g., \spatialsecondelasticitytensor => c
\newcommand{\spatialsecondcomplancetensorsymbol}{\ensuremath{ %
    s %
}} % Spatial second compliance tensor symbol e.g., \spatialsecondcompliancetensorsymbol => s
\newcommand{\spatialsecondcompliancetensor}{\ensuremath{ %
    \tensorfour{\spatialsecondcompliancetensorsymbol} %
}} % Spatial second compliance tensor e.g., \spatialsecondcompliancetensor => s

%
% Linear elasticity
%

\newcommand{\smallstraintensorsymbol}{\ensuremath{ %
    \epsilon%
}} % small strain tensor symbol e.g., \smallstraintensorsymbol => \epsilon
\newcommand{\smallstraintensor}{\ensuremath{ %
    \tensortwo{\smallstraintensorsymbol} %
}} % small strain tensor e.g., \smallstraintensor => \epsilon
\newcommand{\smallrotationtensorsymbol}{\ensuremath{ %
    \omega%
}} % small rotation tensor symbol e.g., \smallstraintensorsymbol => \omega
\newcommand{\smallrotationtensor}{\ensuremath{ %
    \tensortwo{\smallrotationtensorsymbol} %
}} % small rotation tensor e.g., \smallrotationtensor => \omega
\newcommand{\shearstrainsymbol}{\ensuremath{ %
    \gamma%
}} % shear strain symbol e.g., \shearstrainsymbol => \gamma
\newcommand{\shearstresssymbol}{\ensuremath{ %
    \tau%
}} % shear stress symbol e.g., \shearstresssymbol => \tau
\newcommand{\linearstresstensorsymbol}{\ensuremath{ %
    \sigma%
}} % linear stress tensor symbol e.g., \linearstresstensorsymbol => \sigma
\newcommand{\linearstresstensor}{\ensuremath{ %
    \tensortwo{\linearstresstensorsymbol}%
}} % linear stress tensor e.g., \linearstresstensor => \sigma
\newcommand{\linearelasticitytensorsymbol}{\ensuremath{ %
    c%
}} % linear elasticity tensor symbol e.g., \linearelasticitytensorsymbol => c
\newcommand{\linearelasticitytensor}{\ensuremath{ %
    \tensorfour{\linearelasticitytensorsymbol}%
}} % linear elasticity tensor e.g., \linearelasticitytensor => c
\newcommand{\linearcompliancetensorsymbol}{\ensuremath{ %
    s%
}} % linear compliance tensor symbol e.g., \linearcompliancetensorsymbol => s
\newcommand{\linearcompliancetensor}{\ensuremath{ %
    \tensorfour{\linearcompliancetensorsymbol}%
}} % linear compliance tensor e.g., \linearcompliancetensor => s
\newcommand{\linearstrainenergysymbol}{\ensuremath{ %
    \varepsilon %
}} % linear strain energy symbol e.g., \linearstrainenergysymbol => \varepsilon

%
% Bioelectrics
%

\newcommand{\membraneareavolumeratio}{\ensuremath{ %
    A_{m} %
}} % membrane area to volume ratio e.g., \membraneareavolumeration => A_m 
\newcommand{\membranecapacitance}{\ensuremath{ %
    C_{m} %
}} % membrane capacitance e.g., \membranecapacitance => C_m
\newcommand{\transmembranevoltage}{\ensuremath{ %
    {\voltagesymbol}_{m} %
}} % transmembrane voltage e.g., \transmembranevoltage => V_m
\newcommand{\cellstatevariablesymbol}{\ensuremath{ %
    u %
}} % cell state variable symbol  e.g., \cellstatevariablesymbol => u
\newcommand{\cellstatevariablevector}{\ensuremath{ %
    \vectr{u} %
}} % cell state variable vector  e.g., \cellstatevariablevector => u
\newcommand{\transmembranecurrent}{\ensuremath{ %
    {\currentsymbol}_{m} %
}} % transmembrane current e.g., \transmembranecurrent => I_m
\newcommand{\ioniccurrent}{\ensuremath{ %
    {\currentsymbol}_{\text{ion}} %
}} % Ionic current e.g., \ioniccurrent => I_ion
\newcommand{\stimuluscurrent}{\ensuremath{ %
    {\currentsymbol}_{s} %
}} % stimulus current e.g., \stimulus current => I_s
\newcommand{\intracellularpotential}{\ensuremath{ %
    {\potentialsymbol}_{i} %
}} % intra-cellular potential e.g., \intracellularpotential => \phi_i
\newcommand{\intracellularstimuluscurrent}{\ensuremath{ %
    {\stimuluscurrent}_{i} %
}} % intra-cellular stimulus current e.g., \intracellularstimuluscurrent => {I_s}_i
\newcommand{\intracellularcurrentdensityvector}{\ensuremath{ %
    {\currentdensityvector}_{i} %
}} % intra-cellular current density vector e.g., \intracellularcurrentdensityvector => J_i
\newcommand{\intracellularconductivitytensor}{\ensuremath{ %
    {\conductivitytensor}_{i} %
}} % intra-cellular conductivity tensor e.g., \intracellularconductivitytensor => \sigma_i
\newcommand{\extracellularpotential}{\ensuremath{ %
    {\potentialsymbol}_{e} %
}} % extra-cellular potential e.g., \extracellularpotential => \phi_e
\newcommand{\extracellularstimuluscurrent}{\ensuremath{ %
    {\stimuluscurrent}_{e} %
}} % extra-cellular stimulus current e.g., \extracellularstimuluscurrent => {I_s}_e
\newcommand{\extracellularcurrentdensityvector}{\ensuremath{ %
    {\currentdensityvector}_{e} %
}} % extra-cellular current density vector e.g., \extracellularcurrentdensityvector => J_e
\newcommand{\extracellularconductivitytensor}{\ensuremath{ %
    {\conductivitytensor}_{e} %
}} % extra-cellular conductivity tensor e.g., \extracellularconductivitytensor => \sigma_e
\newcommand{\monodomainconductivitytensor}{\ensuremath{ %
    {\conductivitytensor}_{m} %
}} % monodomain conductivity tensor e.g., \monodomainconductivitytensor => \sigma_m
